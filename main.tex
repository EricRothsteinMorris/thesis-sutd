%%%%%%%%%%%%%%%%%%%%%%%%%%%%%%%%%%%%%%%%%
% SUTD Masters/Doctoral Thesis
% LaTeX Template 
% Version 1.0 (29/08/16)
%
% Adapted to SUTD requirements by Martin Ochoa
%
% This template is based on a template downloaded from:
% http://www.LaTeXTemplates.com
%
% Version 2.x major modifications by:
% Vel (vel@latextemplates.com)
%
% which in turn was based on a template by:
% Steve Gunn (http://users.ecs.soton.ac.uk/srg/softwaretools/document/templates/)
% Sunil Patel (http://www.sunilpatel.co.uk/thesis-template/)
% 
%
% Template license: 
% CC BY-NC-SA 3.0 (http://creativecommons.org/licenses/by-nc-sa/3.0/)
% 
%%%%%%%%%%%%%%%%%%%%%%%%%%%%%%%%%%%%%%%%%
 
%----------------------------------------------------------------------------------------
%	PACKAGES AND OTHER DOCUMENT CONFIGURATIONS
%----------------------------------------------------------------------------------------

\documentclass[
11pt, % The default document font size, options: 10pt, 11pt, 12pt
oneside, % Two side (alternating margins) for binding by default, uncomment to switch to one side
english, % ngerman for German
singlespacing, % Single line spacing, alternatives: onehalfspacing or doublespacing
%draft, % Uncomment to enable draft mode (no pictures, no links, overfull hboxes indicated)
%nolistspacing, % If the document is onehalfspacing or doublespacing, uncomment this to set spacing in lists to single
%liststotoc, % Uncomment to add the list of figures/tables/etc to the table of contents
%toctotoc, % Uncomment to add the main table of contents to the table of contents
parskip, % Uncomment to add space between paragraphs
%nohyperref, % Uncomment to not load the hyperref package
headsepline, % Uncomment to get a line under the header
]{MastersDoctoralThesis} % The class file specifying the document structure

\hyphenation{pa-ra-me-tri-sing}
\hyphenation{at-tack-er}
\usepackage[utf8]{inputenc} % Required for inputting international characters
\usepackage[T1]{fontenc} % Output font encoding for international characters

\usepackage{palatino} % Use the Palatino font by default

%\usepackage[backend=bibtex,style=authoryear,natbib=true,]{biblatex} % User the bibtex backend with the authoryear citation style (which resembles APA)
\usepackage[backend=bibtex,style=alphabetic, citestyle=alphabetic, natbib=true]{biblatex} 
%\usepackage[backend=bibtex,style=numeric, citestyle=numeric, natbib=true]{biblatex}  
 
\addbibresource{EGRM.bib} % The filename of the bibliography

\usepackage[autostyle=true]{csquotes} % Required to generate language-dependent quotes in the bibliography
%\usepackage{minted}
\usepackage{lscape}
\usepackage{mathtools}
\usepackage{amsfonts}
\usepackage{todonotes}
\usepackage{amsthm}
\usepackage{amssymb}
\usepackage{amsfonts}
\usepackage{stmaryrd}
\usepackage{mathrsfs}
\usepackage{upgreek}
\usepackage{proof}
\usepackage{subcaption}
\usepackage{wrapfig}
\usepackage{xurl} 
%\usepackage{mdsymbol}
\usepackage{algorithmicx}
%\usepackage{algorithmicx/algorithmicx}
% \usepackage{algorithm}
\usepackage[]{algorithm2e}
\LinesNumbered
\SetKwRepeat{Do}{do}{while}
\SetKw{Break}{break}
\SetKw{Input}{Input:}
\usepackage{algpseudocode}
\usepackage{tikz-cd}
\usepackage{tikz}

\usetikzlibrary{automata, positioning, arrows}


\tikzset{
->, % makes the edges directed
%>=stealth?, % makes the arrow heads bold
node distance=3cm, % specifies the minimum distance between two nodes. Change if necessary.
every state/.style={thick, fill=gray!10}, % sets the properties for each ?state? node
initial text=$ $, % sets the text that appears on the start arrow
}

\theoremstyle{definition}
\newtheorem{definition}{Definition}[section]
\newtheorem{proposition}{Proposition}[section]
\newtheorem{remark}{Remark}[chapter]
 
%CHAPTER INTRODUCTIOn
\newcommand{\branch}[3]{\ensuremath{{#2}\ {+_{#1}}\ {#3}}}
\newcommand{\iterate}[2]{\ensuremath{{#2}^{\left(#1\right)}}}
\newcommand{\bexp}[0]{\ensuremath{\text{BExp}}}
\newcommand{\gexp}[0]{\ensuremath{\text{Exp}}}
\newcommand{\usg}[0]{\ensuremath{\text{u}}}
\newcommand{\eval}[0]{\ensuremath{\mathtt{eval}}}
\newcommand{\sat}[0]{\ensuremath{\mathtt{sat}}}
\newcommand{\sg}[0]{\ensuremath{\text{s}}}
\newcommand{\set}[1]{\ensuremath{\left\{#1\right\}}}
\newcommand{\lbl}[0]{\ensuremath{\text{lbl}}}
\newcommand{\Real}[0]{\ensuremath{\mathbb{R}}}
\newcommand{\Z}[0]{\ensuremath{\mathbb{Z}}}
\newcommand{\Nat}[0]{\ensuremath{\mathbb{N}}}
\newcommand{\Bool}[0]{\ensuremath{2}}%{\ensuremath{\mathbb{B}}}
\newcommand{\letter}[0]{\ensuremath{\left(\text{A-Z\ |\ a-z}\right)}}
\newcommand{\Atom}[0]{\ensuremath{A}}%{\text{At}}}
\newcommand{\GuardedString}[0]{\ensuremath{\text{GS}}}
\newcommand{\RC}[0]{\ensuremath{\text{RC}}}
\newcommand{\Low}[0]{\ensuremath{\text{Low}}}
\newcommand{\Variable}[0]{\ensuremath{\mathscr{V}}}
\newcommand{\Integer}[0]{\ensuremath{\mathbb{Z}}}
\newcommand{\alphanumeric}[0]{\ensuremath{\left(\text{A-Z\ |\ a-z\ |\ 0-9}\right)}}
\newcommand{\alphanumericP}[0]{\ensuremath{\left(\text{A-Z\ |\ a-z\ |\ 0-9\ |\ .\ |\ \_\ |\ \$}\right)}}
\newcommand{\semantics}[1]{\ensuremath{\llbracket #1\rrbracket}}
\newcommand{\valuation}[0]{\ensuremath{\Gamma}}
\newcolumntype{L}{>{$}l<{$}} % math-mode version of "l" column type
\newcolumntype{R}{>{$}r<{$}} % math-mode version of "r" column type
\newcolumntype{C}{>{$}c<{$}} % math-mode version of "c" column type
%\newcommand{\hourglass}[0]{}%{\LARGE\fontspec{Cambria}^^^^231b}
\newcommand{\timeequiv}[0]{\equiv_{\hourglass}}
%END CHAPTER INTRO

\newcommand{\Monad}{{\mathbb{\MFunctor}}}
\newcommand{\Functor}{F}
\newcommand{\DTS}{DTS}
\newcommand{\MFunctor}{T}
\newcommand{\TheLatentBehaviourOf}[2]{{\TheBehaviourOf{#1}^{#2}}}%{{{#1}_{beh}}}
%\newcommand{\TheLatentBehaviourOfIn}[3]{{{\llbracket#1\rrbracket}}_{({#2}}}%{{{#1}_{beh}}}
\newcommand{\TheBehavioursOf}[1]{{{\llbracket\TheFinalCoalgebraOf{#1}\rrbracket}}}
%\newcommand{\TheBehaviourOf}[1]{{{\llbracket#1\rrbracket}}}%{{{#1}_{beh}}}
\newcommand{\TheBehaviourOfIn}[2]{{{\llbracket#1\rrbracket}}_{#2}}%{{{#1}_{beh}}}
\newcommand*{\supp}{{\texttt{supp}}}
%\newcommand{\Powerset}{{\mathscr{P}}}
%\newcommand{\FinitePowerset}{{\Powerset_\omega}}
%\MakeRobust{\Haskell}
%Notation for Sets of Elements
% \newcommand{\SomeSet}[1]{
% 	{\ifthenelse{\equal{#1}{1}}
% 		{{X}}
% 		{\ifthenelse{\equal{#1}{2}}
% 		{{Y}}
% 		{\ifthenelse{\equal{#1}{3}}
% 		{{Z}}
% 		{{#1}}}}
% 	}
% }
% \MakeRobust{\SomeSet}
% \newcommand*{\TheSet}{{\SomeSet{1}}}
% \newcommand{\AsSet}[1]{{{#1_\AsCategory{Set}}}}
% \newcommand*{\0}{{0_\AsCategory{Set}}}
% \newcommand*{\1}{{1_\AsCategory{Set}}}


%-----Commands for annotation
\newcommand{\marker}[1]{
 \noindent{\newline \color{green}
 \framebox[\textwidth][t]{%
  \parbox[t]{0.9\textwidth}{\textcolor{green}{MARKER: #1}}
}}}

\newcommand{\review}[1]{
 \noindent{\newline \color{blue}
 \framebox[\textwidth][t]{%
  \parbox[t]{0.9\textwidth}{\textcolor{blue}{Note to Reviewers: #1}}
}}}

%Commands for...
%COMMANDS BY ERIC



\newcommand{\nat}{{\omega}}
\newcommand{\head}{{\texttt{head}}}
\newcommand{\forget}{{\texttt{forget}}}
\newcommand{\tail}{{\texttt{tail}}}
\newcommand{\unde}{{\texttt{undef}}}
\newcommand{\has}{{\mathbf{has}}}
\newcommand{\out}{{\mathbf{output}}}
\newcommand{\add}{{\mathbf{add}}}
\newcommand{\del}{{\mathbf{del}}}
\newcommand{\ins}{{\mathbf{instruction}}}
\newcommand{\bool}{{\mathbb{B}}}%{\mathbf{\mathtt{bool}}}%
\newcommand{\boolUndef}{\{\true,\false\}\sqcup\{\unde\}}%{{\mathbb{B}}}
\newcommand{\xeq}[1]{\ensuremath{\stackrel{#1}{=}}}
\newcommand{\xneq}[1]{\ensuremath{\stackrel{#1}{\neq}}}
\newcommand{\ofsigma}[1]{{\mathcal{#1}(\Sigma)}}
\newcommand\numberthis{\addtocounter{equation}{1}\tag{\theequation}}
\newcommand{\Lift}{\AsFunctor{Lift}}
\newcommand{\Trans}{\AsFunctor{Tr}}
\newcommand{\Places}{\AsFunctor{Pl}}
\newcommand{\Slift}{\AsFunctor{S}}%\mathpzc{abcdefghijklmnopqrstuvwxyz}}
%\newcommand{\slift}{\mathcal{S}\AsCoalgebra{habbat}\mathcal{L}\AsCoalgebra{ift}}
\newcommand{\AsRule}[1]{{{\textsc{#1}}}}
\MakeRobust{\AsRule}
\newcommand{\AsFunction}[1]{{\textsc{#1}}}



%ASMNotation
\newcommand{\true}{\mathtt{\texttt{true}}}
\newcommand{\false}{\mathtt{\texttt{false}}}
\newcommand{\fst}{\mathtt{\texttt{fst}}}
\newcommand{\snd}{\mathtt{\texttt{snd}}}
\newcommand{\id}{{\texttt{id}}}
\newcommand{\Root}{\mathtt{\texttt{root}}}
\newcommand{\Observation}{\mathtt{\texttt{obs}}}
\newcommand{\Configuration}{\mathtt{\texttt{conf}}}
\newcommand{\Exec}{\mathtt{\texttt{run}}}
\newcommand{\Pack}{\mathtt{\texttt{pack}}}
\newcommand{\Powerset}{{\mathscr{P}}}
\newcommand{\FinitePowerset}{{\Powerset_\omega}}
\newcommand{\ASM}{\AsFunctor{ASM}}
\newcommand{\SeqRec}{\AsFunctor{SeqRec}}
\newcommand*{\Automata}{\AsFunctor{Au}}
\newcommand{\TheAlgorithm}{\AsFunctor{A}}
\newcommand{\TheTrace}{\tau}
\newcommand{\AbstractStates}{\AsFunctor{AbstractStates}}
\newcommand{\TheASM}{\AsFunctor{M}}
\newcommand{\AbstractStateCategory}{\AsCategory{State}}
\newcommand{\TheSignature}{\Sigma}
%\newcommand{\TheVocabulary}{\Pi}
\newcommand{\TheSetOfVariables}{\AsFunctor{V}}
\newcommand{\TheSetOfRuleInstances}{\mathbf{R}}
\newcommand{\TheRuleInstance}{\MakeLowercase{\TheSetOfRuleInstances}}
\newcommand{\BisimilarityIn}[1]{\sim_{#1}}
\newcommand{\Forget}{\mathtt{forget}}
\newcommand{\Learn}{\mathtt{learn}}
\newcommand{\Force}{\mathtt{force}}
\newcommand{\Causal}{\mathtt{causal}}
\newcommand{\TheRule}{r}
\newcommand{\TheVariable}{\MakeLowercase{\TheSetOfVariables}}
\newcommand{\TheSuperuniverse}{\mathcal{S}}
\newcommand{\TheLanguage}{\mathcal{L}}
\newcommand{\TheValue}{s}
\newcommand{\formula}[3]{\llbracket #1 \rrbracket^\AsAlgebra{#2}_#3}
\newcommand{\TheSetOfLocations}{\mathbf{L}}
\newcommand{\TheSetOfUpdates}{\mathbf{U}}
\newcommand{\TheSubsetOfUpdates}{U}
\newcommand{\ASMRules}{\mathbf{\TheSetOfRules}}
\newcommand{\TheLocation}{\MakeLowercase{\TheSetOfLocations}}
\newcommand{\TheRelationLocation}{\MakeLowercase{\TheSetOfRelationLocations}}
\newcommand{\TransitionRule}{\MakeLowercase{\ASMRules}}
\newcommand{\TheSetOfTerms}{\mathbf{T}}
\newcommand{\TheSetOfFormulas}{\mathbf{F}}
\newcommand{\TheTerm}{t}
\newcommand{\TheVariableAssignment}{\zeta}


%Notation for Elements
\newcommand{\SomeElement}[1]{
	{
	\ifthenelse{\equal{#1}{1}}
		{{x}}
		{\ifthenelse{\equal{#1}{2}}
		{{y}}
		{\ifthenelse{\equal{#1}{3}}
		{{z}}
		{{{#1}}}}}
	}
}
\MakeRobust{\SomeElement}
\newcommand{\TheElement}{{\SomeElement{1}}}

%Notation for Sets of Elements
\newcommand{\SomeSet}[1]{
	{\ifthenelse{\equal{#1}{1}}
		{{X}}
		{\ifthenelse{\equal{#1}{2}}
		{{Y}}
		{\ifthenelse{\equal{#1}{3}}
		{{Z}}
		{{#1}}}}
	}
}
\MakeRobust{\SomeSet}
\newcommand{\TheCategoryOfSets}{{\AsCategory{Set}}}
\newcommand{\TheSet}{{\SomeSet{1}}}
\newcommand{\AsSet}[1]{{{#1_\TheCategoryOfSets}}}
\newcommand{\0}{{0_\TheCategoryOfSets}}
\newcommand{\1}{{1_\TheCategoryOfSets}}

\newcommand{\ThePowersetOf}[1]{{\Powerset\!\left({#1}\right)}}
\newcommand{\TheFinitePowersetOf}[1]{{\FinitePowerset\left(#1\right)}}

%Notation for Predicates
\newcommand{\SomePredicate}[1]{
	{\ifthenelse{\equal{#1}{1}}
		{{\AsRule{P}}}
		{\ifthenelse{\equal{#1}{2}}
		{{\AsRule{Q}}}
		{\ifthenelse{\equal{#1}{3}}
		{{\AsRule{R}}}
		{\AsRule{#1}}}}
	}
}
\MakeRobust{\SomePredicate}
\newcommand{\ThePredicate}{{\SomePredicate{1}}}
\newcommand{\TheProperty}{\ThePredicate}
\newcommand{\SomeProperty}[1]{\SomePredicate{#1}}
\newcommand{\AsPredicate}[1]{{{#1_\AsCategory{Pred}}}}
\newcommand{\ThePredicatesOf}[1]{{\mathscr{P}_{#1}}}


%Notation for Relations
\newcommand{\SomeRelation}[1]{
	{\ifthenelse{\equal{#1}{1}}
		{{R}}
		{\ifthenelse{\equal{#1}{2}}
		{{S}}
		{\ifthenelse{\equal{#1}{3}}
		{{T}}
		{{#1}}}}
	}
}
\MakeRobust{\SomeRelation}
\newcommand{\TheRelation}{{\SomeRelation{1}}}
\newcommand{\AsRelation}[1]{{{#1_\AsCategory{Rel}}}}

%Notation for Functions
\newcommand{\SomeFunction}[1]{
	{\ifthenelse{\equal{#1}{1}}
		{f}
		{\ifthenelse{\equal{#1}{2}}
		{g}
		{\ifthenelse{\equal{#1}{3}}
		{h}
		{{#1}}}}
	}
}
\MakeRobust{\SomeFunction}
\newcommand{\TheFunction}{\SomeFunction{1}}
\newcommand{\TheHomomorphism}{{\SomeFunction{3}}}
\newcommand{\ProjectionFunction}{\pi}
\newcommand{\InjectionFunction}{\kappa}

%Notation for Categories
\newcommand{\SomeCategory}[1]{
	{\ifthenelse{\equal{#1}{1}}
		{{\mathbf{C}}}
		{\ifthenelse{\equal{#1}{2}}
		{{\mathbf{D}}}
		{\ifthenelse{\equal{#1}{3}}
		{{\mathbf{E}}}
		{{\mathbf{#1}}}}}
	}
}
\MakeRobust{\SomeCategory}
\newcommand{\TheCategory}{{\SomeCategory{1}}}
\newcommand{\AsCategory}[1]{{\SomeCategory{#1}}}
\newcommand{\Obj}{{\SomeCategory{Obj}}}

%Notation for Functors
\newcommand{\AsFunctor}[1]{\mathpzc{#1}} 
\newcommand{\SomeFunctor}[1]{
	{
	\ifthenelse{\equal{#1}{1}}
		{\AsFunctor{F}}
		{\ifthenelse{\equal{#1}{2}}
		{\AsFunctor{G}}
		{\ifthenelse{\equal{#1}{3}}
		{\AsFunctor{H}}
		{\AsFunctor{{#1}}}}}
	}
}
\MakeRobust{\SomeFunctor}
\newcommand{\TheFunctor}{{\SomeFunctor{1}}}
\newcommand{\Moore}{\AsFunctor{Au}}
\newcommand{\Mealy}{\AsFunctor{Me}}

%Notation for Algebras
\newcommand{\AsAlgebra}[1]{
	{\mathfrak{{#1}}}
}
\newcommand{\SomeAlgebra}[1]
{
	{\ifthenelse{\equal{#1}{1}}
	{\AsAlgebra{A}}
	{\ifthenelse{\equal{#1}{2}}
		{\AsAlgebra{B}}
		{\ifthenelse{\equal{#1}{3}}
			{\AsAlgebra{C}}
			{\PackageWarning{Preamble}{non-standard algebra symbol}#1}
		}
	}
	}
}
\MakeRobust{\SomeAlgebra}
\newcommand{\TheAlgebra}{{\SomeAlgebra{1}}}
\newcommand{\TheInitialAlgebraOf}[1]{{{0}_{#1}}}


%Notation for Coalgebras
\newcommand{\AsCoalgebra}[1]{
	{\mathbb{{#1}}}
}
\newcommand{\SomeCoalgebra}[1]{
	{\AsCoalgebra{\MakeUppercase{\SomeSet{#1}}}}
}
\MakeRobust{\SomeCoalgebra}
\newcommand{\TheCoalgebra}{{\SomeCoalgebra{1}}}
\newcommand{\TheFinalCoalgebraOf}[1]{{{1}_{#1}}}

%Notation for Inputs %\PackageError{Preamble}{non-standard input symbol} //Use \PackageWarning if you want...
\newcommand{\SomeInput}[1]
{
	{\ifthenelse{\equal{#1}{1}} 
	{i}
	{\ifthenelse{\equal{#1}{2}}
		{j}
		{\ifthenelse{\equal{#1}{3}}
			{k}
			{\PackageWarning{Preamble}{non-standard input symbol}#1}
		}
	}
	}
}
\MakeRobust{\SomeInput}
\newcommand{\TheInput}{{\SomeInput{1}}}
\newcommand{\TheHighInput}{\SomeInput{h}} 
\newcommand{\TheLowInput}{\SomeInput{l}} 

%Notation for Sets of Inputs
\newcommand{\SomeSetOfInputs}[1]
{
	\ifthenelse{\equal{#1}{1}}
	{A}
	{\ifthenelse{\equal{#1}{2}}
		{B}
		{\ifthenelse{\equal{#1}{3}}
			{C}
			{\PackageWarning{Preamble}{non-standard set of inputs symbol}#1}
		}
	}
}
\MakeRobust{\SomeSetOfInputs}
\newcommand{\TheSetOfInputs}{I}
\newcommand{\TheSetOfModes}{M}
\newcommand{\TheSetOfHighInputs}{\SomeSetOfInputs{H}}
\newcommand{\TheSetOfLowInputs}{\SomeSetOfInputs{L}}
\newcommand{\TheSemilatticeOfInputs}{{\AsCoalgebra{\TheSetOfInputs}}}

%Notation for Input Sequences
\newcommand{\SomeSequenceOfInputs}[1]{
{
	\ifthenelse{\equal{#1}{1}}
	{w}
	{\ifthenelse{\equal{#1}{2}}
		{u}
		{\ifthenelse{\equal{#1}{3}}
			{v}
			{\PackageWarning{Preamble}{non-standard sequence of inputs symbol}#1}
		}
	}
}
}
\MakeRobust{\SomeSequenceOfInputs}
\newcommand{\TheSequenceOfInputs}{{\SomeSequenceOfInputs{1}}}
\newcommand{\TheSequenceOfHighInputs}{\SomeSequenceOfInputs{\pi}}
\newcommand{\TheSequenceOfLowInputs}{\SomeSequenceOfInputs{\lambda}}
\newcommand{\SomeInputSequence}[1]{\SomeSequenceOfInputs{#1}}
\newcommand{\TheInputSequence}{\TheSequenceOfInputs}

%Notation for Sets of Input Sequences
\newcommand{\SomeSetOfSequencesOfInputs}[1]{\SomeSetOfInputs{#1}^{*}}
\MakeRobust{\SomeSetOfSequencesOfInputs}
\newcommand{\TheSetOfSequencesOfInputs}{{\SomeSetOfSequencesOfInputs{1}}}
\newcommand{\TheSetOfSequencesOfHighInputs}{{\SomeSetOfSequencesOfInputs{H}}}
\newcommand{\TheSetOfSequencesOfLowInputs}{{\SomeSetOfSequencesOfInputs{L}}}
\newcommand{\TheSetOfFiniteInputSequences}{\TheSetOfSequencesOfInputs}
\newcommand{\TheSetOfHighInputSequences}{\TheSetOfSequencesOfHighInputs}
\newcommand{\TheSetOfLowInputSequences}{\TheSetOfSequencesOfLowInputs}
\newcommand{\SomeSetOfInputSequences}[1]{\SomeSetOfSequencesOfInputs{#1}}

%Notation for Input Streams
\newcommand{\SomeStreamOfInputs}[1]{
	\ifthenelse{\equal{#1}{1}}
		{\sigma}
		{\ifthenelse{\equal{#1}{2}}
		{\tau}
		{\ifthenelse{\equal{#1}{3}}
		{\rho}
		{{#1}}}}
	}
\MakeRobust{\SomeStreamOfInputs}
\newcommand{\TheStreamOfInputs}{{\SomeStreamOfInputs{1}}}
\newcommand{\TheInputStream}{\TheStreamOfInputs}
\newcommand{\SomeInputStream}[1]{\SomeStreamOfInputs{#1}}

%Notation for Sets of Input Streams
\newcommand{\SomeSetOfStreamsOfInputs}[1]{\SomeSetOfInputs{#1}^{\nat}}
\MakeRobust{\SomeSetOfStreamsOfInputs}
\newcommand{\TheSetOfStreamsOfInputs}{{\SomeSetOfStreamsOfInputs{1}}}
\newcommand{\TheSetOfStreamsOfHighInputs}{{\SomeSetOfStreamsOfInputs{H}}}
\newcommand{\TheSetOfStreamsOfLowInputs}{{\SomeSetOfStreamsOfInputs{L}}}
\newcommand{\TheSetOfInputStreams}{\TheSetOfStreamsOfInputs}
\newcommand{\TheSetOfHighInputStreams}{\TheSetOfStreamsOfHighInputs}
\newcommand{\TheSetOfLowInputStreams}{\TheSetOfStreamsOfLowInputs}
\newcommand{\SomeSetOfInputStreams}[1]{\SomeSetOfStreamsOfInputs{#1}}

%Notation for Outputs 
\newcommand{\SomeOutput}[1]
{
	\ifthenelse{\equal{#1}{1}}
	{o}
	{\ifthenelse{\equal{#1}{2}}
		{p}
		{\ifthenelse{\equal{#1}{3}}
			{q}
			{\PackageWarning{Preamble}{non-standard output symbol}#1}
		}
	}
}
\MakeRobust{\SomeOutput}
\newcommand{\TheOutput}{\SomeOutput{1}} 
\newcommand{\TheHighOutput}{\SomeOutput{h}} 
\newcommand{\TheLowOutput}{\SomeOutput{l}} 

%Notation for Sets of Outputs
\newcommand{\SomeSetOfOutputs}[1]
{
	\ifthenelse{\equal{#1}{1}}
	{O}
	{\ifthenelse{\equal{#1}{2}}
		{P}
		{\ifthenelse{\equal{#1}{3}}
			{Q}
			{\PackageWarning{Preamble}{non-standard set of outputs symbol}#1}
		}
	}
}
\MakeRobust{\SomeSetOfOutputs}
\newcommand{\TheSetOfOutputs}{{\SomeSetOfOutputs{1}}}
\newcommand{\TheSetOfHighOutputs}{\SomeSetOfOutputs{H}}
\newcommand{\TheSetOfLowOutputs}{\SomeSetOfOutputs{L}}
\newcommand{\TheSemilatticeOfOutputs}{{\AsCoalgebra{\TheSetOfOutputs}}}

%Notation for Output Sequences
\newcommand{\SomeSequenceOfOutputs}[1]
{
	\ifthenelse{\equal{#1}{1}}
	{\underline{w}}
	{\ifthenelse{\equal{#1}{2}}
		{\underline{u}}
		{\ifthenelse{\equal{#1}{3}}
			{\underline{v}}
			{\PackageWarning{Preamble}{non-standard sequence of outputs symbol} \underline{#1}}
		}
	}
}
\MakeRobust{\SomeSequenceOfOutputs}
\newcommand{\TheSequenceOfOutputs}{\SomeSequenceOfOutputs{1}}
\newcommand{\TheSequenceOfHighOutputs}{\SomeSequenceOfOutputs{\pi}}
\newcommand{\TheSequenceOfLowOutputs}{\SomeSequenceOfOutputs{\lambda}}
\newcommand{\SomeOutputSequence}[1]{\SomeSequenceOfOutputs{#1}}


%Notation for Sets of Output Sequences
\newcommand{\SomeSetOfSequencesOfOutputs}[1]{\SomeSetOfOutputs{#1}^{*}}
\MakeRobust{\SomeSetOfSequencesOfOutputs}
\newcommand{\TheSetOfSequencesOfOutputs}{{\SomeSetOfSequencesOfOutputs{1}}}
\newcommand{\TheSetOfOutputSequences}{\TheSetOfSequencesOfOutputs}
\newcommand{\TheSetOfSequencesOfHighOutputs}{{\SomeSetOfSequencesOfOutputs{\underline{H}}}}
\newcommand{\TheSetOfSequencesOfLowOutputs}{{\SomeSetOfSequencesOfOutputs{\underline{L}}}}
\newcommand{\TheSetOfHighOutputSequences}{\TheSetOfSequencesOfHighOutputs}
\newcommand{\TheSetOfLowOutputSequences}{\TheSetOfSequencesOfLowOutputs}

%Notation for Exceptions 
\newcommand{\SomeException}[1]
{
	\ifthenelse{\equal{#1}{1}}
	{e}
	{\ifthenelse{\equal{#1}{2}}
		{d}
		{\ifthenelse{\equal{#1}{3}}
			{c}
			{\PackageWarning{Preamble}{non-standard Exception symbol}#1}
		}
	}
}
\MakeRobust{\SomeException}
\newcommand{\TheException}{\SomeException{1}} 
\newcommand{\TheHighException}{\SomeException{h}} 
\newcommand{\TheLowException}{\SomeException{l}} 

%Notation for Sets of Exceptions
\newcommand{\SomeSetOfExceptions}[1]
{
	\ifthenelse{\equal{#1}{1}}
	{E}
	{\ifthenelse{\equal{#1}{2}}
		{F}
		{\ifthenelse{\equal{#1}{3}}
			{G}
			{\PackageWarning{Preamble}{non-standard set of exceptions symbol}#1}
		}
	}
}
\MakeRobust{\SomeSetOfExceptions}
\newcommand{\TheSetOfExceptions}{{\SomeSetOfExceptions{1}}}
\newcommand{\TheSemilatticeOfExceptions}{{\AsCoalgebra{\TheSetOfExceptions}}}


%Notation for Semirings
\newcommand{\AsSemiring}[1]{\AsCoalgebra{#1}}
\newcommand{\SomeSemiring}[1]{\AsSemiring{\MakeUppercase{\SomeSet{#1}}}}
\MakeRobust{\SomeSemiring}
\newcommand{\TheSemiring}{\SomeSemiring{1}}
\newcommand{\TheRingOfIntegers}{{\AsSemiring{Z}}}
\newcommand{\Addition}{+}
\newcommand{\Multiplication}{\times}
\newcommand{\Zero}{0}
\newcommand{\One}{1}
\newcommand{\Undef}{\perp}

%Notation for FPS
\newcommand{\SomeFPS}[1]{
	\ifthenelse{\equal{#1}{1}}
		{\sigma}
		{\ifthenelse{\equal{#1}{2}}
		{\gamma}
		{\ifthenelse{\equal{#1}{3}}
		{\rho}
		{{#1}}}}
	}
\MakeRobust{\SomeFPS}
\newcommand{\TheFPS}{\SomeFPS{1}}
\newcommand{\TheFPSSemiring}{\AsSemiring{F}}
\newcommand{\FPS}{{\textsc{fps}}}

%Notation for Behaviours
\newcommand{\SomeBehaviour}[1]{{\SomeFPS{#1}}}
\MakeRobust{\SomeBehaviour}
\newcommand{\TheBehaviour}{{\SomeBehaviour{1}}}
\newcommand{\TheBehaviourOf}[1]{{{\llbracket#1\rrbracket}}}%{{{#1}_{beh}}}


%Notation for Sequences
\newcommand{\lseq}{\left<}
\newcommand{\rseq}{\right>}
\newcommand{\AsSequence}[1]{{{\left<#1\right>}}}
\newcommand{\AsTuple}[1]{{\AsSequence{#1}}}
\newcommand{\AsPair}[1]{{\AsTuple{#1}}}

%Notation for IMs
\newcommand{\Countable}{\omega}%\mathbb{N}}
\newcommand{\Indexes}{V}%\mathbb{N}}
\newcommand{\SomeIndexes}{I}
\newcommand{\Index}{\MakeLowercase{\Indexes}}
\newcommand{\SomeIndex}{\MakeLowercase{\SomeIndexes}}
\newcommand{\xOutputs}{{\left(\1+\Outputs\right)}}%{\overline{\Outputs}}
\newcommand{\OutputSequence}{\rho}%{\{\perp\}}
\newcommand{\OutputFunction}{\theta}%{\{\perp\}}
\newcommand{\xOutputSequence}{\rho}%{\{\perp\}}
\newcommand{\IMone}{\IM^{\text{\ding{172}}}}
\newcommand{\Count}{\mathtt{count}}

\newcommand{\plc}{\texttt{PLC}}
\newcommand{\attack}{\texttt{attack}}
\newcommand{\High}{\mathcal{H}}
\newcommand{\True}{\texttt{True}}
\newcommand{\const}{{\Delta}}%{{\AsFunction{const}}}
\newcommand{\TheRequirement}{\ensuremath{\mathcal{P}}}
\newcommand{\TheSetOfRequirements}{\ensuremath{\mathbb{P}}}
\newcommand{\IM}[1]{\ensuremath{\text{IM}\!\AsSequence{#1}}}
\newcommand{\IMCT}{\IM{\Gamma_{\mathcal{K}}}}
\newcommand{\TheSystem}{{\TheBehaviourOf{\cdot}}}


%TIKZ
\usepackage{tikz}
\usepackage{tikz-cd}
\usetikzlibrary{arrows,automata, positioning, petri, shapes, decorations} 
% \usetikzlibrary{arrows}
% \usetikzlibrary{positioning}
% \usetikzlibrary{petri}
% \usetikzlibrary{shapes,snakes}


%DashedArrows
\makeatletter
\newcommand{\da@rightarrow}{\mathchar"0\hexnumber@\symAMSa 4B }
\newcommand{\da@leftarrow}{\mathchar"0\hexnumber@\symAMSa 4C }
\newcommand{\xdashedrightarrow}[2][]{%
  \mathrel{%
    \mathpalette{\da@xarrow{#1}{#2}{}\da@rightarrow{\,}{}}{}%
  }%
}
\newcommand{\xdashedleftarrow}[2][]{%
  \mathrel{%
    \mathpalette{\da@xarrow{#1}{#2}\da@leftarrow{}{}{\,}}{}%
  }%
}
\newcommand{\da@xarrow}[7]{%
  % #1: below
  % #2: above
  % #3: arrow left
  % #4: arrow right
  % #5: space left 
  % #6: space right
  % #7: math style 
  \sbox0{$\ifx#7\scriptstyle\scriptscriptstyle\else\scriptstyle\fi#5#1#6\m@th$}%
  \sbox2{$\ifx#7\scriptstyle\scriptscriptstyle\else\scriptstyle\fi#5#2#6\m@th$}%
  \sbox4{$#7\dabar@\m@th$}%
  \dimen@=\wd0 %
  \ifdim\wd2 >\dimen@
    \dimen@=\wd2 %   
  \fi
  \count@=2 %
  \def\da@bars{\dabar@\dabar@}%
  \@whiledim\count@\wd4<\dimen@\do{%
    \advance\count@\@ne
    \expandafter\def\expandafter\da@bars\expandafter{% 
      \da@bars
      \dabar@ 
    }%
  }%  
  \mathrel{#3}%
  \mathrel{%   
    \mathop{\da@bars}\limits
    \ifx\\#1\\% 
    \else
      _{\copy0}%
    \fi
    \ifx\\#2\\%
    \else
      ^{\copy2}%
    \fi
  }%   
  \mathrel{#4}%
}

%DoublePointed arrows
\makeatletter
\providecommand*{\twoheadrightarrowfill@}{%
  \arrowfill@\relbar\relbar\twoheadrightarrow
}
\providecommand*{\twoheadleftarrowfill@}{%
  \arrowfill@\twoheadleftarrow\relbar\relbar
}
\providecommand*{\xtwoheadrightarrow}[2][]{%
  \ext@arrow 0579\twoheadrightarrowfill@{#1}{#2}%
}
\providecommand*{\xtwoheadleftarrow}[2][]{%
  \ext@arrow 5097\twoheadleftarrowfill@{#1}{#2}%
}
\makeatother

\newtheorem{example}{Example}[chapter]
\newtheorem{question}{Research Question}[chapter]
\newtheorem{theorem}{Theorem}[chapter]
\newtheorem{corollary}{Corollary}[chapter]
%\newenvironment{haskell}[1]{\begin{minted}{haskell}#1}{\end{minted}}


%-----------NOTATION FOR CLASSIFICATION--------------
\usepackage{lscape}
\usepackage{wrapfig}
\usepackage{framed}

\usepackage{tikz}
\usepackage{mathtools}
\usepackage{amssymb} 
\usepackage{todonotes} 
\usepackage{multirow}
\usepackage{mathrsfs}
\usepackage{mathpartir}
\usepackage{soul}
\setcounter{tocdepth}{3}

\newcommand{\Always}{{\Globally}}%{{\mathbf{G}}}
\newcommand{\Globally}{{\square}}%{{\mathbf{G}}}
\newcommand{\vect}[1]{\ensuremath{\vec{#1}}}%{\overrightarrow{\bf{#1}}}}
\newcommand\sbullet[1][.5]{\mathbin{\vcenter{\hbox{\scalebox{#1}{$\bullet$}}}}}


 
%------------NOTATION FOR REPAIR------------------------------
\newcommand{\iteration}[2]{\ensuremath{{#2}^{\left(#1\right)}}}
%\newcommand{\bexp}[0]{\ensuremath{ \mathscr{B}}}


% \newcommand{\sat}[0]{\ensuremath{\mathtt{sat}}}
% \newcommand{\sg}[0]{\ensuremath{\text{s}}}
% \newcommand{\set}[1]{\ensuremath{\left\{#1\right\}}}
% \newcommand{\lbl}[0]{\ensuremath{\text{lbl}}}
% \newcommand{\Real}[0]{\ensuremath{\mathbb{R}}}
% \newcommand{\Bool}[0]{\ensuremath{\mathbb{B}}}
% \newcommand{\Nat}[0]{\ensuremath{\mathbb{N}}}
% \newcommand{\letter}[0]{\ensuremath{\left(\text{A-Z\ |\ a-z}\right)}}
% \newcommand{\Atom}[0]{\ensuremath{\text{At}}}
% \newcommand{\GuardedString}[0]{\ensuremath{\text{GS}}}
% \newcommand{\RC}[0]{\ensuremath{\texttt{RC}}}
\newcommand{\sel}[0]{\ensuremath{\textbf{sel}}}
% \newcommand{\Low}[0]{\ensuremath{\text{Low}}}
\newcommand{\cache}[0]{\ensuremath{\texttt{mem}}}
\newcommand{\Variables}[0]{\ensuremath{\mathscr{V}}}
\newcommand{\Structures}[0]{\ensuremath{\mathscr{S}}}
% \newcommand{\Integer}[0]{\ensuremath{\mathbb{Z}}} 
% \newcommand{\alphanumeric}[0]{\ensuremath{\left(\text{A-Z\ |\ a-z\ |\ 0-9}\right)}}
% \newcommand{\alphanumericP}[0]{\ensuremath{\left(\text{A-Z\ |\ a-z\ |\ 0-9\ |\ .\ |\ \_\ |\ \$}\right)}}
%\newcommand{\semantics}[1]{\ensuremath{\left\llbracket #1\right\rrbracket}}
\newcommand{\irsemantics}[1]{\ensuremath{\semantics{#1}_{\texttt{IR}}}}
\newcommand{\gkatsemantics}[1]{\ensuremath{\semantics{#1}_{\texttt{exp}}}}
%\newcommand{\valuation}[0]{\ensuremath{\Gamma}}
%\newcolumntype{L}{>{$}l<{$}} % math-mode version of "l" column type
%\newcolumntype{R}{>{$}r<{$}} % math-mode version of "r" column type
%\newcolumntype{C}{>{$}c<{$}} % math-mode version of "c" column type
\newcommand{\hourglass}[0]{}%{\LARGE\fontspec{Cambria}^^^^231b}
\makeatletter
%----------------------------------------------------------------------------------------
%	MARGIN SETTINGS
%----------------------------------------------------------------------------------------

\geometry{
	paper=a4paper, % Change to letterpaper for US letter
	inner=2.54cm, % Inner margin
	outer=2.8cm, % Outer margin
	bindingoffset=1cm, % Binding offset
	top=2.5cm, % Top margin
	bottom=2.5cm, % Bottom margin
	%showframe,% show how the type block is set on the page
}

%---------------------------------------------------------------------------------------- 
%	THESIS INFORMATION
%----------------------------------------------------------------------------------------

\thesistitle{Latent Behaviour Analysis}% and its Applications to Security and Design} % Your thesis title, this is used in the title and abstract, print it elsewhere with \ttitle
\supervisor{Dr. Sudipta \textsc{Chattopadhyay}} % Your supervisor's name, this is used in the title page, print it elsewhere with \supname
\examiner{} % Your examiner's name, this is not currently used anywhere in the template, print it elsewhere with \examname
\degree{Doctor of Philosophy} % Your degree name, this is used in the title page and abstract, print it elsewhere with \degreename
\author{Eric G. \textsc{Rothstein-Morris}} % Your name, this is used in the title page and abstract, print it elsewhere with \authorname
\addresses{} % Your address, this is not currently used anywhere in the template, print it elsewhere with \addressname
\university	{SUTD}
\keywords{} % Keywords for your thesis, this is not currently used anywhere in the template, print it elsewhere with \keywordnames
\pillar{Information Systems Technology and Design (ISTD)} % Pillar

\hypersetup{pdftitle=\ttitle} % Set the PDF's title to your title
\hypersetup{pdfauthor=\authorname} % Set the PDF's author to your name
\hypersetup{pdfkeywords=\keywordnames} % Set the PDF's keywords to your keywords

\begin{document}

\frontmatter % Use roman page numbering style (i, ii, iii, iv...) for the pre-content pages

\pagestyle{plain} % Default to the plain heading style until the thesis style is called for the body content

%----------------------------------------------------------------------------------------
%	TITLE PAGE
%----------------------------------------------------------------------------------------

\begin{titlepage}
\begin{center}

\begin{figure}[t]
\centering
\includegraphics[width=0.5\textwidth]{Figures/SUTD}\\
\vspace{1cm}
% \end{figure}
% \hfill\break\\[1cm]%[2.3cm]
% \begin{figure}[!h]
	\centering
    % \begin{tikzcd}[column sep=1.75cm, row sep=1cm]
    %     & 
    %     &\sigma F
    %         \arrow[dr,swap,"\omega"']
    %     &
    %     \\ 
    %     \sigma F
    %         %\arrow[dd, "\simeq","\omega"'] 
    %         %\arrow[dd, "\simeq"', "!"]
	% 		\arrow[dd, "1"]
    %         %\arrow[d, "c"] 
    %     & X  
    %         \arrow[l, dotted, swap,"\TheBehaviourOf{\cdot}_{c}"]
    %         \arrow[dd,"c"] 
    %     & X
    %         \arrow[l, swap, "m"]
    %         \arrow[u, "\TheBehaviourOf{\cdot}_{c}"]
    %         \arrow[r,dotted, swap, "\TheBehaviourOf{\cdot}_{b\circ c\circ m}"]
    %     &\sigma F 
    %         %\arrow[dd, "\simeq","\omega"']
    %         \arrow[dd, "1"]
    %         \arrow[dr, "\textbf{id}"]
    %     \\
    %     &&&&\sigma(F)
    %     \\
    %     F(\sigma F)
    %         %
    %     &F(X)     
    %         \arrow[l, dotted, "F(\TheBehaviourOf{\cdot}_{c})"]
    %         \arrow[r, swap, "b"]
    %     &F(X)
    %         \arrow[d, swap, "F(\TheBehaviourOf{\cdot}_{c})"]
    %         \arrow[r, dotted, "F(\TheBehaviourOf{\cdot}_{b\circ c\circ m})"] 
    %     &
    %     F(\sigma F)
    %         %\arrow[ur, swap, "\omega^{-1}"]
    %         \arrow[ur, swap, "1^{-1}"]
    %     \\
    %     &
    %     &F(\sigma F)
    %         \arrow[ru, swap, "F(\omega)"]
    %     &
    % \end{tikzcd} 
	\begin{tikzcd}[column sep=1.75cm, row sep=1cm]
        &\sigma F
            \arrow[dr,swap,"\omega"']
        &
        \\ 
        X  
            \arrow[dd,"c"] 
        & X
            \arrow[l, swap, "m"]
            \arrow[u, "!_{c}"]
            \arrow[r,dotted, swap, "!_{b\circ c\circ m}"]
        &\sigma F 
            %\arrow[dd, "\simeq","\omega"']
            \arrow[dd, "1_F"]
            \arrow[dr, "\textbf{id}"]
        \\
        &&&\sigma(F)
        \\
        F(X)     
            \arrow[r, swap, "b"]
        &F(X)
            \arrow[d, swap, "F(!_{c})"]
            \arrow[r, dotted, "F(!_{b\circ c\circ m})"] 
        &
        F(\sigma F)
            %\arrow[ur, swap, "\omega^{-1}"]
            \arrow[ur, swap, "1_F^{-1}"]
        \\
        &F(\sigma F)
            \arrow[ru, swap, "F(\omega)"]
        &
    \end{tikzcd} 
\end{figure}
\break
\hfill\break\\[-0.5cm]%[2.3cm]
{\huge \bfseries \ttitle}\\[1cm]%[3cm] % Thesis title

Submitted by\\[0.25cm]%[0.75cm]
\authorname % Author name - remove the \href bracket to remove the link

\vspace{2em}%{4em}

Thesis Advisor\\[0.25cm]%[0.75cm]
\supname % Supervisor name - remove the \href bracket to remove the link

\vspace{2em}%{4em}

\pillarname\\[1cm]%[1.5cm] % Research group name and department name

\large{A thesis submitted to the Singapore University of Technology and Design in fulfillment of the requirement for the degree of \degreename}\\[1cm] % University requirement text


{\large \the\year}\\[4cm] % Date
%\includegraphics{Logo} % University/department logo - uncomment to place it

\vfill
\end{center}
\end{titlepage}

%----------------------------------------------------------------------------------------
%	DECLARATION PAGE
%----------------------------------------------------------------------------------------

\begin{tec}
\addchaptertocentry{\tecname}
\begin{tabular}{ll}
	TEC Chair: & Prof. Jianying Zhou \\
	Main Advisor: & Prof. Sudipta Chattopadhyay \\
	%Co-advisor(s): & Prof. XXXX (if any) \\
	Internal TEC member 1: & Prof. CHEN Binbin\\
	Internal TEC member 2: & Prof. Cyrille Jegourel \\
	%External TEC member 1: & Prof. XXXX (optional) \\
	%External TEC member 2: & Prof. XXXX (optional) \\
\end{tabular}
\end{tec}
\vfill\eject

%----------------------------------------------------------------------------------------
%	DECLARATION PAGE
%----------------------------------------------------------------------------------------
 
\begin{declaration}
\addchaptertocentry{\authorshipname}

\noindent I, \authorname, declare that this thesis titled, \enquote{\ttitle} and the work presented in it are my own. I confirm that:

\begin{itemize}
\item This work was done wholly or mainly while in candidature for a research degree at this University.
\item Where any part of this thesis has previously been submitted for a degree or any other qualification at this University or any other institution, this has been clearly stated.
\item Where I have consulted the published work of others, this is always clearly attributed.
\item Where I have quoted from the work of others, the source is always given. With the exception of such quotations, this thesis is entirely my own work.
\item I have acknowledged all main sources of help.
\item Where the thesis is based on work done by myself jointly with others, I have made clear exactly what was done by others and what I have contributed myself.\\
\end{itemize}

I hereby confirm the following: 
\begin{itemize}
\item I hereby confirm that the thesis work is original and has not been submitted to any other University or Institution for higher degree purposes.
\item I hereby grant SUTD the permission to reproduce and distribute publicly paper and electronic copies of this thesis document in whole or in part in any medium now known or hereafter created in accordance with Policy on Intellectual Property, clause 4.2.2.
\item I have fulfilled all requirements as prescribed by the University and provided 1 copy of my thesis in PDF.
\item I have attached all publications and award list related to the thesis (e.g. journal, conference report and patent).
\item The thesis does / does not (delete accordingly) contain patentable or confidential information.
\item I certify that the thesis has been checked for plagiarism via turnitin/ithenticate. The score is 100\%.
\end{itemize}

\noindent Name and signature:\\
\rule[0.5em]{25em}{0.5pt} % This prints a line for the signature

\noindent Date:\\
\rule[0.5em]{25em}{0.5pt} % This prints a line to write the date
\end{declaration}

% \cleardoublepage

%----------------------------------------------------------------------------------------
%	QUOTATION PAGE
%----------------------------------------------------------------------------------------

% Uncomment for quotation page
% \vspace*{0.2\textheight}
%
% \noindent\enquote{\itshape Thanks to my solid academic training, today I can write hundreds of words on virtually any topic without possessing a shred of information, which is how I got a good job in journalism.}\bigbreak
%
% \hfill Dave Barry

%----------------------------------------------------------------------------------------
%	ABSTRACT PAGE
%----------------------------------------------------------------------------------------

\begin{abstract}
\addchaptertocentry{\abstractname} % Add the abstract to the table of contents
\emph{Latent behaviour analysis} (LBA) is a method to study the changes to observable behaviour in a transition system in the presence of an external force (e.g., an attack or a fault) that affects it. In LBA, we model these external forces as transformations of the state space. We use the framework of universal coalgebra to seamlessly combine the effects of a spatial transformation with the dynamics of a transition systems described as a coalgebra of a functor $F$: the $F$-coalgebra $(X,c\colon X\rightarrow F(X))$ and the spatial transformation $m\colon X\rightarrow X$ combine into a latent coalgebra $(X,c\circ m)$, 
which represents the system $(X,c)$ in the presence of the force modelled by $m$. By varying the spatial transformation $m$, we obtain a family of latent coalgebras that are compatible with the usual verification and testing methods for coalgebras and transition systems, and their analysis returns properties of the original system in the presence of the faults or adversaries modelled by these spatial transformations. 
%More precisely, LBA proposes to model the transition system as an $F$-coalgebra $(X,c)$ and the external force as an endofunction $m\colon X\rightarrow X$; we then study the effects that $m$ has on $(X,c)$ by checking the latent $F$-coalgebra $(X,c\circ m)$. 

Although checking latent coalgebras is a simple idea, it has a variety of practical applications. This thesis shows how to use LBA to study three cybersecurity problems. The first problem is the \emph{classification of attacker models}. We provide a method to systematically generate and compare attacker models for a system based on their potentially harmful effects. The second problem is the \emph{quantification of robustness in Cyber-physical Systems (CPS)}, where we propose a method for the systematic generation of attacks and counterattacks in CPS to quantify the robustness of a system. Finally, we study the problem of \emph{timing side-channel repair}, for which we propose using a spatial transformation that enforces constant memory access patterns and preserves the functionality of the program. 

% The Thesis Abstract is written here (and usually kept to just this page). The page is kept centered vertically so can expand into the blank space above the title too\ldots

\end{abstract}

%----------------------------------------------------------------------------------------
%	Publications
%----------------------------------------------------------------------------------------

\begin{publications}
\addchaptertocentry{\publicationsname} % Add the abstract to the table of contents

Rothstein-Morris E., Murguia C. G. , and Ochoa M. (2017). Design-time Quantification of Integrity in Cyber-physical Systems. In Proceedings of the 2017 Workshop on Programming Languages and Analysis for Security. PLAS '17. Association for Computing Machinery, New York, NY, USA, 63–74. \url{https://doi.org/10.1145/3139337.3139347}

Rothstein-Morris E., Sun J. and Chattopadhyay S. (2020) Systematic Classification of Attackers via Bounded Model Checking. In: Beyer D., Zufferey D. (eds) Verification, Model Checking, and Abstract Interpretation. VMCAI 2020. Lecture Notes in Computer Science, vol 11990. Springer, Cham. \url{https://doi.org/10.1007/978-3-030-39322-9_11}

Rothstein-Morris E., Sun J., Chattopadhyay S. (2021) ORIGAMI: Folding Data Structures to Reduce Timing Side-Channel Leakage. Submitted to the 35th IEEE Computer Security Foundations Symposium. CSF 2022. 

Rothstein-Morris E., Castellanos J.H., Ochoa M. and Chattopadhyay S. (2021). Improving the Robustness of Cyber-Physical System Models via Latent Behaviour Analysis and Testing. Submitted to the ACM Transactions on Privacy and Security. 
\end{publications}

%----------------------------------------------------------------------------------------
%	ACKNOWLEDGEMENTS
%----------------------------------------------------------------------------------------

\begin{acknowledgements}
\addchaptertocentry{\acknowledgementname} % Add the acknowledgements to the table of contents
\todo[inline]{I have so many people to thank...I should start writing this.}
It seems strange to think that almost five years have already passed since I arrived in Singapore. 

I would first like to thank my family. My mother, my father and my brother...

Next, I would like to thank my supervisors Mart\'in Ochoa, Sun Jun, and Sudipta Chattopadhyay. I've had the pleasure to work with each of you, learn from you, and experience through you the world of academia. I thank you for the time and lessons you've given me to make this journey a success. 

I want to especially thank Jorge Cuellar. Jorge, your passion for teaching and research was what ultimately inspired me to pursue a PhD. I am very grateful for those hours we spent together going over parts of this thesis to make it sensible and sober. I still have so much to learn! 

The Passau team taught me valuable lessons in life.




 


\todo[inline]{I will list all of them, indeed. They are important to me. }

% My family provide constant support.
% My friends keep my grip on reality.
% My advisor guides me.
% Is that how I see people? by their relationships with me and not by who they are individually? Man, I need to talk to someone... 

% Maybe writing this helps. 

%The acknowledgments and the people to thank go here, don't forget to include your project advisor\ldots


Thank you Maria Fernanda García, dear Mafe, for always checking on me! Your messages always bring joy and a smile to my face, and they keep me motivated. I'm very grateful to be your friend :) 

\end{acknowledgements}

%----------------------------------------------------------------------------------------
%	LIST OF CONTENTS/FIGURES/TABLES PAGES
%----------------------------------------------------------------------------------------

\tableofcontents % Prints the main table of contents

\listoffigures % Prints the list of figures

\listoftables % Prints the list of tables

%---------------------------------------------------------------------------------------- 
%	ABBREVIATIONS
%----------------------------------------------------------------------------------------

% Uncomment for abbreviations page
% \begin{abbreviations}{ll} % Include a list of abbreviations (a table of two columns)

% \textbf{AIG} & \textbf{A}nd-\textbf{I}nverter \textbf{G}raphs\\
% \textbf{COI} & \textbf{C}one \textbf{O}f \textbf{I}nfluence\\
% \textbf{OIC} & Dual \textbf{C}one \textbf{O}f \textbf{I}nfluence\\
% \textbf{LBA} & \textbf{L}atent \textbf{B}ehaviour \textbf{A}nalysis\\
% \end{abbreviations}

%----------------------------------------------------------------------------------------
%	PHYSICAL CONSTANTS/OTHER DEFINITIONS
%----------------------------------------------------------------------------------------

% Uncomment for physical constants etc
% \begin{constants}{lr@{${}={}$}l} % The list of physical constants is a three column table
%
% % The \SI{}{} command is provided by the siunitx package, see its documentation for instructions on how to use it
%
% 	Speed of Light & $c_{0}$ & \SI{2.99792458e8}{\meter\per\second} (exact)\\
% %Constant Name & $Symbol$ & $Constant Value$ with units\\
%
% \end{constants}

%----------------------------------------------------------------------------------------
%	SYMBOLS
%----------------------------------------------------------------------------------------

% Uncomment for symbols
% \begin{symbols}{lll} % Include a list of Symbols (a three column table)
%
% $a$ & distance & \si{\meter} \\
% $P$ & power & \si{\watt} (\si{\joule\per\second}) \\
% %Symbol & Name & Unit \\
%
% \addlinespace % Gap to separate the Roman symbols from the Greek
%
% $\omega$ & angular frequency & \si{\radian} \\
%
% \end{symbols}

%----------------------------------------------------------------------------------------
%	DEDICATION
%----------------------------------------------------------------------------------------

\dedicatory{To my family and friends. Your constant love and support made this thesis possible.}

%----------------------------------------------------------------------------------------
%	THESIS CONTENT - CHAPTERS
%----------------------------------------------------------------------------------------
 
\mainmatter % Begin numeric (1,2,3...) page numbering

\pagestyle{thesis} % Return the page headers back to the "thesis" style

% Include the chapters of the thesis as separate files from the Chapters folder
% Uncomment the lines as you write the chapters

%!TEX root = ../main.tex
% Chapter Template

\newcommand{\branch}[3]{\ensuremath{{#2}\ {+_{#1}}\ {#3}}}
\newcommand{\iterate}[2]{\ensuremath{{#2}^{\left(#1\right)}}}
\newcommand{\bexp}[0]{\ensuremath{\text{BExp}}}
\newcommand{\gexp}[0]{\ensuremath{\text{Exp}}}
\newcommand{\usg}[0]{\ensuremath{\text{u}}}
\newcommand{\eval}[0]{\ensuremath{\mathtt{eval}}}
\newcommand{\sat}[0]{\ensuremath{\mathtt{sat}}}
\newcommand{\sg}[0]{\ensuremath{\text{s}}}
\newcommand{\set}[1]{\ensuremath{\left\{#1\right\}}}
\newcommand{\lbl}[0]{\ensuremath{\text{lbl}}}
\newcommand{\Real}[0]{\ensuremath{\mathbb{R}}}
\newcommand{\Nat}[0]{\ensuremath{\mathbb{N}}}
\newcommand{\letter}[0]{\ensuremath{\left(\text{A-Z\ |\ a-z}\right)}}
\newcommand{\Atom}[0]{\ensuremath{\text{At}}}
\newcommand{\GuardedString}[0]{\ensuremath{\text{GS}}}
\newcommand{\RC}[0]{\ensuremath{\text{RC}}}
\newcommand{\Low}[0]{\ensuremath{\text{Low}}}
\newcommand{\Variable}[0]{\ensuremath{\mathscr{V}}}
\newcommand{\Integer}[0]{\ensuremath{\mathbb{Z}}}
\newcommand{\alphanumeric}[0]{\ensuremath{\left(\text{A-Z\ |\ a-z\ |\ 0-9}\right)}}
\newcommand{\alphanumericP}[0]{\ensuremath{\left(\text{A-Z\ |\ a-z\ |\ 0-9\ |\ .\ |\ \_\ |\ \$}\right)}}
\newcommand{\semantics}[1]{\ensuremath{\llbracket #1\rrbracket}}
\newcommand{\valuation}[0]{\ensuremath{\Gamma}}
\newcolumntype{L}{>{$}l<{$}} % math-mode version of "l" column type
\newcolumntype{R}{>{$}r<{$}} % math-mode version of "r" column type
\newcolumntype{C}{>{$}c<{$}} % math-mode version of "c" column type
%\newcommand{\hourglass}[0]{}%{\LARGE\fontspec{Cambria}^^^^231b}
\newcommand{\timeequiv}[0]{\equiv_{\hourglass}}

\chapter{Introduction} % Main chapter title
\label{ch:Introduction} % Change X to a consecutive number; for referencing this chapter elsewhere, use \ref{ChapterX}

%----------------------------------------------------------------------------------------
%	SECTION 1
%----------------------------------------------------------------------------------------

\section{Story Time}
We model systems based on what they are supposed to do, but what can we say about the things that they \emph{could} do?

\todo[inline]{Actually, the thesis would be super interesting if potential behaviour problems are presented in terms of a game between the system and the attacker, with the attacker following some sort of reactive strategy, and the system as well. At some point, one or the other wins because it is a deterministic game if the system is deterministic. }
How can we model attackers? How can we model their actions? How can we prepare against them?

Many attacks are discovered through testing: an experienced hacker creates a proof of concept to illustrate how some system may be attacked. Then, an abstraction is created to explain why the attack works, and to suggest countermeasures against it. 
\todo[inline]{You can probably show several papers that display this pattern.} 

Fuzzing is probably one of the most popular techniques for system transformations. 
%-----------------------------------
%	SUBSECTION 1
%-----------------------------------
\subsection{Subsection 1}

Nunc posuere quam at lectus tristique eu ultrices augue venenatis. Vestibulum ante ipsum primis in faucibus orci luctus et ultrices posuere cubilia Curae; Aliquam erat volutpat. Vivamus sodales tortor eget quam adipiscing in vulputate ante ullamcorper. Sed eros ante, lacinia et sollicitudin et, aliquam sit amet augue. In hac habitasse platea dictumst.

%-----------------------------------
%	SUBSECTION 2
%-----------------------------------

\subsection{Subsection 2}
Morbi rutrum odio eget arcu adipiscing sodales. Aenean et purus a est pulvinar pellentesque. Cras in elit neque, quis varius elit. Phasellus fringilla, nibh eu tempus venenatis, dolor elit posuere quam, quis adipiscing urna leo nec orci. Sed nec nulla auctor odio aliquet consequat. Ut nec nulla in ante ullamcorper aliquam at sed dolor. Phasellus fermentum magna in augue gravida cursus. Cras sed pretium lorem. Pellentesque eget ornare odio. Proin accumsan, massa viverra cursus pharetra, ipsum nisi lobortis velit, a malesuada dolor lorem eu neque.

%----------------------------------------------------------------------------------------
%	SECTION 2
%----------------------------------------------------------------------------------------

\section{Main Section 2}

Sed ullamcorper quam eu nisl interdum at interdum enim egestas. Aliquam placerat justo sed lectus lobortis ut porta nisl porttitor. Vestibulum mi dolor, lacinia molestie gravida at, tempus vitae ligula. Donec eget quam sapien, in viverra eros. Donec pellentesque justo a massa fringilla non vestibulum metus vestibulum. Vestibulum in orci quis felis tempor lacinia. Vivamus ornare ultrices facilisis. Ut hendrerit volutpat vulputate. Morbi condimentum venenatis augue, id porta ipsum vulputate in. Curabitur luctus tempus justo. Vestibulum risus lectus, adipiscing nec condimentum quis, condimentum nec nisl. Aliquam dictum sagittis velit sed iaculis. Morbi tristique augue sit amet nulla pulvinar id facilisis ligula mollis. Nam elit libero, tincidunt ut aliquam at, molestie in quam. Aenean rhoncus vehicula hendrerit.
%!TEX root = ../main.tex
% Chapter Template


\chapter{Preliminaries and Notation} % Main chapter title
\label{ch:Preliminaries} % Change X to a consecutive number; for referencing this chapter elsewhere, use \ref{ChapterX}

\todo[inline]{This section has content which I did not develop, but that I use. The content I developed follows in the next chapters.}

%----------------------------------------------------------------------------------------
%	SECTION 1
%----------------------------------------------------------------------------------------
\section{Notation and Constructions on Sets and Functions}
\subsection{Sets and elements}
We denote \emph{sets} by upper-case letters $X,Y,Z,$ etc, and we denote their \emph{elements} by lower-case letters $x, y, z, $ etc. 

We denote the set of natural numbers by $\mathbb{N}$ and the set of natural numbers without zero by $\mathbb{N}^+$. Similarly, we denote the set of real numbers by $\mathbb{R}$ and the set of positive real numbers (excluding zero) by $\mathbb{R}^+$.

We denote the size of a finite set $X$ by $|X|$, and the empty set by $\emptyset$.

\subsection{Functions}
We denote \emph{functions} by lower-case letters $f,g,h,$ etc. A function $f\colon X\rightarrow Y$ maps the elements from the set $X$ to elements of the set $Y$. An \emph{endofunction} is a function $g\colon X\rightarrow X$ which maps a set $X$ to itself. We denote function composition by $\circ$. 
We say that an endofunction $h\colon X \rightarrow X$ has finite support iff $h(x)\neq x$ for a finite number of $x$ in $X$, even if $X$ is infinite.

\subsection{Range set}
A \emph{range set} $n$ where $n\in \mathbb{N}^+$ is a set of $n$ elements, i.e., $n=\set{0,1, \ldots, n-1}$. It is usually clear from the context when $n$ represents a number or a range set. We highlight three important range sets: 0, 1 and 2.

The range set 0 is equal to the empty set $\emptyset$. 
The range set 1, which is equal to the set $\set{0}$, serves as a constant for several set operations (modulo isomorphism).
The range set 2, which is equal to $\set{0,1}$, to model booleans: 0 models $\false$ and $1$ models $\true$. 

\subsection{Subsets $X\rightarrow 2$}
Given a set $X$, we characterise the \emph{subsets of $X$} using functions of type $X\rightarrow 2$; we say that $x\in X$ is an element of the subset characterised by $f\colon X\rightarrow 2$ iff $f(x)=1$. 
 
\subsection{Exponential Set $Y^X$}
The \emph{exponential set} $Y^X$ is the set of all functions of type $X\rightarrow Y$. 
The exponential set $1\rightarrow X$ is isomorphic to $X$, and we use it to select objects from $X$; more precisely, we choose an element $x\in X$ via a function $f\colon 1\rightarrow X$ such that $f(0)=x$. (This seems overcomplicated at first, but we use this equivalence when discussing the following concepts of $F$-algebra and $F$-coalgebra.)

\subsection{Sum Set $X+Y$}
The \emph{sum set} $X + Y$ is the disjoint union of $X$ and $Y$, i.e. 
\begin{align*}
    X + Y\triangleq\set{ \iota_1(x) |x\in X} \cup \set{\iota_2(y) | y\in Y},
\end{align*}
where $\iota_1$ and $\iota_2$ act as different tags.
\subsection{Product Set $X\times Y$}
The \emph{product set} $X\times Y$ is the cartesian product of $X$ and $Y$, i.e. 
\begin{align*}
    X\times Y\triangleq\set{(x,y)|x\in X, y\in Y}.
\end{align*}
The \emph{finite product set} of sets $X_1, \ldots, X_n$ is their cartesian product; i.e., $X_1\times \ldots \times X_n$. We denote product sets by upper-case letters with an arrow above $\vec{X},\vec{Y}, \vec{Z}$, etc.
\subsection{Vector/Tuple $\vec{x}$}
Given a finite product set $\vec{X}=X_1\times \ldots \times X_n$, we denote the elements of $\vec{X}$ by $\vec{x}$, $\vec{y}$, $\vec{z}$, etc. We call these elements \emph{vectors} or, alternatively, \emph{tuples}. 

\subsection{Coordinates $\vec{x}[\pi]$}
Given a finite product set $\vec{X}=X_1\times \ldots \times X_n$, its \emph{coordinates} are $\pi_1, \ldots, \pi_n$, where $\pi_i\colon \vec{X} \rightarrow X_i$ extract the $i$-th value of a vector; i.e., $\pi_i(x_1, \ldots, x_n)\triangleq x_i$, for $1\leq i \leq n$. We often write the expression $\vec{x}[\pi]$ instead of $\pi(\vec{x})$ when $\pi$ is a coordinate. %We sometimes use the alternative notation $\vec{x}_\pi$ to denote $\vec{x}[\pi]$, usually when $\pi$ is a natural number.

We use natural numbers for coordinates when no explicit set of coordinates is provided. More precisely, given $\vec{x}\in X_1\times \ldots \times X_n$ where $\vec{x}=(v_1, \ldots, v_n)$, if no set of coordinates is provided, then we use the coordinates $1, 2, \ldots , n$, so $\vec{x}[i]=v_i$ for $i\in \set{1, \ldots, n}$. 

\subsection{Constant Function $\Delta_x$}
Given sets $X$ and $Y$, each element $x\in X$ lifts to a \emph{constant function} $\Delta_x\colon Y\rightarrow X$, defined by $\Delta_x(y)=x$, for $y\in Y$.

\subsection{Pair Function $(f,g)$}
Given two functions $f\colon X\rightarrow Y$ and $g\colon X\rightarrow Z$, the \emph{pair function} $(f,g)\colon X\rightarrow Y\times Z$ is defined by $(f,g)(x)=(f(x),g(x))$.

\subsection{Product Function $f\times g$}
Given two functions $f\colon X\rightarrow Y$ and $g\colon A\rightarrow B$, the \emph{product function} $f\times g\colon X\times A\rightarrow Y\times B$ is defined by $f\times g(x,a)=(f(x),g(a))$.

\subsection{Function Mapping $f(X)$}
Given a set $X$ and a function $f\colon X\rightarrow Y$, the \emph{mapping of $f$ over $X$}, denoted $f(X)$, is the set defined by $f(X)\triangleq\set{f(x)| x\in X}$.

\section{Category Theory Preliminaries and Notation}

\subsection{Categories, Objects and Arrows}
A \emph{category} $\AsCategory{C}$ is a mathematical concept similar to a graph, comprised of two aspects: a collection of \emph{objects}, denoted $Obj(\AsCategory{C})$, and a collection of \emph{arrows} among those objects, denoted $Arr(\AsCategory{C})$. Each category satisfies the following  rules:
\begin{itemize}
    \item every object $X$ has an identity arrow that maps $X$ to itself, denoted $\id_X$,
    \item given the objects $X, Y$ and $Z$, and the arrows $X\xrightarrow{f}Y$ and $Y\xrightarrow{g}Z$, the arrow $X\xrightarrow{g\circ f}Z$ exists; i.e. the composition of arrows is well defined,
    \item the identity arrows act as units for arrow composition; i.e., for $X\xrightarrow{f}Y$, the equality $\id_{Y}\circ f=f=f\circ \id_{X}$ holds.
    \item the composition of arrows is associative; i.e., $(h\circ g)\circ f = h \circ (g\circ f)$.
\end{itemize}

\subsection{Commutative Diagram}
A \emph{commutative diagram} is a directed graph which visually represents equations. Like any directed graph, source nodes do not have incoming arrows, and sink nodes do not have outgoing arrows, and they represent the beginning and end of equations. For example, given the functions $f\colon X\rightarrow Y$, $g\colon Y\rightarrow Z$ and $h\colon Z\rightarrow A$, the equalities $(h\circ g)\circ f= h\circ g \circ f= h \circ (g\circ f)$ is represented by the diagram shown in Figure~\ref{fig:Preliminaries:CommutativeDiagram}.

\begin{figure}[t] 
    \centering
    \begin{tikzcd}[column sep=1.5cm, row sep=1.5cm]
         X
            \arrow[r,"f"]
            \arrow[rr,bend left, "g\circ f"]
            %\arrow[r,"c"]
            %\arrow[d,"1"']
        &Y
            \arrow[r,"g"]
            \arrow[rr,bend right, "h\circ g"]
        &Z
            \arrow[r,"h"]
        &A
            %\arrow[u,"1^{-1}"']
    \end{tikzcd}
    \caption{Commutative diagram describing the equality $(h\circ g)\circ f= h\circ g \circ f= h \circ (g\circ f)$}
    \label{fig:Preliminaries:CommutativeDiagram} 
\end{figure}
\subsection{Initial and Terminal Objects}
A category $\AsCategory{C}$ has an \emph{initial object}, often denoted by 0, if there exists a unique arrow from $0$ to every object $o$ in the category. Dually, a category $\AsCategory{C}$ has a \emph{final object}, often denoted by 1, if there exists a unique arrow from every object $o$ in the category to 1. 

A category may have several different initial and terminal objects, but they are always unique modulo isomorphism.

\subsection{Functor}
Functors are to categories what arrows are to objects. A \emph{functor} $F$ maps a category $\AsCategory{C}$ to a category $\AsCategory{D}$, mapping $Obj(\AsCategory{C})$ to $Obj(\AsCategory{D})$, and $Arr(\AsCategory{C})$ to $Arr(\AsCategory{D})$ such that the following conditions are satisfied for all objects $X, Y, Z$ and arrows $f,g$.
\begin{itemize}
    \item for $f\colon X\rightarrow Y$, $f$ is mapped to a function of type $g\colon F(X)\rightarrow F(Y)$,
    \item $\id_X$ is mapped to $\id_{F(X)}$; i.e. $F(\id_{X})=\id_{F(X)}$.
    \item if $f\colon X\rightarrow Y$ and $g\colon Y\rightarrow Z$, then $F(g\circ f)=F(g)\circ F(f)$.
\end{itemize}
These conditions ensure functors preserve the structure of the category they are mapping.

\section{The Category of Sets and Functions}
The \emph{category of sets and functions} $\AsCategory{Set}$ is the category whose objects are sets and whose arrows are functions. The empty set $0$ is the initial object, and any set isomorphic to the range set 1 is a terminal object.

\subsection{Currying}
\emph{Currying} is the use of the isomorphism between the sets $Z^{X\times Y}$ and $Z^{Y^X}$ which maps a function $f\colon (X\times Y)\rightarrow Z$ to a function $g\colon X\rightarrow Z^Y$ such that $g(x)(y)=f(x,y)$ for $x\in X$ and $y\in Y$. We use currying to adjust the types between equivalent constructions (e.g. between Automata and $F$-coalgebras).

\subsection{Partial Application}
Given a function $f\colon X\rightarrow Y \rightarrow Z$ and $x\in X$, the \emph{partial application} of $f$ given $x$ is the function $f_x\colon Y \rightarrow Z$, defined for $y\in Y$ by $f_x(y)=f(x)(y)$.

\subsection{Monoids and Sequences} 
A \emph{monoid} is a set $X$ paired with a sum function $+\colon X\rightarrow X\rightarrow X$ and a unit element $0\in X$ such that $0$ is neutral for $+$ and $+$ is associative. We highlight two monoids for a given set $X$: the \emph{free monoid of $X$} and the \emph{endofunctions monoid of $X$}.

The \emph{free monoid of $X$} is the set $X^*$ of finite sequences of elements of $X$, where the sum function is sequence concatenation $\cdot$ and the unit is the empty sequence $\varepsilon$. 

The \emph{endofunctions monoid of $X$} is the set $X^X$ of endofunctions in $X$, where the sum is function composition $\circ$ and the unit is the identity function $\id_X$.

\subsection{Languages}
Given a set $X$, a \emph{language with alphabet $X$} is a subset of $X^*$. 

\subsection{Deterministic Finite Automata (DFA)}
Given a finite set $A$, a \emph{deterministic finite automaton} (DFA) is a tuple $\mathbb{X}=(X,x_0,\delta,F)$, where $X$ is a finite set, $x_0$ is an element of $X$, $\delta\colon X\times A \rightarrow X$ is a transition function, and $F\colon X\rightarrow 2$ is a subset of $X$ which marks accepting states. Given a sequence $w\in A^*$, we say that the automaton $\mathbb{X}$ \emph{accepts} $w$ if and only if the following conditions are satisfied,
\begin{itemize}
    \item if $w=\varepsilon$, then $\mathbb{X}$ accepts $w$ iff $F(x_0)=1$, 
    \item if $w=a\cdot w'$, then $\mathbb{X}$ accepts $w$ iff the automaton $(X,\delta(x_0,a),\delta,F)$ accepts $w'$, with $a\in A$ and $w'\in A^*$.
\end{itemize}
The denotational semantics of a DFA is the language with alphabet $A$ whose sequences it accepts, rejecting the rest.

\subsection{Set Isomorphisms}
\todo[inline]{Homomorphisms and isomorphisms}
% \subsubsection{List of Relevant Isomorphisms}
% \begin{itemize}
%     \item $1\times X \simeq X$,
%     \item $2\times X\simeq X+X$,
%     \item $X^1\simeq X$,
%     \item $X^2\simeq X\times X$,
%     \item $Z^{X\times Y}\simeq Z^{(Y^X)}$,
%     \item $X\times Y \simeq Y\times X$
%     \item $X + Y \simeq Y+ X$
%     \item $|X|=n$ iff $X\simeq n$.
%     \item $\underbrace{\left(X\times\ldots\times X\right)}_\text{$n$ times}\simeq X^n$ (for this case, we say that $n$ is the set of coordinates of $X^n$)
% \end{itemize}
\section{Universal (Co)Algebra Preliminaries and Notation}
\label{sec:Preliminaries:Coalgebras}
% \todo[inline]{ntroduction to the world of (co)algebras and explain why you want to use them. They are a wonderful way to unify the formalism of the thesis.}
% \todo[inline]{Complete here}
%\todo[inline]{*Puts an Edna Mode face* NO MONADS (unless necessary)}
%\subsection{$\Monad$-algebras, $\Functor$-coalgebras and $\lambda$-bialgebras}
\subsection{$F$-(co)algebras}
Given an endofunctor $F$ in a category $\AsCategory{C}$, an \emph{$F$-coalgebra} is a pair $(X,X\xrightarrow{c}F(X))$ of an object $X$ and a morphism $c\colon X\rightarrow F(X)$. We use $F$-coalgebras to model dynamic systems with state. We call $X$ the \emph{carrier} and $c$ the \emph{dynamics}.

Dually, an \emph{$F$-algebra} is a pair $(X,F(X)\xrightarrow{a}X)$.

\subsection{Pointed coalgebras}
Pointed coalgebras are coalgebras with a distinguished state, usually referred to as the \emph{initial state}. Formally, for a functor $F$, a pointed $F$-coalgebra is a triple $(X,c, x_0)$ where $(X,c)$ is an $F$-coalgebra and $(X, x_0)$ is a $\Delta_X$-algebra, i.e., $x_0\colon 1\rightarrow X$, and $x_0(\star)$ characterises the initial state.

\subsection{Automata as Pointed Coalgebras}
DFA which characterise languages with an alphabet $A$ can be modelled by coalgebras of the functor $F(X)=2\times X^A$. Given a DFA $(X,x_0,\delta,F)$, the corresponding $F$-coalgebra is $(X,(F,\delta))$, where $(F,\delta)\colon X\rightarrow 2\times X^2$ is a function pair. We can make this coalgebra pointed by adding a function $x_0\colon 1\rightarrow X$ to mark the initial state.

In general, given sets $I$ and $O$, we can model transition systems whose denotational semantics are functions of type $I^*\rightarrow O$; i.e., systems which process a finite sequence of elements in $I$ and respond with an element in $O$. The corresponding functor in this case is $F(X)=O\times X^I$. An $F-$coalgebra would be of the form $(X,X\xrightarrow{(\gamma,\delta)}O\times X^I)$, with $\gamma\colon X\rightarrow O$ and $\delta\colon X\rightarrow X^I$; the function $o$ lets us explore the component in the state $x$, and $\delta$ lets us perform transitions. We write $x/o$ to denote $\gamma(x)=o$, and we write $x^i$ as a shorthand for $\delta(x)(i)$. 

% \subsection{States and Components}
% We often use vectors/tuples as a states. We access the values inside states by means of projection functions. If the carrier is a product set $\vec{X}=Y_1 \times\ldots Y_n$, we can use, for $j=1..n$, the projection functions $\pi_j\colon \vec{X}\rightarrow Y_j$ defined, for $\vec{x}=(v_1,\ldots,v_n)$, by $\pi_j(\vec{x})\triangleq v_j$. In this case, we say that $\pi_1$ to $\pi_n$ are the \emph{components} of $\vec{X}$ and its elements. Since both coalgebras and components are functions from the carrier, we use bracket notation for component to distinguish them, i.e., we write $\vec{x}[\pi_j]$ instead of $\pi_j(\vec{x})$.  %, or, alternatively the \emph{state variables} of states in $X$. 
% Whenever we write a state or a carrier  with arrows above them (i.e., $\vec{x}$ or $\vec{X}$), we imply that they have more than one component. Note that if the carrier $X$ has only one component, then it must be the identity function $\id\colon X\rightarrow X$. 
% \todo[inline]{Consider $\vec{x}.\pi_j$ too, called dot notation}
%A \emph{$\Monad$-algebra} for a monad $\Monad=(\MFunctor,\eta,\mu)$ is a pair $(\TheSet,\Monad(\TheSet)\xrightarrow{a}\TheSet)$ of an object $\TheSet$ and a morphism $a\colon \Monad(\TheSet)\rightarrow\TheSet$ which, due to the relationship between monads and adjunctions, satisfies two properties: the \emph{multiplication square} and the \emph{unit triangle}, shown in Figure~\ref{fig:MultiplicationSquare}.
%
%%\subsubsection The compositions $a\circ \mu_X$ and $a\circ T(a)$ are equal.
%\begin{figure}[h]
%\centering
%\begin{minipage}{0.45\textwidth}
% \centering
%\begin{tikzcd}
%    T^2(X) \arrow{r}{T(a)} \arrow[swap]{d}{\mu_X} & T(X) \arrow{d}{a} \\
%    T(X) \arrow{r}{a}& X
% \end{tikzcd}
%\end{minipage}
% \begin{minipage}{0.45\textwidth}
% \centering
%\begin{tikzcd}
%    T(X) \arrow{r}{a} \arrow[swap]{dr}{1_{T(X)}} & X\arrow{d}{\eta} \\
%    & T(X)
%  \end{tikzcd}
%  \end{minipage}
%  \caption{Multiplication square (left) and unit triangle (right).}
%\label{fig:MultiplicationSquare}
%\end{figure}
%Given a state structure $S$, we can use the State monad .
%\begin{align}
%State_S(X) = (S\times X)^S 
%\end{align}
%
%\begin{align}
%\eta_X:X\rightarrow  (S\times X)^S\\
%\eta_X(x)= \lambda s \rightarrow (s,x)
%\end{align}
%
%%\begin{align}
%%>>=\colon  (S\rightarrow (T,S))\rightarrow (T\rightarrow (S\rightarrow (Y,S)) \rightarrow  (S\rightarrow (Y,S))\\
%%m >>= f = \lambda s \rightarrow \text{let $(t,s')=m(s)$ in $f(t)(s')$} 
%%\end{align}
%
%\begin{align}
%\mu_X:(S\times (S\times X)^S)^S\rightarrow  (S\times X)^S\\
%\mu_X(\mathbf{x})= \lambda s \rightarrow \text{let $(s',f)=\mathbf{x}(s)$ in f(s')}
%\end{align}
%\todo[inline]{THERE IS NO NEED FOR MONADS AT THE MOMENT. Monads seem to be what maps a type $X$ to an initial/final $F$-coalgebra of a functor where $X$ is constant}
%\subsection{An Intuition of States and Time}
%\subsection{Catamorphisms, Anamorphisms, Final and Initial $F$-Coalgebras}
\subsection{The Category of $F$-coalgebras and $F$-homomorphisms}
Given a functor $F$, the \emph{category of $F$-coalgebras}, denoted $\AsCategory{Coalg_F}$ is the category whose objects are $F$-coalgebras and whose arrows are \emph{$F$-homomorphisms}. An \emph{$F$-homomorphism} $\mathbb{X}\xrightarrow{f} \mathbb{Y}$ between an $F$-coalgebra $\mathbb{X}=(X,c)$ and an $F$-coalgebra $\mathbb{Y}=(Y,d)$ is a function $f\colon X\rightarrow Y$ such that
\begin{align}
    F(f)\circ c = d\circ f.
\end{align}

\subsection{Semantic Map and Final $F$-Coalgebras}
Let $F$ be a functor such that $F(X)=O\times (X)^I$, and let $I$ and $O$ be sets; we define the set $\sigma F\triangleq O^{I^*}$, and we define the function pair $(?, (\cdot)')\colon \sigma F\rightarrow O\times (\sigma F)^I$, for $\phi \in \sigma F$, $i\in I$ and $w\in I^*$, by
\begin{align}
    \phi? \triangleq \phi(\varepsilon),
    \phi'(i)(w) \triangleq \phi(i\cdot w)
\end{align}
Following convention, we write $\phi'(i)$ as $\phi^i$, and the pair $(?, (\cdot)')$ as $1_F$. The pair $(\sigma F, 1_F)$ is a \emph{final $F$-coalgebra}, since it is a final object in the category $\AsCategory{Coalg_F}$.

Given an $F$-coalgebra $(X,c)$ where $c=(\gamma,\delta)$, the unique $F$-homomorphism between $(X,c)$ and $(\sigma F, 1_F)$ is given by a function $!_c\colon X\rightarrow \sigma F$, which is defined for $x\in X$, $i\in I$ and $w\in I^*$ by 
\begin{align}
    !_c(x)(\varepsilon)\triangleq \gamma(x),\quad \text{and} \quad 
    !_c(x)(i\cdot w)\triangleq !_c(x^i)(w).
\end{align}
The function $!_c$ is called the \emph{semantic map} because it maps $x\in X$ to its denotational semantics in $\sigma F$. For example, in the case of DFA as coalgebras, it maps each state $x$ in the carrier of the coalgebra to the language it would be recognised if $x$ were the initial state in the DFA.

\subsection{Bisimilarity}
Given $F$-coalgebras $(X,c)$ and $(Y,d)$, we say that states \emph{$x\in X$ and $y\in Y$ are bisimilar}, denoted $x\sim y$, if and only if $!_c(x)=!_d(y)$. In other words, $x$ and $y$ are bisimilar iff they have the same semantics.

\subsection{Coalgebraic Specification Language}
Given a functor $F(X)=O\times X^I$ where $O$ and $I$ are fixed sets, a \emph{coalgebraic specification language} for $F$ is an $F$-coalgebra $(\mathcal{E},c)$ where $\mathcal{E}$ is a set of expressions whose semantics are given by the semantic map $!_c$.

For example, consider the functor $F(X)=2\times X^2$; we define the set of expressions $\mathcal{E}$ by the grammar
\begin{align}
    e::= ID:0?ID\land1?ID\land\downarrow (0|1)
\end{align}
where $ID$ is a set of identifiers. We define the pair function $c=(\gamma,\delta)\colon \mathcal{E}\rightarrow F(\mathcal{E})$ by 
\begin{align*}
    \gamma(x:0?\land1?z\land\downarrow a)&\triangleq a\\
    \delta(x:0?y\land1?z\land\downarrow a)(0)&\triangleq y\\
    \delta(x:0?y\land1?z\land\downarrow a)(1)&\triangleq z.
\end{align*}
The expression $x:0?x\land1?x\land\downarrow 1$ can be considered to describe a single-state automaton whose single state is  accepting, and it loops to itself on both inputs 0 and 1. The semantic map $!_c$ maps the expression $x:0?x\land1?x\land\downarrow 1$ to the language $2^*$.%, i.e., the language which contains every sequence in $2^*$.
\todo[inline]{Cite Kleene Coalgebra somehow}

\subsection{Completeness}
Given a coalgebraic specification language $L=(\mathcal{E},c)$ for the functor $F(X)=O\times X^I$, we say that $L$ is \emph{complete} if and only if the semantic map $!_c\colon \mathcal{E}\rightarrow \sigma F$ is surjective. In other words, for every behaviour $\phi\in \sigma F$, there exists an expression $e\in \mathcal{E}$ such that $!_c(e)=\phi$.

\subsection{Soundness}
Given a coalgebraic specification language $L=(\mathcal{E},c)$ for the functor $F(X)=O\times X^I$ and a notion of equivalence in $\mathcal{E}$ modelled by a function $\equiv \colon \mathcal{E}\times \mathcal{E}\rightarrow 2$, we say that $L$ is \emph{sound} if and only if, for $e_1, e_2\in \mathcal{E}$, if $e_1\equiv e_2$, then $e_1\sim e_2$; i.e., if two expressions are equivalent, then they have the same behaviour.


\subsection{Behavioural Property}
\label{sec:preliminaries:BehaviouralProperty}
Let $F$ be a functor for which a final $F$-coalgebra $(\sigma F, 1)$ exists. A \emph{behavioural property} $P$ is a function $P\colon \sigma F\rightarrow 2$. Given an arbitrary $F$-coalgebra $(X,c)$, we say that $x\in X$ satisfies the behavioural property $P$ if and only if $(P\circ !_c)(x)=1$.

We use {behavioural properties} to formalise properties of a system. Behavioural properties come in all shapes and forms, and are often described using logics like LTL~\cite{LTL}. CTL~\cite{CTL}, and $\mu$ calculus~\cite{MuCalculus}.

\section{Coinduction}
Coinduction is a dual principle to induction, and we use it to define what things are in terms of how they behave. 
\todo[inline]{I'll come back to this}

\subsection{Coinduction Proof Principle}
\todo[inline]{Coinductive Definitions?}
\subsection{Bisimulation}
% \section{Spectre}
% \label{sec:Preliminaries:Spectre}
% \todo[inline]{And Meltdown?}
% \subsection{Spectre V1}
% %\todo[inline]{}
% \subsection{Spectre V2}

\section{Computational Effects}
\subsection{Termination}
\subsection{Non-determinism}
\subsection{Generalised Determinisation}

\section{Cyber-Physical System} 
\todo[inline]{This section probably makes more sense in its own chapter, when we explain what the latent behaviour of a CPS is.}
\section{Side Channels}
%!TEX root = ../main.tex
% Chapter Template


\chapter{Latent Behaviours} % Main chapter title
\label{ch:LatentBehaviours} % Change X to a consecutive number; for referencing this chapter elsewhere, use \ref{ChapterX}
\todo[inline]{Find a suitable quote?}
\begin{quote}
If you only do what you can do, you will never be more than what you are now.
\end{quote}

\section{Introduction}
\todo[inline]{Give an interesting motivational example: What is the notion of latent behaviour? Why do we care about them? How is it useful to study them?}
\section{Latent $F$-coalgebras and their Behaviours}
Informally, a system reveals a latent behaviour if and only if its states are altered but its dynamics are preserved. To alter states, we use a \emph{transformation} function.
%For the functor $\Functor$, the monad $\Monad=(\MFunctor,\eta,\mu)$, and a $\lambda$-bialgebra $\Bialgebra=(X,a,c)$, we can define the behaviour associated to $\Bialgebra$ by means of the semantic mapping in the category of $\Functor$-coalgebras, i.e., by means of $\TheBehaviourOf{\cdot}\colon \MFunctor(X)\rightarrow \sigma\Functor$. 
%The \emph{latent behaviour} given a transformation coalgebra $\mathbb{M}=(X,X\xrightarrow{(o_m,\delta_m)} F(X))$, is modelled by the function $\TheLatentBehaviourOf{\cdot}{\mathbb{M}}\colon X\rightarrow \sigma F$, which is coinductively defined, for $i\in I$, by
%\begin{align}
%o(\TheLatentBehaviourOf{x}{\mathbb{M}})&\triangleq o_m(x)\\
%\delta(\TheLatentBehaviourOf{x}{\mathbb{M}})(i)&\triangleq \TheLatentBehaviourOf{\delta_m(x)(i)}{\mathbb{M}}.
%\end{align}
%The transformation $a$ skews the behaviour of states to reflect the changes to the state. 

\begin{definition}[Consistent State Transformations]
Given an $F$-coalgebra $(X,c)$, any function of type $m\colon X\rightarrow X$ is a \emph{transformation} of $X$. %A transformation $m\colon X \rightarrow X$ has \emph{finite support} iff $m(x)\neq x$ only for a finite number of $x\in X$. We denote the set of finitely supported transformations by $X^X_\omega$. 
A transformation $m$ is \emph{(behaviourally) consistent} if and only if, whenever $x\sim y$, then $m(x)\sim m(y)$, for all $x,y \in X$. We denote the set of consistent transformations by $X^X|_\sim$. Henceforth, we consider only consistent transformations, unless explicitly mentioned otherwise.
\end{definition}


When we use a transformation function $m\colon X\rightarrow X$ to skew the normal behaviour of $\mathbb{X}$, we reveal the \emph{latent coalgebra of $\mathbb{X}$ under $m$}. 
\begin{definition}[Latent Coalgebra]
Given an $F$-coalgebra $\TheCoalgebra=(X,c)$ and a transformation $m$, the \emph{latent coalgebra under $m$} is $(X,c\circ m)$. 
% \begin{align}
%     \mathbb{X}\circ m\triangleq(X,{(o\circ m, \delta\circ m })).
% \end{align}
The function $\TheLatentBehaviourOfIn{\cdot}{m}{c}\colon X\rightarrow \sigma F$ defines the \emph{latent behaviour} under $m$. The homomorphism $\TheLatentBehaviourOfIn{\cdot}{m}{\mathbb{X}}$ corresponds to the semantic mapping of the $F$-coalgebra $(X,c\circ m)$; that is, for $x\in X$, 
\begin{align}
\TheLatentBehaviourOfIn{x}{c}{m}\triangleq\TheBehaviourOfIn{x}{{c\circ m}}
\end{align} 

\paragraph{Space-Time Duality} 
We now define the set $m(X)\triangleq\{m(x)\ |\ x\in X\}$. We define a dynamical system $(m(X),m\circ c\colon m(X)\rightarrow m(X))$, whose associated set of behaviours is $(m\circ c)_{m(X)}^\infty$.

\begin{figure}[t]
    \centering
    \begin{tikzcd}%[column sep=1.75cm, row sep=1cm]
        (X\times m(X), (c \circ m,m \circ c))
            \arrow[d,swap,"\fst"]
            \arrow[r,swap,"\snd"]
        & (m(X), m \circ c)
            \arrow[d,swap,"m\circ c_{(\cdot)}^\infty"']
        \\ 
        (X,c \circ m)
            \arrow[r,swap,"(c\circ m)_{(\cdot)}^\infty"]
        &  (X^\infty, (\cdot)')
    \end{tikzcd} 
    \caption{Pullback relating $c\circ m$ and $m\circ c$, describing the space-time duality of dynamical systems defined by $\id$-coalgebras.}
\end{figure}

\begin{theorem}[Fundamental Theorem of Latent Behaviour Analysis]
    \label{theo:Fundamental}
    The semantic map-ping of the $F$-coalgebra $(X,c\circ m\colon X\rightarrow F(X))$ factors through the semantic mapping of the $F$-coalgerba $(m(X),c\colon m(X)\rightarrow F(m(X)))$
    \todo[inline]{This probably needs to use bounded functors.}
\end{theorem}
\begin{proof}
    The Epi-mono factorisation from Universal Coalgebra uses the kernel of homomorphisms as the quotient. Here, we replace it by $\equiv_m$, enforcing bisimilarity between $x$ and $m(x)$. Note that $\equiv_{\id}$ is normal bisimilarity.

    The transformation $m\colon X\rightarrow X$ forces the appearances of equivalence classes $\equiv_m\subseteq X\times X$ with respect to $c$, where $x_1\equiv_m x_2$ iff $m(x_1)\sim m(x_2)$, which overrides bisimilarity in $X$ with respect to $c$. The set $X/\equiv_m$ is isomorphic to $m(X)$.
\end{proof}

\begin{figure}[t]
    \centering
    \includegraphics[width=1\textwidth]{Figures/Epi-monofactorisation.png}
    \caption{Reason why the fundamental theorem works.}
    \label{fig:LambdaCurves}
    \end{figure}

\begin{figure}[t]
    \centering
    \includegraphics[width=1\textwidth]{Figures/FundamentalTheo.png}
    \caption{Reason why the fundamental theorem works.}
    \label{fig:LambdaCurves}
    \end{figure}

\begin{proposition}[Behavioural Change]
For any dynamical system described by an $F$-coalgebra 
$(X,c\circ m\colon X\rightarrow F(X))$, where $m\colon X\rightarrow X$, and  $c\colon X\rightarrow F(X)$, there exists a \emph{unique} transformation function $\Delta_m\colon \sigma F\rightarrow \sigma F$ which maps $\TheBehaviourOfIn{x}{c}$ to $\TheLatentBehaviourOfIn{x}{c}{m}$, describing the \emph{behavioural change} of $x$, for all $x\in X$.
\end{proposition}
\begin{proof}
    The function $\Delta_{m}\colon \sigma F\rightarrow \sigma F$ is an $F$-coalgebra since $F(\sigma F)\simeq \sigma F$, so it has a behaviour. 
coinductively defined by 
{\color{red}
\todo[inline]{Transform this into a commutative diagram?}
\begin{align*}
    \Delta_{m}(\TheBehaviourOfIn{x}{c}) \sim_{(c \circ m)} (x)\\
    %\Delta_{m}(c^\infty_x)[1]&\triangleq c^\infty_{m(x)}[1]\\
    \Delta_{m}(c^\infty_x)[t+1]&\triangleq \Delta_{m}\left(c^\infty_{(c\circ m)(x)}\right)[t]
\end{align*}
}
\end{proof}

\section{Dynamical Systems of the $\id$ Functor}
% Theorem~\ref{theo:Fundamental} is a far stretch from the fundamental theorem of arithmetic, which states that every natural number can be decomposed into a multiplication of primes, but it follows a similar reasoning. Normally, we ignore this decomposition, so $\delta=c$ and $m=\id$. 
% \todo[inline]{It would be nice to have ``primes" here, and they may exist. They'll probably }



The \emph{fundamental theorem of latent behaviour analysis} implies the existence of unique transformation function $\Delta_m\colon X^\infty\rightarrow X^\infty$ which maps $c_{x}^\infty$ to both $\delta_{x}^\infty$ and $c_{m(x)}^\infty$ for all $x\in X$. The function $\Delta_m$ models the behavioural change induced by $m$ when applied to the $F$-coalgebra $(X,c)$. This unique function $\delta_{x}^\infty$ and $c_{m(x)}^\infty$ is 
coinductively defined by 
% \todo[inline]{BE CAREFUL HOW YOU DEFINE THE INDICES and if you want behaviours to include $x$, in coalgberas that is not very common.}
\begin{align*}
    \Delta_{m}(c^\infty_x)[0]&\triangleq (c \circ m)(x)\\
    %\Delta_{m}(c^\infty_x)[1]&\triangleq c^\infty_{m(x)}[1]\\
    \Delta_{m}(c^\infty_x)[t+1]&\triangleq \Delta_{m}\left(c^\infty_{(c\circ m)(x)}\right)[t]
\end{align*}
This definition compounds the effect of $m$ over time.
% \begin{align}
%     \Delta_m(c_{x}^\infty)\triangleq(m\circ c)_{x}^\infty
% \end{align}
% which is not a very useful definition. However, if $m$ satisfies some linearity properties, then this function $\Delta_m(c_{x}^\infty)$ can be defined (co)inductively.

For example, consider the system $(\mathbb{N},c=(+1))$ and the operator $m=(*3)$. The corresponding definition is $\Delta_{m}$ by
\begin{align*}
    \Delta_{m}(c^\infty_x)[0]&=1\\%(+1)^\infty_{2x}[1]\\
    (\Delta_{m}(c^\infty_x))[t+1]&=\Delta_{m}(c^\infty_{3x+1})[t]
\end{align*}
%This definition compounds the effect of the operator $(*2)$. 
From $(1,2,3,\ldots)=(+1)^\infty_0$, we obtain the sequence $(1,7,13,40,121,\ldots)$. 
% Now, $\Delta_{(*2)}(\omega)[0]=0$. %, so $((+1)\circ (*2))^\infty_0[0]=0$. 
% Next, $\Delta_{(*2)}(((+1)\circ (*2))^\infty_{1})[0]=1$

%$\Delta_{(*2)}(\omega)[t+1]$ since $\omega=(+1)^\infty_{(2*0)+1}$
\begin{definition}
The operator $m$ is \emph{linear with respect to $c$} if and only if
\begin{align}
    (c\circ m)^\infty_x[t]= (m \circ c)^\infty_x[t].
\end{align}    
\end{definition}

%Functorial operators are quite rare. It means that their effects do not compound over time. 
% (Linear operators have a deep relation to bialgebras of the identity functor, where $c\circ m= m\circ c $.)
% The operator $(*2)$ is not linear with respect to $(+1)$, but the operator $+3$ is. $(0,4,8,)$.

Depending on the properties of $m$ and $\Delta_m$, we might infer whether some behavioural property was preserved for all sequences, i.e., if $P(c^\infty_x)$, then $P((c\circ m)^\infty_x)$. For example, for $m=(*3)$, and $c=(+1)$, the sequence is still strictly increasing. 



%We say that thefunction $\Delta^m$ has a solution if it can be written








THe usefulness of latent behaviour analysis is that you can decompose a behaviour-defining into several components. Each of those components 



\todo[inline]{Latent coalgebras are coalgebras with the state space deformed.}
\todo[inline]{THESE DIAGRAMS ARE WRONG. THE BEH OF C CANNOT APPEAR LIKE THISOR YOU CAN GO VIA M to the end of c}
\end{definition}
\begin{figure}
    \centering
    \begin{tikzcd}[column sep=large]
        \sigma F
            \arrow[d, "\simeq","\omega"'] 
        &X
            \arrow[r, "m"]
            %\arrow[rd, "c\circ m", red]
            \arrow[l, dotted, swap,"\TheBehaviourOf{\cdot}_{c\circ m}"]
        &X 
            \arrow[r, dotted, "\TheBehaviourOf{\cdot}_c"] 
            \arrow[d, "c"] 
        & \sigma F 
            \arrow[d, "\simeq","\omega"'] 
        \\
        F(\sigma F)
        &
        &F(X) 
            \arrow[r, dotted, "F(\TheBehaviourOf{\cdot}_c)"]
            \arrow[ll, dotted, swap,"F(\TheBehaviourOf{\cdot}_{c\circ m})"]     
        &F(\sigma F)
    \end{tikzcd}
    \caption{$(X,c\circ m)$ is an $F$-coalgebra, so it has a unique $F$-homomorphism to the final $F$-coalgebra $(\sigma F, \omega)$. Geometrically, $m$ is a transformation of the state space, so some preserve certain structural properties (e.g. a permutation) while others do not (e.g. a collapsing function to some fixed $x$).}
\end{figure}
\begin{figure}
    \centering
    \begin{tikzcd}[column sep=large]
        \sigma F
            \arrow[d, "\simeq","\omega"'] 
        &&X
            \arrow[r, "m"]
            \arrow[d, "b\circ c\circ m"] 
            %\arrow[rd, "c\circ m", red]
            \arrow[ll, dotted, swap,"\TheBehaviourOf{\cdot}_{b\circ c\circ m}"]
        &X 
            \arrow[rr, dotted, "\TheBehaviourOf{\cdot}_c"] 
            \arrow[d, "c"] 
        && \sigma F 
            \arrow[d, "\simeq","\omega"'] 
        \\
        F(\sigma F) 
        &&F(X) \arrow[ll, dotted, swap,"F(\TheBehaviourOf{\cdot}_{b\circ c\circ m})"]
        &F(X) 
            \arrow[rr, dotted, "F(\TheBehaviourOf{\cdot}_c)"]
            \arrow[l, "b"]     
        &&F(\sigma F)
    \end{tikzcd}
    \caption{Latency using both $m$ and $b$ to reveal latent behaviours: $(X,b\circ c\circ m)$ is still an $F$-coalgebra, so it has a unique $F$-homomorphism to the final $F$-coalgebra $(\sigma F, \omega)$. Metaphorically, consider the function $c$ as a causal model which transforms the cause $x$ into the consequence $c(x)$; the transformation $m$ corresponds to a change of cause, while a transformation $b$ corresponds to a change of consequence. In a more concrete example, assume that we are grading a test, and $c$ is the grading rules. The value $c(x)$ is the grade assigned to the answers $x$. A transformation $m$ would correspond to changing the answers before grading, while a transformation $b$ corresponds to changing the grade after the answers have been graded. Now, if $F(X)$ is the The transformation $b$ needs not respect the properties of $c$, since it is applied after it. Thus, it would be possible to change the grade through $b$ to some value that is impossible to obtain by any possible combination of answers (i.e. for every $m$ that changes answers).}
\end{figure}
\begin{figure}
    \centering
    \begin{tikzcd}[column sep=large]
        \sigma F
            \arrow[d, "\simeq","\omega"'] 
        &X
            \arrow[r, "m"]
            %\arrow[rd, "c\circ m", red]
            \arrow[l, dotted, swap,"\TheBehaviourOf{\cdot}_{c\circ m}"]
        &X 
            \arrow[r, dotted, "\TheBehaviourOf{\cdot}_c"] 
            \arrow[d, "c"] 
        & \sigma F 
            \arrow[d, "\simeq","\omega"'] 
        \\
        F(\sigma F)
        &
        &F(X) 
            \arrow[r, dotted, "F(\TheBehaviourOf{\cdot}_c)"]
            \arrow[ll, dotted, swap,"F(\TheBehaviourOf{\cdot}_{c\circ m})"]     
        &F(\sigma F)
    \end{tikzcd}
    \caption{$(X,T(X)\xrightarrow{c\circ a}F(X))$ is an $FT$-coalgebra, (not quite a bialgebra, but it could be). $T(X)$ defines specifications for $X$, so we could mutate those instead of $X$ directly. This is what the paper by Harrison and Goldstein does \cite{DoJudgeATestByItsCover}}
\end{figure}


{\color{red}The deformation $b$ is not that interesting because it is too flexible. If $F=\id$ then $t$ has the same type as $s$ and they compose, so we can approximate $t.c.s$ with $c.s'$, where $s'=s.t$. If $F(X)$ has only one component, then $t=const \phi$ forces the system to have the behaviour that the attacker wants. More precisely, it can reveal any $F$-coalgebra for that carrier set. The balance would be to allow the attacker of $t$ to influence only a set of components in $F(X)$, just like we do with $s$. Consider the $F$-coalgebras of $()$ for the functor $2x\id^2$; there are only two: $c_0(())=(False,const ())$ and $c_1(())=(True,const ())$. Using state transformations we cannot reveal new behaviours given $c_0$ or $c_1$, but with with behaviour transformations we can: $t=(\texttt{not},\id)$ causes $t.c_0=c_1$ and $t.c_1=c_0$ so you can strictly do more. The question is, do we need more? 

Poetically, $m$ corresponds to a deformation of space, and $b$ corresponds to a deformation of causality.
}

\begin{example}
\label{ex:LatentBehaviour}
Consider the functor $G=2\times \texttt{id}^2$ and a $G$-coalgebra $\mathbb{X}=(2\times2,(\gamma,\delta))$ that can be used to recognise sequences of zeroes and ones that end in two consecutive ones when started at state 0., i.e. the language $(0+1)^*111^*$. We define said $G$-coalgebra $\mathbb{X}$ by
\begin{align}
\gamma(x,y)&\triangleq x \land y;\\
\delta(x,y)(i)&\triangleq(i,x).
\end{align}
Figure~\ref{fig:ExampleLatent} shows the deterministic finite automaton corresponding to this $G$-coalgebra; the coalgebra is not minimal, because the states $(0,0)$ and $(0,1)$ are bisimilar.
% \begin{figure}[t]
% \centering
% \begin{tikzpicture}
% \node[state] (00) {$(0,0)$};
% \node[state, below right of=00] (01) {$(0,1)$};
% \node[state, above right  of=00] (10) {$(1,0)$};
% \node[state, accepting, below right of=10] (11) {$(1,1)$};
% \draw (00) edge[bend left, above] node{1} (10)
% (00) edge[loop above] node{0} (00)
% (01) edge[bend left, above] node[left]{1} (10)
% (01) edge[bend left, above] node{0} (00)
% (10) edge[bend left, above] node{1} (11)
% (10) edge[bend left, above] node[right]{0} (01)
% (11) edge[loop above] node{1} (11)
% (11) edge[bend left, above] node{0} (01)
% ;\end{tikzpicture}
% \caption{Corresponding automaton to the $G-$coalgebra described in Example~\ref{ex:LatentBehaviour}.}
% \label{fig:ExampleLatent}
% \end{figure}

\begin{figure}[t]
    \centering
    \begin{tikzpicture}
        \node[state] (00) {$(0,0)$};
        \node[state, below right of=00] (01) {$(0,1)$};
        \node[state, above right  of=00] (10) {$(1,0)$};
        \node[state, accepting, below right of=10] (11) {$(1,1)$};
        \draw (00) edge[bend left, above] node{1} (10)
        (00) edge[loop above] node{0} (00)
        (01) edge[bend left, above] node{1} (10)
        (01) edge[bend left, above] node{0} (00)
        (10) edge[bend left, above] node{1} (11)
        (10) edge[bend left, above] node{0} (01)
        (11) edge[loop above] node{1} (11)
        (11) edge[bend left, above] node{0} (01)
        ;\end{tikzpicture}
    \caption{Corresponding automaton to the $G-$coalgebra described in Example~\ref{ex:LatentBehaviour}.}
    \label{fig:ExampleLatent}
\end{figure}

This $G$-coalgebra has 256 transformations, given by the $4^4$ endofunctions in $X$. We use these transformations to reveal latent behaviours.%, and every consistent transformation $m$ satisfies $m(0,0)~m(0,1)$. 
\todo[inline]{You can probably graph this with a 3 dimensional graph like a vector field whose axises are input, state, and output. We could use four dimensions so that state is 2-dimensional}
% We present the automata that correspond to latent $G-$coalgebras in Figures~\ref{fig:3.2}--\ref{fig:3.28}. Note that many of these latent coalgebras are not minimal, and many display the same behaviours.

% From 27 endofuctions and three intended behaviours, we can derive the 24 latent behaviours presented in Table~\ref{tab:ExampleLatentBehaviours} as regular expressions. 
{\color{red}
\todo[inline]{Rewrite this!!!}
The existence of latent behaviours opens a new possibility for system repurposing: if we wanted the system to recognise the language $(0+1)^*1$ (i.e., the language of sequences that end in 1), we could mutate our original system using a transformation where $0\mapsto0, 1\mapsto2$ and $2\mapsto2$ (shown in Figure~\ref{fig:3.10}), whilst preserving $0$ as the initial state. Nevertheless, if providing the transformation is within the capabilities of an adversary, it would also mean that they can repurpose our system as well. 
}


\begin{figure}[t]
\centering
\begin{tikzpicture}
    \node[state] (00) {$(0,0)$};
    \node[state, below right of=00] (01) {$(0,1)$};
    \node[state, above right  of=00] (10) {$(1,0)$};
    \node[state, accepting, below right of=10] (11) {$(1,1)$};
    \draw (00) edge[bend left, above] node{1} (10)
    (00) edge[loop above] node{0} (00)
    (01) edge[bend left, above] node{1} (11)
    (01) edge[loop above] node{0} (01)
    (10) edge[loop above] node{1} (10)
    (10) edge[bend left, above] node{0} (00)
    (11) edge[loop above] node{1} (11)
    (11) edge[bend left, above] node{0} (01)
    ;\end{tikzpicture}
\caption{Corresponding automaton to the latent $G-$coalgebra under transformation $(a,b)\mapsto(b,a)$. This transformation is not consistent, since $(0,0)\sim (0,1)$ in the original $F$-coalgebra, but $(0,0)$ and $(1,0)$ are no longer bisimilar}% described in Example~\ref{ex:LatentBehaviour}.}
\label{fig:ExampleInconsistentLatent}
\end{figure}

\begin{figure}[t]
\centering
\begin{tikzpicture}
    \node[state] (00) {$(0,0)$};
    \node[state, below right of=00] (01) {$(0,1)$};
    \node[state, accepting, above right  of=00] (10) {$(1,0)$};
    \node[state, accepting, below right of=10] (11) {$(1,1)$};
    \draw (00) edge[bend left, above] node{1} (10)
    (00) edge[loop above] node{0} (00)
    (01) edge[bend left, above] node{1} (10)
    (01) edge[bend left, above] node{0} (00)
    (10) edge[bend left, above] node{1} (11)
    (10) edge[bend left, above] node{0} (01)
    (11) edge[loop above] node{1} (11)
    (11) edge[bend left, above] node{0} (01)
    ;\end{tikzpicture}
\caption{Corresponding automaton to the latent $G-$coalgebra under transformation $(0,x)\mapsto (0,0)$ and $(1,x)\mapsto (1,1)$. This transformation is consistent, and reveals a new behaviour from the state $(0,0)$, more precisely, the language $(0,1)^*1$ of sequences that end in $1$.}% described in Example~\ref{ex:LatentBehaviour}.}
\label{fig:ExampleConsistentLatent}
\end{figure} 
    
% \begin{table}[t]
% \centering
% \begin{tabular}{|l | l | }
% \hline
% $\rho_{\ref{fig:3.2}.0}$ &  $\emptyset$   \\
% $\rho_{\ref{fig:3.4}.2}$ &  $1^*$ \\
% $\rho_{\ref{fig:3.7}.0}$ &  $(0+1)^*111^*$\\%$(0+10+111^*0)^*111^*$ \\
% $\rho_{\ref{fig:3.7}.1}$ &  $\rho_{\ref{fig:3.7}.0}+1$ \\
% $\rho_{\ref{fig:3.7}.2}$ &  $\rho_{\ref{fig:3.7}.0}+1+\varepsilon$ \\
% $\rho_{\ref{fig:3.8}.1}$ &  $(00^*1+10^*1)^*$ \\
% $\rho_{\ref{fig:3.8}.0}$ &  $0^*1\rho_{\ref{fig:3.8}.1}$ \\
% $\rho_{\ref{fig:3.9}.2}$& $(0+1)^*01$\\
% $\rho_{\ref{fig:3.9}.0}$&   $\rho_{\ref{fig:3.9}.2}+1$\\
% $\rho_{\ref{fig:3.9}.1}$&   $\rho_{\ref{fig:3.9}.2}+1+\varepsilon$\\
% $\rho_{\ref{fig:3.10}.0}$ &  $(0+1)^*1$ \\
% $\rho_{\ref{fig:3.10}.1}$ &  $\rho_{\ref{fig:3.10}.0}+\varepsilon$ \\
% $\rho_{\ref{fig:3.13}.1}$ &  $1^*0\rho_{\ref{fig:3.10}.0}$ \\
% $\rho_{\ref{fig:3.17}.0}$ &  $(0+10+(11)^+(0+10))^*(11)^+$ \\
% $\rho_{\ref{fig:3.17}.1}$ &  $(11)^*+(0+10)\rho_{\ref{fig:3.17}.0}$ \\
% $\rho_{\ref{fig:3.17}.2}$ &  $0\rho_{\ref{fig:3.17}.0}+1\rho_{\ref{fig:3.17}.1}$ \\
% $\rho_{\ref{fig:3.18}.0}$ &  $\varepsilon$ \\
% $\rho_{\ref{fig:3.20}.1}$ &  $(0+1)^*0$ \\
% $\rho_{\ref{fig:3.20}.0}$ &  $\rho_{\ref{fig:3.20}.1}+\varepsilon$ \\
% $\rho_{\ref{fig:3.22}.0}$ &  $(1+0)^*$ \\
% $\rho_{\ref{fig:3.22}.1}$ &  $1^*0(1+0)^*$ \\
% $\rho_{\ref{fig:3.25}.1}$ &  $1(1+0)^*+0(1+0)^*$ \\
% $\rho_{\ref{fig:3.26}.0}$ &  $(0+10+11)^*$ \\
% $\rho_{\ref{fig:3.26}.2}$ &  $(0+1)\rho_{\ref{fig:3.26}.0}$\\
% \hline
% \end{tabular}
% \caption{Latent behaviours for all transformations for the coalgebra that recognises $(0+1)^*11$. }
% \label{tab:ExampleLatentBehaviours}
% \end{table}

% %$\rho_{\ref{fig:3.8}.0}$ &  $$ \\
% %$\rho_{\ref{fig:3.8}.1}$ &  $$ \\
% %$\rho_{\ref{fig:3.8}.2}$ &  $$ \\
% %------------------------------------------------------------------------------------------------------------------------
% \begin{figure}[t]
% \centering
% \begin{tikzpicture}
% \node[state] (q1) {$0$};
% \node[state, right of=q1] (q2) {$1$};
% \node[state, right of=q2] (q3) {$2$};
% \draw (q1) edge[loop above] node{0} (q1)
% (q1) edge[bend left, above] node{1} (q2)
% (q2) edge[bend left, above] node{0} (q1)
% (q2) edge[loop above] node{1} (q2)
% (q3) edge[bend left, above] node{1} (q2)
% (q3) edge[bend left, below] node{0} (q1);
% \end{tikzpicture}
% \caption{Latent coalgebra under graph $G(m)=\set{(0,0),(1,0),(2,0)}$. $\TheLatentBehaviourOf{0}{m}=\TheLatentBehaviourOf{1}{m}=\TheLatentBehaviourOf{2}{m}=\emptyset$ }
% \label{fig:3.2}

% \begin{tikzpicture}
% \node[state] (q1) {$0$};
% \node[state, right of=q1] (q2) {$1$};
% \node[state, right of=q2] (q3) {$2$};
% \draw (q1) edge[loop above] node{0} (q1)
% (q1) edge[bend left, above] node{1} (q2)
% (q2) edge[bend left, above] node{0} (q1)
% (q2) edge[loop above] node{1} (q2)
% (q3) edge[loop above] node{1} (q3)
% (q3) edge[bend left, below] node{0} (q1);
% \end{tikzpicture}
% \caption{Latent coalgebra under graph $G(m)=\set{(0,0),(1,0),(2,1)}$. $\TheLatentBehaviourOf{0}{m}=\TheLatentBehaviourOf{1}{m}=\TheLatentBehaviourOf{2}{m}=\emptyset$}
% \label{fig:3.3}

% \begin{tikzpicture}
% \node[state] (q1) {$0$};
% \node[state, right of=q1] (q2) {$1$};
% \node[state, accepting, right of=q2] (q3) {$2$};
% \draw (q1) edge[loop above] node{0} (q1)
% (q1) edge[bend left, above] node{1} (q2)
% (q2) edge[bend left, above] node{0} (q1)
% (q2) edge[loop above] node{1} (q2)
% (q3) edge[loop above] node{1} (q3)
% (q3) edge[bend left, below] node{0} (q1);
% \end{tikzpicture}
% \caption{Latent coalgebra under graph $G(m)=\set{(0,0),(1,0),(2,2)}$. $\TheLatentBehaviourOf{0}{m}=\TheLatentBehaviourOf{1}{m}=\emptyset$, $\TheLatentBehaviourOf{2}{m}=1^*$}
% \label{fig:3.4}
% \end{figure}

% \begin{figure}[t]
% \captionsetup{singlelinecheck=off}
% \centering
% \begin{tikzpicture}
% \node[state] (q1) {$0$};
% \node[state, right of=q1] (q2) {$1$};
% \node[state, right of=q2] (q3) {$2$};
% \draw (q1) edge[loop above] node{0} (q1)
% (q1) edge[bend left, above] node{1} (q2)
% (q2) edge[bend left, above] node{0} (q1)
% (q2) edge[bend left, above] node{1} (q3)
% (q3) edge[bend left, above] node{1} (q2)
% (q3) edge[bend left, below] node{0} (q1);
% \end{tikzpicture}
% \caption{Latent coalgebra under graph $G(m)=\set{(0,0),(1,1),(2,0)}$ . $\TheLatentBehaviourOf{0}{m}=\TheLatentBehaviourOf{1}{m}=\TheLatentBehaviourOf{2}{m}=\emptyset$}

% \begin{tikzpicture}
% \node[state] (q1) {$0$};
% \node[state, right of=q1] (q2) {$1$};
% \node[state, right of=q2] (q3) {$2$};
% \draw (q1) edge[loop above] node{0} (q1)
% (q1) edge[bend left, above] node{1} (q2)
% (q2) edge[bend left, above] node{0} (q1)
% (q2) edge[bend left, above] node{1} (q3)
% (q3) edge[loop above] node{1} (q3)
% (q3) edge[bend left, below] node{0} (q1);
% \end{tikzpicture}
% \caption{Latent coalgebra under graph $G(m)=\set{(0,0),(1,1),(2,1)}$ . $\TheLatentBehaviourOf{0}{m}=\TheLatentBehaviourOf{1}{m}=\TheLatentBehaviourOf{2}{m}=\emptyset$}

% \begin{tikzpicture}
% \node[state] (q1) {$0$};
% \node[state, right of=q1] (q2) {$1$};
% \node[state,accepting, right of=q2] (q3) {$2$};
% \draw (q1) edge[loop above] node{0} (q1)
% (q1) edge[bend left, above] node{1} (q2)
% (q2) edge[bend left, above] node{0} (q1)
% (q2) edge[bend left, above] node{1} (q3)
% (q3) edge[loop above] node{1} (q3)
% (q3) edge[bend left, below] node{0} (q1);
% \end{tikzpicture}
% \caption{Latent coalgebra under graph $G(m)=\set{(0,0),(1,1),(2,2)}$ (i.e. $m=\texttt{id}$). Latent behaviours under $m$ are the original behaviours: 
% \protect\begin{align*}
% 	\TheLatentBehaviourOf{0}{m}&=(0+1)^*111^*, \\
% 	\TheLatentBehaviourOf{1}{m}&=1+\TheLatentBehaviourOf{0}{m}\\
% 	\TheLatentBehaviourOf{2}{m}&=\varepsilon + 1+ \TheLatentBehaviourOf{0}{m}
% \protect\end{align*}
% }
% \label{fig:3.7}
% \end{figure}



% \begin{figure}[t]
% \captionsetup{singlelinecheck=off}
% \centering
% \begin{tikzpicture}
% \node[state] (q1) {$0$};
% \node[state,accepting, right of=q1] (q2) {$1$};
% \node[state, right of=q2] (q3) {$2$};
% \draw (q1) edge[loop above] node{0} (q1)
% (q1) edge[bend left, above] node{1} (q2)
% (q2) edge[bend left, above] node{0} (q1)
% (q2) edge[bend left, above] node{1} (q3)
% (q3) edge[bend left, above] node{1} (q2)
% (q3) edge[bend left, below] node{0} (q1);
% \end{tikzpicture}
% \caption{Latent coalgebra under graph $G(m)=\set{(0,0),(1,2),(2,0)}$. 
% \protect\begin{align*}
% \TheLatentBehaviourOf{0}{m}&=\TheLatentBehaviourOf{2}{m}=0^*1\TheLatentBehaviourOf{1}{m},\\
% \TheLatentBehaviourOf{1}{m}&=(00^*1+10^*1)^*
% \protect\end{align*}
% \label{fig:3.8}
% }

% \begin{tikzpicture}
% \node[state] (q1) {$0$};
% \node[state,accepting, right of=q1] (q2) {$1$};
% \node[state, right of=q2] (q3) {$2$};
% \draw (q1) edge[loop above] node{0} (q1)
% (q1) edge[bend left, above] node{1} (q2)
% (q2) edge[bend left, above] node{0} (q1)
% (q2) edge[bend left, above] node{1} (q3)
% (q3) edge[loop above] node{1} (q3)
% (q3) edge[bend left, below] node{0} (q1);
% \end{tikzpicture}
% \caption{Latent coalgebra under graph $G(m)=\set{(0,0),(1,2),(2,1)}$.
% \protect\begin{align*}
% \TheLatentBehaviourOf{0}{m}&=1+\TheLatentBehaviourOf{2}{m},\\
% \TheLatentBehaviourOf{1}{m}&=\varepsilon+1+\TheLatentBehaviourOf{2}{m},\\
% \TheLatentBehaviourOf{2}{m}&=(0+1)^*01
% \protect\end{align*}
% \label{fig:3.9}
% }

% \begin{tikzpicture}
% \node[state] (q1) {$0$};
% \node[state,accepting, right of=q1] (q2) {$1$};
% \node[state, accepting,right of=q2] (q3) {$2$};
% \draw (q1) edge[loop above] node{0} (q1)
% (q1) edge[bend left, above] node{1} (q2)
% (q2) edge[bend left, above] node{0} (q1)
% (q2) edge[bend left, above] node{1} (q3)
% (q3) edge[loop above] node{1} (q3)
% (q3) edge[bend left, below] node{0} (q1);
% \end{tikzpicture}
% \caption{Latent coalgebra under graph $G(m)=\set{(0,0),(1,2),(2,2)}$.
% \protect\begin{align*}
% \TheLatentBehaviourOf{0}{m}&=(0+1)^*1,\\ 
% \TheLatentBehaviourOf{1}{m}&=\TheLatentBehaviourOf{2}{m}=\varepsilon + \TheLatentBehaviourOf{0}{m}
% \protect\end{align*}
% \label{fig:3.10}
% }
% \end{figure}

% \begin{figure}[t]
% \captionsetup{singlelinecheck=off}
% \centering
% \begin{tikzpicture}
% \node[state] (q1) {$0$};
% \node[state, right of=q1] (q2) {$1$};
% \node[state, right of=q2] (q3) {$2$};
% \draw (q1) edge[loop above] node{0} (q1)
% (q1) edge[bend left, above] node{1} (q3)
% (q2) edge[bend left, above] node{0} (q1)
% (q2) edge[loop right] node{1} (q2)
% (q3) edge[bend left, above] node{1} (q2)
% (q3) edge[bend left, below] node{0} (q1);
% \end{tikzpicture}
% \caption{Latent coalgebra under graph $G(m)=\set{(0,1),(1,0),(2,0)}$. $\TheLatentBehaviourOf{0}{m}=
% \TheLatentBehaviourOf{1}{m}=\TheLatentBehaviourOf{2}{m}=\emptyset$}

% \begin{tikzpicture}
% \node[state] (q1) {$0$};
% \node[state, right of=q1] (q2) {$1$};
% \node[state, right of=q2] (q3) {$2$};
% \draw (q1) edge[loop above] node{0} (q1)
% (q1) edge[bend left, above] node{1} (q3)
% (q2) edge[bend left, above] node{0} (q1)
% (q2) edge[loop right] node{1} (q2)
% (q3) edge[loop above] node{1} (q3)
% (q3) edge[bend left, below] node{0} (q1);
% \end{tikzpicture}
% \caption{Latent coalgebra under graph $G(m)=\set{(0,1),(1,0),(2,1)}$. $\TheLatentBehaviourOf{0}{m}=
% \TheLatentBehaviourOf{1}{m}=\TheLatentBehaviourOf{2}{m}=\emptyset$}


% \begin{tikzpicture}
% \node[state] (q1) {$0$};
% \node[state, right of=q1] (q2) {$1$};
% \node[state,accepting, right of=q2] (q3) {$2$};
% \draw (q1) edge[loop above] node{0} (q1)
% (q1) edge[bend left, above] node{1} (q3)
% (q2) edge[bend left, above] node{0} (q1)
% (q2) edge[loop right] node{1} (q2)
% (q3) edge[loop above] node{1} (q3)
% (q3) edge[bend left, below] node{0} (q1);
% \end{tikzpicture}
% \caption{Latent coalgebra under graph $G(m)=\set{(0,1),(1,0),(2,2)}$. 
% \protect\begin{align*}
% \TheLatentBehaviourOf{0}{m}&=(0+1)^*1,\\ 
% \TheLatentBehaviourOf{1}{m}&=1^*0\TheLatentBehaviourOf{0}{m},\\
% \TheLatentBehaviourOf{2}{m}&=\varepsilon+\TheLatentBehaviourOf{0}{m}
% %\TheLatentBehaviourOf{2}{m}&=1^*+\TheLatentBehaviourOf{1}{m}
% \protect\end{align*}
% }
% \label{fig:3.13}
% \end{figure}

% \begin{figure}[t]
% \captionsetup{singlelinecheck=off}
% \centering
% \begin{tikzpicture}
% \node[state] (q1) {$0$};
% \node[state, right of=q1] (q2) {$1$};
% \node[state, right of=q2] (q3) {$2$};
% \draw (q1) edge[loop above] node{0} (q1)
% (q1) edge[bend left, above] node{1} (q3)
% (q2) edge[bend left, above] node{0} (q1)
% (q2) edge[bend left, above] node{1} (q3)
% (q3) edge[bend left, above] node{1} (q2)
% (q3) edge[bend left, below] node{0} (q1);
% \end{tikzpicture}
% \caption{Latent coalgebra under graph $G(m)=\set{(0,1),(1,1),(2,0)}$. $\TheLatentBehaviourOf{0}{m}=
% \TheLatentBehaviourOf{1}{m}=\TheLatentBehaviourOf{2}{m}=\emptyset$}


% \begin{tikzpicture}
% \node[state] (q1) {$0$};
% \node[state, right of=q1] (q2) {$1$};
% \node[state, right of=q2] (q3) {$2$};
% \draw (q1) edge[loop above] node{0} (q1)
% (q1) edge[bend left, above] node{1} (q3)
% (q2) edge[bend left, above] node{0} (q1)
% (q2) edge[bend left, above] node{1} (q3)
% (q3) edge[loop above] node{1} (q3)
% (q3) edge[bend left, below] node{0} (q1);
% \end{tikzpicture}
% \caption{Latent coalgebra under graph $G(m)=\set{(0,1),(1,1),(2,1)}$. $\TheLatentBehaviourOf{0}{m}=
% \TheLatentBehaviourOf{1}{m}=\TheLatentBehaviourOf{2}{m}=\emptyset$}


% \begin{tikzpicture}
% \node[state] (q1) {$0$};
% \node[state, right of=q1] (q2) {$1$};
% \node[state, accepting, right of=q2] (q3) {$2$};
% \draw (q1) edge[loop above] node{0} (q1)
% (q1) edge[bend left, above] node{1} (q3)
% (q2) edge[bend left, above] node{0} (q1)
% (q2) edge[bend left, above] node{1} (q3)
% (q3) edge[loop above] node{1} (q3)
% (q3) edge[bend left, below] node{0} (q1);
% \end{tikzpicture}
% \caption{Latent coalgebra under graph $G(m)=\set{(0,1),(1,1),(2,2)}$.
% \protect\begin{align*}
% \TheLatentBehaviourOf{0}{m}&=\TheLatentBehaviourOf{1}{m}=(0+1)^*1,\\ 
% \TheLatentBehaviourOf{2}{m}&=\TheLatentBehaviourOf{0}{m}+\varepsilon
% \protect\end{align*}
% }
% \end{figure}

% \begin{figure}[t]
% \centering
% \captionsetup{singlelinecheck=off}
% \begin{tikzpicture}
% \node[state] (q1) {$0$};
% \node[state,accepting, right of=q1] (q2) {$1$};
% \node[state, right of=q2] (q3) {$2$}; 
% \draw (q1) edge[loop above] node{0} (q1)
% (q1) edge[bend left, above] node{1} (q3)
% (q2) edge[bend left, above] node{0} (q1)
% (q2) edge[bend left, above] node{1} (q3)
% (q3) edge[bend left, above] node{1} (q2)
% (q3) edge[bend left, below] node{0} (q1);
% \end{tikzpicture}
% \caption{Latent coalgebra under graph $G(m)=\set{(0,1),(1,2),(2,0)}$.
% \protect\begin{align*}
% \TheLatentBehaviourOf{0}{m}&=(0+10+(11)^+(0+10))^*(11)^+ \\ %(0+1(0+1(11)^*(0+10)))^*11(11)^* ,\\ 
% \TheLatentBehaviourOf{1}{m}&=(11)^*+(0+10)\TheLatentBehaviourOf{0}{m},\\
% \TheLatentBehaviourOf{2}{m}&=0\TheLatentBehaviourOf{0}{m}+1\TheLatentBehaviourOf{1}{m}
% \protect\end{align*}
% \label{fig:3.17}
% }

% \begin{tikzpicture}
% \node[state] (q1) {$0$};
% \node[state,accepting, right of=q1] (q2) {$1$};
% \node[state, right of=q2] (q3) {$2$};
% \draw (q1) edge[loop above] node{0} (q1)
% (q1) edge[bend left, above] node{1} (q3)
% (q2) edge[bend left, above] node{0} (q1)
% (q2) edge[bend left, above] node{1} (q3)
% (q3) edge[loop above] node{1} (q3)
% (q3) edge[bend left, below] node{0} (q1);
% \end{tikzpicture}
% \caption{Latent coalgebra under graph $G(m)=\set{(0,1),(1,2),(2,1)}$.
% \protect\begin{align*}
% \TheLatentBehaviourOf{0}{m}&=\TheLatentBehaviourOf{2}{m}=\emptyset,\\ 
% \TheLatentBehaviourOf{1}{m}&=\varepsilon
% \protect\end{align*}
% \label{fig:3.18}
% }

% \begin{tikzpicture}
% \node[state] (q1) {$0$};
% \node[state,accepting, right of=q1] (q2) {$1$};
% \node[state, accepting, right of=q2] (q3) {$2$};
% \draw (q1) edge[loop above] node{0} (q1)
% (q1) edge[bend left, above] node{1} (q3)
% (q2) edge[bend left, above] node{0} (q1)
% (q2) edge[bend left, above] node{1} (q3)
% (q3) edge[loop above] node{1} (q3)
% (q3) edge[bend left, below] node{0} (q1);
% \end{tikzpicture}
% \caption{Latent coalgebra under graph $G(m)=\set{(0,1),(1,2),(2,2)}$.
% \protect\begin{align*}
% \TheLatentBehaviourOf{0}{m}&=(0+1)^*1,\\
% \TheLatentBehaviourOf{1}{m}&=\TheLatentBehaviourOf{2}{m}=\varepsilon+\TheLatentBehaviourOf{0}{m}
% \protect\end{align*}
% }
% \end{figure}

% \begin{figure}[t]
% \captionsetup{singlelinecheck=off}
% \centering
% \begin{tikzpicture}
% \node[state, accepting] (q1) {$0$};
% \node[state, right of=q1] (q2) {$1$};
% \node[state, right of=q2] (q3) {$2$};
% \draw (q1) edge[loop above] node{0} (q1)
% (q1) edge[bend left, above] node{1} (q3)
% (q2) edge[bend left, above] node{0} (q1)
% (q2) edge[loop right] node{1} (q2)
% (q3) edge[bend left, above] node{1} (q2)
% (q3) edge[bend left, below] node{0} (q1);
% \end{tikzpicture}
% \caption{Latent coalgebra under graph $G(m)=\set{(0,2),(1,0),(2,0)}$. 
% \protect\begin{align*}
% \TheLatentBehaviourOf{0}{m}&=(0+1)^*0+\varepsilon ,\\ 
% \TheLatentBehaviourOf{1}{m}&=(0+1)^*0
% \protect\end{align*} 
% \label{fig:3.20}
% }

% \begin{tikzpicture}
% \node[state, accepting] (q1) {$0$};
% \node[state, right of=q1] (q2) {$1$};
% \node[state, right of=q2] (q3) {$2$};
% \draw (q1) edge[loop above] node{0} (q1)
% (q1) edge[bend left, above] node{1} (q3)
% (q2) edge[bend left, above] node{0} (q1)
% (q2) edge[loop right] node{1} (q2)
% (q3) edge[loop above] node{1} (q3)
% (q3) edge[bend left, below] node{0} (q1);
% \end{tikzpicture}
% \caption{Latent coalgebra under graph $G(m)=\set{(0,2),(1,0),(2,1)}$.
% \protect\begin{align*}
% \TheLatentBehaviourOf{0}{m}&=(0+1)^*0+\varepsilon ,\\ 
% \TheLatentBehaviourOf{1}{m}&=(0+1)^*0
% \protect\end{align*}
% }

% \begin{tikzpicture}
% \node[state, accepting] (q1) {$0$};
% \node[state, right of=q1] (q2) {$1$};
% \node[state, accepting, right of=q2] (q3) {$2$};
% \draw (q1) edge[loop above] node{0} (q1)
% (q1) edge[bend left, above] node{1} (q3)
% (q2) edge[bend left, above] node{0} (q1)
% (q2) edge[loop right] node{1} (q2)
% (q3) edge[loop above] node{1} (q3)
% (q3) edge[bend left, below] node{0} (q1);
% \end{tikzpicture}
% \caption{Latent coalgebra under graph $G(m)=\set{(0,2),(1,0),(2,2)}$. 
% \protect\begin{align*}
% \TheLatentBehaviourOf{0}{m}&=\TheLatentBehaviourOf{2}{m}=(0+1)^*,\\ 
% \TheLatentBehaviourOf{1}{m}&=1^*0\TheLatentBehaviourOf{0}{m}
% \protect\end{align*}
% \label{fig:3.22}
% }
% \end{figure}

% \begin{figure}[t]
% \captionsetup{singlelinecheck=off}
% \centering
% \begin{tikzpicture}
% \node[state, accepting] (q1) {$0$};
% \node[state, right of=q1] (q2) {$1$};
% \node[state, right of=q2] (q3) {$2$};
% \draw (q1) edge[loop above] node{0} (q1)
% (q1) edge[bend left, above] node{1} (q3)
% (q2) edge[bend left, above] node{0} (q1)
% (q2) edge[bend left, above] node{1} (q3)
% (q3) edge[bend left, above] node{1} (q2)
% (q3) edge[bend left, below] node{0} (q1);
% \end{tikzpicture}
% \caption{Latent coalgebra under graph $G(m)=\set{(0,2),(1,1),(2,0)}$.
% \protect\begin{align*}
% \TheLatentBehaviourOf{0}{m}&=(0+1)^*0+\varepsilon,\\ 
% \TheLatentBehaviourOf{1}{m}&=\TheLatentBehaviourOf{2}{m}=(0+1)^*0\\
% \protect\end{align*}
% }

% \begin{tikzpicture}
% \node[state, accepting] (q1) {$0$};
% \node[state, right of=q1] (q2) {$1$};
% \node[state, right of=q2] (q3) {$2$};
% \draw (q1) edge[loop above] node{0} (q1)
% (q1) edge[bend left, above] node{1} (q3)
% (q2) edge[bend left, above] node{0} (q1)
% (q2) edge[bend left, above] node{1} (q3)
% (q3) edge[loop above] node{1} (q3)
% (q3) edge[bend left, below] node{0} (q1);
% \end{tikzpicture}
% \caption{Latent coalgebra under graph $G(m)=\set{(0,2),(1,1),(2,1)}$. 
% \protect\begin{align*}
% \TheLatentBehaviourOf{0}{m}&=(0+1)^*0+\varepsilon,\\ 
% \TheLatentBehaviourOf{1}{m}&=\TheLatentBehaviourOf{2}{m}=(0+1)^*0\\
% \protect\end{align*}
% }

% \begin{tikzpicture}
% \node[state, accepting] (q1) {$0$};
% \node[state, right of=q1] (q2) {$1$};
% \node[state, accepting, right of=q2] (q3) {$2$};
% \draw (q1) edge[loop above] node{0} (q1)
% (q1) edge[bend left, above] node{1} (q3)
% (q2) edge[bend left, above] node{0} (q1)
% (q2) edge[bend left, above] node{1} (q3)
% (q3) edge[loop above] node{1} (q3)
% (q3) edge[bend left, below] node{0} (q1);
% \end{tikzpicture}
% \caption{Latent coalgebra under graph $G(m)=\set{(0,2),(1,1),(2,2)}$ . 
% \protect\begin{align*}
% \TheLatentBehaviourOf{0}{m}&=\TheLatentBehaviourOf{2}{m}=(0+1)^*,\\ 
% \TheLatentBehaviourOf{1}{m}&=0\TheLatentBehaviourOf{0}{m}+1\TheLatentBehaviourOf{2}{m}
% \protect\end{align*}
% \label{fig:3.25}
% }
% \end{figure}

% \begin{figure}[t]
% \captionsetup{singlelinecheck=off}
% \centering
% \begin{tikzpicture}
% \node[state,accepting] (q1) {$0$};
% \node[state, accepting, right of=q1] (q2) {$1$};
% \node[state, right of=q2] (q3) {$2$};
% \draw (q1) edge[loop above] node{0} (q1)
% (q1) edge[bend left, above] node{1} (q3)
% (q2) edge[bend left, above] node{0} (q1)
% (q2) edge[bend left, above] node{1} (q3)
% (q3) edge[bend left, above] node{1} (q2)
% (q3) edge[bend left, below] node{0} (q1);
% \end{tikzpicture}
% \caption{Latent coalgebra under graph $G(m)=\set{(0,2),(1,2),(2,0)}$. 
% \protect\begin{align*}
% \TheLatentBehaviourOf{0}{m}&=\TheLatentBehaviourOf{1}{m}=(0+10+11)^*,\\ 
% \TheLatentBehaviourOf{2}{m}&=(0+1)\TheLatentBehaviourOf{0}{m}
% \protect\end{align*}
% \label{fig:3.26}
% }

% \begin{tikzpicture}
% \node[state,accepting] (q1) {$0$};
% \node[state, accepting, right of=q1] (q2) {$1$};
% \node[state, right of=q2] (q3) {$2$};
% \draw (q1) edge[loop above] node{0} (q1)
% (q1) edge[bend left, above] node{1} (q3)
% (q2) edge[bend left, above] node{0} (q1)
% (q2) edge[bend left, above] node{1} (q3)
% (q3) edge[loop above] node{1} (q3)
% (q3) edge[bend left, below] node{0} (q1);
% \end{tikzpicture}
% \caption{Latent coalgebra under graph $G(m)=\set{(0,2),(1,2),(2,1)}$.
% \protect\begin{align*}
% \TheLatentBehaviourOf{0}{m}&=\TheLatentBehaviourOf{1}{m}=(0+1)^*0+\varepsilon\\ 
% \TheLatentBehaviourOf{2}{m}&=(0+1)^*0
% \protect\end{align*}
% }

% \begin{tikzpicture}
% \node[state,accepting] (q1) {$0$};
% \node[state, accepting, right of=q1] (q2) {$1$};
% \node[state, accepting, right of=q2] (q3) {$2$};
% \draw (q1) edge[loop above] node{0} (q1)
% (q1) edge[bend left, above] node{1} (q3)
% (q2) edge[bend left, above] node{0} (q1)
% (q2) edge[bend left, above] node{1} (q3)
% (q3) edge[loop above] node{1} (q3)
% (q3) edge[bend left, below] node{0} (q1);
% \end{tikzpicture}
% \caption{Latent coalgebra under graph $G(m)=\set{(0,2),(1,2),(2,2)}$. 
% \protect\begin{align*}
% \TheLatentBehaviourOf{0}{m}&=\TheLatentBehaviourOf{1}{m}=\TheLatentBehaviourOf{2}{m}=(0+1)^*
% \protect\end{align*}
% }
% \label{fig:3.28}
% \end{figure}

\end{example}
\todo[inline]{If we want $a$ to change with time there is no need to do anything fancy! We can enhance the carrier by doing $X'=X\times \mathbb{N}$ or $X'=X\times [X\rightarrow 2]$; with this, $X$ is enhanced by a natural number counter or a set of conditions to make the dynamics of the transformation coalgebra more interesting. Maybe there is even no need to do changes, it all depends on how informative $X$ is. let's see}

\section{Latent Vulnerabilities}
Not all behaviours are latent for a given $F$-coalgebra. In Example~\ref{ex:LatentBehaviour}, we see that no transformation can yield the language $0^*$ as a latent behaviour. We see latent behaviours as targets for attackers, and we would like to know if there is a way for an attacker to cause the system to shift to a particular latent behaviour by means of a transformation. For that purpose, we introduce the definition of {latent vulnerability problems}.

\begin{definition}
A \emph{latent vulnerability problem} consists of a given a pointed $F$-coalgebra $\mathbb{X}=(X,c,x_0)$ and a given behaviour $\rho\in \sigma F$. To solve this problem, we need to either 1) find a transformation $m$ which proves that $\rho$ is the latent behaviour of $x_0$ under $m$ or 2) prove that no such $m$ can exist.
\end{definition}


\begin{align}
\gamma(m(x))&=\gamma_\omega(\rho)\\
\delta(m(x))(i)&\sim \delta_\omega(\rho)(i), \forall i\in I
\end{align}

% \todo[inline]{This means that the problem becomes a "searching for the next candidate at every step" problem. Alternatively, you could learn to compose solutions too. }
% \todo[inline]{Maybe we can also do something exciting: eventuality -> at some point in time, the behaviour you want becomes apparent, i.e. $\TheBehaviourOf{x_0}^w=\rho$ for some $w$}

 
\section{Attackers and Attacks}
There is an attacker model associated with latent vulnerabilities: attackers that can change the value of state variables, but not the program that defines the system. This later attacker would correspond to one that can arbitrarily change the coalgebra, and is in that sense a much stronger attacker. If the attacker can enforce any behaviour they want, it becomes too powerful to defend against; all coalgebras are vulnerable, and no fix is available. However, an attacker that exploits a latent vulnerability is a bit in the middle: it can change the coalgebra by means of a transformation/attack function, but only through those. In that sense, not every behaviour is at the reach of the attacker, only those who are latent.

We can do a further refinement: we can imagine attackers that control particular components of a state by partitioning states into controllable parts and observable parts
\todo[inline]{Is there a difference? We could show that every attacker that controls a component induces a transformation, and all transformations can be modelled by attackers that control a certain number of components. Right?}

\begin{definition}[Attack]
Given an $F$-coalgebra $(X,c)$, the set of its possible attacks corresponds to the set of consistent transformations of $X$, i.e., $X^X_\sim$. 
\end{definition}
Different $F$-coalgebras have different sets of attacks. In our framework, a minimal $F$-coalgebra $(X,c)$ has as many attacks as $|X|^{|X|}$.

Intuitively, an attacker can be modelled by the set of attacks they have at their disposal. However, we avoid defining attackers by enumerating their attacks, since there may be infinite, and instead we will define them by the set of components of the system that they control. 

\begin{definition}[Attacker]
Given a functor $F$ and an $F$-coalgebra $\mathbb{X}=(\vec{X},c)$, such that $\vec{X}$ has components $\Pi=\set{\pi_1, \ldots, \pi_n}$, an \emph{attacker} of $\mathbb{X}$, say $A$, is a {finite} subset of $\Pi$ whose semantics is a set of attacks defined by the function $\texttt{attacks}\colon \TheFinitePowersetOf{\Pi}\rightarrow\ThePowersetOf{X^X_\omega}$ as follows. 
For $j\in \set{1, \ldots,  n}$ and all $x\in X$,
\begin{align}
%\texttt{attacks}(A)=\set{f\in X^X_\omega | \text{ $f$ is consistent, and if $\pi_j \not\in A$, then $f(x)[\pi_j]=x[\pi_j]$}}.
\texttt{attacks}(A)=\set{f\in X^X_\omega | \text{ if $\pi_j \not\in A$, then $f(x)[\pi_j]=x[\pi_j]$}}.
\end{align}
%{\color{red}
%equivalently
%\begin{align}
%\texttt{attacks}(A)=\set{f\in X^X_\omega | \text{if $f(x)[\pi_j]\neq x[\pi_j]$, then $\pi_j \in A$, for all $x\in X$}}.
%\end{align}
%}
In other words, $A$ has access to the set of attacks that can only affect the components that $A$ has direct control of.
 \end{definition}
 \todo[inline]{Commented is the definition with ranges, but it is kinda complicated. For spectre, we define the strategy first, then we think of what attacker we are, and for vulnerabilities, we will define the attacker first, and then we find the strategy.}
% 
% \begin{definition}[Attacker]
%Given a functor $F$ and an $F$-coalgebra $\mathbb{X}=(\vec{X},c)$, such that $\vec{X}$ has components $\Pi=\set{\pi_1, \ldots, \pi_n}$, an \emph{attacker} of $\mathbb{X}$, say $A$, is a pairing of a {finite} subset of $\Pi$, namely $A[\Pi]$, and a map $A[\texttt{range}]$ that associates each component in $A[\Pi]$ with a range of possible values; formally, $A[\texttt{range}(\pi_j)]\subseteq \Pi_j$, for all $\pi_j \in A[\Pi]$. %Formally, the type of attackers is $\TheFinitePowersetOf{\Pi} \times \ThePowersetOf{\Pi \times \vec{X}}$
%
%The semantics of an attacker is given by the set of attacks defined by the function $\texttt{attacks}\colon \texttt{Attackers}\rightarrow\ThePowersetOf{X^X_\omega}$ as follows. 
%For $j\in \set{1, \ldots,  n}$,  
%\begin{align}
%\texttt{attacks}(A)&\triangleq\set{f\in X^X_\omega | \text{ if $\pi_j \not\in A$, then $f(x)[\pi_j]=x[\pi_j]$, for all $x\in X$}}\cap\\
%&\set{f\in X^X_\omega | \text{ if $\pi_j \in A$, then $f(x)[\pi_j]\in \texttt{control}(A)(\pi_j)$, for all $x\in X$}}
%\end{align}
%{\color{red}
%equivalently
%\begin{align}
%\texttt{attacks}(A)=\set{f\in X^X_\omega | \text{if $f(x)[\pi_j]\neq x[\pi_j]$, then $\pi_j \in A$, for all $x\in X$}}.
%\end{align}
%}
%In other words, we associate to $A$ the set of attacks where, if $A$ does not contain the component $\pi_j$, then the attacks cannot change the value of $\pi_j$ when mutating states.
% \end{definition}
% 
% 
 \todo[inline]{We can refine this definition of attackers with $(\pi_j, Y_j)$, with $Y_j\subseteq X_j$, instead of just $\pi_j$. HoWEVER, I highly encourage against this, because it overcomplicates things unnecessarily.}
 \todo[inline]{I love writing $j\in n$, but this means enumeration should start from 0...I can probably use that in my language though}
 \begin{example}[Attackers of Example~\ref{ex:LatentBehaviour} and their Semantics]
 The carrier $2\times 2$ has two components, $\fst$ and $\snd$, so there are four attackers: $\emptyset$, $\set{\fst}$, $\set{\snd}$ and $\set{\fst,\snd}$. The attacks of the attacker $\emptyset$ do not mutate any component, i.e., $\texttt{attacks}(\emptyset)=\set{\id}$, and the attacks of $\set{\fst}$ and $\set{\snd}$ only affect the first/second component, respectively. The attacker $\set{\fst,\snd}$ has access to all attacks.
 \end{example}
 
 \subsection{Attacks over Discrete Time Systems}
 \todo[inline]{DO NOT FORGET THAT YOU CAN ASSUME $x(t)\neq x(t+k)$ for all $k$, because otherwise, you would have taken a different decision before!!! it's like going on a loop on a chessboard.}
Recall that we model discrete time systems with pointed $\DTS$-coalgebras of the functor $\DTS=O\times X^I$ for some fixed sets $O$ and $I$. Instead of describing attacks as some general mapping over the carrier set, we might want to describe attacks as changes in the current state. We call this notion of individual transformation a \emph{strategy}, and it is formalised as follows:
\begin{definition}[Strategy]
Given a $\DTS$-coalgebra $\TheCoalgebra=(X,\gamma,\delta,x_0)$, a \emph{strategy} $\lambda$ consists of a sequence of attacks...
\todo[inline]{Why do I want this?? Strategies are useful representations of attacks...but why?}
\end{definition}

\begin{proposition}[Every Strategy is an Attack]
For all $\DTS$-coalgebras $\TheCoalgebra=(X,\gamma,\delta,x_0)$ and every strategy $\lambda\in\Lambda$\todo{complete}
\end{proposition}
 
 \section{A Partial Order of Attackers}
\todo[inline]{We know how to quantify attackers in terms of exact latent behaviours, but latent behaviours themselves have an order (they are languages i.e. sets). What about the relationship among them? Do stronger attackers need to recognise more languages, what if a language implies another? how do attackers relate there then?}
\todo[inline]{Introduce the emulation problem.}

\begin{definition}[Attack Ordering]
Given two attacks $m_1, m_2$ and an $F$-coalgebra $\TheCoalgebra$, we say that $m_1\lesssim_\TheCoalgebra m_2$ if and only if $\supp(m_1) \subseteq \supp(m_2)$, and $m_1(x)\sim_\TheCoalgebra m_2(x)$ for all $x\in \supp(m_1)$. In other words, any $x$ that $m_1$ mutates can also be mutated by $m_2$ in the same way, in the context of the coalgebra $\TheCoalgebra$. We omit the subscript $\TheCoalgebra$ when it is clear from the context, or it is irrelevant.
\end{definition}
This notion of attack ordering is a bit more general than one that uses equality (i.e., requiring $m_1(x)= m_2(x)$ instead of $m_1(x)\sim_\TheCoalgebra m_2(x)$), as it lets us focus directly on behaviour, and not on structure. If the coalgebra $\TheCoalgebra$ is minimal, then $m_1(x)= m_2(x)$ if and only if $m_1(x)\sim_\TheCoalgebra m_2(x)$, due to the coinduction proof principle.

\begin{definition}[Attack Execution]
Given an attacker $A$, an $F$-coalgebra $\TheCoalgebra=(X,c)$, and an arbitrary attack $m\in X_\omega^X$, we say that \emph{$A$ can execute the attack $m$} if and only if there exists an attack $m'\in \texttt{attacks}(A)$ such that $m\lesssim m'$.
%\begin{align}
%m' (x)\sim m (x), \quad \text{for all $x\in \supp(m)$}.
%\end{align}
%In such a case, we write $m'\lesssim m$.
\end{definition}

\begin{definition}[Attacker Ordering]
Let $\TheCoalgebra$ be an $F$-coalgebra whose components are $\Pi$. The partial order relation $\leq$ in the set of attackers $\TheFinitePowersetOf{\Pi}$ by its set inclusion, i.e. $A_1 \leq A_2$ if and only if $A_1 \subseteq A_2$. 
\end{definition}
This attacker ordering is \emph{monotonic} with respect to the \texttt{attacks} function: if $A_1 \leq A_2$, then any attack carried out by $A_1$ can be executed by $A_2$. %With it, we can define minimal attackers.

Monotonicity offers us mainly two advantages: 1) we can plan the order of attackers to be checked, and 2) we can propagate the results in case a solution fails.
\todo[inline]{Explain this better. You really need to have clear notions of solution and problems.}
% \begin{definition}[Minimal Attacker]
% \todo[inline]{Probably one of the most esoteric definitions here. It is wrt a particular problem? a particular attack, a property?}
% \end{definition}

\todo[inline]{Should we just focus on DTS? NO!  The DTS I wanted to use does not have loops, so it is clear that we can keep doing this in the current modelling.}

\begin{definition}[Emulation Problem]
We say 
\end{definition}
 
 \section{Solving Latent Vulnerability Problems}
 
 \subsection{Alternative formulations}
 \begin{definition}[Target Value]
Given an $F$-coalgebra $\TheCoalgebra=(X,c)$ with components in $\Pi$, a \emph{target value} is a proposition $x[\pi]==v$, modelled by the triple $(x,\pi,v)$ where $x\in X$, $\pi\in \Pi$ and $v$ is a value in the range of component $\pi$.
\end{definition}
 
\subsection{Exhaustive Search}
Given an attacker $A$ of an $F$-coalgebra $\TheCoalgebra=(X,c)$ and a target behaviour $\sigma$, we could perform an exhaustive search over the domain of its attacks if the carrier $X$ is finite.  A naive, exhaustive search would choose an attack $m$, mutate $\TheCoalgebra$, then perform a bisimulation check to see if there exists a state $x$ such that its latent behaviour $\TheLatentBehaviourOfIn{x}{m}{\TheCoalgebra}$ is equal to $\sigma$.

Since the search space is finite, if we cannot find an attack $m$ within the capabilities of $A$, then the system $\TheCoalgebra$ is safe with respect to $A$.

\begin{example}
\todo[inline]{For the example we can show that it can be solved}
\end{example}


\subsection{Using SMT Solvers}
Due to our formulation, we have a perfect information deterministic game, like Chomp~\cite{Chomp}. Our idea is to leverage an SMT solver to give us a winning move for the attacker, or prove that such move does not exist. For the remainder of this section, we focus on $\DTS$-coalgebras.

\todo[inline]{Rephrase this as a reachability problem. i.e. instead of a target value, use a set of goal states defined by the target values.}
\todo[inline]{HERE I AM}
\begin{definition}[Winning Latency Games]
Let $A$ be an  attacker, $\TheCoalgebra=(X,\gamma,\delta,x_0)$ be a pointed $\DTS$-coalgebra, and consider a set of target values $(\pi_1,v_1),...,(\pi_n,v_n)$. We say that the attacker $A$ \emph{wins the latency game in zero steps} if and only if there exists an attack $m$ in $\texttt{attacks}(A)$ such that $m(x_0)[p_k]=v_k$, for $k=1..n$. If a solution is found in zero steps, then all target components $\pi_1$ to $\pi_n$ must be in control of $A$, i.e., $\set{\pi_1, \ldots, \pi_n}\subseteq A$.

The attacker $A$ \emph{wins the latency game in $\tau$ steps given the inputs $(i_1, \ldots, i_\tau)$ with the attack $m$ at state $x_{\tau}$} if and only if there exists a sequence of states $(x_1, \ldots, x_\tau)$, where
\begin{align}
%m_{\tau+1}(x_{\tau +1})=x_\tau(
x_{t+1}=\delta(m (x_{t}))(i_t), \quad \text{for $t=0..{\tau-1}$}
\end{align}
and
\begin{align}
m(x_\tau)[p_k]=v_k,\quad \text{ for $k=1..n$.}
\end{align}
\end{definition}


\begin{proposition}[Composition through Emulation]
\todo[inline]{There are several ways to compose attackers: through $\delta$ and through $m$. The message I want to carry across is the following: if you know that you can solve a latency game from a state $x$, then 
 I already have a notion of winning over time, so I just need to emulate the attack of other attackers to reach my goal.}
If an attacker $A$ can win a latency game in $\tau$ steps given the inputs $i_1, \ldots i_{\tau}$ in the pointed $\DTS$-coalgebra $(X,\gamma,\delta, x_{0})$ with an attack $m$, then an attacker $A'$ can win the same latency game in $\tau+t$ steps given the input $i_1, i_2, \ldots i_{\tau+1}$ in the pointed $\DTS$-coalgebra $(X,\gamma,\delta, x_{-t})$ if and only if 
\begin{align}
\delta
\end{align}

%Given a pointed $\DTS$-coalgebra $(X,\gamma,\delta, x_0)$, an attacker $A_0$ can win a given latency game in ${\tau+1}$ steps given the inputs $(i_1, \ldots, i_\tau, i_{\tau+1})$ with some attack $m$ at some state $x_{\tau+1}$ if and only if there exists an attacker $A_{\tau}$ that can win the given latency game in one step in the pointed $\DTS$-coalgebra $(X,m\gamma,\delta, x_{\tau})$.
\end{proposition}
\begin{proof}
The main objective of $A_0$ is to reach state $x_{\tau}$ and from there 
\end{proof}


%\begin{proposition}
%\label{sec:Incompleteness}
%Given an arbitrary $F$-coalgebra $\mathbb{X}=(X,c)$ and a behaviour $\rho\in \sigma F$, there may not exist a transformation $m\colon X\rightarrow X$ such that $\rho$ is a latent behaviour of $\mathbb{X}$ under $m$.
%\end{proposition}
%\begin{proof}
%We provide a counterexample for the opposite proposition, that is, there exists a transformation $m\colon X\rightarrow X$ such that $\rho$ is a latent behaviour of $\mathbb{X}$ under $m$, for all $F$-coalgebras and all behaviours in $\sigma F$. 
%Consider, for the functor $F=2\times \texttt{id}^2$, a single-state $F$-coalgebra that accepts all sequences; this $F$-coalgebra cannot display a latent behaviour different from its original behaviour.
%\end{proof}

We would like to consider two problems related to latent behaviours: 
\begin{itemize}
\item How can we use transformations ourselves to repurpose a system that is already been defined?
\item How can an attacker use transformations to force a behaviour they want?
\end{itemize}
\todo[inline]{Note that, ultimately, both questions need a method to solve an equation for a transformation. That is, given a target behaviour and an a source coalgebra, how do you solve the problem of finding a transformation that helps you display the behaviour you want? Is it even possible?}




%!TEX root = ../main.tex
\chapter{Systematic Classification of Attacker Models using Latent-Behaviour Analysis}
\label{ch:Classification}
\todo[inline]{This chapter needs to be readapted to align better with latent behaviours, but it is probably the easiest to align: if the attacker can find a spatial transformation that triggers a requirement failure, then there is an attack that the attacker can use.}
\todo[inline]{This section is in red because it needs to be rewritten into context}
\section{Introduction}
\label{sec:Classification:ClassificationProblem}
\paragraph{Problem Statement}
Some systems are designed to provide security guarantees in the presence of attackers. For example, the Diffie-Hellman key agreement protocol guarantees \emph{perfect forward secrecy} (PFS) \cite{Gunther1990,Menezes1996}. PFS is the security property which guarantees that the session key remains secret even if the long-term keys are compromised. These security guarantees are only valid in the context of the attacker models for which they were proven; it is unknown whether those guarantees apply for stronger or incomparable attacker models. For instance, PFS describes an attacker model, say $\mathcal{M}_{DH}$, that can compromise the long-term keys \emph{and only those}, and it also describes a property (i.e., the confidentiality of the session keys) that is guaranteed in the presence of an attacker that fits the model $\mathcal{M}_{DH}$. However, if we consider a stronger attacker model (e.g., an attacker that can directly compromise the session key), then Diffie-Hellman can no longer guarantee the confidentiality of the session keys. It is difficult to provide any guarantees against an attacker model that is too proficient/powerful, so it is in the interest of the system designer to choose an adequate attacker model that puts the security guarantees of the system in the context of realistic and relevant attackers. 
 
While $\mathcal{M}_{DH}$ is an attacker model that describes attackers who compromise confidentiality, we are interested in attacker models that characterise attackers who compromise integrity. We associate the security guarantees with respect to attacker models that target integrity with \emph{robustness}. For now, consider the following research question: \textbf{RQ1)} given a system and a list of security requirements, how do we systematically generate attackers that can potentially break these requirements, and how do we check if these attackers are successful? 
\todo[inline]{This question should appear in the introduction of the thesis to motivate applications. This should not be the first time we see it.}

We started to approach this question by first modelling systems and attacker models using concepts from Chapter~\ref{ch:LatentBehaviours}. In summary, if an $F$-coalgebra $(\vec{X},c)$ whose carrier is a product type $\vec{X}$ with coordinates $C$ models the system, we can systematically generate attacker models by characterising them using sets of coordinates following Definition~\ref{def:Latent:AttackerModel}. In this chapter, we model AIGs as $F$-coalgebras of some functor $F$, and we systematically generate attacker models for them. This addresses the part of the research question which is concerned with attacker generation. We then use bounded model checking to partially verify whether these attackers are successful in breaking any requirements.

%We approach this question at a high level for a system $S$ with a set $C$ of $n$ components, and a set of security requirements $R$ as follows. 
\paragraph{AIGs and their Attacker Models} AIGs are systems which limit the actions of attackers by design, since all variables are boolean. Consequently, an attacker has one of three options when it comes to transforming a component $c\in C$ at state $\vec{x}$: 1) the attacker leaves $\vec{x}[c]$ as is, applying $\id$ in this components, 2) the attacker forces $\vec{x}[c]$ to be 0, applying $\Delta_0$ in component $c$, or 3) the attacker applies $\Delta_1$ in component $c$, forcing $\vec{x}[x]$ to be 1. As a matter of fact, we ignore the transformation $\id$, since it is always possible for the attacker to emulate $\id$ by applying $\Delta_{\vec{x}[c]}$ to component $c$ at state $\vec{x}$.

Each subset of $C$ characterises an attacker model; more precisely, a subset $A\subseteq C$ characterises an attacker which can transform the components in $A$ by applying $\Delta_0$ or $\Delta_1$. The application of $\Delta_0$ and $\Delta_1$ may be different for each 

then there are $2^C$ different attacker models, and each attacker $A\subseteq C$ has $2^{|A|}$ attacks at their disposal. 

% Let $A$ be a subset of $C$; the set $A$ models an attacker that can interact with an AIG $S$ by means transforming each component $c\in A$. More precisely, for every component $c$ in $A$, the attacker can change the value of $c$ at any time and any number of times during execution, possibly following an attack strategy. 

Considering the exponential size of the set of attackers (i.e., $2^{n}$), a brute-force approach to checking whether each of those attackers breaks each requirement in $R$ is inefficient for two reasons: 1) an attacker $A$ may only affect an isolated part of the system, so requirements that refer to other parts of the system should not be affected by the presence of $A$, and 2) if some attacker $B$ affects the system in a similar way to $A$ (e.g., if they control a similar set of components), then the knowledge we obtain while verifying the system in the presence of $A$ may be useful when verifying the system in the presence of $B$. These two reasons motivate a second research question: \textbf{RQ2)} which techniques can help us to \emph{classify} attackers, i.e., to map each attacker to the set of requirements that it breaks? 

{\color{red}
To answer these two research questions in a more concrete and practical context, we study systems modelled by \emph{And-inverter Graphs (AIGs)} (see  \cite{AIGs,AIGs2}). AIGs describe hardware models at the bit-level \cite{AIGER}, and have attracted the attention of industry partners including IBM and Intel \cite{HWMCC2014BM}. Due to being systems described at bit-level, AIGs present a convenient system model to study the problem of attacker classification, because the range of actions that attackers have over components is greatly restricted: either the attacker leaves the value of the component as it is, or the attacker negates its current value. However, this approach can be generalised to other systems by considering non-binary ranges for components, and by allowing attackers to choose any value in those ranges.

\subsubsection{Contributions.} In this paper, we provide:
\begin{itemize}
\item a formalisation of attackers of AIGs and how they interact with systems, 
\item a methodology to perform bounded model checking while considering the presence of attackers,
\item a set of heuristics that characterise attacker frontiers for invariant properties using bounded model checking,
\item experimental evidence of the effectiveness of the proposed methodology and heuristics.
\end{itemize}

\section{AIGs as $F$-coalgebras}
\label{sec:Classification:LatentBehaviours}
The first application of latent-behaviour analysis is the systematic generation and comparison of attacker models for \emph{And-inverter Graphs} (AIG) systems, which are used to model hardware systems at the level of bits~\cite{??}. 
\todo[inline]{Put AIGs in the preliminaries...}
Conceptually, an AIG is a circuit which has some memory locations, some external inputs, and a wiring which connects memory locations and input locations to output locations. We can imagine a state of an AIG to be a vector $\vec{x}\in 2^V$ where $V$ is the set of memory locations or \emph{latches} such that when the system receives an input word $\vec{i}$ with coordinates in the set of inputs $W$, it computes a vector of \emph{gates} $\vec{g}\in 2^G$ which depend on both $\vec{x}$ and $\vec{i}$; to monitor security, the AIG produces a vector of outputs $\vec{o}\in 2^E$, where $E$ is a set of invariant security requirements.

An interesting property of AIGs is that they update their components sequentially: each gate can only compute its value once its inputs have computed their values, and gates can depend on other gates, latches and inputs. They also produce an 

In this chapter, we consider very simple attackers; these attackers can only flip bits, but given the intricacy of AIGs, a single bit flip can propagate 

%-------------------------------------------------------------------------------------------------------------------------------------------------
\section{Preliminaries}
\label{sec:preliminaries}
In this section, we provide the foundation necessary to formally present 
the problem of model checking {And-inverter Graph} (AIGs) in the presence of attackers. 
Let $\Bool=\set{0,1}$ be the set of booleans/bits. 
\begin{figure}[!t]
\begin{minipage}{0.6\textwidth}
\begin{framed}
\includegraphics[width=\textwidth]{Example.pdf}
\end{framed}
\end{minipage}
\begin{minipage}{0.35\textwidth}
\centering
\begin{framed}
\includegraphics[width=\textwidth]{Attack2.pdf}\\
\end{framed}
\begin{framed}
\includegraphics[width=\textwidth]{Attack1.pdf}
\end{framed}
\end{minipage}
\caption{\textbf{Left:} Recall the AIG from Figure ~\ref{fig:Preliminaries:AIGExample} describing a system with two
inputs $w_1$ and $w_2$ (green boxes), one latch $v_1$ with initial value 1 (grey box), two gates $g_1$ and $g_2$ (gray circles), and three invariant requirements $r_1=\Always g_1$, $r_2=\Always \lnot g_2$ and $r_3=\Always v_1$ (red circles). 
The arrows represent logical dependencies, and bullets in the arrows imply negation.
\textbf{Right above}: an attacker that controls latch $v_1$ can set its initial value to 0 to break $r_2$ and $r_3$ in 0 steps. This attacker uses a spatial transformation to implement this attack.
\textbf{Right below}: an attacker that controls gate $g_2$ can set its value to 1 at time 0 to break $r_2$ in 0 steps and $r_3$ in 1 step, because the value of $v_1$ at time 1 is 0. This attacker uses a dynamics transformation, since gates are not part of the state of AIGs.}%. An arrow with $!$ means negation}
\label{fig:Classification:Example}
\end{figure}

\section{Motivational Example}
\label{sec:Example}
In this section, we provide a motivational example of the problem of model checking compromised systems, and we illustrate how to classify attackers given a list of security requirements. Consider a scenario where an attacker $A$ controls the gate $g_2$ of Example~\ref{ex:simple}. By controlling $g_2$, we mean that $A$ can set the value of $g_2(t)$ at will for all $t\geq 0$. Since $r_2=\Always \lnot g_2$, it is possible for $A$ to break $r_2$ by setting $g_2(0)$ to $1$. We note that the same strategy works to break both requirements, but it need not be in the general case; i.e., an attacker may have one strategy to break one requirement, and a different strategy to break another. $A$ can also break $r_3=\Always v_1$, because, if $A$ sets $g_2(0)$ to 1, then $v_1(1)$ is equal to $0$. Since the original system fails to enforce $r_1$, we say that $A$ has the \emph{power to break the requirements} $r_1$, $r_2$ and $r_3$. Now, consider a different attacker $B$ which only controls the gate $g_1$. No matter what value $B$ chooses for $g_1(t)$ for all $t$, it is impossible for $B$ to break $r_2$ or $r_3$, so we say that $B$ only has the power to break $r_1$. 

If we allow attackers to control any number of components, then there are $8$ different attackers, described by the subsets of $\set{v_1,g_1,g_2}$. We do not consider attackers that control inputs, because the model checking of invariant properties requires the property to hold for all inputs, so giving control of inputs to an attacker does not make it more powerful (i.e. the attacker cannot break more requirements than it already could without the inputs). Figure~\ref{fig:borders} illustrates the classification of attackers depending on whether they can break a given requirement or not. Based on it, we can provide the following security guarantees: 1) the system cannot enforce $r_1$, and 2) that the system can only enforce $r_2$ and $r_3$ in the presence of attackers that are as capable to interact with the system as $\set{g_1}$ (i.e. they only control $g_1$ or nothing).

According to the classification, attacker $\set{g_2}$ is as powerful as the attacker $\set{v_1,g_1,g_2}$, since both attackers can break the same requirements $r_1$, $r_2$ and $r_3$. This information may be useful to the designer of the system, because it may prioritise attackers that control less components but are as powerful as attackers that control more when deploying defensive mechanisms.

\begin{figure}[!t]
\centering
\begin{minipage}{0.45\textwidth}
	\centering
{
\scriptsize
\begin{framed}
\begin{tikzpicture}[node distance=1cm,>=stealth',auto, pin distance=0.1cm]
\tikzstyle{safe}=[rectangle,thick,draw=black!75,fill=green!20,minimum size=2mm]
\tikzstyle{fail}=[rectangle,thick,draw=black!75,fill=red!20,minimum size=2mm]
\tikzstyle{dnc}=[rectangle,thick,draw=black!75,fill=red!20,minimum size=2mm]
\node [dnc] (all) {$\set{v_1,g_1,g_2}$};
\node [dnc] (v1g2) [below= 0.5cm of all] {$\set{v_1,g_2}$}
edge [post](all);
\node [dnc] (v1g1) [left = 0.5cm of v1g2] {$\set{v_1,g_1}$}
edge [post](all);
\node [dnc] (g1g2) [right = 0.5cm of v1g2] {$\set{g_1,g_2}$}
edge [post](all);
\node [safe] (g1) [below= 0.5cm of v1g2] {$\set{g_1}$}
edge [post](v1g1)
edge [post](g1g2);
\node [fail] (v1) [left = 0.5cm of g1] {$\set{v_1}$}
edge [post,color=red!50](v1g1)
edge [post,color=red!50](v1g2);
\node [fail] (g2) [right = 0.5cm of g1] {$\set{g_2}$}
edge [post,color=red!50](v1g2)
edge [post,color=red!50](g1g2);
\node [safe] (empty) [below= 0.5cm of g1] {$\set{}$}
edge [post](v1)
edge [post](g1)
edge [post](g2);
 \end{tikzpicture}
 \end{framed}}

\end{minipage}
\begin{minipage}{0.45\textwidth}
	\centering
{\scriptsize
\begin{framed}
\begin{tikzpicture}[node distance=1cm,>=stealth',auto, pin distance=0.1cm]
\tikzstyle{safe}=[rectangle,thick,draw=black!75,fill=green!20,minimum size=2mm]
\tikzstyle{fail}=[rectangle,thick,draw=black!75,fill=red!20,minimum size=2mm]
\tikzstyle{fail2}=[diamond, draw=black!75,fill=red!20,minimum size=1mm]
\tikzstyle{dnc}=[rectangle,thick,draw=black!75,fill=red!20,minimum size=2mm]
\node [dnc] (all) {$\set{v_1,g_1,g_2}$};
\node [dnc] (v1g2) [below= 0.5cm of all] {$\set{v_1,g_2}$}
edge [post](all);
\node [dnc] (v1g1) [left = 0.5cm of v1g2] {$\set{v_1,g_1}$}
edge [post](all);
\node [dnc] (g1g2) [right = 0.5cm of v1g2] {$\set{g_1,g_2}$}
edge [post](all);
\node [dnc] (g1) [below= 0.5cm of v1g2] {$\set{g_1}$}
edge [post](v1g1)
edge [post](g1g2);
\node [dnc] (v1) [left = 0.5cm of g1] {$\set{v_1}$}
edge [post](v1g1)
edge [post](v1g2);
\node [dnc] (g2) [right = 0.5cm of g1] {$\set{g_2}$}
edge [post](v1g2)
edge [post](g1g2);
\node [fail] (empty) [below= 0.5cm of g1] {$\set{}$}
edge [post,color=red!50](v1)
edge [post,color=red!50](g1)
edge [post,color=red!50](g2);
 \end{tikzpicture}
  \end{framed}}
\end{minipage}
\caption{Left: classification of attackers for requirements $r_2$ and $r_3$. Right: classification of attackers for requirement $r_1$. A green attacker cannot break the requirement, while a red attacker can.}
\label{fig:borders}
\end{figure}

%-------------------------------------------------------------------------------------------------------------------------------------------------
\section{Bounded Model Checking of Compromised Systems}
\label{sec:bmc}
We recall the research questions that motivate this work: \textbf{RQ1)} given a system and a list of security requirements, how do we systematically generate attackers that can potentially break these requirements, and how do we verify if they are successful? and \textbf{RQ2)} which techniques can help us \emph{classify} attackers, i.e., to map each attacker to the set of requirements that it breaks? 
In this section, we aim to answer these research questions on a theoretical level by formalising the problem of attacker classification via bounded model checking AIGs in the presence of attackers. More precisely, to answer \textbf{RQ1}, we formalise attackers and their interactions with systems, and we show how to systematically generate bounded model checking problems that solve whether some given attackers can break some given requirements. We then propose two methods for the classification of attackers: 1) a brute-force method that creates a model checking problem for each attacker-requirement pair, and 2) a method that incrementally empowers attackers to find ``minimal attackers,'' since minimal attackers represent large portions of the universe of attackers thanks to a monotonicity relation between the set of components controlled by the attacker and the set of requirements that the attacker can break. The latter method is a theoretical approach to answer \textbf{RQ2}, while its practical usefulness is evaluated in Section~\ref{sec:evaluation}.

\subsection{Attackers and Compromised Systems}
Since an AIG describes a system of equations, to incorporate the actions of an attacker $A$ into the system, we modify the equations that are associated to the components controlled by $A$. Let $S=(W,V,G)$ be a system described by an AIG, let $R=\set{r_1, \ldots, r_n}$ be a set of invariant requirements for $S$, and let $C=W\cup V\cup G$ be the set of components of $S$. By definition, an \emph{attacker} $A$ is any subset of $C$. If a component $c$ belongs to an attacker $A$, then $A$ has the \emph{capability to interact with $S$ through} $c$. We modify the equations of every latch $v\in V$ to be parametrised by an attacker $A$ as follows: the original transition equation $v(t+1)=e(t)$ and the initial equation $v(0)=b$ changes to
\begin{align}
\label{eq:badLatch}
v(t+1) = \begin{cases}
e(t), \quad &\text{if $v\not \in A$;}\\
A_v(t+1), \quad &\text{otherwise},
\end{cases}
\quad 
v(0)= \begin{cases}
b, \quad &\text{if $v \not \in A$;}\\
A_v(0), \quad &\text{otherwise.}
\end{cases}
\end{align}
where $A_v(t)$ is a value chosen by the attacker $A$ at time $t$. Similarly, we modify the equation of gate $g\in G$ as follows: the original equation $g(t)=e_1(t)\land e_2(t)$ changes to
\begin{align}
\label{eq:badGate}
g(t) = \begin{cases}
e_1(t)\land e_2(t), \quad &\text{if $g\not \in A$;}\\
A_g(t), \quad &\text{otherwise},
\end{cases}
\end{align}
where $A_g(t)$ is, again, a value chosen by the attacker $A$ at time $t$. We use $A[S]$ to denote the system $S$ under the influence of attacker $A$; i.e., $A[S]$ is the modified system of equations. 

An \emph{attack} $\vect{a}\colon A\rightarrow \Bool$ is a map of components in $A$ to booleans. An \emph{attack strategy} is a finite sequence of attacks $(\vect{a}_0, \vect{a}_1, \ldots, \vect{a}_t)$ that fixes the values of all $A_c(k)$ (used in the equations above) by $A_c(k)=\vect{a}_k(c)$, with $c\in A$ and $0 \leq k \leq t$. %Since we have changed the semantics of the system, it is possible that $A[S]$ only satisfies a subset of requirements in $R$. 
\begin{definition}[Broken Requirement]
\label{def:brokenRequirement}
Given a requirement $r\in R$ with $r=\Always e$, we say that $A$ \emph{breaks the requirement $r$} if and only if there exists a sequence of inputs of length $k$ and an attack strategy of length $k$ such that $e(k)$ is false. We denote the set of requirements that $A$ breaks by  $A[R]$. 
\end{definition}
Finally, we define two partial orders for attackers: \textbf{1)} an attacker $A_i$ is strictly less \emph{capable} (to interact with the system) than an attacker $A_j$ in the 
context of $S$ %, denoted $A_i \leq A_j$ 
iff $A_i\subseteq A_j$ and $A_i \neq A_j$. The {attacker $A_i$ is equally capable to attacker $A_j$ iff $A_i= A_j$}; and \textbf{2)} an attacker $A_i$ is strictly less \emph{powerful} than an attacker $A_j$ in the 
context of $S$ and $R$ %, denoted $A_i \leq A_j$ 
iff $A_i[R]\subseteq A_j[R]$ and $A_i[R]\neq A_j[R]$. Similarly, {attacker $A_i$ is equally powerful to attacker $A_j$ iff $A_i[R]= A_j[R]$}. We simply state that $A_i$ is less capable than $A_j$ if $S$ is clear from the context. Similarly, we simply say that $A_i$ is less powerful than $A_j$ if $S$ and $R$ are clear from the context.

%\subsection{Model Checking in the Presence of Attackers}
We can now properly present the problem of \emph{attacker classification}.
\begin{definition} [Attacker Classification via Model Checking]
\label{def:AttackerQuantification}
Given a system $S$, a set of requirements $R$, and a set of $h$ attackers 
$\set{A_1, \ldots , A_h}$, for every attacker $A$, we compute the set $A[R]$ of requirements that $A$ can break by performing model checking of each requirement in $R$ on the compromised system $A[S]$. 
\end{definition}

%The duality between attackers and sets of components enables the following assertions: a set of components $B$ is \emph{safe} if and only if we can prove that attacker $B$ cannot break any requirements, i.e., $B[R]=\emptyset$; otherwise, the set of components $B$ is  \emph{vulnerable}. 
Definition~\ref{def:AttackerQuantification} assumes that exhaustive model checking is possible for $S$ and the compromised versions $A[S]$ for all attackers $A$. However, if exhaustive model checking is not possible (e.g., due to time limitations or memory restrictions), we consider an alternative formulation for \emph{Bounded Model Checking} (BMC): 
\begin{definition} [Attacker Classification via Bounded Model Checking]
\label{def:BoundedModelCheckingOfSystems}
Let  $S$ be a system, $R$ be a set of requirements, and $t$ be a natural number. Given a set of attackers $\set{A_1, \ldots , A_h}$, for each attacker $A$ , we compute the set $A[R]$ of requirements that $A$ can break \emph{using a strategy of length up to $t$} on the compromised system $A[S]$. 
\end{definition}

In the following, we show how to construct a SAT formula that describes the attacker classification problem via bounded model checking. 

\subsection{A SAT Formula for BMC up to $t$ Steps}
 For a requirement $r=\Always e$ and a time step $t\geq 0$, we are interested in finding an assignment of sources and attacker actions (i.e., an attack strategy) such that, for $0\leq k \leq t$, the value of $e(k)$ is false. We define the proposition $\mathtt{goal}(r,t)$ by
\begin{align}
\mathtt{goal}(\Always e,t) \triangleq \bigvee_{k=0}^t{\lnot e(k)}, 
\end{align}
 
We must inform the SAT solver of the equalities and dependencies between expressions given by the definition of the AIG (e.g., that $e(k) \Leftrightarrow \lnot v_1(k)$). Inspired by the work of Biere \emph{et al.} \cite{BMCWithoutBDDs}, we transform the equations into a Conjunctive Normal Form formula (CNF) that the SAT solver can work with using  \emph{Tseitin encoding} \cite{TseitinEncoding}. Each equation of the form 
\begin{align*}
&v(0)= \begin{cases}
b, \quad &\text{if $v \not \in A$;}\\
A_v(0), \quad &\text{otherwise,}
\end{cases}
\\\text{becomes }
&{\left(v^{\downarrow} \lor (v(0) \Leftrightarrow b ) \right)\land \left(\lnot v^{\downarrow} \lor (v(0) \Leftrightarrow A_v(0)) \right)}
\end{align*}
where $v^{\downarrow}$ is a literal that marks whether the latch $v$ is an element of the attacker $A$ currently being checked; i.e., we assume that $v^{\downarrow}$ is true if $v\in A$, and we assume that $v^{\downarrow}$ is false if $v\not \in A$. Consequently, if $v\not \in A$, then $v(0) \Leftrightarrow b$ must be true, and if $v \in A$, then $v(0)  \Leftrightarrow A_v(0) $ must be true. We denote this new proposition by $\mathtt{encode}(v,0)$, and it characterises the initial state of $v$.

Similarly, for $0\leq k<t$, each equation of the form 
\begin{align*}
&v(k+1) = \begin{cases}
e(k), \quad &\text{if $v\not \in A$;}\\
A_v(k+1), \quad &\text{otherwise},
\end{cases}\quad
\\\text{becomes }
&\left(v^{\downarrow} \lor (v(k+1) \Leftrightarrow e(k) ) \right)\land \left(\lnot v^{\downarrow} \lor (v(k+1) \Leftrightarrow A_v(k+1)) \right).
\end{align*}
We denote this new proposition by $\mathtt{encode}(v,k)$. %We now use the Tseitin encoding of $p \Leftrightarrow (q \land r)$, i.e., $(p \lor \lnot q \lor \lnot r)\land (\lnot p \lor q)\land  (\lnot p \lor r)$, to encode gates. 
Finally, for $0\leq k \leq t$, each equation of the form
\begin{align*}
&g(k) = \begin{cases}
e_1(k)\land e_2(k), \quad &\text{if $g\not \in A$;}\\
A_g(k), \quad &\text{otherwise},
\end{cases}\\\text{becomes }
&\left(g^{\downarrow} \lor (g(k) \Leftrightarrow e_1(k)\land e_2(k) ) \right)\land \left(\lnot g^{\downarrow} \lor (g(k) \Leftrightarrow A_g(k)) \right),
\end{align*}
where $g^{\downarrow}$ is a literal that marks whether the gate $g$ is an element of the attacker $A$ currently being checked in a similar way that the literal $v^\downarrow$ works for the latch $v$. We denote this new proposition by $\mathtt{encode}(g,k)$.
%Consider a component $c$ and an attacker $A$. Under the new system of equations, if $c\in A$, then we assume $c^{\downarrow}$ and the value of component $c$ at time $k$ depends only on the literal $A_c(k)$; if $c\not\in A$, then we assume $\lnot c^{\downarrow}$ and we use the original semantics of the system to determine the value of $c$. 

To perform SAT solving, we need to find an assignment of inputs in $W$ and attacker actions for each component $c$ in $A$ over $t$ steps; thus, we need to assign at least ${|W\times A|\times t}$ literals. The SAT problem for checking whether requirement $r$ is safe up to $t$ steps, denoted $\mathtt{check}(r,t)$, is defined by 
\begin{align}
\label{eq:naiveCheck}
\mathtt{check}(r,t)\triangleq\mathtt{goal}(r,t)\land\! \bigwedge_{c\in (V \cup G)}\left( \bigwedge_{k=0}^{t}{\mathtt{encode}(c,k)}\right).
\end{align}
\begin{proposition}
\label{prop:Correctness}
For a given attacker $A$ and a requirement $r=\Always e$, if we assume the literal $c^{\downarrow}$ for all $c \in A$ and we assume $\lnot x^{\downarrow}$ for all $x\not\in A$ (i.e., $x\in (V \cup G)-A$), then $A$ can break the requirement $r$ in $t$ steps (or less) if and only if $\mathtt{check}(r,t)$ is satisfiable.
\end{proposition}
\begin{proof}

We first show that if $A$ can break the requirement in $t$ steps or less, then $\mathtt{check}(r,t)$ is satisfiable. Since $A$ breaks $r=\Always e$ in $t$ steps or less, then, by Definition~\ref{def:brokenRequirement}, there exists an assignment of inputs $(\vect{w}_0, \ldots, \vect{w}_k)$ and an attacker strategy $(\vect{a}_0, \ldots, \vect{a}_k)$ which causes $e(k)$ to be false for some $k\leq t$; this means that $\mathtt{goal}(r,t)$ is satisfiable, which, in turn, makes $\mathtt{check}(r,t)$ satisfiable.

We now show that if $\mathtt{check}(r,t)$ is satisfiable, then $A$ can break the requirement. If $\mathtt{check}(r,t)$ is satisfiable then $\mathtt{goal}(r,t)$ is satisfiable, and $e(k)$ is false for some $k\leq t$. Consequently, there is an assignment of inputs $\vect{w}(k)$ and attacker actions $A_c(k)$, such that the $\mathtt{encode}(c,k)$ propositions are satisfied for all $c\in A$. With $\vect{a}_k(c)=A_c(k)$ and $\vect{w}_k =\vect{w}(k)$, we obtain a witness input sequence and a witness attack strategy which proves that $A$ can break $r$ in $k$ steps (i.e., in $t$ steps or less since $k\leq t$).
\qed
\end{proof}

Algorithm~\ref{alg:BadQuantification} describes a naive strategy to compute the sets $A[R]$ for each attacker $A$; i.e. the set of requirements that $A$ breaks in $t$ steps (or less). Algorithm~\ref{alg:BadQuantification} works by solving, for each of the $2^{|V \cup G|}$ different attackers, a set of $|R|$ SAT problems, 
{
each of which has a size of at least $\mathcal{O}\left({|C|\times t}\right)$ on the worst case.}

\begin{figure}[!t]
\centering
{
%\begin{framed}
\begin{algorithm}[H]
 \KwData{system $S=(W,V,G)$, a time step $t\geq 0$, a set of requirements $R$.}
 \KwResult{A map that maps the attacker $A$ to $A[R]$.}
Map $\mathcal{H}$\;
\For{\!\!\textbf{each} $r \in R$}
{
\For{\!\!\textbf{each} $A$ such that $A \subseteq (V\cup G)$}
	{
		\If{$\mathtt{check}(r,t)$ is satisfiable while assuming $c^\downarrow$ for all $c\in A$}
		{
			insert $r$ in $\mathcal{H}(A)$\;
		}
	 }
}
 \Return $\mathcal{H}$\;
 \caption{Naive attacker classification algorithm.}
 \label{alg:BadQuantification}
\end{algorithm}
%\end{framed}
}
\end{figure}

In the rest of the section, we propose two sound heuristics in an attempt to improve Algorithm~\ref{alg:BadQuantification}: the first technique aims to reduce the size of the SAT formula, while the other aims to record and propagate the results of verifications among the set of attackers so that some calls to the SAT solver can be avoided.

\subsection{Isolation and Monotonicity}
The first strategy involves relying on \emph{isolation} to prove that it is impossible for a given attacker to break some requirements. To formally capture this notion, we first extend the notion of IOC to attackers. The IOC of an attacker $A$, denoted $\blacktriangle(A)$, is defined by the union of IOCs of the components in $A$; more precisely, $\blacktriangle(A) \triangleq \bigcup\set{\blacktriangle(c)|c \in A}.$

Informally, isolation happens whenever the IOC of $A$ is disjoint from the COI of $r$, implying that $A$ cannot interact with $r$.
\begin{proposition}[Isolation]
\label{theo:isolation}
Let $A$ be an attacker and $r$ be a requirement that is satisfied in the absence of $A$. If $\blacktriangle(A)\cap \blacktriangledown(r)=\emptyset$, then $A$ cannot break $r$.\end{proposition}
\begin{proof}

For the attacker $A$ to break the requirement $r$, there must be a component $c \in \blacktriangledown(r)$ whose behaviour was affected by the presence of $A$, and whose change of behaviour caused $r$ to fail. However, for $A$ to affect the behaviour of $c$, there must be a dependency between the variables directly controlled by $A$ and $c$, since $A$ only chooses actions over the components it controls; implying that $c\in \blacktriangle(A)$. This contradicts the premise that the IOC of $A$ and the COI of $r$ are disjoint, so the component $c$ cannot exist. \qed
\end{proof}
Isolation reduces the SAT formula by dismissing attackers that are outside the COI of the requirement to be verified. Isolation works similarly to \emph{COI reduction} (see~\cite{ToSplitOrToGroup,GraphLabelingForEfficientCOIComputation,HandbookOfSatisfiability,HandbookOfModelChecking,OptimizedModelCheckingOfMultipleProperties}), and it transforms Equation \ref{eq:naiveCheck} into
\begin{align}
\label{eq:isolation}
\mathtt{check}(r,t)\triangleq\mathtt{goal}(r,t)\land\! \bigwedge_{c\in (\blacktriangledown(r)- W)}\left( \bigwedge_{k=0}^{t}{\mathtt{encode}(c,k)}\right)
\end{align}

The second strategy uses \emph{monotonicity} relation between capabilities and power of attackers.
\begin{proposition}[Monotonicity]
\label{theo:monotonicity}
For attackers $A$ and $B$ and a set of requirements $R$, if $A\subseteq B$, then $A[R]\subseteq B[R]$.
\end{proposition}
\begin{proof}
If $A$ is a subset of $B$, then attacker $B$ can always choose the same attack strategies that $A$ used to break the requirements in $A[R]$; thus, $A[R]$ must be a subset of $B[R]$.
 \qed
\end{proof}
Monotonicity allows us to define the notion of {minimal (successful) attackers} for a requirement $r$: attacker $A$ is a \emph{minimal attacker} for requirement $r$ if and only if $A$ breaks $r$, and there is no attacker $B\subset A$ such that $B$ also breaks $r$. In the remainder of this section, we expand on this notion, and we describe a methodology for attacker classification that focuses on the identification of these minimal attackers. %We expand on this notion in the following sections.

\subsection{Minimal (Successful) Attackers}
The set of minimal attackers for a requirement $r$ partitions the set of attackers into those that break $r$ and those who do not. Any attacker that is more capable than a minimal attacker is guaranteed to break $r$ by monotonicity (cf. Proposition~\ref{theo:monotonicity}), and any attacker that is less capable than a minimal attacker cannot break $r$; otherwise, this less capable attacker would be a minimal attacker. Consequently, we can reduce the problem of attacker classification to the problem of finding the minimal attackers for all requirements.

\subsubsection{Existence of a Minimal Attacker.} Thanks to isolation (cf. Proposition~\ref{theo:isolation}) we can guarantee that a requirement $r$ that is safe in the absence of an attacker $A$ remains safe in the presence of $A$ if $\blacktriangledown(r) \cap \blacktriangle(A)$ is empty. Thus, for each requirement $r\in R$, the set of attackers that could break $r$ is $\ThePowersetOf{\blacktriangledown(r)-W}$. Out of all the attackers of $r$, the most capable attacker is $\blacktriangledown(r)-W$, so we can test whether there \emph{exists} any attacker that can break $r$ in $t$ steps by solving $\mathtt{check}(r,t)$ against attacker $\blacktriangledown(r)-W$. 
For succinctness, we henceforth denote the attacker $\blacktriangledown(r)-W$ by $r^{max}$.
\begin{corollary}
From monotonicity and isolation (cf. Propositions \ref{theo:monotonicity} and \ref{theo:isolation}), if attacker $r^{max}$ cannot break the requirement $r$, then there are no minimal attackers for $r$. Equivalently, if $r^{max}$ cannot break $r$, then $r$ does not belong to any set of broken requirements $A[R]$.
\end{corollary}
\begin{figure}[!h]
\centering
{
%\begin{framed}
\begin{algorithm}[H]
 \KwData{system $S$, a requirement $r$, and a time step $t\geq 0$. }
\KwResult{set $M$ of \emph{minimal} attackers for $r$, bounded by $t$.}
{
\If{$check(r,t)$ is \textbf{not} satisfiable while assuming $c^\downarrow$ for all $c\in r^{max}$}
	{	
		\Return $\emptyset$\;
	}
}
Set: $P=\set{\emptyset}$, $M= \emptyset$; \quad /\!/($P$ contains the empty attacker $\emptyset$)\\
\While{$P$ is not empty}
{
	extract $A$ from $P$ such that the size of $A$ is minimal\;
	\If{\textbf{not} (exists $B \in M \text{ such that } B \subseteq A)$}
	{	
        	\eIf{$\mathtt{check}(r,t)$ is satisfiable when assuming $c^\downarrow$ for all $c\in A$}
        	{
        		insert $A$ in $M$\;
        	}
        	{
        		 \For{\!\!\textbf{each} $c \in (r^{max} - A)$}
        		 {
        			insert $A \cup \set{c}$ in $P$\;
        	 	}
        
        	}
	}
	
}
\Return $M$\;

 \caption{The $\mathtt{MinimalAttackers}$ algorithm.}
 \label{alg:CheckRequirement}
\end{algorithm}
%\end{framed}
}
\vspace{-0.5cm}
\end{figure}
\subsubsection{Finding Minimal Attackers.} 
After having confirmed that at least one minimal attacker for $r$ exists, we can focus on finding them. Our strategy consist of systematically increasing the capabilities of attackers that fail to break the requirement $r$ until they do. Algorithm~\ref{alg:CheckRequirement} describes this empowering procedure to computes the set of minimal attackers for a requirement $r$, which we call $\mathtt{MinimalAttackers}$. As mentioned, we first check to see if a minimal attacker exists (Lines 1-3); then we start evaluating attackers in an orderly fashion by always choosing the smallest attackers in the set of pending attackers $P$ (Lines 5-16). Line 7 uses monotonicity to discard the attacker $A$ if there is a successful attacker $B$ with $B\subseteq A$. Line 8 checks if the attacker $A$ can break $r$ in $t$ steps (or less); if so, then $A$ is a minimal attacker for $r$ and is included in $M$ (Line 9); otherwise, we empower $A$ with a new component $c$, and we add these new attackers to $P$ (Lines 11-13). We note that Line 11 relies on isolation, since we only add components that belong to the COI of $r$.

\begin{quote}
We recall the motivational example from Section~\ref{sec:Example}. Consider the computation of $\mathtt{MinimalAttackers}$ for requirement $r_2$. In this case, $r_2^{max}$ is $\set{g1,g2,v_1}$, which is able to break $r_2$, confirming the existence of (at least) a minimal attacker (Lines 1-3). We start to look for minimal attackers by checking the attacker $\emptyset$ (Lines 5-8); after we see that it fails to break $r_2$, we conclude that $\emptyset$ is not a minimal attacker and that we need to increase its capabilities. We then derive the attackers $\set{g_1}, \set{g_2}$ and $\set{v_1}$ by adding one non-isolated component to $\emptyset$, and we put them into the set of pending attackers (Lines 11-13). For attackers $\set{v_1}$ and $\set{g_2}$, we know that they can break the requirement $r_2$, so they get added to the set of minimal attackers, and are not empowered (Line 9); however, for attacker $\set{g_1}$, since it fails to break $r_2$, we increase its capabilities and we generate attackers $\set{v_1, g_1}$ and $\set{g_1, g_2}$. Finally, for these two latter attackers, since the minimal attackers $\set{v_1}$ and $\set{g_2}$ have already been identified, the check in Line 7 fails, and they are dismissed from the set of pending attackers, since they cannot be minimal. The algorithm finishes with $M=\set{\set{v_1}, \set{g_2}}$.
\end{quote}


Algorithm~\ref{alg:MinimalAttackers} applies Algorithm~\ref{alg:CheckRequirement} to each requirement; it collects all minimal attackers in the set $\mathcal{M}$ and initialises the attacker classification map $\mathcal{H}$. Finally, Algorithm~\ref{alg:GoodQuantification} exploits monotonicity to compute the classification of each attacker $A$ by aggregating the requirements broken by the minimal attackers that are subsets of $A$.

\begin{quote}
For the motivational example in Section~\ref{sec:Example}, Algorithm~\ref{alg:MinimalAttackers} returns $\mathcal{M}=\set{\emptyset,\set{v_1}, \set{g_2}}$ and $\mathcal{H}=\set{(\emptyset,\set{r_1}), (\set{v_1},\set{r_2, r_3}),  (\set{g_2},\set{r_2, r_3})}$. From there,  Algorithm~\ref{alg:GoodQuantification} completes the map $\mathcal{H}$, and returns
\begin{align*}
\mathcal{H}=\{&(\emptyset,\set{r_1}), (\set{v_1},\set{r_1,r_2, r_3}),  (\set{g_1},\set{r_1}),(\set{g_2},\set{r_1,r_2, r_3}), \\
&(\set{v_1,g_2},\set{r_1,r_2, r_3}),(\set{v_1,g_2},\set{r_1,r_2, r_3}),\\
&(\set{g_1,g_2},\set{r_1,r_2, r_3}),(\set{v_1,g_1,g_2},\set{r_1,r_2, r_3})\}
\end{align*}
\end{quote}
\vspace{-0.5cm}

\begin{figure}[!t]
%\begin{framed}
\centering
{
\begin{algorithm}[H]
 \KwData{system $S$, a time step $t\geq 0$, and a set of requirements $R$.}
 \KwResult{Set of all minimal attackers $\mathcal{M}$ and an initial classification map $\mathcal{H}$.}
Set: $\mathcal{M}=\emptyset$\;
Map: $\mathcal{H}$\;
\For{$r \in R$}
	{
		\For{$A \in \mathtt{MinimalAttackers}(S,t,r)$}
		{
			insert $r$ in $\mathcal{H}(A)$\;
			insert $A$ in $\mathcal{M}$\;
		}
	 }
 \Return $(\mathcal{M},\mathcal{H})$\;
 \caption{The $\mathtt{AllMinimalAttackers}$ algorithm. }
 \label{alg:MinimalAttackers}
\end{algorithm}}
%\end{framed}
\vspace{-0.5cm}
\end{figure}


\begin{figure}[!t]
\centering
{
%\begin{framed}
\begin{algorithm}[H]
 \KwData{system $S=(W,V,G)$, a time step $t\geq 0$, a set of requirements $R$.}
 \KwResult{A map $\mathcal{H}$ that maps the attacker $A$ to $A[R]$.}

$(\mathcal{M},\mathcal{H})=\mathtt{AllMinimalAttackers}(S,t,R)$\;
\For{\!\!\textbf{each} $A \subseteq {(V\cup G)}$}
{
\For{\!\!\textbf{each} $A' \in \mathcal{M}$}
	{
		\If{$A' \subseteq A$}
		{
			insert all elements of $\mathcal{H}(A')$ in $\mathcal{H}(A)$\;
		}
	 }
}
 \Return $\mathcal{H}$\;
 \caption{Improved classification algorithm. We assume that $\mathcal{H}$ initially maps every $A$ to the empty set.}
 \label{alg:GoodQuantification}
\end{algorithm}
%\end{framed}
}
\vspace{-0.5cm}
\end{figure}

\subsection{On Soundness and Completeness}
\label{sec:completeness}
Just like any bounded model checking problem, if the time parameter $t$ is below the \emph{completeness threshold} (see \cite{EfficientComputationOfRecurrenceDiameters}), the resulting attacker classification up to $t$ steps could be \emph{incomplete}. More precisely, an attacker classification up to $t$ steps may prove that an attacker $A$ cannot break some requirement $r$ with a strategy up to $t$ steps, while in reality $A$ can break $r$ by using a strategy whose length is strictly greater than $t$. There are practical reasons that justify the use of a time parameter that is lower than the completeness threshold: 1) computing the exact completeness threshold is often as hard as solving the model-checking problem \cite{HandbookOfModelChecking}, so an approximation is taken instead; and 2), the complexity of the classification problem growths exponentially with $t$ in the worst case, since the size of the SAT formulae grow with $t$, and there is an exponential number of attackers that need to be classified by making calls to the SAT solver. A classification that uses a $t$ below the completeness threshold, while possibly incomplete, is \emph{sound}, i.e., it does not falsely report that an attacker can break a requirement when in reality it cannot. In Section~\ref{sec:discussion} we discuss possible alternatives to overcome incompleteness, but we leave a definite solution as future work.

We also consider the possibility of limiting the maximum size of minimal attackers to approximate the problem of attacker classification. The result of a classification whose minimal sets are limited to have up to $z$ elements is also sound but incomplete, since does not identify minimal attackers that have more than $z$ elements. We show in Section~\ref{sec:evaluation} that, even with restricted minimal attackers, it is possible to obtain a high coverage of the universe of attackers.
 
\subsection{Requirement Clustering}
\emph{Property clustering} \cite{ToSplitOrToGroup,HandbookOfSatisfiability,OptimizedModelCheckingOfMultipleProperties} is a state-of-the-art technique for the model checking of multiple properties. Clustering allows the SAT solver to reuse information when solving a similar instance of the same problem, but under different assumptions. To create clusters for attacker classification, we combine the SAT problems whose COI is similar (i.e., requirements that have a Jaccard index close to 1), and incrementally enable and disable properties during verification. More precisely, to use clustering, instead of computing $\mathtt{goal}(r,t)$ for a single requirement, we compute $\mathtt{goal}(Y,t)$ for a cluster $Y$ of requirements, defined by 
\begin{align}
%\mathtt{goal}(Y,t) \triangleq \bigwedge_{\Always e\in Y} ((\lnot r^\downarrow) \lor \bigvee_{k=0}^t{\lnot e(k)}).
\mathtt{goal}(Y,t) \triangleq \bigwedge_{r\in Y} (\lnot r^\downarrow \lor \mathtt{goal}(r,t)).
\end{align}
where $r^\downarrow$ is a new literal that plays a similar role to the ones used for gates and latches; i.e., we assume $r^\downarrow$ when we want to find the minimal attackers for $r$, and we assume $\lnot y^\downarrow$ for all other requirements $y \in Y$. 

The SAT problem for checking whether the cluster of requirements $Y$ is safe up to $t$ steps is
\begin{align} 
\label{eq:MasterEq}
\mathtt{check}(Y,t)\triangleq\mathtt{goal}(Y,t)\land\! \bigwedge_{c\in (\blacktriangledown(Y)- W)}\left( \bigwedge_{k=0}^{t}{\mathtt{encode}(c,k)}\right),
\end{align}
where $\blacktriangledown(Y)=\bigcup \set{\blacktriangledown(r)| r\in Y}$.

%-------------------------------------------------------------------------------------------------------------------------------------------------
\section{Evaluation}
\label{sec:evaluation}
In this section, we perform experiments to evaluate how effective is the use of isolation and monotonicity for the classification of attackers, and we evaluate the completeness of partial classifications for different time steps. 

For evaluation, we use a sample of AIG benchmarks from past Hardware Model-Checking Competitions (see \cite{HWMCC2011,HWMCC2013}), from their multiple-property verification track. Each benchmark has an associated list of invariants to be verified which, for the purposes of this evaluation, we interpret as the set of security requirements. As of 2014, the benchmark set was composed of 230 different instances, coming from both academia and industrial settings \cite{HWMCC2014BM}. We quote from \cite{HWMCC2014BM}:
\begin{quote}
``Among industrial entries, 145 instances
belong to the SixthSense family (6s*, provided by IBM), 24 are Intel benchmarks (intel*),
and 24 are Oski benchmarks. Among the academic related benchmarks, the set includes 13
instances provided by Robert (Bob) Brayton (bob*), 4 benchmarks coming from Politecnico
di Torino (pdt*) and 15 Beem (beem*). Additionally, 5 more circuits, already present in
previous competitions, complete the set.''
\end{quote}
All experiments are performed on a quad core MacBook with 2.9 GHz Intel Core i7 and 16GB RAM, and we use the SAT solver CaDiCaL version 1.0.3 \cite{Cadical}. The source code of the artefact is available at \cite{aig-ac-asset}.

We separate our evaluation in two parts: 1) a comparative study where we evaluate the effectiveness of using of monotonicity and isolation for attacker classification in several benchmarks, and 2) a case study, where we apply our classification methodology to a single benchmark --\texttt{pdtvsarmultip}-- and we study the results of varying the time parameters for partial classification.

\subsection{Evaluating Methodologies}
Given a set of competing classification methodologies $\mathcal{M}_1,\ldots, \mathcal{M}_n$ (e.g., Algorithm~\ref{alg:BadQuantification} and Algorithm~\ref{alg:GoodQuantification}), each methodology is given the same set of benchmarks $S_1,\ldots, S_m$, each with its respective set of requirements $R_1, \ldots, R_m$. To evaluate a methodology $\mathcal{M}$ on a benchmark $S=(W,V,G)$ with a set of requirements $R$, we allow $\mathcal{M}$ to ``learn'' for about 10 minutes per requirement by making calls to the SAT solver, and produce a (partial) attacker classification $\mathcal{H}$.  Afterwards, we compute the coverage metric obtained by $\mathcal{M}$, defined as follows.

\begin{definition}[Coverage]
Let $\ThePowersetOf{V}$ be the set of all attackers, and let $\mathcal{H}$ be the attacker classification produced by the methodology $\mathcal{M}$. We recall that $\mathcal{H}$ is a map that maps each attacker $A$ to a set of requirements, and in the ideal case, $\mathcal{H}(A)=A[R]$, for each attacker $A$. The \emph{attacker coverage obtained by methodology $\mathcal{M}$ for a requirement $r$} is the percentage of attackers $A\in\ThePowersetOf{V}$ for which we can correctly determine whether $A$ breaks $r$ by computing $r\in \mathcal{H}(A)$ (i.e., we do not allow guessing and we do not allow making new calls to the SAT solver).
\end{definition}

We also measure the execution time of the classification per requirement. More precisely, the time it takes for the methodology to find minimal attackers, capped at about 10 minutes per requirement. We force stop the classification for each requirement if a timeout occurs, but not while the SAT solver is running (i.e., we do not interrupt the SAT solver), which is why sometimes the reported time exceeds 10 minutes.

\subsection{Effectiveness of Isolation and Monotonicity}
To test the effectiveness of isolation and monotonicity, we selected a small sample of seven benchmarks. For each benchmark, we test four variations of our methodology: 
\begin{enumerate}
\item{$(+IS, +MO)$}: Algorithm~\ref{alg:GoodQuantification}, which uses both isolation and monotonicity
\item{$(+IS,-MO)$}: Algorithm~\ref{alg:GoodQuantification} but removing the check for monotonicity on Algorithm~\ref{alg:CheckRequirement}, Line 7;
\item{$(-IS,+MO)$}: Algorithm~\ref{alg:GoodQuantification} but using Equation~\ref{eq:isolation} instead of Equation~\ref{eq:MasterEq} to remove isolation while preserving monotonicity; and 
\item{$(+IS,+MO)$} Algorithm~\ref{alg:BadQuantification}, which does not use isolation nor monotonicity. 
\end{enumerate}

{The benchmarks we selected have an average of 173 inputs, 8306 gates, 517 latches, and 80 requirements. } Under our formulation of attackers, these benchmarks have on average $2^{8823}$ attackers%, compared to 80 properties to be checked in model checking competitions
. However, since an attacker that controls a gate $g$ can be emulated by an attacker that controls all latches in the sources of $g$, we restrict attackers to be comprised of only latches; reducing the size of the set of attackers from $2^{8823}$ to $2^{517}$ on average per benchmark. Furthermore, we arbitrarily restrict the number of components that minimal attackers may control to a maximum of 3, which implies that, on a worst case scenario, we need to make a maximum of $80\times\sum_{k=0}^3 \binom {517}k$ calls to the SAT solver per benchmark. We also arbitrarily define the time step parameter $t$ to be 10.

Figure~\ref{fig:AverageCoverage} illustrates the average coverage for the four different methodologies, for each of the seven benchmarks. The exact coverage values are reported in the Appendix of \cite{aig-ac-arxiv} (an extended version of this article). We see that our methodology consistently obtains the best coverage of all the other methodologies, with the exception of benchmark $\mathtt{6s155}$, where the methodology that removes isolation triumphs over ours. We attribute this exception to the way the SAT solver reuses knowledge when working incrementally; it seems that, for $(-IS,+MO)$, the SAT solver can reuse more knowledge than for $(+IS,+MO)$, which is why $(-IS,+MO)$ can discover more minimal attackers in average than $(+IS,+MO)$. 


We observe that the most significant element in play to obtain a high coverage is the use of monotonicity. Methodologies that use monotonicity always obtain better results than their counterparts without monotonicity. Isolation does not show a trend for increasing coverage, but has an impact in terms of classification time. Figure~\ref{fig:AverageExecTime} presents the average classification time per requirement for the benchmarks under the different methodologies. We note that removing isolation often increases the average classification time of classification methodologies; the only exception --benchmark $\mathtt{6s325}$-- reports a smaller time because the SAT solver ran out of memory during SAT solving about $50\%$ of the time, which caused an early termination of the classification procedure. This early termination also reflects on the comparatively low coverage for the method $(-IS,+MO)$ in this benchmark, reported on Figure~\ref{fig:AverageCoverage}.

\begin{figure}[!t]
%\begin{minipage}{0.25\textwidth}
\centering
\includegraphics[width=\textwidth]{AverageCoverage}
\caption{Average requirement coverage per benchmark. A missing bar indicates a value that is approximately 0.}
\label{fig:AverageCoverage}
\vspace{0.5cm}
\includegraphics[width=\textwidth]{AverageExecutionTime}
\caption{Average classification time per requirement per benchmark. A missing bar indicates a value that is approximately 0. }
\label{fig:AverageExecTime}
%\end{minipage}
\end{figure}

\subsection{Partial Classification of the \texttt{pdtvsarmultip} Benchmark}
The benchmark \texttt{pdtvsarmultip} has 17 inputs, %giving values to the variables $v_{1}$ to $v_{17}$; it has 
130 latches, %ranging from $v_{18}$ to $v_{147}$; it has 
2743 gates, %ranging from $v_{148}$ to $v_{2743}$; 
and has an associated list of 33 invariant properties, out of which 31 are unique and we interpret as the list of security requirements. 
Since we are only considering attackers that control latches, there are a total of $2^{130}$ attackers that need to be classified for the 31 security requirements.

We consider 6 scenarios for partial classification up to $t$, with $t$ taking values in $\set{0,1,5,10,20,30}$. For each requirement, we obtain the execution time of classification (ms), the size of the set of source latches for the requirement (\#C), the number of minimal attackers found (\#Min.), the total number of calls to the SAT solver (\#SAT), the average number of components per minimal attacker (\#C./Min) and the coverage for the requirement (Cov.). We present the average of these measures in Table~\ref{tab:pdt}. 

Normally, the attacker classification behaves in a similar way to what is reported for requirement $ \Always \lnot g_{2177}$, shown in Table~\ref{tab:pdt2324}. More precisely, coverage steadily increases and stabilises as we increase $t$. However, we like to highlight two interesting phenomena that may occur: 1) coverage {may} \emph{decrease} as we increase the time step (e.g., as shown in Table~\ref{tab:pdt2367}), and 2) the number of minimal attackers decreases while the coverage increases, as shown in Tables~\ref{tab:pdt2324} and ~\ref{tab:pdt2367}. 


Case 1) occurs because the set of attackers that can effectively interact with the system at time 0 is rather small, i.e., $2^6$, while the set of attackers that can affect the system at times 0 and 1 has size $2^{26}$. The size of this set increases with time until it stabilises at $2^{66}$, which is the size of the set of attackers that cannot be dismissed by isolation. 

Case 2) occurs because the minimal attackers that are found for smaller time steps represent a small percentage of the set of attackers that can affect the system, so there is very little we can learn by using monotonicity. More precisely, those minimal attackers control a relatively large set of components, which they need to be successful in breaking requirements, as shown in Step 5, column \#C./Min in Tables~\ref{tab:pdt2324} and~\ref{tab:pdt2367}. By considering more time steps, we are allowing attackers that control less components to further propagate their actions through the system, which enables attack strategies that were unsuccessful for smaller choices of time steps.

\begin{table}[!t]
\centering
\caption{Average measures for all requirements per time steps.}
\begin{tabular}{|c|c|c|c|c|c|c|}
\hline
Steps & ms &  \#C. & \#Min. & \#SAT & \#C./Min. & Cov.\\
\hline
0 & 683.1290323 & 34.96774194 & 2.451612903 & 16328.77419 & 1.4 & 0.527277594\\
1 & 2387.548387 & 46.22580645 & 6.387096774 & 24420 & 1.650232484 & 0.572533254\\
5 & 5229.935484 & 58.93548387 & 44.90322581 & 28355.29032 & 1.639645689 & 0.84956949\\
10 & 24967.12903 & 58.93548387 & 151.1935484 & 25566.54839 & 1.460869285 & 0.918973269\\
20 & 13632.51613 & 58.93548387 & 17.67741935 & 20849.70968 & 1.176272506 & 0.979354259\\
30 & 12208.25806 & 58.93548387 & 15.93548387 & 20798.16129 & 1.104563895 & 0.979354274\\
\hline
\end{tabular}
\vspace{0.5cm}
\label{tab:pdt}
\centering
%\begin{minipage}{0.65\textwidth}
\begin{minipage}{0.45\textwidth}
\caption{Coverage for $\Always \lnot g_{2177}$.}
\resizebox{\textwidth}{!}{
\begin{tabular}{|c|c|c|c|c|c|c|}
\hline
\multicolumn{7}{|c|}{$\Always \lnot g_{2177}$} \\
\hline
Steps & ms & \#C & \#Min. & \#SAT & \#C./Min. & Cov.\\
\hline
0 & 895 & 59 & 0 & 34281 & -- & 5.94E-14\\
1 & 2187 & 66 & 10 & 47378 & 2 & 0.499511\\
5 & 1735 & 66 & 205 & 12476 & 1.912195 & 0.999997\\
10 & 968 & 66 & 27 & 9948 & 1 & 0.999999\\
20 & 1275 & 66 & 27 & 9948 & 1 & 0.999999\\
30 & 1819 & 66 & 27 & 9948 & 1 & 0.999999\\
 \hline
\end{tabular}
}
\label{tab:pdt2324}
\end{minipage}
\begin{minipage}{0.45\textwidth}
\caption{Coverage for $\Always \lnot g_{2220}$.}
\resizebox{\textwidth}{!}{
\begin{tabular}{|c|c|c|c|c|c|c|}
\hline
\multicolumn{7}{|c|}{$\Always \lnot g_{2220}$} \\
\hline
Steps & ms &  \#C. & \#Min. & \#SAT & \#C./Min. & Cov.\\
\hline
0 & 1 & 6 & 1 & 28 & 1 & 0.90625\\
1 & 86 & 26 & 1 & 2628 & 1 & 0.500039\\
5 & 4511 & 67 & 17 & 47664 & 2.588235 & 0.852539\\
10 & 3226 & 67 & 6 & 37889 & 1 & 0.984375\\
20 & 3355 & 67 & 6 & 37889 & 1 & 0.984375\\
30 & 3562 & 67 & 6 & 37889 & 1 & 0.984375\\
 \hline
\end{tabular}
}
\label{tab:pdt2367}
\end{minipage}
\end{table}
By taking an average over all requirements, we observe that coverage seems to steadily increase as we increase the number of steps for the classification, as reported in Table~\ref{tab:pdt}, column Cov. The low coverage for small $t$ is due to the restriction on the size of minimal attackers. More precisely, for small $t$, attackers can only use short strategies, which limits their interaction with the system; we expect attackers to control a large number of components if they want to successfully influence a requirement in this single time step, and since we restricted our search to attackers of size 3 maximum, these larger minimal attackers are not found (e.g., as reported in Table~\ref{tab:pdt2324} for Step 0).

We conclude that experimental evidence favours the use of both monotonicity and isolation for the classification of attackers, although  some exceptions may occur for the use of isolation. Nevertheless, these two techniques help our classification methodology $(+IS,+MO)$ consistently obtain significantly better coverage when compared to the naive methodology $(-IS,-MO)$.
%-------------------------------------------------------------------------------------------------------------------------------------------------

\section{Related Work}
\label{sec:discussion}

\subsubsection{On Defining Attackers.} Describing an adequate attacker model to contextualise the security guarantees of a system is not a trivial task. Some attacker models may be adequate to provide guarantees over one property (e.g. confidentiality), but not for a different one (e.g., integrity). Additionally, depending on the nature of the system and the security properties being studied, it is sensible to describe attackers at different levels of abstraction. For instance, in the case of security protocols, Basin and Cremers define attackers in~\cite{KnowYourEnemy} as combinations of compromise rules that span over three dimensions:  \emph{whose} data is compromised, \emph{which} kind of data it is, and \emph{when} the compromise occurs. In the case of Cyber-physical Systems (CPS), works like \cite{Giraldo2018,Simei} model attackers as sets of components (e.g., some sensors or  actuators), while other works like \cite{IFCPSSec,Cardenas2011,Urbina2016} model attackers that can arbitrarily manipulate any control inputs and any sensor measurements at will, as long as they avoid detection. In the same context of CPS, Rocchetto and Tippenhauer \cite{CPSProfiles} model attackers more abstractly as combinations of quantifiable traits (e.g., insider knowledge, access to tools, and financial support), which, when provided a compatible system model, ideally fully define how the attacker can interact with the system. 

Our methodology for the definition of attackers combines aspects from~\cite{KnowYourEnemy,Giraldo2018} and \cite{Simei}. The authors of~\cite{KnowYourEnemy} define symbolic attackers and a set of rules that describe how the attackers affect the system, which is sensible since many cryptographic protocols are described symbolically. Our methodology describes attackers as sets of components (staying closer to the definitions of attackers in \cite{Giraldo2018} and \cite{Simei}), and has a lower level of abstraction since we describe the semantics of attacker actions in terms of how they change the functional behaviour of the AIG, and not in terms of what they ultimately represent. This lower level of abstraction lets us systematically and exhaustively generate attackers by simply having a benchmark description, but it limits the results of the analysis to the benchmark; Basin and Cremers can compare among different protocol implementations, because attackers have the same semantics even amongst different protocols. If we had an abstraction function from sets of gates and latches to symbolic notions (e.g., ``components in charge of encryption'', or ``components in charge of redundancy''), then it could be possible to compare results amongst different AIGs.

\subsubsection{On Efficient Classification.} 
The works by Cabodi, Camurati and Quer \cite{GraphLabelingForEfficientCOIComputation}, Cabodi et. al \cite{ToSplitOrToGroup}, and Cabodi and Nocco \cite{OptimizedModelCheckingOfMultipleProperties} present several useful techniques that can be used to improve the performance of model checking when verifying multiple properties, including COI reduction and property clustering. We also mention the work by Goldberg et al. \cite{JustAssume} where they consider the problem of efficiently checking a set of safety properties $P_1$ to $P_k$ by individually checking each property while assuming that all other properties are valid. Ultimately, all these works inspired us to incrementally check requirements in the same cluster, helping us transform Equation~\ref{eq:naiveCheck} into Equation~\ref{eq:MasterEq}. Nevertheless, we note that all these techniques are described for model checking systems in the absence of attackers, which is why we needed to introduce the notions of isolation and monotonicity to account for them. Additionally, it may be possible to use or incorporate other techniques that improve the efficiency of BMC in general (e.g., interpolation \cite{Interpolation}).

\subsubsection{On Completeness.} As mentioned in Section~\ref{sec:completeness}, if the time parameter for the classification is below the {completeness threshold}, the resulting attacker classification is most likely {incomplete}. To guarantee completeness, it may be possible to adapt existing termination methods (see \cite{ProvingMorePropertiesWithBMC}) to consider attackers. Alternatively, methods that compute a good approximation of the completeness threshold (see \cite{EfficientComputationOfRecurrenceDiameters}) which guarantee the precision of resulting the coverage should help improve the completeness of attacker classifications. If we consider alternative verification verification techniques, then IC3 \cite{IC3,IC32} and PDR \cite{IC3}, which have seen some success in hardware model checking, may address the limitation of boundedness. Finally, interpolation \cite{Interpolation} could also help finding a guarantee of completeness. 

\subsubsection{On Verifying Non-Safety Properties.} In this work, we focused our analysis exclusively on safety properties of the form $\Always e$. However, we believe that it is possible to extend this methodology to other types of properties, it is possible to efficiently encode Linear Temporal Logic formulae for bounded model checking \cite{BMCWithoutBDDs,lmcs:2236}. The formulations of the SAT problem change for the different nature of the formulae, but both isolation and monotonicity should remain valid heuristics, since they ultimately refer to how strategies of attackers can be inferred, not how they are constructed.  

%-------------------------------------------------------------------------------------------------------------------------------------------------
\section{Conclusion and Future Work}
\label{sec:conclusion}
In this work, we present a methodology to model check systems in the presence of attackers with the objective of mapping each attacker to the list of security requirements that it breaks. This mapping of attackers creates a classification for them, defining equivalence classes of attackers by the set of requirements that they can break. The system can then be considered safe in the presence of attackers that cannot break any requirement. While it is possible to perform a classification of attackers by exhaustively performing model checking, the exponential size of the set of attackers renders this naive approach impractical. Thus, we rely on ordering relations between attackers to efficiently classify a large percentage of them, and we demonstrate empirically by applying our methodology to a set of benchmarks that describe hardware systems at a bit level.

In our view, ensuring the completeness of the attacker classification is the most relevant direction for future work. Unlike complete classifications, incomplete classifications cannot provide guarantees that work in the general case if minimal attackers are not found. We also note that the effectiveness of monotonicity for classification is directly related to finding minimal attackers. Consequently, our methodology may benefit from any other method that helps in the identification of those minimal attackers. In particular, we are interested in checking the effectiveness of an approach where, instead of empowering attackers, we try to {reduce} successful attackers into minimal attackers by removing unnecessary capabilities. This, formally, is an \emph{actual causality analysis} \cite{ActualCausality} of successful attackers.
}
%-------------------------------------------------------------------------------------------------------------------------------------------------
%!TEX root = ../main.tex
\chapter{Improving the Robustness of Cyber-Physical System Models}
\label{ch:CPSRobustness}
\begin{quote}
    "Attacks are random faults that are not really random." -- Carlos Murguia.  
\end{quote}
\todo[inline]{Here comes the paper we are going to send to TOPS. I am not going to put it here to prevent duplication and inconsistencies.}


\todo[inline]{Connector from the classification chapter}
We now show how to study systems with non-binary ranges for coordinates. 
%!TEX root = ../main.tex
\chapter{Transforming Programs for Timing Side-Channel Repair}
\label{ch:SideChannelRepair}
\todo[inline]{Here comes the paper we sent to CSF. It needs to be re-written for consistency}
\todo[inline]{This chapter is interesting because we are going to target programs as states. }



\section{Introduction}
\label{sec:introduction} 
We now consider a scenario where we are applying a spatial transformations to reveal a latent behaviour that satisfies a behavioural property we are interested in. For this case study, we choose a system whose states are programs and whose observations are the state of some cache memory, and we use spatial transformations to repair side-channel attacks that leak information via the cache.
\todo[inline]{Not very convinced about this motivation...}

% \todo[inline]{Candidate conferences: FM 2021 (abs 30 apr/6 may), CSF 2022 (may 14)}
\paragraph*{What are timing side-channel attacks?} Timing attacks are among the best known side-channel attacks~\cite{timing-channel-survey} 
to ex-filtrate secret information from a program. %The basic idea behind timing attacks is to first observe the execution time of a program. Subsequently, such timing information is used to compute or constrain the values of a secret input processed by the program. 
Basic timing side-channel attacks aim to establish a relationship between inputs and execution time, which can be done by attackers who have a copy of the program. After running the program with different inputs, the attacker has a model of this input to execution-time relationship. Attackers then observe the total execution time of the same program run by the victim, and can infer the value of the secret input. 

More advanced timing attacks exploit micro-architectural features, e.g. caches, and they require the attacker be able to interact with these micro-architectural aspects in the machine where the victim executes the vulnerable program. Spectre style attacks~\cite{Spectre}, as discovered in 2018, are sophisticated attacks where the attacker exploits timing covert channels to ex-filtrate secret information loaded by speculative execution in the cache by reverse engineering the value of a secret input after observing cache hit/miss timings~\cite{timing-cache} or by computing the cache lines being accessed~\cite{prime-probe}. While Spectre attacks rely on speculative execution to load secrets into the cache, poorly implemented cryptographic software could directly leak secrets via memory access patterns, even when speculative execution is disabled.
%Therefore, closing the timing channels in programs is of critical importance.

%\textbf{} 
\paragraph*{Repairing Leakage due to Memory Access Patterns (MAP)} 
If we assume that attackers can only infer information from the behaviour of the program counter, then the automatic repair of programs with timing side-channel vulnerabilities is a well understood problem; moreover, existing solutions \cite{SCEliminator,MSESC,Racoon} do an acceptable job at closing timing side-channels while preserving functionality and preventing the appearance of undesired side-effects (e.g. unsafe memory accesses). %minimising the risk of exploitation. 
However, once we empower the attacker to manipulate micro-architectural aspects, especially those that breach memory isolation like the ones used for Spectre~\cite{Spectre} and Meltdown~\cite{Meltdown}, the repairing of vulnerable programs remains an open problem \cite{timing-channel-survey}. 
%heavily relying on \emph{bitslicing}. 

In this work, we are particularly interested in repairing programs that leak information when a data structures is accessed using a secret value. These programs are vulnerable to an attacker that cannot read the contents of the cache directly but can manipulate and observe the state of the cache using attacks like Flush+Reload~\cite{Flush+Reload} or Prime+Probe\cite{prime-probe} to infer secrets.

Consider the example program \textbf{P}, defined by%\verb+if A[s] then x:=1 else x:=0+
% \begin{verbatim}
    \begin{align*}
        \text{if $A[s]$ then $x:=1$ else $y:=0$,}
    \end{align*}
% \end{verbatim} 
which reveals $A[s]$ under the \emph{baseline leakage model}, because the attacker can infer the value of $A[s]$ by following the program counter. The baseline leakage model assumes that the valuations of branch conditions are exposed to attackers (see \cite[\S3, Example 1]{usenix_ctp_verification}), so the program \textbf{P} is unsafe if \texttt{A[s]} is a secret, but it is safe if \texttt{A[s]} is public (e.g. if \texttt{A[s]} is declassified data)%for every possible value of the secret $s$
. Existing repair tools \cite{SCEliminator,MSESC,Racoon} offer strong security guarantees against the baseline leakage model, and can repair the program \textbf{P} with respect to this leakage model, yielding the linear-code program \textbf{T(P)}, defined by
\begin{align*}
    x:=CTSel(A[s],1,x);y:=CTSel(A[s],y,0),
\end{align*}
% \begin{verbatim}
%                 x:=CTSel(A[s],1,x)
%                 y:=CTSel(A[s],y,0),
% \end{verbatim} 
where $CTSel(c,a,b)$ is a constant-time selector which returns $a$ if $c$ is true, or $b$ otherwise. The program \textbf{T(P)} is arguably safer with respect to the baseline leakage model, since the behaviour of the program counter no longer reveals the value of $A[s]$. 

Now, if we consider a leakage model which considers Memory Access Patterns (MAP), then the program \textbf{T(P)} leaks $s$, because the attacker can still infer $s$ by probing the cache. 
% \todo[inline]{I am not going to touch Spectre as motivation, more like related work. }
% This leakage model suits Spectre V1 attacks \cite{Spectre}, since it uses speculative loads to execute \texttt{A[s]}. 
% \todo[inline]{/I am not going to touch Spectre as motivation, more like related work. }
Existing compiler-based repair tools \cite{SCEliminator,MSESC,Racoon} offer weak, inefficient, or no guarantees at all against this leakage model: Racoon \cite{Racoon} implements ORAM, which is quite taxing in terms of performance (it induces an overhead with geometric mean ~16x), \textsc{Sc-Eliminator} uses data structure preloading, but it is an unsound repair if the attacker can manipulate the cache, and the methodology presented in \cite{MSESC} does not repair against this leakage model. %, 
%and existing language-based repair approaches~\cite{FaCT} simply reject programs that access data structures with secrets. 
This lack of effective and efficient repair guarantees is the main motivation for our work. 

%The repair procedure from \cite{MSESC} only offers security guarantees if the \textbf{P} has the same memory footprint (it does not, because)\texttt{SC-eliminator} offers some guarantees 
% Existing repair solution address the program above under the baseline leakage model, but do not repair it under the memory access patterns leakage model.

\paragraph*{Our Contributions}
The authors of ~\cite{WhatYouCisWhatYouGet} describe a philosophy which aims to delegate the compiler the enforcement of timing side-channel freedom. We follow this philosophy, and we propose a set of repair rules to close the timing side-channel created by accessing fixed-size data structures indexed by secret information. Our repair rules offer strong security guarantees with respect to the leakage model that accounts for memory access patterns. In a nutshell, we replace each read access $y:=A[x]$ and each write access $A[x]:=y$ by a linear program that explores $A$, systematically loading it in memory, guaranteeing that $y=A[x]$ in the case of a read, and that $A[x]=y$ in the case of a write. These operations are similar to \emph{fold} high-order functions, which is why we name our repair rules ORIGAMI.

We implement ORIGAMI as an LLVM opt pass to make it compatible with other target-independent compiler optimizations. The tool lets the compiler first perform optimisations, and then the tool applies the ORIGAMI repair rules to the resulting intermediate representation code. Our pass ensures that compiled programs have a constant memory footprint when indexing fixed-size array-like data structures using secrets. Our proposed solution only requires minimal annotation to the source code (i.e., to inform the compiler which variables and function arguments are secrets, and for loops whose bounds cannot be automatically derived by the compiler) and provides theoretical guarantees that the transformed code is secure under the memory access pattern leakage model. 

There are a couple of clear limitations when repairing programs with ORIGAMI, (e.g. ORIGAMI can obfuscate a read access to a data structure whose values are secret pointers, but fails to protect the program if such resulting pointer is later used to load or store a value) which we discuss in Section~\ref{sec:Limitations}. These limitations illustrate the impossibility of repairing every program program with respect to the the memory access patterns leakage model while keeping the program functional, efficient, and secure. 

%These l if the secrets to be protected are pointers, or the if the secrets have unbounded values and we use them to access a dynamic data structure. To improve efficiency, we suggest a couple of optimisations that increase efficiency at the cost of security.

% This rule is embedded by the compiler as part of an optimisation pass.
% %is often translated to \verb+x:=ctSel(A[s],1,0)+ by the enforcement rules used for basic timing side-channel attacks. 
% While this repair rule protects the value of \texttt{A[s]}, it does not protect value of $s$. Reading or writing \texttt{A[s]} loads it in the cache, but that does not mean that the other parts of $A$ are loaded. This technique is used in the typical Spectre V1 from \cite{Spectre}
% \begin{verbatim}
%     PUT EXAMPLE HERE
% \end{verbatim}
% To solve this problem, whenever you access a data structure using a secret, you must do it in a way that its memory access pattern is constant. One way to solve it is by \emph{bit-slicing} the data structure A. 

% \todo[inline]{Here, examples from Section 3 of the USENIX paper are quite good, but they do not say how to solve them.}



% To provide a formal foundation in reasoning about TSCF in arbitrary programs, we use  
% Guarded Kleene Algebra with Tests (GKAT). Specifically, the axioms and rules for GKAT 
% expressions let us reason about the equivalence of programs. In this setting, the notion of 
% semantic equivalence is defined over uninterpreted actions using languages of \emph{guarded strings}. Guarded strings 
% are an intercalation of a logical atom and an action, which can be seen as the concatenation 
% of $\set{\texttt{pre}}\set{\texttt{action}}\set{\texttt{pos}}$ elements, where \texttt{pre} and \texttt{pos} represent the precondition and postcondition of the \texttt{action} (any \texttt{pre} and any \texttt{pos} as we work with uninterpreted actions, but a \texttt{pos} and a \texttt{pre} need to be compatible to be concatenated). 
% GKAT has been used to reason about compiler optimizations~\cite{KATForCompilers}. 
{
This paper is structured as follows: we first provide a brief background in Section~\ref{sec:Preliminaries}. In Section~\ref{sec:TSCF}, we formally define timing side-channel freedom (TSCF) under the MAP leakage model, which is the property that we want to enforce. We present the ORIGAMI repair rules in Section~\ref{sec:ORIGAMI}, and we prove that they enforce TSCF under MAP for a language of while programs; we also discuss the limitations of this enforcement. In section ~\ref{sec:Evaluation}, we evaluate an implementation of the ORIGAMI rules as LLVM optimization passes; we use these passes to enforce TSCF in LLVM-IR after all other compiler optimizations have taken place. We apply ORIGAMI to a small toy example and to real cryptographic ciphers from OpenSSL~\cite{OpenSSL} and to GDK library routines~\cite{gdklib,gdklib}, and we evaluate all the repaired programs using GEM5 -- a cycle accurate simulator for x86 processor -- to empirically show that all repaired programs satisfy TSCF with respect to MAP leakage. We then compare ORIGAMI against related work  in Section~\ref{sec:RelatedWork}, and we conclude in 
Section~\ref{sec:Conclusion}. 
%\section{Motivational Example}
}
\section{Preliminaries}
\label{sec:Preliminaries}
In this section, we provide the definitions and notation that we use through this work. 
 
\subsection{Timing Side Channel Freedom Enforcement}
\paragraph*{Why are timing side-channel vulnerabilities so hard to fix?}
Both the programming languages and the computer security communities understand fairly well where timing differences could be introduced during the compilation and execution processes of software, and they tackle the problem of enforcing \emph{timing side-channel freedom (TSCF)} using a layered approach. 
Unfortunately,
timing differences can be (unintentionally) introduced at every step of the compilation process, and they propagate to the following stages. %Moreover, all TSCF enforcement machinery relies on assumptions of the m

For illustration purposes, let us consider a simplified version of the compilation and execution process. A program starts out as \emph{source code}, which is then given to a compiler. 
The compiler often creates an \emph{intermediate representation} (IR) of the program (e.g. a control flow graph (CFG)), which the compiler then optimises by using %(usually idempotent)
transformation rules that can be applied to \emph{any} IR (e.g. dead-code elimination). We call the entity in charge of creation and optimisation of the IR the \emph{front-end} of the compiler. The \emph{back-end} of the compiler then compiles the optimised IR into a microarchitecture-dependent \emph{low-level representation} (LLR), performs microarchitecture-dependent optimisations, and then creates the executable. Finally, the executable runs on the microarchitecture by following the sequence of instructions in the executable. 

At the \emph{source code} level, a developer who does not follow constant-time programming guidelines, e.g., CryptoCoding~\cite{CryptoCoding}, can introduce timing differences by, e.g., using loops with input-dependent bounds, or by terminating early if a branch condition is satisfied, e.g. in a base case of a recursive functions. 
At the \emph{IR} level, %since there are no CTP guidelines for compilers to follow, and 
since compilers often optimise for performance, they may introduce timing differences via optimisations at the IR level, just like a programmer would at source level. To make matters more complicated, the compiler may even remove TSCF countermeasures introduced at the source code level if it deems them non-optimal, which is why developers of crypto algorithms disable compiler optimisations, or even choose to avoid compilers altogether and instead directly implement crypto routines in assembly \cite{timing-channel-survey}.

Then, the back-end repeats the story of the front-end: it creates the LLR, optimises it and creates the executable, but its own optimisations may remove any TSCF enforcement introduced in the IR, and it may itself introduce timing differences. 
Finally, even if the back-end does not itself introduce timing differences or removes countermeasures added at the previous stages, the microarchitecture may manifest timing differences during program execution; this may be because a micro-architectural instruction can vary its execution time depending on its parameters (e.g. multiplication), or due to out-of-order execution and speculative execution. In that sense, any TSCF enforcement introduced at early stages can be made irrelevant at later stages.  

% \todo[inline]{All sources of timing differences: executions that vary on instructions due to branching, and executions that use instructions that vary on inputs. The other one is not because of instructions but because of the microarchitectural state during execution, in particular the cache. }
\subsection{Leakage Models}
We find the notion of leakage models used in \texttt{ct-verif}~\cite{usenix_ctp_verification} particularly enlightening. Although their leakage models are defined based on LLVM rather than machine code, %their tool \texttt{ct-verif} provides empirical evidence of their usefulness, and t
they argue in~\cite[\S5]{usenix_ctp_verification} that ``LLVM
assembly code produced just before code generation [is] sufficiently
similar to \emph{any} target-machine's assembly code to
provide a high level of confidence.'' 

In the following, we provide the intuition behind three useful leakage models, what it means for a program to leak secrets with respect to them, and insights on what repairing a program with respect to each model entails.

\paragraph*{Baseline Leakage Model} This leakage model reveals to the attacker the valuations of branch conditions. More precisely, the program 
%\begin{verbatim}
\begin{align*}
    \text{if $c$ then $p_1$ else $p_2$}
\end{align*}
%\end{verbatim} 
reveals the valuation of $c$, and the program
% \begin{verbatim}
%     while c do p,
% \end{verbatim} 
\begin{align*}
    \text{while $c$ do $p$}
\end{align*}
reveals the valuation of $c$. This is the \emph{baseline} leakage model because it is implied by all other leakage models.

A program leaks secrets with respect to this model if secrets influence the behaviour of the \emph{program counter}, which is why it is also known as the \emph{program counter security model} \cite{Molnar}. To repair a program with respect to this model, secret-dependent branches are linearised by replacing conditionals with constant-time selectors and loops are fully unrolled. This causes the behaviour of the program counter to be independent of value of secrets.

\paragraph*{Memory Access Patterns Leakage Model (MAP)}
in addition to revealing the valuation of branch conditions, the MAP model reveals the indices used to access data structures. More precisely, the programs
% \begin{verbatim}
%     A[x]:=B[y]
% \end{verbatim} 
\begin{align*}
    A[x]:=y,\quad \text{and} \quad y:=A[x]
\end{align*}
each reveals $x$ because different indices may have different memory access patterns (e.g. when the cache lines for $A[x]$ and $A[x']$ are different), and the attacker can infer this information. Thus, a program leaks secrets under the MAP model if they are used to access data structures \cite{usenix_ctp_verification}. This leakage model is related to {memory trace obliviousness} \cite{MemoryTraceOblivious}, which requires constant behaviour of the memory for all public-equivalent traces. To avoid leakage under the MAP model, we must not use secrets when accessing data structures, and we must enforce a secret-independent behaviour on the program counter (to avoid leakage following the baseline model).

\textsc{SC-Eliminator} proposes the use of preloading and must-hit analysis to repair programs so that they satisfy TSCF under the MAP leakage model. 
Unfortunately, this repair implicitly assumes that the state of the cache during must-hit analysis is the same as when the program executes, which is a problematic assumption if we consider that the attacker can also manipulate the cache using Prime+Probe and Flush+Reload attacks. 
Instead of preloading, we propose a new repair rule where instructions accessing data structures using secrets, i.e.$A[x]:=y$ and $y:=A[x]$, have the constant memory access patterns, thus preventing leaks under the MAP. The details of this solution are explained in detail in Section~\ref{sec:ORIGAMI}.
 
\paragraph*{Operand Sensitive Leakage Model (OS)} For completeness, we include the leakage model that distinguishes operations whose execution time are sensitive to inputs. The program $y:=f(x)$ leaks its parameter $x$ if its total execution time depends on $x$. This leakage model is the most general of the three models presented, as it implies the MAP and baseline models. Repairing programs with respect to the OS model at the source or compiler level is extremely challenging, because some operations offered by the micro-architecture leak their parameters (e.g. division and multiplication); thus, solutions for the OS model may need to be target dependent. Enforcing TSCF with respect to this leakage model is outside the scope of this work.

\subsection{TSCF Under MAP Leakage - Informally}
\label{sec:MAP}
The MAP leakage model represents an attacker that is able to use the timing of hits and misses in the cache to indirectly obtain values from the cache, similar to what attackers relying on Spectre attacks do to recover the secrets loaded in memory. More precisely, if an array-like structure $A$ is large enough to require several cache lines for it to be fully loaded in the cache, then caching $A[s]$ only fills the cache line that corresponds to the index $s$ and its neighbouring values. An attacker can gain information about $s$ by probing the cache, testing which parts of $A$ result in a cache hits and which ones do not. 

For example, under the MAP leakage model, the program 
\begin{align*}
    \text{if $s<size_A$ then $x:=B[A[s]]$}
\end{align*} 
%\verb+if s<size_A then x:=B[A[s]]+ 
reveals both $s$ and $A[s]$, because the state of the cache is different for different values of $s$. This program does not reveal $B[A[s]]$, only the indices used to access it.

% \todo[inline]{Do we really want to do GKAT expressions? They did help me solve an engineering problem in LLVM, but they might not be super justified here. We can probably benefit better from a small imperative language.}

\subsection{Guarded Kleene Algebra with Tests (GKAT)}
\emph{Guarded Kleene Algebra with Tests} (GKAT) is a modern formalism that offers a propositional abstraction of imperative while programs with uninterpreted actions~\cite{GKAT}. The specialty of GKAT is to enable reasoning about properties of programs \emph{by merely looking at their structure and not at their (functional) semantics}. This makes GKAT interesting for modelling transformations at the compiler level~\cite{KATForCompilers}, because a general-purpose compiler should not need know the exact semantics of a program to optimise it; in general, the compiler should look for structural patterns which enable optimisations, just as GKAT does for reasoning. 

We use GKAT to provide a formal foundation in reasoning about TSCF for arbitrary programs. Specifically, the axioms and rules for GKAT 
expressions help us reason about the equivalence of programs. In this setting, the notion of 
semantic equivalence is defined over uninterpreted actions using languages of \emph{guarded strings}, defined using \emph{actions} and \emph{tests}. 

% Through this work, we use GKAT expressions play the role of intermediate representations of programs, and rules defined over them model optimisation passes run by a compiler.
%This which makes them an ideal formalism for reasoning about optimizations in a compiler~\cite{KATForCompilers}. 
 %We also use GKAT to formalise abstract leakage models.
% \todo[inline]{Should we work with the small imperative language presented in GKAT? I think we should} 
\paragraph*{Actions, Tests and Expressions}
Every GKAT is parametrised by a set of abstract \emph{actions} $\Sigma$ and a finite set of abstract \emph{primitive tests} $T$. We assume $T$ and $\Sigma$ are disjoint and non-empty. A test $t\in T$ is an atomic proposition about the state of the program, and the execution of an action $p \in \Sigma$ can affect the state. We form \emph{GKAT expressions} (GKATx) with the grammar presented in Figure~\ref{tab:GKAT}. 

% For example, the expression 
% \begin{align*}
% \branch{\left(x<a1\_size\right)}{\left(\texttt{temp }\&=\texttt{ a2[a1[x] * 512]}\right)}{1}
% \end{align*}
% models the example program of Section~\ref{sec:Spectre}.
% , and the expression
% \begin{align*}
% i:=0\cdot r:=1\cdot \iteration{i<n}{\left(\left(\branch{$s$[i]\neq\texttt{g}[i]}{(r:=0)}{1}\right)\cdot (i:=i+1)\right)}\cdot r
% \end{align*}
% models the example program of Section~\ref{sec:TimingAttacks}, where $r$ is the variable that would be returned. %We remark that there no notion of \texttt{break} for GKATx, however, we can defer 

%$\branch{b}{p}{1}$ models \textbf{if} $b$ \textbf{then} $p$ and $\iteration{b}{\left(p\cdot q\right)}$ models \textbf{while} $b$ \textbf{do} $p;q$ \textbf{end}.

\paragraph*{Atoms}
An \emph{atom} is a truth assignment of all the tests in $T$.  %Formally, an atom is a non-zero minimal element in the free boolean algebra in $T$ \cite{KAT}.
We denote atoms by $\alpha, \beta,$ and $\gamma$, and the set of atoms by $\Atom$. For example, if $T=\set{t_1,t_2}$, the boolean expressions $\alpha=\overline{t_1}\cdot \overline{t_2}$, $\beta=\overline{t_1}\cdot {t_2}$, $\gamma=t_1\cdot \overline{t_2}$, and $\delta=t_1\cdot {t_2}$ are all the atoms, where $\overline{t_i}$ is the complement of $t_i$, for $i\in \set{1,2}$.

Guarded strings 
are an intercalation of a logical atom and an action, which can be seen as the concatenation 
of $\set{\texttt{pre}}\set{\texttt{action}}\set{\texttt{pos}}$ elements, where \texttt{pre} and \texttt{pos} represent the precondition and postcondition of the \texttt{action} (any \texttt{pre} and any \texttt{pos} as we work with uninterpreted actions, but a \texttt{pos} and a \texttt{pre} need to be compatible to be concatenated). 

\paragraph*{Guarded Strings} A \emph{guarded string} %is an interleaved sequence of \emph{atoms} and actions, which models how actions change the state of the system. Formally, a \emph{guarded string} 
$g$ is an element of the set $\GuardedString := \Atom \cdot \left(\Sigma\cdot \Atom\right)^{*}$, and it models a trace of an abstract program. %A language of guarded strings $L\subseteq \GuardedString$ has \emph{determinacy} iff for all $x,y\in L$, if $x$ and $y$ agree on their first $n$ atoms, then they agree on their first $n$ actions.
%\begin{example}
 %\end{example}
To compose guarded strings, we use \emph{fusion product} $\diamond\colon \GuardedString\times\GuardedString\rightarrow\GuardedString$, a partial function defined by
\begin{align}
w\alpha \diamond \beta v \triangleq 
	\begin{cases}
		w \alpha v, & \text{if $\alpha=\beta$};\\
		\text{undefined,}& \text{otherwise.}
	\end{cases}
\end{align}
The fusion product of sets $L_1,L_2\subseteq \GuardedString$ is defined by $L_1\diamond L_2$, where 

\begin{align*}
    L_1\diamond L_2 \triangleq \set{g_1\diamond g_2\ |\ g_1\in L_1,g_2\in L_2,\text{ and $g_1\diamond g_2$ is defined}}.
\end{align*}
\paragraph*{Language-based Semantics}
The \emph{language-based semantics} of a GKATx is a {set of guarded strings} (i.e., a language) defined by%and it can be found in \cite[\S2.2]{GKAT}.\todo{Is this ok?}
\begin{align*}
\semantics{p} &\triangleq \set{\alpha p \beta\ |\ \alpha,\beta\in \Atom},\\
%
\semantics{b} &\triangleq  \set{\alpha\ |\ \alpha\in \Atom \text{ and }\alpha \Rightarrow b},\\
%
\semantics{f\cdot g} &\triangleq \semantics{f}\diamond\semantics{g},\\
%\irsemantics{\branch{b}{f}{g}} &\triangleq \irsemantics{f}\xleftarrow{b}\irsemantics{1}\xrightarrow{\bar{b}}\irsemantics{g}\\
%
%
\semantics{\branch{b}{f}{g}} &\triangleq (\semantics{b}\diamond\semantics{f})\cup ((\Atom-\semantics{b})\diamond \semantics{g}),\\
%
%
\semantics{ \iteration{b}{e}} &\triangleq  \bigcup_{n\geq 0}\left(\semantics{b}\diamond\semantics{e}\right)^{n}\diamond (\Atom-\semantics{b}),
%
%
\end{align*}
where $L^0\triangleq\Atom$ and $L^{n+1}\triangleq L^{n} \diamond L$, for $L \subseteq \GuardedString$. Since actions in GKATx are uninterpreted, language-based semantics are an over-approximation of functional semantics. %i.e., once actions and tests are instantiated, they are guaranteed to exist within the abstract semantics.
% \paragraph*{Axiom}
% A \emph{GKAT axiom} establishes equivalences between syntactically different GKATx.  yet semantically equivalent; e.g., $b\cdot\left(\branch{b}{e}{f} \right) \equiv b\cdot e$.
\paragraph*{Rules}
A \emph{GKAT rule} is an equivalence that lets us transform an arbitrary GKATx into a GKATx that is syntactically different, yet semantically equivalent; e.g., $b\cdot\left(\branch{b}{e}{f} \right) \equiv b\cdot e$.

\begin{figure}[t]
\centering
\begin{tabular}{lr}
\begin{tabular}{rcll}
%\begin{tabular}{RCLL}
\multicolumn{4}{l}{$b,c,d, \in \bexp::=$} \\
  &    $|$   &   0   		&  \textbf{False}   \\
  &    $|$   &   1   		&   \textbf{True}   \\
  &    $|$   &    $t\in T$  		&   $t$  \\
  &    $|$   &    $b \cdot c$  	&  $b\ \textbf{and}\ c$   \\
  &    $|$   &    $b+c$  		&  $b\ \textbf{or}\ c $  \\
  &    $|$   &    $\bar{b}$  	& $\textbf{not}\ b$
\end{tabular}&
%\caption{Boolean expressions}
%\label{tab:GKAT_Boolean_Expressions}
%\end{table}
%\end{minipage}%
%\begin{minipage}{.25\textwidth}
%\begin{tabular}{RCLL}
%\begin{table}[t]
\begin{tabular}{rcll}
\multicolumn{4}{l}{$e,f,g, \in \gexp::=$} \\
  &    $|$   &  $p \in \Sigma$  		&  \textbf{do} $p$ \\
  &    $|$   &  $ b \in \bexp$   		& \textbf{assert} $b$ \\
  &    $|$   &  $  e \cdot f $ 			& $e${;}$f$   \\
  &    $|$   &  $ \branch{b}{f}{g}$ 	& \textbf{if} $b$ \textbf{then}$f$\textbf{else} $g$\\
  &    $|$   &  $  \iteration{b}{e} $		& \textbf{while} $b$ \textbf{do} $e$
\end{tabular}
\end{tabular}
\caption{\textbf{Left:} boolean expressions.  \textbf{Right:} GKAT expressions (from \cite{GKAT}).}
\label{tab:GKAT}
\end{figure}
\section{Formalising TSCF with MAP}
\label{sec:TSCF}
Informally, TSCF means that all secret-dependent program traces have very similar \emph{execution time}. If we model program traces using guarded strings, then the execution time of a program trace is the sum of the execution time required by each of its actions. To quantify the execution time of actions, we use a \emph{time metric}.  %A program is secure if all its secret-dependent program traces have very similar time consumption patterns. 
%We formally capture the notion of time consumption using \emph{time metrics}. 

A \emph{time metric} associates each actions in a program trace with a time consumption. Under the MAP leakage model, an action $p$ may consume different amounts of time depending on the state of the cache when $p$ is executed: if $a$ results in several cache hits then it consumes less time than it would if $a$ resulted in several cache misses. Thus, the memory access pattern of a program trace directly impacts its execution time. We formalise this notion with the{ metric} $\cache$.

%\end{definition

\paragraph*{The $\cache$ Time Metric}
%|a|$ where $|a|$ is the number of arithmetic operations required to evaluate the arithmetic expression $a$. 
Let $\Variables(p)$ be the variables used in the action $p$, let $\omega\colon \GuardedString\times \Variables\rightarrow \Bool$ be a function where $\omega(g,v)$ states if the variable $v$ is in the cache after executing $g$ (or after executing nothing if $g\in \Atom$), let $\texttt{hit}_g(p)\triangleq\set{v\in \Variables(p)| \omega(g,v)}$ be the set of variables in $p$ that are present in the cache, and let $\texttt{miss}_g(p)\triangleq\set{v\not\in \Variables(p)| \omega(g,v)}$; we define $\cache(g)(p)$, the \emph{MAP of $p$ (with respect to $g$)}, %\colon \GuardedString\rightarrow\Sigma\rightarrow \Real^+$
by
\begin{align*}
\cache(g)(p)=%\begin{cases}
%\texttt{time}(p)+{\eta}*(|\Variables(p)-\texttt{hit}(p)|)%, & \text{if $x\in \Atom$}\\
%{}*(
    % \begin{cases}
    %     \eta\times\texttt{miss}_g(p)+\mu\times\texttt{hit}_g(p)&\text{ if $g\in \Atom$}\\
        \eta\times\texttt{miss}_g(p)+\mu\times\texttt{hit}_g(p) 
    %     &\text{ otherwise}.
    % \end{cases}
%)%, & \text{if $x\in \Atom$}\\
%\end{cases}
\end{align*}
where $\eta,\mu\in \Real^+$ are constants with $\eta$ being much greater than $\mu$, respectively modelling the time to load a variable from memory and the time to load a variable from the cache.

Given a GS $g$, the \emph{MAP of $g$}, denoted $\RC_\cache(g)$, is the accumulated MAP of its actions; formally,
\begin{align*}
\RC_\cache(g)&\triangleq 
\begin{cases}
0, &\ \text{if $g \in\Atom$;}\\
\RC_\cache(h)+\cache(h)(p),&\ \text{if $g=h \cdot p\cdot \alpha $;}\\
\end{cases}
\end{align*}
The metric $\cache$ is \emph{causal}, since the resource consumption of actions depends on the history of the execution. 

Now, consider two values for a secret $s$, say $s=0$ or $s=n$; the $\cache$ values of $x:=A[s]$ when executing with an empty cache is 
\begin{align*}
    \cache(1)(x:=A[s])&=\begin{cases}
        \eta\times\set{x,A[0],s}+\mu\times\emptyset,\quad \text{   if $s=0$}\\
        \eta\times\set{x,A[n],s}+\mu\times\emptyset,\quad \text{   if $s=n$}.
    \end{cases}
\end{align*}
The secret $s$ leaks because the MAP of the expression $x:=A[s]$ depends on the value of $s$. When loading $A[0]$ or $A[n]$ into the cache, if they touch different regions of the cache, the attacker can infer the value of $s$ by looking at which regions corresponding to $A$ are loaded. 

A preloading strategy, as the one shown in~\cite{SCEliminator}, places the contents of the structure $A$ in the cache so that $\cache(1)(x:=A[s])=\eta\times\emptyset+\mu\times\set{x,A[s],s}$ for all $s$, implying that the attacker cannot infer information from checking which regions are loaded when we use a secret as an index in $A$, since all relevant regions are loaded if $A$ fits in the cache. However, recall that the attacker can flush the contents of the cache after preloading, so preloading is not a viable strategy: as long as there is an instruction $x:=A[s]$ or $A[s]:=x$, a preloading strategy is not sound with respect to TSCFs.

%Only in very specific conditions (e.g. $A$ fits entirely in a cache line), the attacker 
% Due to its causal nature, the metric \cache is not compositional; i.e., $\RC_\cache(g\cdot h)\neq \RC_\cache(g)+\RC_\cache(h)$. This means we cannot compute the MAP of a program by just adding the independently computed MAPs parts.
% \todo[inline]{This is interesting but not very helpful.}
% %\todo[inline]{This part in red... are there relevant cases where this is not true? Answer: actually, yes, the memory footprint may probably be one of these cases (more precisely, memory footprint should also consider the interaction environment in shared caches... it can get really complicated if we want to model everything...)}
% For example, consider a swapping program %\texttt{swap(x,y)}, 
% defined by %\verb+
% \begin{verbatim}
%         swap(x,y)= z:=x; x:=y; y:=z.   
% \end{verbatim}
% The first action \verb+z:=x+ has a time consumption of $\cache(\varepsilon)(z:=x)=\eta\times\set{x,z}$. The second action has a time consumption of $\cache(z:=x)(x:=y)=\eta\times\set{y}+\mu\times\set{x}$, since $x$ is a hit and \texttt{y} is a miss. Finally, $\cache(z:=x; x:=y)(y:=z)=\mu\times\set{y,z}$. The time consumption of \texttt{swap(x,y)} under the \cache metric is $\RC_\cache(\texttt{swap(x,y)})$, which is equal to the sequence 
% \begin{align*}
%     [\eta\times\set{x,z}+\mu\times\emptyset,\eta\times\set{y}+\mu\times\set{x},\eta\times\emptyset+\mu\times\set{y,z}]
% \end{align*}
% However, $\RC_\cache(\texttt{swap(x,y)}\cdot \texttt{swap(x,y)})$ is equal to the sequence 
% \begin{align*}
%     &[\eta\times\set{x,z},\eta\times\set{y}+\mu\times\set{x},\mu\times\set{y,z}]\\
%     &+[\mu\times\set{x,z}, \mu\times\set{x,y}, \mu\times\set{y,z}],
% \end{align*}
% which illustrates the non-compositional nature of the metric \cache.



\paragraph*{Constant MAPs}     
Given a language of guarded strings $L\subseteq \GuardedString$, we say that $L$ has \emph{constant resource consumption with respect to $\cache$} if and only if $\RC_\cache(g_1)=\RC_\cache(g_2)$ for all $g_1, g_2 \in L$. 

Constant resource consumption in its current form requires \emph{all} GS in the language $L$ to have the same resource consumption. However, a program that has no secret values trivially satisfy this property, since there are no secrets that could be leaked. Thus, we need to enrich the current definition of constant resource consumption so that we only require traces that exclusively differ on secret values have the same resource consumption. As it is standard (see e.g., \cite{Molnar,usenix_ctp_verification}), we introduce a notion of \emph{public equality}. 

Let $\Variables$ be the set of public variables, and let $\semantics{\Variables}$ be the set of valuation functions which map variables to values; we extend the notion of guarded strings so that they are now generated by the grammar
\begin{align*}
\GuardedString := \left(\semantics{\Variables}\times\Atom\right) \cdot \left(\Sigma\cdot \left(\semantics{\Variables}\times\Atom\right)\right)^{*}.
\end{align*}
Now, guarded strings satisfy either the pattern $(\valuation,\alpha)$ or the pattern $(\valuation,\alpha)\cdot p\cdot g$, where $\valuation$ is a valuation of the public variables, $\alpha$ is an atom and $g$ is a guarded string.

\paragraph%\begin{definition}[
    {Public Equality}
%Whenever two valuations $\valuation_1$ and $\valuation_2$ are equal on their public variables, we denote it by $\valuation_1=_p\valuation_2$. 
Two guarded strings $g_1$ and $g_2$ are equal on their public variables, denoted $g_1=_p g_2$, if and only if their \emph{initial} valuations are equal on public variables. 
%\end{definition}

%\todo[inline]{There is an important notion in GSLangs that is called determinacy. In a few words, determinacy of a language is: if any two words in the language coincide on their first $n$ atoms, then they also coincide on their first $n$ actions (or their lack of them). This makes sense for deterministic programs because the initial state (represented by the initial atom), should determine the entire trace (hence the name, determinacy). Languages generated by GKATx satisfy determinacy.  Thus, it is not necessary to require variable assignments to be equal on the first $k$ step.}
We now define a formal notion of constant resource consumption, which we apply to the $\cache$ metric.
\begin{definition}[TSCF with MAP]
    \label{def:MemoryTSCF}
Given a language of guarded strings $L\subseteq \GuardedString$, we say that $L$ has \emph{secure constant time consumption guarantees with respect to $\cache$} if and only if, for all $g_1, g_2 \in L:$
\begin{align*}
\label{eq:WSRC}
g_1=_p g_2\Rightarrow\RC_{\cache}(g_1)=\RC_{\cache}(g_2).
\end{align*}
Equivalently, we say that $L$ satisfies TSCF.
\end{definition}
This definition naturally extends to GKATx. A GKATx $e$ satisfies TSCF if and only if $\semantics{e}$ satisfies TSCF. 
%\begin{definition}[

%\end{definition}
% Constant resource consumption lets us model attackers that can measure the resource consumption of prefixes of the execution trace. These are very powerful attackers, because they can stop the execution of the victim program, measure the current resource consumption, let the victim program execute again, stop it again and measure the new resource consumption. 


\paragraph*{Attacker model} Definition \ref{def:MemoryTSCF} characterises an attacker model where 
%The \emph{attacker models} respective to these metrics are the following: 
% For \texttt{time}, we consider an attacker that can manipulate public inputs and, at the end of the execution, measure the total number of instructions. For \cache, we consider an 
the attacker that can choose all the values of public variable before execution of every GKATx and, at the end of the execution of the GKATx, can measure the total number of hits and misses to the cache. Additionally, the attacker can execute Prime+Probe~\cite{prime-probe} or Flush+Reload~\cite{Flush+Reload} attacks, either remotely or locally. More precisely, this attacker can see which addresses are loaded in the cache but cannot see their contents, and can flush the cache at will. This attacker is similar to the attacker which exploits a Spectre V1 vulnerability to extract secrets via the cache, but our attacker does not rely on speculative execution (speculative execution is beyond the scope of this work). 

In the following sections, we provide a concrete GKAT for an enriched language of while programs. Using this GKAT, we describe existing repair rules used by other repair tools to enforce TSCF under the baseline leakage model. We discuss why these rules are not enough to enforce TSCF under the MAP leakage model, and we propose our own set of rules to fill this gap.

% We use the GKAT to characterise a couple of non-TSCF expressions, and , we discuss why they are not enough when studied under Definition~\ref{def:MemoryTSCF}, and we propose a set of rules of our own to fill this gap.%\emph{weak} 
%constant resource consumption for the two metrics. 
%We want the $\cache$ metric to depend on previous actions, since it is enough to model the behaviour of a cache memory with a capacity proportional to $k$.

%\todo[inline]{@Sudipta: Formally, TSCF is one of either weak or strong secure resource consumption when the RC function is execution time. I have not chosen which because there is a notable difference: a program with an infinite loop is always weak secure, but could fail to be strong secure.}
%{\color{red}We are going to choose weak resource consumption for the implementation, which is the standard notion in several works. We remark that it is important to make this distinction, because strategies to secure against weak attackers might not work for stronger attackers.}
%\end{minipage}

%\section{A Concrete GKAT for LLVM-IR}
 %\cache
\section{Repairing TSCF under MAP with ORIGAMI}
\label{sec:ORIGAMI}
Existing program repair solutions for TSCF \cite{Racoon,SCEliminator,MSESC} offer high-overhead, unsound guarantees, or no guarantees at all with respect to the MAP leakage model. Racoon \cite{Racoon} implements ORAM to enforce TSCF in the MAP leakage model. While secure, the use of ORAM is quite taxing in terms of performance overhead. This overhead is unfortunate, which is why we look for other software/compiler-based alternatives to repair TSCF for the MAP model. \textsc{SC-Eliminator} \cite{SCEliminator} uses must-hit analysis and data structure preloading to enforce TSCF in the MAP leakage model. However, the solution presented is unsound when we consider that the attacker can manipulate the cache. More precisely, \textsc{SC-Eliminator} implicitly assumes that the state of the cache during the must-hit analysis is maintained through execution, which is not true since the attacker can flush the cache after data structures have been preloaded, rendering the must-hit assumptions invalid. Finally, the methodology presented in \cite{MSESC} only offers security guarantees with respect to the baseline leakage model, and not with respect to the MAP leakage model.

In the following, we present an enriched language of while programs, and we study the possible causes for different MAPs. First, we define a concrete GKAT and give more precise definitions about what their time consumption means. Then, we formally present the ORIGAMI program transformation rules, and we justify their soundness formally; i.e., we prove that they enforce TSCF for the MAP leakage model. Then, given that the MAP model implies the baseline model, we include for completeness the well-known repair rules used to repair branches and loops. We proceed to discuss limitations of enforcement, and finally we show a natural extension of ORIGAMI from arrays to multidimensional fixed-size data structures.

\subsection{A Concrete GKAT}
\label{sec:SideChannels:ConcreteGKAT}
We want to keep the GKAT as abstract as possible, but we do need to define when a variable hits or misses the cache. 

We start small by considering uni-dimensional arrays. Let $\Variables$ be the set of variables names; let $\Structures$ be the set of names of arrays. We define the set of \emph{atomic expressions} $\mathscr{A}$ by the grammar
{\small{\begin{align*}      
a\in \mathscr{A}::=
\ &x\in \Variables\ |\ %i \in {\vec{\Bool}}\ |\ 
 n \in \mathbb{N}
 %\ |\ \vec{S}[a]\text{ with $\vec{S}\in \Structures, a\in \mathscr{A}$}
 \ |\ \\
 &\vec{S}[x]\text{ with $\vec{S}\in \Structures, x\in \Variables$}
 \ |\ \\
 &\vec{S}[n]\text{ with $\vec{S}\in \Structures, n\in \mathbb{N}$}.
%\ |\ a_1>>a_2%\text{ (if $a\not\in \vec{\Bool}$)} 
%  &|\ a_1+a_2\ |\ a_1-a_2\ |\ a_1\times a_2\ |\ a_1 \div a_2\ |\ a_1 \textbf{ mod } a_2\\
% & |\ a_1<<a_2 \ |\ a_1>>a_2%\ |\ a_1\land a_2 \ |\ a_1\lor a_2 \ |\ a_1\oplus a_2 .
\end{align*}}}

We define the set of \emph{tests} $T$ by the grammar 
{\small{\begin{align*}
    t\in T&::=\ \   a_1=a_2, \text{ with $a_1,a_2\in \mathscr{A}$.}
    %\textbf{false}\ |\ \textbf{true}\ |\ 
    % %& |\ a_1<a_2\ |\ a_1>a_2\ |\ 
    % & |\ \textbf{not } b\ |\ b_1 \textbf{ and } b_2\ |\ b_1 \textbf{ or } b_2\ |\ b_1 \textbf{ xor } b_2
    \end{align*}}}
% {\small{\begin{align*}
% b\in \mathscr{B}&::= \textbf{false}\ |\ \textbf{true}\ |\ 
% %& |\ a_1<a_2\ |\ a_1>a_2\ |\ 
% a_1=a_2 \text{ with $a_1,a_2\in \mathscr{A}$}\  \\
% & |\ \textbf{not } b\ |\ b_1 \textbf{ and } b_2\ |\ b_1 \textbf{ or } b_2\ |\ b_1 \textbf{ xor } b_2
% \end{align*}}}

We define the set of \emph{actions} $\Sigma$ by the grammar
{\small{\begin{align*}      
p\in \Sigma::=
%&\text{ where $a_1,a_2\in \mathscr{A}$}\ |\ \\
a_1&:=a_2& &\text{ where $a_1,a_2\in \mathscr{A}$}\ |\ \\
%a&:=x &\text{ where $x\in \Variables, k\in \Variables\cup\mathbb{N}$}\ |\ \\
a&:=\sel(b,a_1,a_2)&&\text{ where $a_1,a_2,a_3\in \mathscr{A}, b\in \mathscr{B}$} 
%  &|\ a_1+a_2\ |\ a_1-a_2\ |\ a_1\times a_2\ |\ a_1 \div a_2\ |\ a_1 \textbf{ mod } a_2\\
% & |\ a_1<<a_2 \ |\ a_1>>a_2%\ |\ a_1\land a_2 \ |\ a_1\lor a_2 \ |\ a_1\oplus a_2 .
\end{align*}}}
% We instantiate the set of actions and the set of tests by
% \begin{align*}
% \Sigma\triangleq\mathscr{E}, \quad\text{and}\quad 
% T\triangleq\mathscr{B}.%\set{b |\ a_1,a_2\in \mathscr{A}}.
% \end{align*}

The semantics of the language is standard, and we omit the details since they are largely irrelevant for the purposes of enforcing TSCF under the MAP leakage model. More precisely, we care when and where we load elements in the cache, independently of the functional semantics of an action. Nevertheless, we assume that the compiler discards expressions that are not well-formed (e.g. $1:=A[x]$). We consider nested expressions to be using syntactic sugar; e.g. $A[x]:=B[C[x]]$ is equivalent to
\begin{align*}
    x_1:=C[x]\cdot x_2:=B[x_1]\cdot A[x]:=x_2.
\end{align*}


%\subsection{\texttt{cache} Values }

% The essence of GKAT axioms  can replace expressions as long as they are semantically equivalent. For example, the sequencing axiom~\cite{GKAT} states that $e\cdot 1 \equiv e$. 
% \begin{align*}
%     %\texttt{cache}(
%         A[x]:=B[C[x]]\equiv %\texttt{cache}(
%             x_1:=C[x]\cdot x_2:=B[x_1]\cdot A[x]=x_2
% \end{align*}

%Given a variable name $\vec{x}\in \Variables$ and an  arithmetic expression $a \in \mathscr{A}$, t
% We use the expression $\vec{n}[a]$ to denote standard array notation. The exact semantics of the language are not critical for the purposes of enforcing TSCF under the MAP model, which is why we leave them abstract. More precisely, we only care when and when we load elements in the cache, and not about the functional semantics of an action. 
% Note that $\vec{n}[0],\vec{n}[1],$ etc. can be considered independent variables, since all operations are defined on elements of ${\vec{\Bool}}$.


% We define the set of \emph{select expressions} $\mathscr{S}$ by the grammar
% {\small{\begin{align*}
% s\in \mathscr{S} ::= \textbf{sel}(b,e_1,e_2)%\textbf{skip} \ |\ 
% \end{align*}}}
% with $e_1,e_2 \in  \mathscr{A}\cup\mathscr{B}\cup\mathscr{S}$ (they both have the same type).

% %ogether, $\mathscr{A}$, $\mathscr{B}$ and $\mathscr{S}$ form t
% The \emph{set of expressions} is $\mathscr{E}\triangleq\mathscr{A}\cup\mathscr{B}\cup\mathscr{S}$. %Now that we have expressions, we can
% Finally, we instantiate the set of actions and the set of tests with
% \begin{align*}
% \Sigma\triangleq\set{x:=e\ |\ x\in \Variables, e\in \mathscr{E}},\text{ and } T\triangleq\mathscr{B}.%\set{b |\ a_1,a_2\in \mathscr{A}}.
% \end{align*}

% The exact semantics of the language are not critical for the purposes of enforcing TSCF, since sound transformation rules guarantee that rules preserve semantics. However, we need resource metrics to formally define TSCF for LLVM-IR.
\subsection{ORIGAMI Rules}
It is impossible to repair programs that are not TSCF with traditional GKAT rules. Since GKAT rules preserve semantics, non-TSCF expressions are never transformed into TSCF expressions via these rules. 
%when the semantics of a GKATx is a non-TSCF language of GSs, then no matter how we transform the GKATx using GKAT rules, the semantics of the resulting GKATx is the same as the original GKATx. 
Thus, for the purposes of enforcing TSCF, we assume two equivalences between actions and GKATX: the \emph{read equivalence} $(y:=A[x])\equiv\ \mathscr{O}(y:=A[x])$ and the \emph{write equivalence} $(A[x]:=y)\equiv\ \mathscr{O}(A[x]:=y)$, where 
%To justify the soundness of the following repair rules,  one that lets us prove the soundness of the ORIGAMI rule $(y:=A[x])\leadsto\ \mathscr{O}(y:=A[x])$, 
\begin{align*}
    \mathscr{O}(y:=A[x])\triangleq\  
    &acc:=\sel(x=0,A[0],acc)\cdot \\
    &acc:=\sel(x=1,A[1],acc)\cdot \\
    &\ldots\\
    &acc:=\sel(x=n,A[n],acc)\cdot \\
    &y:=acc,
\end{align*} and 
\begin{align*}
    \mathscr{O}(A[x]:=y)\triangleq\  
    &A[0]:=\sel(x=0,y,A[0])\cdot \\
    &A[1]:=\sel(x=1,y,A[1])\cdot \\
    &\ldots\\
    &A[n]:=\sel(x=n,y,A[n]).
\end{align*}
and $A$ has a fixed size of $n+1$. The \emph{read ORIGAMI rule} is transforming $(y:=A[x])$ into $\mathscr{O}(y:=A[x])$, denoted 
\begin{align}
    (y:=A[x])\leadsto\ \mathscr{O}(y:=A[x]),
\end{align}
and the \emph{write ORIGAMI rule} is 
\begin{align}
    (A[x]:=y)\leadsto\ \mathscr{O}(A[x]:=y).
\end{align}

Functionally, these read and write rules are sound: $\mathscr{O}(y:=A[x])$ implements an iteration over $A$, loading every value $A[i]$ but only storing it in $acc$ if $i=x$, and $\mathscr{O}(A[x]:=y)$ also implements an iteration, which loads $A[i]$ and stores it back at $A[i]$ if $i\neq x$, or loads $A[i]$ but stores $y$ otherwise.

We now provide an argument for why these rules are sound with respect to TSCF. We assume that $0\leq x \leq n$, which we consider a reasonable assumption since in many languages the semantics of the expression $A[x]$ where $x>n$ is undefined or triggers an exception. 
% For the read equivalence, independently of the original value of $acc$, the action $acc:=\sel(x=i,A[i],acc)$ works as an identity over $acc$, i.e., $sel(x=i,A[i],acc)=acc$, if and only if $x\neq i$; since $0\leq x \leq n$, there is one and only one index $i$ for which $sel(x=i,A[i],acc)=A[i]$, i.e., when $x=i$. Thus, $acc$ preserves its initial value until $x=i$, when it gets updated to $A[x]$; this new value is preserved until the end of the execution, since we do not update the value of $x$ through the expression. 
To show why $\mathscr{O}(y:=A[x])$ is TSCF with respect to the MAP leakage, we do a proof by induction on the size of $A$.
\begin{theorem}
    \label{theo:read}
    Let $A$ be an array-like data structure of size $n+1$ with $n\in \mathbb{N}$, then $\semantics{\mathscr{O}(y:=A[x])}$ satisfies TSCF with MAP.
\end{theorem}
\begin{proof}
    The intuition behind the proof is the following: this ORIGAMI rule expands a single access to the data structure $A$ into a constant sequence of accesses to $A$ with fixed parameters, folding a $\sel$ instruction to accumulate the value that would be computed by the access $A[x]$; since the new sequence of accesses to $A$ is independent of $x$, the MAP of $\mathscr{O}(y:=A[x])$ does not leak $x$. 

    %Assume a cache configuration set by an adversary. 
    To prove that $\semantics{\mathscr{O}(y:=A[x])}$ satisfies TSCF, we do a proof by induction.
    For $n=0$, $\mathscr{O}(y:=A[x])$ is equal to $acc:=\sel(x=0,A[x],acc)\cdot y:=acc$. Now, since $\cache$ only depends on the actions of GSs and not on its atoms, we can take the symbolic GS $g=\alpha\cdot acc:=\sel(x=0,A[x],acc)\cdot \beta \cdot y:=acc\cdot \gamma$, where $\alpha, \beta$ and $\gamma$ are symbolic variables for atoms, and compute $\RC_{\cache}(g)$. Assume that $h=\alpha\cdot acc:=\sel(x=0,A[x],acc)\cdot \beta$,  the value of $\RC_{\cache}(g)$ in the base case where $n=0$ is
    \begin{align*}
        \RC_{\cache}(g)&=\RC_{\cache}(h)+\cache(h)(y:=acc) \\
        &=\cache(\alpha)(acc:=\sel(x=0,A[x],acc))\\
        &+\cache(h)(y:=acc)
    \end{align*}
which is given by the sequence
    \begin{align*}
        \RC_{\cache}(g)&=[\eta\times\set{acc,x,A[0]}+\mu\times \emptyset,\\
        &\eta\times\set{y}+\mu\times\set{acc}].
    \end{align*}
In the inductive case where $n>0$, we have
    \begin{align*}
        \RC_{\cache}(g)&=[\eta\times\set{acc,x,A[0]}+\mu\times \emptyset,\\
        &\eta\times\set{A[1]}+\mu\times \set{acc,x},\\
        &\eta\times\set{A[2]}+\mu\times \set{acc,x},\\
        &\ldots\\
        &\eta\times\set{A[n]}+\mu\times \set{acc,x},\\
        &\eta\times\set{y}+\mu\times\set{acc}]
    \end{align*}
We remark that if the attacker clears the cache before the $i$-th instruction $acc:=\sel(x=i+1,A[x],acc)$, then the MAP changes from $\eta\times\set{A[2]}+\mu\times \set{acc,x}$ to $\eta\times\set{acc,x,A[i]}+\mu\times \emptyset$, as if it were a base case; the secret $x$ does not leak because we are loading $A[i]$ and not $A[x]$, and the attacker cannot decide whether $i=x$ (the attacker can only observe whether $i$ and $x$ are loaded, but not their values). We conclude that the MAP is indistinguishable for all values of $x$, and only depends on the size of $A$ (i.e., $n+1$). 
%     
%     \begin{align*}
%         &\semantics{acc:=\sel(x=0,A[x],acc)\cdot y:=acc}\\
%         &=\semantics{acc:=\sel(x=0,A[x],acc)}\diamond\semantics{y:=acc}
%     \end{align*} 
    
\end{proof}
%A detailed proof can be found in the Appendix. 

A similar argument to Theorem~\ref{theo:read} can be made for rules of the form $\mathscr{O}(A[x]:=y)$.
\begin{theorem}
    \label{theo:write}
    Let $A$ be an array-like data structure of size $n+1$ with $n\in \mathbb{N}$, then $\semantics{\mathscr{O}(A[x]:=y)}$ satisfies TSCF with MAP.
\end{theorem}
\begin{proof}
    This ORIGAMI rule also expands the single access $A[x]$ into a constant sequence of accesses to $A$ with fixed parameters, but this time it loads from every position of $A$, and it stores a value chosen via a $\sel$; again, since this new sequence of accesses to $A$ is independent of $x$, the MAP of $\mathscr{O}(A[x]:=y)$ does not leak $x$. 

    %Assume a cache configuration set by an adversary. 
    To prove that $\semantics{\mathscr{O}(A[x]:=y)}$ satisfies TSCF, we do a proof by induction.
    For $n=0$, $\mathscr{O}(A[x]:=y)$ is equal to $A[0]:=\sel(x=0,y,A[x])$. We take symbolic GS $g=\alpha\cdot A[0]:=\sel(x=0,y,A[x])\cdot \beta$, where $\alpha$ and $\beta$ are symbolic variables for atoms, and we compute $\RC_{\cache}(g)$ in the base case where $n=0$ as follows:
    \begin{align*}
        \RC_{\cache}(g)
        &=\cache(\alpha)(A[0]:=\sel(x=0,y,A[0]))
    \end{align*}
which is given by the sequence
    \begin{align*}
        \RC_{\cache}(g)&=[\eta\times\set{x,y,A[0]}+\mu\times \set{A[0]}].
    \end{align*}
In the inductive case where $n>0$, we have
    \begin{align*}
        \RC_{\cache}(g)&=[\eta\times\set{x,y,A[0]}+\mu\times \set{A[0]},\\
        &\eta\times\set{A[1]}+\mu\times \set{x,y,A[1]},\\
        &\eta\times\set{A[2]}+\mu\times \set{x,y,A[2]},\\
        &\ldots\\
        &\eta\times\set{A[n]}+\mu\times \set{x,y,A[n]}]
    \end{align*}
If the attacker clears the cache before the $i$-th instruction $A[i]:=\sel(x=i,y,A[i])$, then the MAP changes from $\eta\times\set{A[i]}+\mu\times \set{x,y,A[i]}$ to $\eta\times\set{x,y,A[i]}+\mu\times \set{A[i]}$, as if it were a base case. We again conclude that the MAP is indistinguishable for all values of $x$, and only depends on the size of $A$
%     
%     \begin{align*}
%         &\semantics{acc:=\sel(x=0,A[x],acc)\cdot y:=acc}\\
%         &=\semantics{acc:=\sel(x=0,A[x],acc)}\diamond\semantics{y:=acc}
%     \end{align*} 
    
\end{proof}
%A detailed proof can be found in the Appendix. 
\subsection{ORIGAMI Rules as Spatial Transformations}
\todo[inline]{I'm sketching this section, needs to be heavily rewritten}
The ORIGAMI rules are functions that take a GKATX as an input and produce a GKATX as an output. We model the ORIGAMI rules as endofunctions in the set $\mathcal{E}$ of GKAT expressions for the concrete GKAT presented in Section~\ref{sec:SideChannels:ConcreteGKAT}, which is a coalgebraic specification language itself. 

The idea of ORIGAMI is to replace all programs with dangerous instructions (e.g. $x:=A[s]$ and $A[s]:=x$) with programs that have the same functionality, but do not have those dangerous instructions. 

\todo[inline]{Do we model the attacker or do we model the origami rules as spatial transformations?} Modelling the attacker is probably the easiest.

\todo[inline]{It really depends what we want to model as observation and what not.}
For coalgebraic modelling, we need to define over which aspects of the system we want to define behavioural properties. We could do something interesting: we put the state of the cache as part of the system, and we have variable assignments and cache state, and the idea is that even if the attacker does something, (modelled by a transformation of the cache) the attacker does not learn any new information. 

The problem with this is that we don't really care about what happens on the cache, what we care about is that we assume that there is in fact leakage if we index with a sensitive variable. The 



% \todo[inline]{Ok, I have the following plan:}
% \begin{itemize}
%     % \item Model the state as a pair $(p,c)$ where $p$ is the program and $c$ is the cache. 
%     % \item A program is a sequence of the concrete GKAT instructions. Each instruction has a set of variables
%     % \item The variables that appear in an action $a$ are \emph{touched} by $a$
%     % \item If a variable is touched, and it is in the cache, then we have a hit, otherwise it is a miss
%     % \item What we really want is to never index
% \end{itemize}

We define the coalgebra $(\mathscr{E}, (\gamma,\delta))$ where $(\gamma,\delta)\colon \mathscr{E}\rightarrow 1+2^V\times \mathscr{E}$ which tells us for each variable $v\in V$ whether $v$ is a miss or a hit in the cache, respectively modelled by $\gamma(v)=0$ and $\gamma(v)=1$. This is an $F$-coalgebra of the functor $F(X)=1+2^V\times X$. The final $F$-coalgebra is $(\sigma F,1_\gamma, 1_\delta)$ where $\sigma F = (1+2^V)^\mathbb{N}$ such that for $\phi\in \sigma F$, if $\phi(n)=\iota_1(0)$, then $\phi(n+k)=\iota_1(0)$ for all $n,k\in \nat$; the value $\iota_1(0)$ models termination. 






\todo[inline]{We can model the actions of the attacker tampering with the cache by extending the state, but do we wan to do that?}

ORIGAMI rules are idempotent: given a GKATX $e$, the programs $\mathscr{O}(e)$ and $(\mathscr{O}\circ\mathscr{O})(e)$
% Let $\mathcal{E}$ be the set of GKAT expressions for the concrete GKAT presented in Section~\ref{sec:SideChannels:ConcreteGKAT}.

{\color{red}


\section{Repair Rules for the Baseline Model}
Since the MAP implies the baseline model, we must repair programs also with respect to the baseline model. The baseline repair rules are well known, but we briefly mention them here for completeness. Existing repair tools, including \cite{Racoon,SCEliminator,MSESC}, are in general capable of repairing programs with respect to the baseline leakage model using the following repair rules, whose basic strategy consists of removing branches via {predication} and by unrolling loops once they have been fixed to have a constant number of iterations. 

\paragraph*{Branch predication} Given a program of the form 
% \begin{verbatim}
\begin{align*}
    \text{if $b$ then $x:=e_1$ else $x:=e_2$}
\end{align*}
% \end{verbatim}
whose corresponding GKATx is $\branch{b}{(x:=e_1)}{(x:=e_2)}$, the \emph{predication repair rule} replaces the branch by the select instruction $x:=\sel(b,e_1,e_2)$. The tools \cite{Racoon,SCEliminator,MSESC} implement this rule to remove leaks due to branching.
There are a couple of minor complications; repairing the GKATx $\branch{c\neq0}{(a:=b/c)}{1}$
% \begin{verbatim}
%     if c!=0 then a:=b/c else a:=a
% \end{verbatim}
% and
% \begin{verbatim}
%     if n<size_A then x:=A[n] else x:=x
% \end{verbatim}
could introduce runtime exceptions that were previously prevented by the conditional (e.g. division by zero or accessing an array out of bounds). However, it suffices to take a couple extra precautions before repair, as shown in \cite{Racoon} and \cite{MSESC}. 
%We remark that ORIGAMI rules remove all sensitive out-of-bounds accesses: by rep

%The predication rule linearises the branch. 

% The semantics of the expression \verb+CTSel(c,e1,e2)+ are as follows. First, the expressions \texttt{e1} and \texttt{e2} are fully evaluated; then,  it returns \texttt{e1} if \texttt{c} holds or \texttt{e2} if it does not. 

% 

\paragraph*{Loop Unrolling} Given a program of the form 
\begin{align*}
    \text{while $b$ do $e$}
\end{align*}
% \begin{verbatim}
    
% \end{verbatim}
which corresponds to the GKATx $\iteration{b}{e}$, the \emph{loop unrolling rule} replaces the program with a sequence of linearisable branches 
\begin{align*}
    \underbrace{\left(\branch{b}{e}{1}\right)\cdot\left(\branch{b}{e}{1}\right)\cdot\ldots\cdot\left(\branch{b}{e}{1}\right)}_\text{$k$ times}.
    \end{align*}
where $k$ is the maximum number of iterations that the loop could execute for all secrets. 

The major complication to apply the loop unrolling rule is determining $k$. Racoon does not protect information leaks from loop trip counts \cite{Racoon}, so they cannot repair programs with loops whose trip count is originally secret dependent. \texttt{SC-eliminator} arbitrarily chooses a value from a list of fixed sizes (e.g. 64, 128, 256) depending on static analysis, \cite{MSESC} assumes that loops are already unrolled, and \cite{Racoon} do not protect against information leaks from loop trip counts. 

\paragraph*{Limitations of Predication and Unrolling}
Neither predication nor unrolling repair programs with respect to the MAP leakage model. The loopless and branchless program $x:=A[s]$ where $s$ is a secret and $A$ is a data structure has different memory access patterns for the different values of $s$; this limitation is the main motivation for ORIGAMI.

\paragraph*{Vulnerable to OS Leakage} Since the MAP leakage model is weaker than the OS leakage model, ORIGAMI does not repair timing side-channel vulnerabilities that arise by the use of operations whose execution time varies depending on their parameters. We only repair leakage that occurs due to accesses to data structures using secret indices.

\subsection{Limitations of ORIGAMI}
\label{sec:Limitations}
There are a couple of limitations and side effects to applying the ORIGAMI rules. 

\paragraph*{Secret Pointers} While we can fold a data structure whose data are secret pointers without revealing the value of the resulting pointer or the value used to access it, once this resulting pointer is loaded, it is leaked to the attacker via the cache. Thus, ORIGAMI should not be used to repair programs which contain secret pointers that are naively accessed. %Fixed-size multi-dimensional data structures do not perform intermediate pointer %This does not occur when folding a multi-dimensional data-structures; accessing these multi-dimensional data structures can be soundly repaired without leaking information using ORIGAMI.

\paragraph*{Secret Data Structure Sizes} The MAP of $\mathscr{O}(y:=A[x])$ and of $\mathscr{O}(A[x]:=y)$ depends on the size of $A$. This adds a caveat for using ORIGAMI to enforce TSCF under MAP: the size of data structures indexed by secrets must be public, since the size of $A$ can be inferred via timing. We believe this caveat is acceptable, since several cryptographic algorithms fix the size of the data structures that hold secrets (e.g. AES has key sizes 128, 192 or 256 bits). We also believe this case is dual to trying to repair a loop by unrolling it if the loop bound depends on an unbounded secret: it is impossible to repair without compromising its functionality\cite{SCEliminator,MSESC}.

\paragraph*{Out-of-bounds Accesses} As a side-effect of applying the ORIGAMI rules, accessing a data structure $A$ of size $n$ using a sensitive index $x$ such that $x>n$ no longer results in a runtime error. The repaired version of $y:=A[x]$ simply sets $y$ to the initial value of the accumulator $acc$ if $x>n$. Similarly, a repaired version of $A[x]:=y$ where $x>n$ never stores $y$; it simply loads all $A[i]$ and writes them back. Consequently, these new behaviours leak to the attacker the information $x>n$, assuming that $n$ is public. This limitation is arguably artificial, since the original program possibly also leaks this information through a runtime error.

\paragraph*{Vulnerabilities Beyond MAP} As we previously stated, ORIGAMI does not offer security guarantees for programs that are vulnerable due to speculative execution or that are vulnerable in the OS leakage model, since these have leakage and attacker models that are beyond the scope of this work.
% and we focused on protecting the value of the secret when it is leaked via timing under the MAP leakage model, not via runtime errors.

% \todo[inline]{Probably the most important section, where we blame the parts we could not do to the general problem of enforcement. Pointers CANNOT be secrets, since their value leaks when you read them from memory. Structures whose values are pointers can be folded, but the value of the secret (i.e., the pointer) leaks once we resolve the pointer and load the pointee in memory.}

\subsection{Multidimensional ORIGAMI}
So far, we presented and illustrated ORIGAMI using array-like data structures $A$ of fixed size $n$. However, ORIGAMI can be applied to repair programs that use any iterable structure of fixed size. These data structures include, e.g., C structures and multidimensional arrays. 

For example, we can apply ORIGAMI to a fixed-size multidimensional structure $A$ of size $(n+1)\times (m+1)$ by systematically folding all of its $(n+1)\times (m+1)$ elements. More precisely, to repair $y:=A[i][j]$, we compute
\begin{align*}
    \mathscr{O}(y:=A[i][j])\triangleq\  
    &acc:=\sel(i=0\land j=0,A[0][0],acc)\cdot \\
    &acc:=\sel(i=0\land j=1,A[0][1],acc)\cdot \\
    &\ldots\\
    &acc:=\sel(i=0\land j=m,A[0][m],acc)\cdot \\
    &acc:=\sel(i=1\land j=0,A[1][0],acc)\cdot \\
    &\ldots\\
    &acc:=\sel(i=1\land j=m,A[1][m],acc)\cdot \\
    &\ldots\\
    % &acc:=\sel(i=n\land j=0,A[n][0],acc)\cdot \\
    % &\ldots\\
    &acc:=\sel(i=n\land j=m,A[n][m],acc)\cdot \\
    &y:=acc.
\end{align*}
Similarly, to repair $A[i][j]:=y$, we compute
\begin{align*}
    \mathscr{O}(A[i][j]:=y)\triangleq\  
    &A[0][0]:=\sel(i=0\land j=0,y,A[0][0])\cdot \\
    &A[0][1]:=\sel(i=0\land j=1,y,A[0][1])\cdot \\
    &\ldots\\
    &A[0][m]:=\sel(i=0\land j=m,y,A[0][m])\cdot \\
    &A[1][0]:=\sel(i=1\land j=0,y,A[1][0])\cdot \\
    &\ldots\\
    &A[1][m]:=\sel(i=1\land j=m,y,A[1][m)\cdot \\
    &\ldots\\
    % &A[n][0]:=\sel(i=n\land j=0,y,A[n][0])\cdot \\
    % &\ldots\\
    &A[n][m]:=\sel(i=n\land j=m,y,A[n][m]).
\end{align*}

For higher dimensions the ORIGAMI rules extend similarly, and they can be customised if some indices are secrets and some are not. Informally, we only need to fold those indices that hold secrets, e.g., if a read access $y:=A[i][k]$ uses a secret $i$ and a public value $k$, it suffices to fold as follows: 
\begin{align*}
    \mathscr{O}(y:=A[i][k])\triangleq\  
    &acc:=\sel(i=0,A[0][k],acc)\cdot \\
    &acc:=\sel(i=1,A[1][k],acc)\cdot \\
    &\ldots\\
    &acc:=\sel(i=n,A[n][k],acc)\cdot \\
    &y:=acc.
\end{align*}
We can use the index $k$ directly because it is public, i.e., if it leaks via the cache, the attacker does not learn any new information about the secret $i$.

\section{Evaluation}
\label{sec:Evaluation}
Three research questions motivate an empirical evaluation:
\paragraph*{RQ1} How effectively can we enforce TSCF?
\paragraph*{RQ2} How efficiently can we enforce TSCF?
\paragraph*{RQ3} How do our enforcement results vary when compiler optimizations are considered?
%We recall the research questions that motivate this work:
%
% \todo[inline]{i.e., what is the variance in the repaired executables} 
%\paragraph*{RQ2}  How efficient is our transformation to eliminate side channels?
%\

%We implement our repair rules as LLVM 12 \texttt{opt} passes. 
We divide the evaluation of ORIGAMI in two sections: 1), a simple litmus tests written in C which illustrates what to expect when repairing programs with ORIGAMI, and 2) a set of five test programs from representative libraries (also written in C): we test two cryptographic libraries (AES from basic crypto \cite{AESBasic}, DES and RC4 from OpenSSL \cite{OpenSSL}), the functions \texttt{gdk-keyname} and \texttt{gdk-keyuni} from the GDK Linux library.%, and nine different instances of the STAC suite \cite{STACBenchmarks} from six of its benchmarks.

\subsection{Implementation}
\label{sec:Tool} 
We implement the ORIGAMI repair rules presented in Section~\ref{sec:ORIGAMI} using LLVM (version 13) as optimisation passes that we can apply with the LLVM \texttt{opt} tool. With this, we can let Clang perform code optimisations to the C programs, and then we apply the ORIGAMI rules, so that they are not undone by further code optimisation. 

We rely on annotations to inform ORIGAMI which variables are secret. We provide the details of the tainting procedure in Section~\ref{sec:taint}. 

The compilation chain is as follows: using \texttt{clang} with either the \texttt{-\texttt{O0}} or the \texttt{-\texttt{O3}} optimisation flag, we create an intermediate representation in LLVM-IR. This IR representation has been already optimised by \texttt{clang}, so we can apply the ORIGAMI rules without fearing them being optimised out. Finally, to obtain an executable, we use \texttt{llc} with disabled optimizations (i.e., using the \texttt{-\texttt{O0}} flag) to compile and link the repaired LLVM-IR into the repaired executable. This is arguably the weakest link in our compilation chain, but our experiments evidence that ORIGAMI rules are preserved by this final compilation step.
%\todo[inline]{Can remove next parag if spectre example is clear after repair and division by zero is also clear.}
%Division by zero and reading from non-allocated memory are side effects that occur when predication of statements is used. Because we did not want to change the implementation of LLVM-IR instructions directly, we countered those side effects by adding additional checks. In the case of division by zero, we first check if the denominator is zero, and if it is, we replace it by one. In the case of accessing an element outside of an array, we rely on the \texttt{inbounds} parameter of the \texttt{GetElementPtr} instruction, which returns a \texttt{poison} value (i.e., an error value) if the indices lie outside of the array, if it is a poison value, we replace the invalid pointer for one that points to the first position of the array. Thus, our tool cannot repair cases where sensitive arrays are empty. 

\subsection{Experimental Setup}
To evaluate the ORIGAMI repair rules, we use Gem5 \cite{gem5}. Gem5 is a highly-configurable simulator which encompasses system-level architecture and processor microarchitecture. The setup for the simulated environment in Gem5 consists of a single simple timing CPU with a 1Gz processor, possessing only a level one 2-set associative cache with a cache-line size of 64 bytes, the data cache has size 8kB and the instruction cache has size 16kB and is 2-set associative.

Gem5 provides statistical data of the execution, including the number of CPU cycles, the number of committed instructions, and the number of hits and misses on both data and instruction caches (among other metrics). We measure the variance of these metrics to evaluate RQ1. We use the growth factor $m_{repaired}/m_{original}$ of each metric $m$ to evaluate RQ2. Finally, to address RQ3, we repair executables compiled with \texttt{O0} and with \texttt{O3} to see if there are any significant differences.

 %We use a tournament type branch predictor for speculative execution.
%  We remark that we also ran our experiments without branch speculation, but it did not show any significant changes to the metrics presented with branch speculation enabled, and thus we decided to omit them.

For each benchmark, we generate a set of five random secret inputs. %four random secret inputs, and we craft a fixed, worst case secret input. 
We use these five instances of the secret to obtain the following metrics: the number of CPU cycles, the number of committed instructions, and the number of hits and misses on both data and instruction caches. An implementation is \emph{vulnerable with respect to the baseline leakage model} if it shows variance in the numbers of committed instructions. An implementation is \emph{vulnerable with respect to the MAP leakage model} if it shows variance in the numbers of cache hits or misses.  Finally, an implementation is \emph{vulnerable with respect to the OS leakage model} if it shows variance in the numbers of CPU cycles. 

%\todo[inline]{Technically, you would need to know which sections of the cache are loaded to prove that you are always loading the same.}
We say that the ORIGAMI rules are effective at repairing programs with respect to the MAP leakage model if for all original programs that are vulnerable with respect to MAP, they are no longer vulnerable with respect to MAP after repair. We remark that ORIGAMI does not repair with respect to the OS leakage model, so we do not expect the variance of CPU cycles in repaired programs to be zero.

\subsection{About taint analysis} 
\label{sec:taint}
% We implement basic taint propagation at the IR level to identify which portions of the program need to be repaired; i.e., which variables depend on a secret either via data dependence or control dependence. That way, we can identify which data structures are being accessed using secret indices.
We follow the philosophy presented by the authors of ~\cite{WhatYouCisWhatYouGet}, where the programmer collaborates with the compiler to enforce TSCF. Thus, the programmer must provide the information that the compiler cannot derive on its own. Unfortunately, we did not find any reliable tool to perform taint propagation and analysis from the source language C to LLVM-IR, which is why we implement one. %Due to the complexity of performing adequate taint analysis, we insist that the tool used in this evaluation issimply a prototype and not a fully fledged repair tool.

The programmer needs to do the following annotations to the source code: 1) annotate secret variables at their declaration, 2) annotate sensitive loops by giving them a maximum \emph{constant integer} loop bound, and 3) annotate functions that manipulate secrets such that the compiler does not inline. For the latter, the programmer uses the annotation
\begin{quote}
    \texttt{\_\_attribute\_\_((noinline))}. 
\end{quote}

To implement taint propagation from C to LLVM-IR, we use the annotation 
\begin{quote}
    \texttt{\_\_attribute\_\_(annotate("secret"))} 
\end{quote}
to mark variables that contain secrets. We refer to these annotated variables as \emph{taint sources}. At the IR level, we identify the taint sources, and we propagate the taint using both data and control flow dependencies, which we obtain from analysis passes offered directly by LLVM (which we assume are reliable). Our taint analysis is not inter-procedural, so programmers must annotate variables and parameters in each function.

Because ORIGAMI not only applies the ORIGAMI rules but also branch linearisation and loop unrolling, whenever ORIGAMI finds a loop whose bound depends on a secret, ORIGAMI rejects the program and asks the programmer to provide the annotation 
\begin{quote}
    \texttt{\_\_attribute\_\_(annotate("bound=N"))}    
\end{quote}
 on the respective taint source, where $N$ is the maximum number of times the loop could iterate given any secret. 

 %This is not ideal because it increases the number of points of failure (i.e., the number of places where the programmer needs to annotate variables).  %, but we aimed to over-approximate tainting: in the worst case where we over-taint, public variables are marked as secret by the programmer, and the program is repaired unnecessarily.

%is more control over the granularity of secrets
%
%we can repair functions individually and not entire programs, yielding a compositional solution. 

%\todo[inline]{Check that this corresponds to the final metrics.}

\subsection{Litmus Test}
We consider a small program to show how ORIGAMI repairs programs with iterated array reads and writes. In the following, let $A[10][10]$ be a two-dimensional data structure of size $10\times10$ with randomly generated integer values; now, consider the program
{
\begin{verbatim}
    int secret1 = N % 10, secret2 = M % 10;
    for (int i=0; i<secret1; i++)
        for (int j=0; j<secret2; j++)
            A[j][i]=A[i][j];
\end{verbatim}
}
where \texttt{N} and \texttt{M} are randomly generated secret integers. This program accesses $A$ to read and write different portions of it, which is what ORIGAMI aims to repair.
%This program performs copies and transposes the submatrix \texttt{A[0][0]} to \texttt{A[N][M]} onto. 

This program is vulnerable with respect to both the baseline leakage model and the MAP leakage model. The loop bound of the outer loop depends on \texttt{N} and loop bound of the inner loop depends on \texttt{M}. Moreover, the first access we perform in the data structure is \texttt{A[N][M]}, which leaks both \texttt{N} and \texttt{M} according to the MAP leakage model. ORIGAMI needs to fix both the dependence of loop bounds on secrets and the MAPs for this program so that it is constant. 

We remark that Racoon does not repair programs with sensitive loop trip counts like this litmus test, so we believe that their mean 16x performance overhead only applies when repairing programs which do not have sensitive loops. Thus, we cannot conclusively compare ORIGAMI to Racoon in terms of performance overhead for this litmus test; we postpone the comparison with Racoon for the following benchmarks, which require little or no loop unrolling to be repaired. We present the results of the repair in Table~\ref{tab:Litmus}. 

\begin{table}[t]
    \caption{Simulation Results for the litmus test}
    \label{tab:Litmus}
       \centering
       \resizebox{\linewidth}{!}{
          \begin{tabular}{|l||l|l|l|l|l||l|l|l|l|l|}
       \hline
           &  \multicolumn{5}{|c||}{\texttt{O0}} &  \multicolumn{5}{|c|}{\texttt{O3}} \\ 
            {Metric} &  \multicolumn{3}{|c|}{Average} & \multicolumn{2}{|c||}{Variance}  &  \multicolumn{3}{|c|}{Average} & \multicolumn{2}{|c|}{Variance} \\ 
            & Original & Repair & Factor & Original & Repair & Original & Repair & Factor & Original & Repair \\ \hline
            \hline
Size            & 16504    & 884856  & 53.615 & 0         & 0 & 16504    & 6082680  & 368.558 & 0        & 0 \\ \hline
NumCycles       & 606469.2 & 3098976 & 5.11   & 1908459.2 & 0 & 604672.8 & 18269050 & 30.213  & 169281.2 & 0 \\ \hline
Commited Insts  & 84967    & 299321  & 3.523  & 75442.5   & 0 & 84677    & 1283111  & 15.153  & 9292     & 0 \\ \hline
Icache hits     & 114734.2 & 397059  & 3.461  & 197395.7  & 0 & 114214.2 & 1829849  & 16.021  & 13854.2  & 0 \\ \hline
Icache misses   & 931      & 14535   & 15.612 & 0         & 0 & 930.6    & 95739    & 102.879 & 0.8      & 0 \\ \hline
Icache accesses & 115665.2 & 411594  & 3.558  & 197395.7  & 0 & 115144.8 & 1925588  & 16.723  & 14011.2  & 0 \\ \hline
Dcache hits     & 114734.2 & 397059  & 3.461  & 197395.7  & 0 & 114214.2 & 1829849  & 16.021  & 13854.2  & 0 \\ \hline
Dcache misses   & 931      & 14535   & 15.612 & 0         & 0 & 930.6    & 95739    & 102.879 & 0.8      & 0 \\ \hline
Dcache accesses & 115665.2 & 411594  & 3.558  & 197395.7  & 0 & 115144.8 & 1925588  & 16.723  & 14011.2  & 0 \\ \hline
       \end{tabular}
       }
       \vspace{-0.2cm}
   \end{table}

\paragraph{Results Analysis}
From the metrics of the original programs, both in \texttt{O0} and \texttt{O3}, the program displays high variance in all metrics except the number of cache misses. After applying ORIGAMI, the variance is zero in all metrics, which confirms that the repair tool is homogenising accesses to $A$, and it is ensuring that no information leaks due to MAPs. 

We remark that the number of CPU cycles (column \texttt{NumCycles}) has a variance of zero because the repaired program does not use operations whose execution time depends on parameters. 
In the following benchmarks, we see that ORIGAMI cannot always induce a variance of zero for this metric, since ORIGAMI does not repair the cause of this leakage (i.e., use of functions whose execution time varies with their parameters).

We also remark that the overhead factor of repairing the litmus test program (column \texttt{Factor}) is extraordinary: 5.11x for \texttt{O0} and ~30.2x for \texttt{O3}. This large overhead factor is understandable; to repair this program, ORIGAMI must force both loops to have the same number of iterations independently of the secret given, so the program transforms from one with an average of 25 assignments per execution to a program which always performs 100 assignments per execution. By applying loop unrolling, ORIGAMI prevents secrets from leaking via the program counter. Then, ORIGAMI folds each of those 100 assignments over $A$, otherwise the secrets leak via the cache under the MAP leakage model. 

We remark that the excessive overhead incurred by repairing the program is mainly due to loop unrolling and not due to the ORIGAMI rules. To confirm this claim, we apply ORIGAMI to the following benchmarks, which require little to no loop unrolling to be repaired. Without loop unrolling, we can better observe the effective impact of only applying the ORIGAMI rules to repair a program.

% In the following, we present a set of five test programs from representative libraries, which should also be in the domain of programs that Racoon can repair, and we compare the ORIGAMI tool with Racoon.

% \todo[inline]{Complete this analysis.}
% On an average case, we expect the original program to perform $4\times4$ reads and writes. However, the fixed program should perform a read and a write for every position in the array, so $8\times8$ reads and writes. However, a fold must be performed for each read and one fold for each write, so the repaired program will access memory at least $(8\times8)\times(8\times8)\times(8\times8)$ times. Nevertheless, the performance penalty is only a factor of ??x in the \texttt{O0} case, and of ??x in the \texttt{O3} case.

% originally, the program executes $(16-\texttt{N})\times(16-\texttt{M})$ times 



% \subsection{Summary of Results for the Litmus Tests}
% \label{sec:LismusSummary}

% For all studied STAC benchmarks and both optimization levels, our repair rules produce a binary that shows no variance in any of the relevant metrics that concern the resource metrics \texttt{time} and $\cache$. This empirically shows that our repair rules are sound for programs that can be modelled by the GKAT presented in Section~\ref{sec:LLVM-IRGKAT}.



% Again, we remark that instructions at the microarchitectural level may take different time for different inputs (e.g. multiplication), and our repair rules assume all micro-architectural instructions take constant time (except load and store), which is why we use the amount of committed instructions instead of the number of CPU cycles when we evaluate the repair rules for the \texttt{time} metric.}
% \todo[inline]{Maybe we should move the vulnerable benchmarks up first. It makes the evaluation more interesting, I feel.}
\subsection{AES, OpenSSL and GDK}
We now describe the other examples that we used for benchmarking, and briefly explain the results of the experiments for each of them. 

\subsubsection{Advanced Encryption Standard (AES)}
AES is a specification for data encryption to establish secure communication. AES receives a plaintext message % $m$
 and as an array of size 16 bytes % $k$
  as secret inputs. These settings are enough to produce variations in the number of CPU instructions and hits/misses in the cache \cite{Chalice}. We repair the AES implementation available at \cite{AESBasic}. %for the key expansion and encryption functions. 
  We present the results of the repair in Table~\ref{tab:AESResult}. 

\paragraph*{Results analysis} since we are repairing a cryptographic encryption function, it is expected that it was developed using CTP guidelines; thus, the original version satisfies TSCF with respect to the MAP leakage model. However, the repair procedure slightly affects the program positively in terms of performance for the \texttt{O0} case, where compiler optimisations are disabled. The reason behind it is that our implementation of ORIGAMI requires the control flow graph (CFG) to match the structure of GKAT expressions, and during this transformation, the LLVM pass manager might optimise some parts of the code, and this optimisation compensates the overhead.

In the \texttt{O3} case, when compiler optimisations are enabled, we see that ORIGAMI does impact the performance of optimised code, although only by a factor of $3.2\%$ if we consider the number of CPU committed instructions to quantify time (following the baseline leakage model). At the least, ORIGAMI does not add significant overhead to MAP TSCF compliant code.

\subsubsection{OpenSSL Data Encryption Standard (DES)}
DES is a symmetric-key algorithm for data encryption which receives as secret inputs a plaintext % $m$
 and an array of 8 bytes% $k$
. %Similarly to AES, we generate 5 random secret $k$ inputs, but leave the secret plaintext input $m$ fixed. 
We repair the OpenSSL implementation of DES available at \cite{OpenSSL} and present the results of the repair in Table~\ref{tab:DESResult}. %We repair the functions \texttt{DES\_set_key_unchecked} and \texttt{DES\_set_key_unchecked}

\paragraph*{Results analysis} unlike the AES \texttt{O0} case, the modifications introduced by ORIGAMI are not compensated by the slight optimisations used when transforming the CFG. We see a performance impact of $37\%$ in the \texttt{O0} case and an impact of $1.7\%$ in the \texttt{O3} case. This benchmark illustrates the benefits of having the compiler be the one in charge of enforcing TSCF, and how repair rules benefit from compiler optimisations. In particular, \texttt{O3} combines pointer computation operations (\texttt{GetElementPtr} instructions); e.g., to compute the pointer of $A[i][j]$, \texttt{O0} would first try to compute the pointer of $A[i]$ and then compute $A[i][j]$ relative to the pointer of $A[i][0]$, while \texttt{O3} would directly compute $A[i][j]$ relative to the pointer of $A[0][0]$. Due to implementation details, ORIGAMI in \texttt{O0} aggregates the partial pointer computations, heavily impacting performance. However, in the \texttt{O3} case, ORIGAMI need not aggregate partial computations, and the repair has a smaller performance impact.
 
\subsubsection{OpenSSL RC4}
This benchmark is a stream cipher that can be used for data encryption. Although simple, several vulnerabilities have been reported, and the use of RC4 been prohibited in standard implementations of TLS \cite{rfc7465}. RC4 receives as secret inputs an array of 10 bytes % $k$
and a plaintext% $m$
 .% Again, we leave the plaintext $m$ fixed, and we generate 5 random secrets $k$.
  We repair the OpenSSL implementation of RC4 available at \cite{OpenSSL} and present the results of the repair in Table~\ref{tab:RC4Result}. 
 
\paragraph*{Results analysis} the original program of RC4 satisfies the OS leakage model, so it needs no repair. This case illustrates how applying ORIGAMI to safe code preserves TSCF with respect to the original program, and it does not add significant performance penalties, since the only transformation the ORIGAMI tool is doing is some CFG restructuring so that it matches a GKAT. In this case, the penalty is $3.6\%$ for \texttt{O0} and $5.1\%$ for \texttt{O3}. Arguably, these type of programs do not benefit from ORIGAMI, and are better off being validated by a TSCF verifier like \cite{usenix_ctp_verification} and compiled using a certified compiler like CompCert \cite{CompCert}.

\subsubsection{GDK Library - \texttt{gdk\_unicode\_to\_keyval}}
We study the function \texttt{gdk\_unicode\_to\_keyval} of the Linux GDK library. This function converts a secret ISO10646 character to a key symbol~\cite{gdklib}. The secret input % $k$
is a single unsigned integer value. This benchmark uses a binary search that depends on a secret input, and it loads from a structure using a secret index during that binary search.
 % We generate 5 random secret $k$ inputs for this experiment.
  The results of the repair of this benchmark appear in Table~\ref{tab:gdklibResult}.

\paragraph*{Results analysis} the original program is not secure with respect to the baseline model, but even if we fix with respect to the baseline leakage model, we still have to deal with the MAP difference for each secret. ORIGAMI repairs both vulnerabilities. Interestingly, in the \texttt{O0} case, ORIGAMI shows that the number of CPU cycles is constant, while in the \texttt{O3} case, there are variations. However, this is just a coincidence, since ORIGAMI does not repair programs with respect to the OS leakage model. %We attribute this difference that in the \texttt{O0} case, only constant operations are being used, or variable operations are used but they are used in all cases. In the \texttt{O3} case, it may be that non-constant operations are called, and their parameters are different through the executions. Nevertheless, 

\subsubsection{GDK Library - \texttt{gdk\_keyval\_name}}
Lastly, we simulate the function \texttt{gdk\_keyval\_name} of the Linux GDK library. This function converts a key value into a symbolic name \cite{gdklib}. The secret input $k$ is a single unsigned integer value. 
%This function converts a secret GDK key symbol $k$ to the corresponding ISO10646 Unicode character \cite{gdklib}. 
Again, for this test case, we generate 5 random secret $k$ inputs. The results of repairing appear in Table~\ref{tab:gdklibNameResult}. 
% \todo[inline]{Discuss about how you can still use secret pointers as long as you do not load them (they are equivalent to integers in this case)}

\paragraph*{Results analysis} this benchmark behaves like \texttt{gdk\_unicode\_to\_keyval}. 
The original program uses a binary search which depends on a secret, causing it to have timing side-channel vulnerabilities. ORIGAMI unfolds the binary search, and ensures that accesses to the structure holding the value-to-symbolic name pairs is loaded adequately. 

\subsection{Evaluation Conclusion} %of Results for AES, OpenSSL and GDK}

Usually, the code used for cryptographic libraries follows CryptoCoding guidelines~\cite{CryptoCoding}. This is reflected in some original binaries satisfying TSCF for at least one optimization level. This experiment lets us observe more precisely the impact of repairing secure code. We see that the impact factor remains relatively close to one for all the metrics in the benchmarks, which is reassuring: secure code should not need repair, and if it gets repaired, the impact should be minimal. We conclude that it is then viable for programmers to follow CryptoCoding guidelines, and then use ORIGAMI as a safety net without harshly impacting performance. By taking the number of CPU Cycles as a measure of time, the overhead of ORIGAMI in the benchmarks has a geometric mean of ~1.24x when applied to programs with a fixed loop count, which is significantly better than the ~16x overhead caused by Racoon. 

ORIGAMI can also repair vulnerable programs like the litmus test, \texttt{gdk\_unicode\_to\_keyval} and \texttt{gdk\_keyval\_name}. 
However, repairing programs which contain loops whose trip counts depend on secrets can largely increase the size and execution time of programs. For example, the litmus test program uses secrets which range from 0 to 9, and loop iterate a different number of times depending on the value of the secrets. ORIGAMI enforces TSCF without sacrificing functionality by forcing the program to run the outer loop 10 times and the inner loop 10 times no matter which secrets are provided, obfuscating no-ops to preserve functionality with $sel$ instructions. If we were to have more nested loops, this effect would compound exponentially; however, this is unavoidable, otherwise the secrets would leaks via the timing channel. 

% The usefulness of ORIGAMI is more significant when it is applied to code that is not written using CryptoCoding guidelines, as shown in the litmus test. However, the impact of repairing programs is rather significant, which is why we would still encourage programmers that want to use ORIGAMI to follow CryptoCoding guidelines to reduce the repair overhead.

% \todo[inline]{Update if we add more benchmarks}.
% (1.428*1.094*1.051*1.036*1.032*0.921*1.37*1.017*1.014*1.778)-1

More than the applicability and usefulness of our tool, with this evaluation, we attest the viability of the philosophy of~\cite{WhatYouCisWhatYouGet}: compilers can be empowered to close timing-side channels automatically, and programmers need only worry about providing the right information to the compiler; independently of whether the programmer follows CryptoCoding guidelines or not.
 
% We would like to remark that the repaired version of OpenSSL DES (Table~\ref{tab:DESResult}) displays more variance in the number of CPU cycles than the original executable. We believe this happens as our repaired version may increase the variation in CPU cycles due to the increased number of instructions. Recall that we do not target to repair a low level instruction, yet such a low level instruction may show variation in timing for different inputs.
%We believe that this happens because some operations in LLVM-IR may need more CPU cycles depending on the inputs provided (e.g. shifting and multiplication), and this repair could be aggregating these operations more than in other benchmarks.
%\todo[inline]{@Sudipta: A natural question from a reviewer could be ``by why on this benchmark and not on others?'' and I would be very puzzled. I'll have to study the different micro-architectural aspects and try to find the culprit.}

% so most of the repair work is invested in preloading data structures indexed by secrets. However, there may be cases where our tool cannot infer the static size of the data structure (e.g. an array that is embedded in a structure), and the tool skips adding the preloading. This is most likely what causes the small variance in the Dcache hits and misses when repairing the binary that was compiled with no optimizations enabled \texttt{O0}. 
%{\color{blue}
%\todo[inline]{These results could be generalised for all benchmarks, and they can be part of the discussion section.}
%Although this implementation of AES is written using CryptoCoding guidelines, it illustrates how enabling compiler optimization could introduce variance in terms of cache hits and misses. It is worth remarking the following:
%\paragraph*{Variance in NumCycles after repair with optimizations disabled:} we attribute this variance to other micro-architectural aspects that are not considered by our repair rules (e.g., speculative load and stores). These, however, are not present when compiling with optimizations enabled.
%\paragraph*{Lack of sensitive branches and loops:} since this benchmark follows CryptoCoding guidelines, it avoids using branches and loops that depend on secrets. Thus, the preloading of data structures becomes the predominant repair rule. Empirically, we see that the number of cache hits and misses now has a variance of zero, but the number of memory cache accesses increases significantly. The large increase of Dcache misses is most likely due to misses during preloading.
%}

\begin{table}[t]
    \caption{Simulation Results for AES}
    \label{tab:AESResult}
       \centering
       \resizebox{\linewidth}{!}{
          \begin{tabular}{|l||l|l|l|l|l||l|l|l|l|l|}
       \hline
           &  \multicolumn{5}{|c||}{\texttt{O0}} &  \multicolumn{5}{|c|}{\texttt{O3}} \\ 
            {Metric} &  \multicolumn{3}{|c|}{Average} & \multicolumn{2}{|c||}{Variance}  &  \multicolumn{3}{|c|}{Average} & \multicolumn{2}{|c|}{Variance} \\ 
            & Original & Repair & Factor & Original & Repair & Original & Repair & Factor & Original & Repair \\ \hline
            \hline
            Size            & 26648    & 22608    & 0.848 & 0        & 0      & 22320    & 30744  & 1.377 & 0      & 0      \\ \hline
            NumCycles       & 660317.6 & 619675.2 & 0.938 & 172858.8 & 6919.2 & 617122.8 & 632704 & 1.025 & 9447.2 & 128468 \\ \hline
            Commited Insts  & 95836    & 88221    & 0.921 & 0        & 0      & 87370    & 90187  & 1.032 & 0      & 0      \\ \hline
            Icache hits     & 129047   & 118874   & 0.921 & 0        & 0      & 117596   & 122196 & 1.039 & 0      & 0      \\ \hline
            Icache misses   & 997      & 942      & 0.945 & 0        & 0      & 942      & 965    & 1.024 & 0      & 0      \\ \hline
            Icache accesses & 130044   & 119816   & 0.921 & 0        & 0      & 118538   & 123161 & 1.039 & 0      & 0      \\ \hline
            Dcache hits     & 129047   & 118874   & 0.921 & 0        & 0      & 117596   & 122196 & 1.039 & 0      & 0      \\ \hline
            Dcache misses   & 997      & 942      & 0.945 & 0        & 0      & 942      & 965    & 1.024 & 0      & 0      \\ \hline
            Dcache accesses & 130044   & 119816   & 0.921 & 0        & 0      & 118538   & 123161 & 1.039 & 0      & 0      \\ \hline
       \end{tabular}
       }
       \vspace{-0.2cm}
   \end{table}
%
%\begin{table}[t]
%  \begin{center}
%    \caption{Simulation results for AES original binaries}
%    \label{tab:AESOriginal}
%    \csvautotabular{AESOriginalResults.csv}
%   \end{center}
%\end{table}
%
%\begin{table}[t]
%  \begin{center}
%    \caption{Simulation results for AES repaired binaries}
%    \label{tab:AESRepair}
%    \csvautotabular{AESRepairedResults.csv}
%   \end{center}
%\end{table}

\begin{table}[t]
 \caption{Simulation Results for OpenSSL DES}
 \label{tab:DESResult}
    \centering
    \resizebox{\linewidth}{!}{
   	\begin{tabular}{|l||l|l|l|l|l||l|l|l|l|l|}
    \hline
        &  \multicolumn{5}{|c||}{\texttt{O0}} &  \multicolumn{5}{|c|}{\texttt{O3}} \\ 
         {Metric} &  \multicolumn{3}{|c|}{Average} & \multicolumn{2}{|c||}{Variance}  &  \multicolumn{3}{|c|}{Average} & \multicolumn{2}{|c|}{Variance} \\ 
     & Original & Repair & Factor & Original & Repair & Original & Repair & Factor & Original & Repair \\ \hline
     \hline
Size            & 37720    & 361264    & 9.578 & 0       & 0        & 28888    & 32984    & 1.142 & 0       & 0        \\ \hline
NumCycles       & 631279.6 & 1085718.8 & 1.72  & 15402.8 & 270293.2 & 610879.6 & 620995.2 & 1.017 & 10270.8 & 277777.2 \\ \hline
Commited Insts  & 88047    & 120661    & 1.37  & 0       & 0        & 85776    & 87256    & 1.017 & 0       & 0        \\ \hline
Icache hits     & 118456   & 164702    & 1.39  & 0       & 0        & 115513   & 117432   & 1.017 & 0       & 0        \\ \hline
Icache misses   & 1005     & 3539      & 3.521 & 0       & 0        & 952      & 968      & 1.017 & 0       & 0        \\ \hline
Icache accesses & 119461   & 168241    & 1.408 & 0       & 0        & 116465   & 118400   & 1.017 & 0       & 0        \\ \hline
Dcache hits     & 118456   & 164702    & 1.39  & 0       & 0        & 115513   & 117432   & 1.017 & 0       & 0        \\ \hline
Dcache misses   & 1005     & 3539      & 3.521 & 0       & 0        & 952      & 968      & 1.017 & 0       & 0        \\ \hline
Dcache accesses & 119461   & 168241    & 1.408 & 0       & 0        & 116465   & 118400   & 1.017 & 0       & 0        \\ \hline
    \end{tabular}
    }
\end{table}
%The Data Encryption Standard (DES) is a symmetric-key algorithm for data encryption. The DES algorithm receives as inputs a secret plaintext $m$ and a secret array of 8 bytes $k$. 
%
%We test the OpenSSL implementation of DES available at \cite{OpenSSL}. Similarly to AES, we generate 5 random secret $k$ inputs, but leave the  secret plaintext input $m$ fixed. The results of repairing DES binaries appear in Table~\ref{tab:DESResult}.
%
%Similarly to AES, this code follows CryptoCoding guidelines, so most of the repair work is invested in preloading data structures indexed by secrets. However, there may be cases where our tool cannot infer the static size of the data structure (e.g. an array that is embedded in a structure), and the tool skips adding the preloading. This is most likely what causes the small variance in the Dcache hits and misses when repairing the binary that was compiled with no optimizations enabled \texttt{O0}. 
%
\begin{table}[t]
 \caption{Simulation Results for OpenSSL RC4}
 \label{tab:RC4Result}
    \centering
    \resizebox{\linewidth}{!}{
   	\begin{tabular}{|l||l|l|l|l|l||l|l|l|l|l|}
    \hline
        &  \multicolumn{5}{|c||}{\texttt{O0}} &  \multicolumn{5}{|c|}{\texttt{O3}} \\ 
         {Metric} &  \multicolumn{3}{|c|}{Average} & \multicolumn{2}{|c||}{Variance}  &  \multicolumn{3}{|c|}{Average} & \multicolumn{2}{|c|}{Variance} \\ 
     & Original & Repair & Factor & Original & Repair & Original & Repair & Factor & Original & Repair \\ \hline  
    Size & 20856 & 20856 & 1 & 0 & 0 & 16760 & 20856 & 1.244 & 0 & 0 \\ \hline
        NumCycles & 649490 & 660708 & 1.017 & 0 & 0 & 616062 & 635596 & 1.032 & 0 & 0 \\ \hline
        Commited Insts & 96169 & 99648 & 1.036 & 0 & 0 & 87699 & 92180 & 1.051 & 0 & 0 \\ \hline
        Icache hits & 128847 & 133321 & 1.035 & 0 & 0 & 118333 & 124891 & 1.055 & 0 & 0 \\ \hline
        Icache misses & 960 & 966 & 1.006 & 0 & 0 & 942 & 976 & 1.036 & 0 & 0 \\ \hline
        Icache accesses & 129807 & 134287 & 1.035 & 0 & 0 & 119275 & 125867 & 1.055 & 0 & 0 \\ \hline
        Dcache hits & 128847 & 133321 & 1.035 & 0 & 0 & 118333 & 124891 & 1.055 & 0 & 0 \\ \hline
        Dcache misses & 960 & 966 & 1.006 & 0 & 0 & 942 & 976 & 1.036 & 0 & 0 \\ \hline
        Dcache accesses & 129807 & 134287 & 1.035 & 0 & 0 & 119275 & 125867 & 1.055 & 0 & 0 \\ \hline
    \end{tabular}
    }
\end{table}



\begin{table}[t]
 \caption{Simulation Results for \texttt{gdk\_unicode\_to\_keyval}}
 \label{tab:gdklibResult}
    \centering
    \resizebox{\linewidth}{!}{
   	\begin{tabular}{|l||l|l|l|l|l||l|l|l|l|l|}
    \hline
        &  \multicolumn{5}{|c||}{\texttt{O0}} &  \multicolumn{5}{|c|}{\texttt{O3}} \\ 
         {Metric} &  \multicolumn{3}{|c|}{Average} & \multicolumn{2}{|c||}{Variance}  &  \multicolumn{3}{|c|}{Average} & \multicolumn{2}{|c|}{Variance} \\
         & Original & Repair & Factor & Original & Repair & Original & Repair & Factor & Original & Repair \\ \hline
         \hline
         Size            & 22520    & 47096  & 2.091 & 0       & 0 & 20520    & 192552    & 9.384 & 0        & 0        \\ \hline
         NumCycles       & 594394.4 & 672250 & 1.131 & 24780.8 & 0 & 594569.6 & 1044676.4 & 1.757 & 229836.8 & 126774.8 \\ \hline
         Commited Insts  & 83921    & 91832  & 1.094 & 125     & 0 & 83820.4  & 119655    & 1.428 & 3678.8   & 0        \\ \hline
         Icache hits     & 113138   & 122786 & 1.085 & 245     & 0 & 112984.4 & 162846    & 1.441 & 5671.8   & 0        \\ \hline
         Icache misses   & 905      & 1318   & 1.456 & 0       & 0 & 901      & 3612      & 4.009 & 0        & 0        \\ \hline
         Icache accesses & 114043   & 124104 & 1.088 & 245     & 0 & 113885.4 & 166458    & 1.462 & 5671.8   & 0        \\ \hline
         Dcache hits     & 113138   & 122786 & 1.085 & 245     & 0 & 112984.4 & 162846    & 1.441 & 5671.8   & 0        \\ \hline
         Dcache misses   & 905      & 1318   & 1.456 & 0       & 0 & 901      & 3612      & 4.009 & 0        & 0        \\ \hline
         Dcache accesses & 114043   & 124104 & 1.088 & 245     & 0 & 113885.4 & 166458    & 1.462 & 5671.8   & 0        \\ \hline\hline
    \end{tabular}
    }
\end{table}

\begin{table}[t]
 \caption{Simulation Results for \texttt{gdk\_keyval\_name}}
 \label{tab:gdklibNameResult}
    \centering
    \resizebox{\linewidth}{!}{
   	\begin{tabular}{|l||l|l|l|l|l||l|l|l|l|l|}
    \hline
        &  \multicolumn{5}{|c||}{\texttt{O0}} &  \multicolumn{5}{|c|}{\texttt{O3}} \\ 
         {Metric} &  \multicolumn{3}{|c|}{Average} & \multicolumn{2}{|c||}{Variance}  &  \multicolumn{3}{|c|}{Average} & \multicolumn{2}{|c|}{Variance} \\ 
         & Original & Repair & Factor & Original & Repair & Original & Repair & Factor & Original & Repair \\ \hline
         \hline
Size            & 49456     & 53552  & 1.083 & 0        & 0 & 49080     & 368568     & 7.51  & 0       & 0          \\ \hline
NumCycles       & 595192.48 & 606580 & 1.019 & 16235.09 & 0 & 593816.56 & 1429166.72 & 2.407 & 1635.51 & 1023769.29 \\ \hline
Commited Insts  & 83977.32  & 85130  & 1.014 & 51.14    & 0 & 83848.12  & 149118     & 1.778 & 11.86   & 0          \\ \hline
Icache hits     & 113206.12 & 114574 & 1.012 & 94.61    & 0 & 113021.44 & 205123     & 1.815 & 17.17   & 0          \\ \hline
Icache misses   & 901       & 970    & 1.077 & 0        & 0 & 899       & 5913       & 6.577 & 0       & 0          \\ \hline
Icache accesses & 114107.12 & 115544 & 1.013 & 94.61    & 0 & 113920.44 & 211036     & 1.852 & 17.17   & 0          \\ \hline
Dcache hits     & 113206.12 & 114574 & 1.012 & 94.61    & 0 & 113021.44 & 205123     & 1.815 & 17.17   & 0          \\ \hline
Dcache misses   & 901       & 970    & 1.077 & 0        & 0 & 899       & 5913       & 6.577 & 0       & 0          \\ \hline
Dcache accesses & 114107.12 & 115544 & 1.013 & 94.61    & 0 & 113920.44 & 211036     & 1.852 & 17.17   & 0          \\ \hline
    \end{tabular}
    }
\end{table}

% \section{Discussion}
% \label{sec:Limitations}
% In this section, we briefly discuss the limitations and possible extensions of our approach.
% %The crypto benchmarks are written following CryptoCoding guidelines, and their results illustrate how enabling compiler optimization could introduce variance in terms of cache hits and misses. Thus, it is worth remarking the following:
% %\paragraph*{Variance in NumCycles after repair with optimizations disabled:} we attribute this variance to other micro-architectural aspects that are not considered by our repair rules (e.g., speculative load and stores). These, however, are not present when compiling with optimizations enabled.
% %\paragraph*{Lack of sensitive branches and loops:} when a benchmark follows CryptoCoding guidelines, it avoids using branches and loops that depend on secrets. Thus, the preloading of data structures becomes the predominant repair rule. Empirically, we see that the number of cache hits and misses now has a variance of zero, but the number of memory cache accesses increases significantly. The large increase of Dcache misses is most likely due to misses during preloading.
\section{Related Work}
\label{sec:RelatedWork}
%\todo[inline]{CSF papers on side channels, would be good to cite}

% \paragraph*{Detection and repair of timing side-channels}
The existing solutions for closing timing side-channels proposed by researchers in both computer security and programming languages can be classified into two complementary main categories: \emph{verification and testing} solutions and \emph{enforcement} solutions. {Verification and testing} solutions address the problem of checking whether a system has any TSC vulnerabilities. {Enforcement} solutions ensure that the resulting systems have no TSC vulnerabilities. %Additionally, verification and enforcement techniques work at different levels; some work at \emph{source/language level}, others at \emph{intermediate/compiler level} and others at \emph{binary/assembly level}.

\paragraph*{Verification Solutions}
Verification solutions help us determine whether a system has timing side-channel vulnerabilities, but they do not offer automated repair solutions. There are verification solutions for source (e.g., ABPV13~\cite{ABPV13},Low$^*$\cite{LowStar} and \cite{7958606}), intermediate  (e.g., \cite{usenix_ctp_verification,FlowTracker}), assembly (e.g., \cite{Jasmin, Vale}) and binary (e.g., \cite{CacheAudit,KMO12}) levels. We refer the interested reader to Barbosa et al. \cite[\S IV]{timing-channel-survey}, which contains an overview of tools for side-channel resistance. 

%We consider the methodology presented in \cite{SecureCompilation} to be part of verification solutions. 
The verification procedure presented in \cite{SecureCompilation} does not determine whether programs satisfy TSCF or not; instead, they verify whether compilers preserve the cryptographic constant-time properties of programs during compilation. This is a far better guarantee that just trusting the compilers to do so. They do not address the problem of program repair.

ORIGAMI is a static enforcement repair solution, so it does not fit the verification category. However, we believe that ORIGAMI could be used in conjunction with verification solutions to check which programs effectively need repair before applying ORIGAMI unnecessarily. 

\paragraph*{Testing Solutions}
Testing solutions like ct-fuzz~\cite{ct-fuzz} and ~\cite{stvr.1718} aim to provide counterexamples that violate TSCF. By a counterexample, we mean two sensitive inputs that could cause a program to display different execution times. However, these testing solutions do not repair programs in case that they find a counterexample. Similarly to verification solutions, we believe that ORIGAMI can be used in conjunction with these testing solutions.%Testing solutions may prove that a program violates TSCF faster than verification tools, but they cannot convincingly prove that the program satisfies TSCF. Moreover, these solutions also do not repair programs in case that they find a counterexample.

\paragraph*{Enforcement solutions}
Enforcement solutions are quite diverse, and can be further subdivided into two categories: \emph{runtime enforcement}  and \emph{static enforcement}. 

Runtime enforcement solutions, like \textsc{Schmit}~\cite{Schmit}, alter the \emph{execution} of programs to dynamically reduce the leakage of existing side channels. Static enforcement solutions modify the programs before execution, and do not monitor their execution.

Static enforcement solutions can be further subdivided into two categories: \emph{synthesis} and \emph{repair}. \emph{Synthesis} solutions apply a security-by-design principle, so systems generated by this tools satisfy TSCF. Among synthesis tools we find FaCT \cite{FaCT} and CompCert \cite{CompCert}. 
FaCT is a C-like DSL that always generates constant-time LLVM bitcode, and CompCert is a certified compiler that ensures that properties satisfied by the original program are preserved in the compiled program. We appreciate the security-by-design aspect offered by FaCT: if only secure programs can be written, then TSCF is automatically enforced. However, this guarantee only holds if optimization passes are disabled, since they might introduce TSC vulnerabilities. Moreover, FaCT simply reject programs that access data structures with secrets. Unlike FaCT, we let the compiler apply any optimizations it deems necessary, and then we use the ORIGAMI rules to ensure the resulting code satisfies TSCF under the MAP leakage model. 

\emph{Program repair} solutions like Racoon\cite{Racoon}, \textsc{SC-Eliminator}~\cite{SCEliminator} and \cite{MSESC} often modify programs at the IR level (i.e., they modify the front-end of the compiler). All static enforcement solutions assume that actors further ahead in the compilation and execution processes do not undo their modifications to the program. Other repair solutions, like oo7 \cite{oo7} work at lower levels, but they protect programs against Spectre V1 attacks, and not against leakage caused directly by the program.

Our tool is mostly similar to Racoon, \textsc{SC-Eliminator} and the solution proposed in \cite{MSESC} because we do static enforcement; however, ORIGAMI offers an advantage over each of these three solutions. The tool in~\cite{MSESC} does not repair programs with respect to the MAP leakage model (they repair with respect to the baseline model), the rules proposed by \textsc{SC-Eliminator} are unsound in the context of the MAP leakage model, and while Racoon does offer some protection against the MAP leakage model, the performance overhead has a geometric mean of 16x, while ORIGAMI has a performance overhead of ~1.24x when repairing programs that do not require loop unrolling. 

\paragraph*{Shared limitations} similarly to Racoon and \textsc{SC-eliminator}, ORIGAMI does not repair recursive calls, does not offer security guarantees over side-effects (e.g. I/O system calls), and does not repair programs whose timing side-channel vulnerability stems from the usage of functions whose execution time depends on their parameters (i.e., programs vulnerable in the OS leakage model). 

\paragraph*{Spectre V1} Spectre V1 attacks~\cite{Spectre} rely on the speculative execution of a gadget of the form $x:=B[A[s]]$ to leak secrets into the cache. Under the MAP leakage model, $x:=B[A[s]]$ reveals $A[s]$ when the program \textbf{if} $s<size_A$ \textbf{then} $x:=B[A[s]]$ executes, so if some secret value $k$ is speculatively accessed via the $A[s']$, and then loaded in memory by $B[A[s']]$, the attacker can reverse engineer the value of $A[s']$, which is $k$. While Spectre attacks load secrets in the cache through speculative execution, we focus on repairing programs that load secrets directly via memory access patterns. 

ORIGAMI does not defend against Spectre type attacks. Our security guarantees only apply for non-speculative behaviours of the program. To repair programs that exhibit speculative leaks, we recommend to use a methodology that finds speculative leaks (e.g.~\cite{pitchfork}) or a methodology that prevents speculative leaks by cutting dataflows from sensitive sources to speculative sinks (e.g. \textsc{Blade}~\cite{Blade}).

% Our tool %works at the level of functions, and it 
% relies on programmer annotations to inform the compiler which variables could be secret. In this sense, while the programmer no longer needs to follow CTP guidelines, they have to carefully annotate variables, as otherwise the program will not be properly repaired.
% % \paragraph*{\textsc{SC-Eliminator}, Racoon, MSESC and ORIGAMI}
% The advantages of ORIGAMI with respect to these tools are less overhead and sound enforcement of TSCF. 


% \text

% % Both Racoon and SC-Eliminator do not protect against , since they do not completely eliminate sensitive loops; our tool unfolds those sensitive loops and removes any sensitive branches, turning the sensitive parts of the program that contain loops into straight-line code. We do share some limitations with these tools, though: we do not repair calls to external libraries, we do not repair recursive calls, and we do not repair side-effects in general (e.g. I/O statements). 


% {\color{red}
% %We take a closer look at the differences and similarities between our tool and both Racoon and SC-Eliminator.



% SC-Eliminator performs code standardisation to ease program repair, but their notion of standardised code is informal. We improve on this by using the GKAT formalism and with well-nested CFGs. By transforming CFGs into their well-nested form, we can repair programs that contain loops with multi-level \texttt{break} and/or \texttt{continue} instructions, and, although these programs do not present a fundamental challenge to Racoon, the latter does not repair them.

% We remark that SC-Eliminator creates programs where the bound of sensitive loops is independent of sensitive variables, but these loops have an arbitrary bound, proportional to the number of bits of sensitive variables. This may compromise the functionality of the repaired program, since a looped statement may need be executed more than this arbitrary bound. %(several of the STAC benchmarks display this behaviour). 
% We empower the programmer via annotations so they report the number of times the loop executes in a worse-case scenario. This responsibility could be removed from the programmer if the compiler can infer this bound automatically. \vspace{-0.1cm}
% \todo[inline]{One way to solve it is by \emph{bit-slicing} the data structure A. How does bit-slicing relate to folding??}

% MSESC as shown in Figure~\ref{fig:wrongLeakageModel}, 
% \begin{figure}[h]
%     \centering
%     \includegraphics[width=\linewidth]{figs/LuigiFail.png}
%     \caption{Taken from \cite{MSESC}. The transformed program \texttt{foo} (below) illustrates the repair rule proposed in \cite{MSESC}. This rule is sound if \texttt{i} is not a secret. However, if \texttt{i} is a secret, then the value of \texttt{i} leaks if \texttt{i<n} under the leakage model that accounts for memory access patterns, which states that the value of indices used to access data structures are leaked via the cache.} %Spectre V1 attackers use this gadget to recover secrets that were leaked into the cache via speculative execution.}%
%     %\Description{Reason why .}
%     \label{fig:wrongLeakageModel}
% \end{figure}

 %Nevertheless, the programmer can safely over-annotate functions
%Poorly annotated programs are not repaired correctly. 
%We can support public outputs, but the responsibility rests with the programmer: our function-level taint analysis. We prevent the inlining of functions, so calls to function without annotations are considered to work over public inputs and need not be repaired.s
%, and if it does nsot pass the check, then the compiler can enforce TSCF by applying the repair rules.

% Finally, since our methodology uses no tool other than the compiler, we need to use general rules to enforce TSCF. These rules, while sound, are not optimal, and they could be improved by using additional tools and information. For example, if the compiler has a model of the memory (i.e., target-specific information), then it could reduce the number of preload instructions and still enforce TSCF. 
% \todo[inline]{@Sudipta. Can you please make sure that the following is technically correct?}

% in this scenario, it is technically possible to cache \texttt{k} by computing \texttt{A[size\_A]}. 
% In normal circumstances, the caching of \texttt{A[size\_A]} never occurs
% %if \texttt{s>=size\_A}, then the instruction . 
% However, during speculative execution, a misprediction of the branch executes \texttt{x:=B[A[size\_A]]}, caching \texttt{k}. 
% An attacker can then check which sections of \texttt{B} result in cache hits, and it can infer the value of \texttt{k} using a Flush+Reload sequence \cite{Flush+Reload}. Our repair rules do not protect against these speculative execution attacks, because it is still possible to speculatively execute \texttt{x:=B[A[size\_A]]} even when all $s$, $A$ and \texttt{B} are considered public values in the program \verb+if s<size_A then x:=B[A[s]]+branch. %We guarantee that as long as the value being loaded is within the original boundaries of the data structure, the

\section{Conclusion}
\label{sec:Conclusion}
Although ORIGAMI offers a sound approach to enforcing timing side-channel freedom (TSCF) with respect to the memory access pattern (MAP) leakage model, there is still a long way to go until true TSCF is achieved. Most notably, ORIGAMI does not repair programs that are vulnerable with respect to the operand sensitive leakage model (OS), but adds another layer of security by repairing programs that are safe with respect to the baseline leakage model, but unsafe with respect to MAP leakage. 

While we share the philosophy of ~\cite{WhatYouCisWhatYouGet}, and we expect compilers to take over the task of enforcement of TSCF instead of the programmer, we believe that as long as hardware does not offer strong TSCF guarantees which can be depended on and rightfully used by compilers, any software based solution for TSCF enforcement, including ORIGAMI, will remain conditional, and thus not entirely reliable. Nevertheless, while this type of hardware is available, software-based solutions can mitigate timing side-channel vulnerabilities.
} 
%!TEX root = ../main.tex
\chapter{Discussion and Conclusion}
Latent behaviour analysis 

We present a small conclusion in each of the chapters of the application module of the thesis, i.e., Chapters~\ref{ch:Classification}, ~\ref{ch:CPSRobustness} and ~\ref{ch:SideChannelRepair}. In this chapter, we discuss and provide a conclusion of LBA as a paradigm for the study of systems, and we discuss possible avenues for future work.

\section{Generality}
it seems like a pretty useful technique in general! 



\section{Predictability} 
By only using spatial transformations $m\colon X\rightarrow X$ and not dynamics transformations $b\colon F(X)\rightarrow F(X)$, we force every latent coalgebra $(X,c\circ m)$ to somehow factor through their original coalgebra $(X,c)$. While these latent coalgebras are clearly and undoubtedly related to their original coalgebra, the relations between the original behaviours and the revealed latent behaviours are, in general, hard to predict from the spatial transformations. We attribute this unpredictability to the lack of restrictions over the choice of spatial transformations. 

In geometry, we can predict the effect of some transformations on objects before we apply them, e.g., symmetries. We believe that there should be a non-trivial family of spatial transformations whose impact on the behaviour of systems is predictable before they are used (the trivial families of spatial transformations are constants and identity transformations). Our intuition points towards spatial transformations that naturally lift to dynamics transformation; more precisely, the transformations where the latent coalgebra $(X, F(m)\circ c)$ and $(X,c\circ m)$ have the same behaviours are predictable. The systems that satisfy this property are known as \emph{bialgebras} \cite{JacobsBook}. We leave formal proof of the predictability of the effects of spatial transformation in these systems as future work.

\section{A Proof Principle}
In this thesis, we studied the effects that a single spatial transformation $m\colon X\rightarrow X$ has on an $F$-coalgebra $(X,c)$; however, since $m$ is an element of the monoid of endofunctions in $X$, it can surely be non-uniquely decomposed into a finite sequence of transformations $m_1, m_2, \ldots m_n$ where $m=m_n\circ\ldots \circ m_2\circ m_1$ each with $m_i\colon X\rightarrow X$, where all $m_i$ preserve a behavioural property $Q$. If that is the case, then $(X,c\circ m)$ should satisfy the property $Q$. This proof principle follows the line of certified compilers (e.g., CompCert~\cite{CompCert}) that can transform one system into another while preserving their behavioural properties.

This proof principle seems particularly useful for a system $(X,d)$ that can be described as a latent coalgebra of a system $(X,c)$ and a spatial transformation $m$ (i.e., $d=c\circ m$), where $(X,c)$ is a system where proving the property $Q$ is relatively simple, but proving it in $(X,d)$ is daunting. 
For example, consider the system in Figure~\ref{fig:FirstLatent}, which is a latent system of the one in Figure~\ref{fig:IntroVectorSpace}% revealed by the spatial transformation in Figure~\ref{fig:SpatialDeformation}
; perhaps it is easier to prove a property of the system in Figure~\ref{fig:FirstLatent} by showing that it holds in the system in Figure~\ref{fig:IntroVectorSpace} and showing that the spatial transformation in Figure~\ref{fig:SpatialDeformation} preserves it, instead of proving such property directly. 


\section{System Repurposing}
How can we use transformations ourselves to repurpose a system that is already been defined?
% \todo[inline]{General directions to continue this line of work. There are mainly three directions at the moment}
% \section{Applications to Security Analysis}
% \todo[inline]{A wonderful assertion: no spatial transformation breaks the property behaviour of a system. This implies that an attacker needs to change the program itself, and that model is beyond state integrity.}
% \section{Applications to Program Repair}
% \todo[inline]{We can ourselves morph the state. If we find that some transformations, like the ones we used for program repair, enforce a behavioural property but respect functionality, then we can apply them!}
% \section{Applications to Program Synthesis}
% \todo[inline]{A curious problem: if you have a set of ``gadgets'' defined by an $F$-coalgebra, can you combine them to satisfy a specification?}
\section{Future work}
\todo[inline]{This should go in the last chapter!}
We would like to consider two problems related to latent behaviours: 
\begin{itemize}
\item How can we use transformations ourselves to repurpose a system that is already been defined?
\item How can an attacker use transformations to force a behaviour they want?
\end{itemize}
\todo[inline]{Note that, ultimately, both questions need a method to solve an equation for a transformation. That is, given a target behaviour and an a source coalgebra, how do you solve the problem of finding a transformation that helps you display the behaviour you want? Is it even possible?}
%\begin{figure}
    \centering
    \begin{tikzcd}[column sep=large]
        \sigma F
            \arrow[d, "\simeq","\omega"'] 
        &X
            \arrow[r, "m"]
            \arrow[rr, bend left, "\TheBehaviourOf{\cdot}_{c}"]
            %\arrow[rd, "c\circ m"]
            \arrow[l, dotted, swap,"\TheBehaviourOf{\cdot}_{c\circ m}"]
        &X 
            %\arrow[r, dotted, "\TheBehaviourOf{\cdot}_c"] 
            \arrow[d, "c"] 
        & \sigma F  
            \arrow[r, "w"]
            %\arrow[rd,"\rho"]
            %\arrow[d, "\simeq","\omega"']
        & \sigma F 
            \arrow[d, "\simeq","\omega"']
        \\
        F(\sigma F)
        &
        &F(X) 
            %\arrow[r, dotted, "F(\TheBehaviourOf{\cdot}_c)"]
            \arrow[ll, dotted, swap,"F(\TheBehaviourOf{\cdot}_{c\circ m})"] 
            \arrow[rr, bend right, "F(\TheBehaviourOf{\cdot}_{c})"]    
        &%F(\sigma F)
            %\arrow[r, "F(w)"] 
        &F(\sigma F)
    \end{tikzcd}
    \caption{This commutes if $w(\TheBehaviourOf{x}_{c})= \TheBehaviourOf{x}_{c\circ m}$.}
\end{figure}
\newpage
\begin{figure}
    \centering
    \begin{tikzcd}[column sep=large]
        \sigma F
            \arrow[dd, "\simeq","\omega"'] 
        & X
            \arrow[d, "m"]
            \arrow[r, "\TheBehaviourOf{\cdot}_{c}"]
            \arrow[l, dotted, swap,"\TheBehaviourOf{\cdot}_{c\circ m}"]
        &\sigma F
            \arrow[d,"\tau"']
        \\ 
            %\arrow[d, "c"] 
        & X   
            \arrow[d, "c"] 
        & \sigma F 
            \arrow[d, "\simeq","\omega"']
        \\
        F(\sigma F)
            %
        &F(X) 
            \arrow[r, "F(\TheBehaviourOf{\cdot}_{c})"]    
            \arrow[l, dotted, swap,"F(\TheBehaviourOf{\cdot}_{c\circ m})"] 
        &F(\sigma F)
    \end{tikzcd}
    \caption{This commutes if $\tau(\TheBehaviourOf{x}_{c})= \TheBehaviourOf{x}_{c\circ m}$.}
\end{figure}
\newpage
\begin{figure}
    \centering
    \begin{tikzcd}[column sep=large, row sep=large]
        & X
            \arrow[d, swap, "m"]
            \arrow[r, "\TheBehaviourOf{\cdot}_{c}"]
            \arrow[drr,dotted, swap, "\TheBehaviourOf{\cdot}_{c\circ m}"]
        &\sigma F
            \arrow[dr,swap,"\delta_m"']
        &
        \\ 
        \sigma F
            \arrow[d, "\simeq","\omega"'] 
            %\arrow[d, "c"] 
        & X  
            \arrow[l, dotted, swap,"\TheBehaviourOf{\cdot}_{c}"]
            \arrow[d,swap, "c"] 
        & 
        &\sigma F 
            \arrow[dd, "\simeq","\omega"']
        \\
        F(\sigma F)
            %
        &F(X) 
            \arrow[r, "F(\TheBehaviourOf{\cdot}_{c})"]    
            \arrow[l, dotted, swap,"F(\TheBehaviourOf{\cdot}_{c})"]
            \arrow[drr, dotted, swap,"F(\TheBehaviourOf{\cdot}_{c\circ m})"] 
        &F(\sigma F)
            \arrow[dr, "F(\delta_m)"]
        \\
        &
        &
        &F(\sigma F)
    \end{tikzcd}
    \caption{This commutes if $\delta_m(\TheBehaviourOf{x}_{c})= \TheBehaviourOf{x}_{c\circ m}$.}
\end{figure}
\newpage
\begin{figure}
    \centering
    \begin{tikzcd}[column sep=2cm, row sep=1cm]
        & 
        &\sigma F
            \arrow[dr,swap,"\Delta^m_b"']
        &
        \\ 
        \sigma F
            %\arrow[dd, "\simeq","\omega"'] 
            \arrow[dd, "\simeq"', "!"]
            %\arrow[d, "c"] 
        & X  
            \arrow[l, dotted, swap,"\TheBehaviourOf{\cdot}_{c}"]
            \arrow[dd,"c"] 
        & X
            \arrow[l, swap, "m"]
            \arrow[u, "\TheBehaviourOf{\cdot}_{c}"]
            \arrow[r,dotted, swap, "\TheBehaviourOf{\cdot}_{b\circ c\circ m}"]
        &\sigma F 
            %\arrow[dd, "\simeq","\omega"']
            \arrow[dd, "!"]
            \arrow[dr, "\textbf{id}"]
        \\
        &&&&\sigma(F)
        \\
        F(\sigma F)
            %
        &F(X)     
            \arrow[l, dotted, "F(\TheBehaviourOf{\cdot}_{c})"]
            \arrow[r, swap, "b"]
        &F(X)
            \arrow[d, swap, "F(\TheBehaviourOf{\cdot}_{c})"]
            \arrow[r, dotted, "F(\TheBehaviourOf{\cdot}_{b\circ c\circ m})"] 
        &
        F(\sigma F)
            %\arrow[ur, swap, "\omega^{-1}"]
            \arrow[ur, swap, "!^{-1}"]
        \\
        &
        &F(\sigma F)
            \arrow[ru, swap, "F(\Delta^m_b)"]
        &
    \end{tikzcd}
    \caption{Latent Behaviours}
\end{figure}
\newpage

Let $w$ be the transformation in $\sigma F$ defined by 
\begin{align}
    w(\TheBehaviourOf{x}_{c})= \TheBehaviourOf{x}_{c\circ m},
\end{align}
Actually, $w$ is the solution to some behavioural equation; $w$ has type $\sigma F \rightarrow \sigma F$, but $\sigma F\simeq F(\sigma F)$, so we can define a $F$-coalgebra $\rho\colon \sigma F\rightarrow F(\sigma F)$ such that $w=\omega^{-1}\circ\rho$, equivalently $\rho= \omega \circ w$.

Question: is $h$ a homomorphism from $(X, c \circ m)$ to $(\sigma F, F(w)\circ \omega )$?

We need to prove then that 
\begin{align}
    F(w)\circ \omega \circ h = F(h)\circ c\circ m
\end{align}  
Ok, so, for this to be true, we need to define what $h$ is, let us consider a couple of candidates. Assume $h=\TheBehaviourOf{\cdot}_{c}$. 

\begin{align}
    F(w)\circ \omega \circ \TheBehaviourOf{\cdot}_{c} &= F(\TheBehaviourOf{\cdot}_{c})\circ c\circ m\\
    F(w)\circ F(\TheBehaviourOf{\cdot}_{c}) \circ c &= F(\TheBehaviourOf{\cdot}_{c})\circ c\circ m\\
    F(w\circ \TheBehaviourOf{\cdot}_{c}) \circ c &= F(\TheBehaviourOf{\cdot}_{c})\circ c\circ m\\
    F(\TheBehaviourOf{x}_{c\circ m}) \circ c &= F(\TheBehaviourOf{\cdot}_{c})\circ c\circ m\\
\end{align} 
Noppers. 
\newpage
Let us just manipulate. 
\begin{align}
    F(w)\circ \omega \circ h 
\end{align}  
Assume $h=\TheBehaviourOf{\cdot}_{c}\circ m$. Then
\begin{align}
    F(w)\circ \omega \circ (\TheBehaviourOf{\cdot}_{c}\circ m) &=
    F(w)\circ (\omega \circ \TheBehaviourOf{\cdot}_{c})\circ m \\
    &=F(w)\circ F(\TheBehaviourOf{\cdot}_{c})\circ c \circ m \\
    &=F(w\circ \TheBehaviourOf{\cdot}_{c})\circ c \circ m \\
    &=F(\TheBehaviourOf{\cdot}_{c\circ m})\circ c \circ m\\
    &=\omega \circ (\TheBehaviourOf{\cdot}_{c\circ m})
\end{align} 
This looks nicer....

\newpage
Let us just manipulate. 
\begin{align}
    \omega \circ w \circ h 
\end{align}  
Assume $h=\TheBehaviourOf{\cdot}_{c}$. Then

\begin{align}
    (\omega \circ w \circ \TheBehaviourOf{\cdot}_{c})(x) &=
    (\omega \circ \TheBehaviourOf{\cdot}_{c\circ m })(x)\\
    &= F(\TheBehaviourOf{\cdot}_{c\circ m })\circ c \circ m\\
\end{align} 
\newpage
However, we know that 

\begin{align}
    F(\TheBehaviourOf{\cdot}_{c\circ m})\circ c\circ m =\omega \circ \TheBehaviourOf{\cdot}_{c\circ m}
\end{align} 
so $w$ would have to be the identity for this diagram to commute. If we instead try $h=\TheBehaviourOf{\cdot}_{c}$, then we need to prove that 
\begin{align}
    \omega \circ w \circ \TheBehaviourOf{\cdot}_{c} = F(\TheBehaviourOf{\cdot}_{c})\circ c\circ m
\end{align} 
we know that

\begin{align}
    \omega \circ \TheBehaviourOf{\cdot}_{c} = F(\TheBehaviourOf{\cdot}_{c})\circ c
\end{align} 
so we can substitute
\begin{align}
    \omega \circ w \circ \TheBehaviourOf{\cdot}_{c} = \omega \circ \TheBehaviourOf{\cdot}_{c} \circ m
\end{align} 
Now, the intuition of $w$ is that it corresponds to the change of behaviour, so we can substitute
\begin{align}
    \omega \circ \TheBehaviourOf{\cdot}_{c\circ m} = \omega \circ \TheBehaviourOf{\cdot}_{c} \circ m
\end{align} 
but this is not nice, because this corresponds to applying $m$ and then the original behaviour yields the latent behaviour. This is not true. 

Let us now consider $h=\TheBehaviourOf{\cdot}_{c}\circ m$. To prove
\begin{align}
    \omega \circ w \circ h = F(h)\circ c\circ m
\end{align}  
we need to prove 
\begin{align}
    \omega \circ w \circ \TheBehaviourOf{\cdot}_{c}\circ m = F(\TheBehaviourOf{\cdot}_{c}\circ m)\circ c\circ m
\end{align}
Now, let us first resolve $F(\TheBehaviourOf{\cdot}_{c}\circ m)$
\begin{align}
    \omega \circ w \circ \TheBehaviourOf{\cdot}_{c}\circ m = F(\TheBehaviourOf{\cdot}_{c})\circ F(m)\circ c\circ m
\end{align}
definition of $w$
\begin{align}
    \omega \circ\TheBehaviourOf{\cdot}_{c\circ m}\circ m = F(\TheBehaviourOf{\cdot}_{c})\circ F(m)\circ c\circ m
\end{align}
We substitute $ \omega \circ\TheBehaviourOf{\cdot}_{c\circ m}$ because $\TheBehaviourOf{\cdot}_{c\circ m}$ is a homomorphism
\begin{align}
    F(\TheBehaviourOf{\cdot}_{c\circ m})\circ c\circ m \circ m = F(\TheBehaviourOf{\cdot}_{c})\circ F(m)\circ c\circ m
\end{align}
This makes even less sense.
\newpage
\begin{figure}[t]
    \centering
    \begin{tikzcd}[column sep=large]
        X \arrow[r, dotted, "\TheBehaviourOf{\cdot}_c"] \arrow[d, "c", red] & \sigma F \arrow[d, "\simeq"] \\
        F(X) \arrow[r, dotted, "F(\TheBehaviourOf{\cdot}_c)"]     & F(\sigma F)
    \end{tikzcd}\\
    \centering
    \begin{tikzcd}[column sep=large]
        X \arrow[r, dotted, "\TheBehaviourOf{\cdot}^{m}_{c}=\TheBehaviourOf{\cdot}_{c\circ m}"] \arrow[d, "m", blue] & \sigma F \arrow[dd, "\simeq"] \\
        X \arrow[d, "c", red] & \  \\
        F(X) \arrow[r, dotted, "F\left(\TheBehaviourOf{\cdot}_{c\circ m}\right)"]     & F(\sigma F)
    \end{tikzcd}
\end{figure}
\begin{figure}[t]
    \centering
    \begin{tikzcd}
        &X 
            %\arrow[r, dotted, "\TheBehaviourOf{\cdot}^{m}_{c}=\TheBehaviourOf{\cdot}_{c\circ m}"] 
            \arrow[r, dotted, "\TheBehaviourOf{\cdot}_{c\circ m}"] 
            \arrow[d, "m", blue] 
        & \sigma F 
            \arrow[dd, "\simeq"] 
        \\
        \sigma F 
            \arrow[d, "\simeq"] 
            \arrow[rru, "w", bend left=45, purple] 
        & X 
            \arrow[d, "c", red] 
            \arrow[l, dotted, "\TheBehaviourOf{\cdot}_c"] 
            \arrow[ur, "?", olive]
            & 
        \\
        F(\sigma F)
        & F(X) 
            \arrow[l, dotted, "F(\TheBehaviourOf{\cdot}_c)"]
            \arrow[r, dotted, "F\left(\TheBehaviourOf{\cdot}_{c\circ m}\right)"]    
        & F(\sigma F)
    \end{tikzcd}
\end{figure}

\begin{figure}[t]
    % \begin{tikzcd}[column sep=large]
    %     X 
    %         %\arrow[r, dotted, "\TheBehaviourOf{\cdot}^{m}_{c}=\TheBehaviourOf{\cdot}_{c\circ m}"] 
    %         \arrow[r, dotted, "\TheBehaviourOf{\cdot}_{c\circ m}"] 
    %         \arrow[d, "m", blue] 
    %     &\sigma F   
    %         \arrow[d, "id", olive]
    %     \\
    %     X 
    %         \arrow[d, "c", red] 
    %         %\arrow[r, "\TheBehaviourOf{\cdot}_c"] 
    %     &\sigma F 
    %         \arrow[d, "\simeq"] 
    %         %\arrow[u, "w", olive]
    %     \\
    %     F(X) 
    %         %\arrow[r, dotted, "F(\TheBehaviourOf{\cdot}_{c})"]
    %         \arrow[r, dotted, "F\left(\TheBehaviourOf{\cdot}_{c\circ m}\right)"]    
    %     &F(\sigma F)
    %      %F(\sigma F)
    % \end{tikzcd}\\
    % \begin{tikzcd}[column sep=large]
    %     X 
    %         \arrow[rd, dotted, "\TheBehaviourOf{\cdot}_{c\circ m}"] 
    %         \arrow[r, "\TheBehaviourOf{\cdot}_{c}"] 
    %         \arrow[d, "m", blue] 
    %     &\sigma F   
    %         \arrow[d, "w", olive]
    %     \\
    %     X 
    %         \arrow[d, "c", red] 
    %         %\arrow[r, "\TheBehaviourOf{\cdot}_c"] 
    %     &\sigma F 
    %         \arrow[dd, "\simeq"] 
    %         %\arrow[u, "w", olive]
    %     \\
    %     F(X) 
    %         %\arrow[r, dotted, "F(\TheBehaviourOf{\cdot}_{c})"]
    %         \arrow[rd, dotted, swap, "F\left(\TheBehaviourOf{\cdot}_{c\circ m}\right)"]    
    %     &
    %     \\
    %     &
    %     F(\sigma F)
    %      %F(\sigma F)
    % \end{tikzcd}
    % \caption{This says $\TheBehaviourOf{\cdot}_{c\circ m}=w\circ\TheBehaviourOf{\cdot}_{c}$}
    % \begin{tikzcd}[column sep=large]
    %     X 
    %         \arrow[r, "\TheBehaviourOf{\cdot}_{c\circ m}"]  
    %         \arrow[d, "m", blue] 
    %     &\sigma F   
    %         %\arrow[d, "w", olive]
    %     \\
    %     X 
    %         \arrow[d, "c", red] 
    %         \arrow[r, dotted, "\TheBehaviourOf{\cdot}_{c}"]
    %     &\sigma F 
    %         \arrow[d, "\simeq"] 
    %         \arrow[u, "w", olive]
    %     \\
    %     F(X) 
    %         %\arrow[r, dotted, "F(\TheBehaviourOf{\cdot}_{c})"]
    %         \arrow[r, dotted, swap, "F\left(\TheBehaviourOf{\cdot}_{c}\right)"]    
    %     &F(\sigma F)
    %     % \\
    %     % &
    %     % F(\sigma F)
    % \end{tikzcd}
    % \caption{This says $\TheBehaviourOf{\cdot}_{c\circ m}=w\circ\TheBehaviourOf{\cdot}_{c}\circ m$. THIS IS INCORRECT. The thing is: that would be if $m$ is applied only once. }
    \begin{tikzcd}[column sep=large]
        X 
            %\arrow[r, "\TheBehaviourOf{\cdot}_{c\circ m}"]  
            \arrow[rr, dotted, bend left, "\TheBehaviourOf{\cdot}_{c\circ m}"] 
            \arrow[d, "m", blue] 
        &\sigma F   
            \arrow[d, "m'", olive]
            \arrow[r, dotted, "\TheBehaviourOf{\cdot}_{\omega\circ m'}"] 
        &\sigma F
            \arrow[dd, "\simeq", "\omega"']
        \\
        X 
            \arrow[d, "c", red] 
            \arrow[r, dotted, "\TheBehaviourOf{\cdot}_{c}"]
        &\sigma F 
            \arrow[d, "\simeq", "\omega"' ] 
            %\arrow[ur, "w"]
            %\arrow[u, "w", olive]
        &\\
        F(X) 
            %\arrow[r, dotted, "F(\TheBehaviourOf{\cdot}_{c})"]
            \arrow[r, dotted, swap, "F\left(\TheBehaviourOf{\cdot}_{c}\right)"]  
            \arrow[rr, dotted, swap, bend right, "F(\TheBehaviourOf{\cdot}_{c\circ m})"]   
        &F(\sigma F)
            \arrow[r, dotted, swap, "F(\TheBehaviourOf{\cdot}_{\omega\circ m'})"]
        &F(\sigma F)
        % \\
        % &
        % F(\sigma F)
    \end{tikzcd}
    %\caption{This says $k\circ \TheBehaviourOf{\cdot}_{c\circ m}=\TheBehaviourOf{\cdot}_{c}\circ m$. }
\end{figure}
An example:
\begin{figure}[t]
    \begin{tikzpicture}
        \node[state, accepting] (00) {$(0,0)$};
        \node[state, accepting, below right of=00] (01) {$(0,1)$};
        \node[state, accepting, above right  of=00] (10) {$(1,0)$};
        \node[state, accepting, below right of=10] (11) {$(1,1)$};
        \draw (00) edge[bend left, above] node{1} (11)
        (00) edge[bend left, above] node{0} (01)
        (01) edge[bend left, above] node{1} (11)
        (01) edge[loop above] node{0} (01)
        (10) edge[bend left, above] node{1} (11)
        (10) edge[bend left, above] node{0} (01)
        (11) edge[loop above] node{1} (11)
        (11) edge[bend left, above] node{0} (01)
        ;\end{tikzpicture}\\
    \begin{tikzcd}[column sep=large]
        (0,0) 
            %\arrow[r, "\TheBehaviourOf{\cdot}_{c\circ m}"]  
            \arrow[rr, mapsto, dotted, bend left, "\TheBehaviourOf{\cdot}_{c\circ m}"] 
            \arrow[d, mapsto, "{\texttt{const}(1,1)}", blue] 
        &\sigma F   
            \arrow[d, "\texttt{const}(\TheBehaviourOf{(1,1)})", olive]
            \arrow[r, dotted, "\TheBehaviourOf{\cdot}_{\omega\circ w}"] 
        &(0,1)^*
            \arrow[dd, "\simeq", "\omega"']
        \\
        (1,1) 
            \arrow[d, "c", red] 
            \arrow[r, dotted, "\TheBehaviourOf{\cdot}_{c}"]
        &\TheBehaviourOf{(1,1)}%1^*+(0,1)^*111^*
            \arrow[d, "\simeq", "\omega"' ] 
            %\arrow[u, "w", olive]
        &\\
        {(1,(\cdot,1))} 
            %\arrow[r, dotted, "F(\TheBehaviourOf{\cdot}_{c})"]
            \arrow[r, dotted, swap, "F\left(\TheBehaviourOf{\cdot}_{c}\right)"]  
            \arrow[rr, dotted, swap, bend right, "F(\TheBehaviourOf{\cdot}_{c\circ m})"]   
        &{(1,\TheBehaviourOf{(\cdot,1)}))}
            \arrow[r, dotted, swap, "F(\TheBehaviourOf{\cdot}_{\omega\circ w})"]
        &{(1,(0,1)^*))}
        % \\
        % &
        % F(\sigma F)
    \end{tikzcd}
    \caption{This says $k\circ \TheBehaviourOf{\cdot}_{c\circ m}=\TheBehaviourOf{\cdot}_{c}\circ m$. }
\end{figure}


% \begin{figure}[t]
%     \centering
%     \begin{tikzcd}
%         X 
%             \arrow[rrd, dotted, "\TheBehaviourOf{\cdot}^{m}_{c}=\TheBehaviourOf{\cdot}_{c\circ m}", bend left] 
%             \arrow[dr, "m", blue] 
%         &
%         & \\
%         &X
%         &\sigma F
%             %\arrow[dd, "\simeq"] \\
%         %X \arrow[d, "c", red] \arrow[l, dotted, "\TheBehaviourOf{\cdot}_c"] & \  \\
%         % F(\sigma F)& F(X) \arrow[l, dotted, "F(\TheBehaviourOf{\cdot}_c)"]    \arrow[r, dotted, "F\left(\TheBehaviourOf{\cdot}_{c\circ m}\right)"]     & F(\sigma F)
%     \end{tikzcd}
% \end{figure}
is $w$ coinductively defined? no. For $w$ to be defined coinductively, it would need to be an $F$-coalgebra homomorphism, between $(\sigma F,\simeq)$ and $(\sigma F,\simeq)$. However, since $(\sigma F,\simeq)$ is final in the category of $F$-coalgebras, the only $F$-homomorphism would be $id_{\sigma F}$, so $w$ can only be defined coinductively if $w=id_{\sigma F}$. It is, instead a transformation of $\sigma F$, defined by 
\begin{align}
    w(\TheBehaviourOf{m(x)}_{c})&\triangleq\TheBehaviourOf{x}_{c\circ m}\\
    \TheBehaviourOf{x}_{c\circ m}&=(w\circ\TheBehaviourOf{-}_{c}\circ m)(x)
\end{align}

\begin{figure}[t]
    \begin{tikzcd}[column sep=large]
        X
        \arrow[r, "m", blue] 
        %\arrow[dr, "c \circ m", red]  
        \arrow[dd, dotted, "\TheBehaviourOf{\cdot}_{c\circ m}"]
        &X 
            \arrow[d, "c", red] 
            \arrow[r, dotted, "\TheBehaviourOf{\cdot}_{c}"]
        &\sigma F 
            \arrow[d, "\simeq", "\omega"',red ] 
            %\arrow[ur, "w"]
            %\arrow[u, "w", olive]
        \\
        &
        F(X) 
            %\arrow[r, dotted, "F(\TheBehaviourOf{\cdot}_{c})"]
            \arrow[r, dotted, swap, "F\left(\TheBehaviourOf{\cdot}_{c}\right)"]  
            \arrow[d, dotted, swap, "F(\TheBehaviourOf{\cdot}_{c \circ m})"]  
        &F(\sigma F)
        \\
        \sigma F
            \arrow[r, "\simeq", "\omega"',red] 
        &F(\sigma F)
        &
        % &
        % F(\sigma F)
    \end{tikzcd}\end{figure}
\begin{figure}
    \centering
    \begin{tikzcd}[column sep=large]
        \sigma F
            \arrow[d, "\simeq","\omega"', red] 
        &X
            \arrow[r, "m", blue]
            %\arrow[rd, "c\circ m", red]
            \arrow[l, dotted, swap,"\TheBehaviourOf{\cdot}_{c\circ m}"]
        &X 
            \arrow[r, dotted, "\TheBehaviourOf{\cdot}_c"] 
            \arrow[d, "c", red] 
        & \sigma F 
            \arrow[d, "\simeq","\omega"', red] 
        \\
        F(\sigma F)
        &
        &F(X) 
            \arrow[r, dotted, "F(\TheBehaviourOf{\cdot}_c)"]
            \arrow[ll, dotted, swap,"F(\TheBehaviourOf{\cdot}_{c\circ m})"]     
        &F(\sigma F)
    \end{tikzcd}
    %\caption{This says $k\circ \TheBehaviourOf{\cdot}_{c\circ m}=\TheBehaviourOf{\cdot}_{c}\circ m$. }
\end{figure}

\begin{figure}
    \centering
    \begin{tikzcd}[column sep=large]
        \sigma F
            \arrow[d, "\simeq","\omega"'] 
        &X
            \arrow[r, "m"]
            %\arrow[rd, "c\circ m", red]
            \arrow[l, dotted, swap,"\TheBehaviourOf{\cdot}_{c\circ m}"]
        &X 
            \arrow[r, dotted, "\TheBehaviourOf{\cdot}_c"] 
            \arrow[d, "c"] 
        & \sigma F 
            \arrow[d, "\simeq","\omega"'] 
        \\
        F(\sigma F)
        &
        &F(X) 
            \arrow[r, dotted, "F(\TheBehaviourOf{\cdot}_c)"]
            \arrow[ll, dotted, swap,"F(\TheBehaviourOf{\cdot}_{c\circ m})"]     
        &F(\sigma F)
    \end{tikzcd}
    %\caption{This says $k\circ \TheBehaviourOf{\cdot}_{c\circ m}=\TheBehaviourOf{\cdot}_{c}\circ m$. }
\end{figure}

\begin{figure}
    \centering
    \begin{tikzcd}[column sep=large]
        \sigma F
            \arrow[d, "\simeq","\omega"'] 
        &X
            \arrow[r, "m"]
            %\arrow[rd, "c\circ m", red]
            \arrow[l, dotted, swap,"\TheBehaviourOf{\cdot}_{c\circ m}"]
        &X 
            \arrow[r, dotted, "\TheBehaviourOf{\cdot}_c"] 
            \arrow[d, "c"] 
        & \sigma F 
            \arrow[d, "\simeq","\omega"'] 
        \\
        F(\sigma F)
        &
        &F(X) 
            \arrow[r, dotted, "F(\TheBehaviourOf{\cdot}_c)"]
            \arrow[ll, dotted, swap,"F(\TheBehaviourOf{\cdot}_{c\circ m})"]     
        &F(\sigma F)
    \end{tikzcd}
    %\caption{This says $k\circ \TheBehaviourOf{\cdot}_{c\circ m}=\TheBehaviourOf{\cdot}_{c}\circ m$. }
\end{figure}
%%!TEX root = ../main.tex
% Chapter Template

%\newcommand{\branch}[3]{\ensuremath{{#2}\ {+_{#1}}\ {#3}}}
%\newcommand{\iterate}[2]{\ensuremath{{#2}^{\left(#1\right)}}}
%\newcommand{\bexp}[0]{\ensuremath{\text{BExp}}}
%\newcommand{\gexp}[0]{\ensuremath{\text{Exp}}}
%\newcommand{\usg}[0]{\ensuremath{\text{u}}}
%\newcommand{\eval}[0]{\ensuremath{\mathtt{eval}}}
%\newcommand{\sat}[0]{\ensuremath{\mathtt{sat}}}
%\newcommand{\sg}[0]{\ensuremath{\text{s}}}
%\newcommand{\set}[1]{\ensuremath{\left\{#1\right\}}}
%\newcommand{\lbl}[0]{\ensuremath{\text{lbl}}}
%\newcommand{\Real}[0]{\ensuremath{\mathbb{R}}}
%\newcommand{\Nat}[0]{\ensuremath{\mathbb{N}}}
%\newcommand{\letter}[0]{\ensuremath{\left(\text{A-Z\ |\ a-z}\right)}}
%\newcommand{\Atom}[0]{\ensuremath{\text{At}}}
%\newcommand{\GuardedString}[0]{\ensuremath{\text{GS}}}
%\newcommand{\RC}[0]{\ensuremath{\text{RC}}}
%\newcommand{\Low}[0]{\ensuremath{\text{Low}}}
%\newcommand{\Variable}[0]{\ensuremath{\mathscr{V}}}
%\newcommand{\Integer}[0]{\ensuremath{\mathbb{Z}}}
%\newcommand{\alphanumeric}[0]{\ensuremath{\left(\text{A-Z\ |\ a-z\ |\ 0-9}\right)}}
%\newcommand{\alphanumericP}[0]{\ensuremath{\left(\text{A-Z\ |\ a-z\ |\ 0-9\ |\ .\ |\ \_\ |\ \$}\right)}}
%\newcommand{\semantics}[1]{\ensuremath{\llbracket #1\rrbracket}}
%\newcommand{\valuation}[0]{\ensuremath{\Gamma}}
%\newcolumntype{L}{>{$}l<{$}} % math-mode version of "l" column type
%\newcolumntype{R}{>{$}r<{$}} % math-mode version of "r" column type
%\newcolumntype{C}{>{$}c<{$}} % math-mode version of "c" column type
%%\newcommand{\hourglass}[0]{}%{\LARGE\fontspec{Cambria}^^^^231b}
%\newcommand{\timeequiv}[0]{\equiv_{\hourglass}}

\chapter{Side-Channel Repair} % Main chapter title

\label{ChapterGKAT} % Change X to a consecutive number; for referencing this chapter elsewhere, use \ref{ChapterX}

%----------------------------------------------------------------------------------------
%	SECTION 1
%----------------------------------------------------------------------------------------

\section{General Idea}
In the following, whenever we write \emph{program}, we mean \emph{non-concurrent program}
The premises of our work are the following:
\begin{itemize}
\item Every program can be linearised using a set of \emph{predication axioms}.
\end{itemize}

\section{Related work and how do we distinguish from them}
\begin{itemize}
\item Ashay Rane: we do not use \emph{cmove} as a primitive instruction, since it \todo{prove this?}may be vulnerable to spectre attacks
\end{itemize}

\section{CTP (Cryptocoding) Axioms, and our solution}
Taken from \cite{cryptocoding}.
\begin{itemize}
\item Compare secrets in constant time.
\begin{itemize}
\item We implement a constant-time comparison function for each type at the LLVM IR. Do note that there are several comparison operations: leq, geq, neq, etc. Ideally, we have a way to implement them at the IR level, but this might not be possible; in other words, there may be the need to leave these as primitive operations that are translated differently for the different architectures. This latter approach has its advantages though: there is room for optimization and we can ensure that it, in fact, gets translated to a sequence of constant time instructions; however, it requires more muscle, i.e., we need to be super careful on how we translate them so that they are, in fact, constant time. 
\end{itemize}
\item Avoid branchings controlled by secret data
\begin{itemize}
\item The suggested solution: 
\begin{quote}
Timing leaks may be mitigated by introducing dummy operations in branches of the program in order to ensure a constant execution time. It is however more reliable to avoid branchings altogether, for example by implementing the conditional operation as a straight-line program. To select between two inputs a and b depending on a selection bit (a la CTSelect). 
\end{quote}
\item We implement predication axioms that allow us to transform control structures into arithmetic expressions using Shannon expansions.
\end{itemize}
\item Avoid table look-ups indexed by secret data
\begin{itemize}
\item The suggested solution: 
\begin{quote}
Replace table look-up with sequences of constant-time logical operations, for example by bitslicing look-ups (as used in NaCl's implementation of AES-CTR, or in Serpent. For AES, constant-time non-bitsliced implementations are also possible, but are much slower. (check blog for resources on how to do this).
\end{quote}
\item The two available options is to somehow implement a rudimentary version of ORAM or make lookups on a set generated by the secret (which should probably be similar to what bit slicing is).
\end{itemize}
\item Avoid secret-dependent loop bounds
\begin{itemize}
\item The suggested solution: 
\begin{quote}
Make sure that all loops are bounded by a constant (or at least a non-secret variable). In particular, make sure, as far as possible, that loop bounds and their potential underflow or overflow are independent of user-controlled input (you may have heard of the Heartbleed bug).
\end{quote}
\item Our solution: each secret has an associated iteration bound (i.e., loops that depend on that secret will be unwinded that number amount of times).
\end{itemize}
\item Prevent compiler interference with security-critical operations (i.e., compilers can mess up your patches)
\begin{itemize}

\item The suggested solution: 
\begin{quote}
Look at the assembly code produced and check that all instructions are there. (This will not be possible for typical application sizes, but should be considered for security-sensitive code.)

Know what optimizations your compiler can do, and carefully consider the effect of each one on security programming patterns. In particular, be careful of optimizations that can remove code or branches, and code that prevents errors which "should be impossible" if the rest of the program is correct.

When possible, consider disabling compiler optimizations that can eliminate or weaken security checks.

Note that such workarounds may not be sufficient and can still be optimized out.
\end{quote}
\item Our solution: we suggest that CTFixes happen as latter as possible in the compilation chain. We can only offer guarantees at the LLVM IR level, the rest is target specific (this is acceptable, since the repair paper works at the same level).
\end{itemize}

\item Prevent confusion between secure and insecure APIs
\item Avoid mixing security and abstraction levels of cryptographic primitives in the same API layer
\item Use unsigned bytes to represent binary data
\begin{quote}
Some languages in the C family have separate signed and unsigned integer types. For C in particular, the signedness of the type char is implementation-defined. This can lead to problematic code.
\end{quote}
\begin{itemize}
\item Their suggestion:
\begin{quote}
In languages with signed and unsigned byte types, implementations should always use the unsigned byte type to represent bytestrings in their APIs.
\end{quote}
\item our solution: LLVM IR does uses both signed and unsigned. We need an axioms to only yield unsigned byte types.
\end{itemize}
\item Clean memory of secret data
\begin{itemize}
\item Our suggestion: although this might take a lot of time, it is a good idea to reset all input secrets and their intermediate operations. 
\end{itemize}
\item Use strong randomness
\begin{itemize}
\item Our suggestion: we cannot really enforce this. We assume it. (TBH I dunno how does randomness work on the LLVM IR).
\end{itemize}
\item Always typecast shifted values\todo{read blog}.
\end{itemize}

Do note also the following:
\begin{itemize}
\item Casting operations at higher level may be dangerous: they might introduce branching at the lower level for some architectures (there is an example where a \_Bool is casted into an uint\_32 in the blog.
\end{itemize}
\section{Formalisation}
We use Guarded Kleene Algebra with Tests (GKAT) \cite{GKAT} to define our transformation rules. the GKAT syntax is as follows:
\newline
\begin{minipage}{0.5\textwidth}
\begin{align*}
\begin{tabular}{RCLL}
\multicolumn{4}{L}{b,c,d, \in \bexp::=} \\
  &    |   &  0   		&  \textbf{False}   \\
  &    |   &   1   		&   \textbf{True}   \\
  &    |   &    t\in T  		&   t  \\
  &    |   &    b \cdot c  	&  b \ \textbf{and}\ c   \\
  &    |   &    b+c  		&   b \ \textbf{or}\ c   \\
  &    |   &    \bar{b}  	& \textbf{not}\ b
\end{tabular}
\end{align*}
\end{minipage}%
\begin{minipage}{.5\textwidth}
   \begin{align*}
\begin{tabular}{RCLL}
\multicolumn{4}{L}{e,f,g, \in \gexp::=} \\
  &    |   &  p \in \Sigma  		&  \textbf{do } p \\
  &    |   &   b \in \bexp   		&   \textbf{assert } b \\
  &    |   &    e \cdot f  			&  e{;}f   \\
  &    |   &   \branch{b}{f}{g} 	&   \textbf{if }b \textbf{ then }f\textbf{ else }g   \\
  &    |   &    \iterate{b}{e} 		& \textbf{while }b\textbf{ do }e
\end{tabular}
\end{align*}
\end{minipage}
\paragraph{}
We take a layered approach to constant time programming (CTP); initially, we consider actions $a\in \Sigma$ to be abstract (i.e., only symbols), then, we give semantics to actions by mapping them to a set of instructions in LLVM IR. This layered approach allows us to exclusively focus on program structure, and how branching and iteration could introduce side channels.

\subsection{Language-Based Semantics CTP}
The language-based semantics of a GKAT expression $e$ corresponds to a language of \emph{guarded strings} \cite{GKAT}. Informally, a guarded string is an interleaved sequence of \emph{(logical) atoms} and actions, which model how actions change the state of the system. Informally, an \emph{atom} is a valuation of all tests in the set of tests $T$ (e.g., if $T=\{t_1,t_2\}$, the expressions  $\overline{t_1}\land \overline{t_2}$, $\overline{t_1}\land {t_2}$, $t_1\land \overline{t_2}$, and $t_1\land {t_2}$ are the atoms). Formally, an atom is a non-zero minimal element in the free boolean algebra in $T$ \cite{KAT}. We denote atoms by $\alpha, \beta,$ and $\gamma$, and the set of atoms by $\Atom$. The formal definition of a guarded string is as follows.

\begin{definition}[Guarded String]
Given a set of actions $\Sigma$ and a set of tests $T$, a \emph{guarded string} $g$ is an element of the set $\GuardedString := \Atom \cdot \left(\Sigma\cdot \Atom\right)^{*}$. 
\end{definition}

\todo[inline]{Here is where you explain what the semantics of GKAT expressions are in terms of guarded strings, and you say that infinite loops yield an empty language, which is super weird, right?}

\begin{definition}[Resource Metrics and Resource Consumption]
A \emph{resource metric} is a map $m\colon \GuardedString \rightarrow \Sigma \rightarrow \Real^{+}$. Given a resource metric $m$ and a guarded string $g$, the value of $m(g)(a)$ is the {resource consumption for the action $a$ after executing all the actions in $g$}. The \emph{resource consumption} of $g$, denoted $\RC(g)$, is the accumulated resource consumption of its actions; formally,
\begin{align}
\RC(g)&\triangleq 
\begin{cases}
0, &\quad \text{if $g \in\Atom$;}\\
\RC(h) + m(h)(a),&\quad \text{if $g=h \cdot a\cdot \alpha $;}\\
\end{cases}
\end{align}

{\color{red}Henceforth, we will assume that resource metrics are history-independent, i.e., resource consumption is independent of previous actions or initial conditions, meaning that resource metrics are instead maps of type $\Sigma \rightarrow \Real^{+}$.  }
\todo[inline]{Are there relevant cases where this is not true? Answer: actually, the memory footprint may probably be one of these cases (more precisely, memory footprint should also consider the interaction environment in shared caches... it can get really complicated if we want to model everything...)}

\end{definition}

\begin{definition}[Weak Constant Time]
Given a language of guarded strings $L\subseteq \GuardedString$, we say that $L$ has \emph{weak constant time} if and only if, for all $g_1, g_2 \in L$, $\RC(g_1)=\RC(g_2)$. 
\end{definition}

We also define a strong notion of constant time to study more modular attackers.
\begin{definition}[Strong Constant Time]
Given a language of guarded strings $L\subseteq \GuardedString$, we say that two guarded strings $g_1, g_2 \in L$ are \emph{strong constant time equivalent}, denoted $g_1 \timeequiv g_2$, if and only if,
\begin{align}
g_1 \timeequiv g_2 \triangleq \begin{cases}
\textbf{true}, &\quad \text{if $g_1\in \Atom$ and $g_2\in \Atom$,}\\
g_1' \timeequiv g_2' \land m(g'_1)(a_1)=m(g'_2)(a_2),&\quad \text{if $g_i=g_i' \cdot a_i\cdot \alpha_i $ for $i=1,2$; }
\end{cases}
\end{align}
We say that  $L$ has \emph{strong constant time} if and only if, for all $g_1, g_2 \in L$, $g_1\timeequiv g_2$.
\end{definition}

To obtain the notion of constant time that is used for security analysis, we need to match traces that are only equal on their public variables. Atoms themselves may match even when public values are different (e.g. if $p$ is public, the test $p\leq 128$ is satisfied by any public value below 128); thus, we need to make the state of public variables explicit. For such purposes, let $\Variable$ be the set of variables, and let $\semantics{\Variable}$ be the set of valuation functions which map variables to values; we override the definition of guarded strings so that the set of guarded strings is now generated by the grammar
\begin{align}
\GuardedString := \left(\semantics{\Variable}\times\Atom\right) \cdot \left(\Sigma\cdot \left(\semantics{\Variable}\times\Atom\right)\right)^{*}.
\end{align}
Thus, guarded strings now satisfy either the pattern $(\valuation,\alpha)$ or the pattern $(\valuation,\alpha)\cdot a\cdot h$, where $\valuation$ is a valuation of the variables, $\alpha$ is an atom and $h$ is a guarded string.

\begin{remark}
{If the domains of variables are finite, this formulation reduces to the original formulation as follows: for each variable $p$ and each of its possible values $v$, we include the proposition $p=v$ in the set of tests.}
\end{remark}
\begin{definition}[Public Equality]
Whenever two valuations $\valuation_1$ and $\valuation_2$ are equal on their public variables, we denote it by $\valuation_1=_p\valuation_2$. Two guarded strings $g_1$ and $g_2$ are equal on their public variables, denoted $g_1=_p g_2$, if and only if their \emph{initial} valuations are equal on public variables. 
\end{definition}
Depending on the attacker model, we might choose one of the following definitions to suit our needs best:
\begin{definition}[Secure Weak Constant Time]
Given a language of guarded strings $L\subseteq \GuardedString$, we say that $L$ has \emph{secure weak constant time} guarantees if and only if, for all $g_1, g_2 \in L:$
\begin{align}
g_1=_p g_2\Rightarrow\RC(g_1)=\RC(g_2).
\end{align}
\end{definition}

\begin{definition}[Secure Strong Constant Time]
Given a language of guarded strings $L\subseteq \GuardedString$, we say that $L$ has \emph{secure strong constant time} guarantees if and only if, for all $g_1, g_2 \in L:$
\begin{align}
g_1=_p g_2\Rightarrow g_1\timeequiv g_2.
\end{align}
\end{definition}

\section{The IMP Language}
We explore the simple imperative programming language with variable assignments and boolean expressions from \cite{GKAT}: IMP; the language is defined as follows:
\begin{align*}
{\small
\begin{tabular}{L RL}
\text{-- arithmetic expressions:} & a\in \mathscr{A}\!\!\!\!\!\! &::= x\in \Variable\ |\ n \in \Nat\ |\ a_1+a_2\ |\ a_1-a_2\ |\ a_1\times a_2\\
\text{-- boolean expressions:} & b\in \mathscr{B}\!\!\!\!\!\! &::= \textbf{false}\ |\ \textbf{true}\ |\ a_1<a_2\ |\ \textbf{not } b\ |\ b_1 \textbf{ and } b_2\ |\ b_1 \textbf{ or } b_2\\
\text{-- commands:} & c\in \mathscr{C}\!\!\!\!\!\! &::= \textbf{skip}\ |\ x:=a\ |\ c_1;c_2\ |\ \textbf{if }b\text{ then }c_1\text{ else }c_2\ |\ \textbf{while } b\textbf{ do }c
\end{tabular}
}
\end{align*}

As stated in \cite{GKAT}, this language can be modelled in GKAT using actions for assignments and primitive tests for comparisons as follows:
\begin{align}
\Sigma=\set{x:=a\ |\ x\in \Variable, a\in \mathscr{A}}, \quad T=\set{a_1 < a_2\ |\ a_1,a_2\in \mathscr{A}}.
\end{align}
Following \cite{GKAT}, we interpret GKAT expressions over the space of variable assignments $\semantics{\Variable}\triangleq\Variable \rightarrow\Nat$:
\begin{align*}
\eval(x:=a)&\triangleq \set{(\sigma, \sigma[x:=n])\ |\  \sigma \in \semantics{\Variable}, n=\mathscr{A}\semantics{a}_\sigma},\\
\sigma[x:=n]&\triangleq \uplambda y \ldotp 
	\begin{cases}
		n, \quad &\text{if $y=x$};\\
		\sigma(y), \quad&\text{otherwise},
	\end{cases}\\
\sat(a_1<a_2)&\triangleq \set{\sigma \in \semantics{\Variable}\ |\ \mathscr{A}\semantics{a_1}_{\sigma} < \mathscr{A}\semantics{a_2}_{\sigma}},
\end{align*}
%where $\mathscr{A}\semantics{a}_\sigma\colon \semantics{\Variable}\rightarrow \Real_\perp$ denotes arithmetic evaluation of the expression $a$ in the context of $\sigma$.
where $\mathscr{A}\semantics{a}_\sigma$ denotes the arithmetic evaluation of $a$ in the context of $\sigma$. Finally, sequential composition, conditionals and while loops are modelled by their GKAT counterparts, and $\mathbf{skip}$ is modelled by 1.


\section{Memory Footprint for the IMP Language}
A memory footprint considers how variables are contiguously allocated in memory. To capture the notion of a memory footprint, we are going to give a dimension to variables so they now work as finite vectors indexed by integers.  Given a variable $\vec{x}\in \Variable$ and an arithmetic expression $a \in \mathscr{A}$, the expression $\vec{x}[a]$ refers to the value stored in $\vec{x}$ at position $a$.

This treatment of variables as vectors not change the expressivity of the IMP language, and it is purely to address the fact that, for several computer architectures, when we read a memory location (e.g, $\vec{x}[a]$), its contiguous locations are also loaded into the cache (e.g., $\vec{x}[a+1]$ to $\vec{x}[a+k-1]$, where $k$ is the cache line size).

\subsection{Secure Constant Memory}
Now that actions have a more concrete semantics, we can describe how we expect a program in IMP to interact with cache memory. We follow \cite{Chattopadhyay} and describe caches using three main parameters: cache line size (CLS) in bytes, the number of cache sets (CS) and an associativity and replacement policy (P). 

Our main challenge in defining a resource consumption function to model how actions affect the state of the cache stems from the case where caches are shared by different processes. If we want our transformed programs to be resilient to attacks like \textsc{Flush+Reload}, then 

\section{The IMP Language with Division}
We modify the simple imperative programming language with variable assignments and boolean expressions from \cite{GKAT} --IMP-- so that it includes division; the language is defined as follows:
\begin{align*}
{\small
\begin{tabular}{R RL}
\text{arithmetic expressions} & a\in \mathscr{A}\!\!\!\!\!\! &::= x\in \Variable\ |\ n \in \Real_\perp\ |\ a_1+a_2\ |\ a_1-a_2\ |\ a_1\times a_2\ |\ a_1 \div a_2\\
\text{boolean expressions} & b\in \mathscr{B}\!\!\!\!\!\! &::= \textbf{false}\ |\ \textbf{true}\ |\ a_1<a_2\ |\ a_1-a_2\ |\ \textbf{not } b\ |\ b_1 \textbf{ and } b_2\ |\ b_1 \textbf{ or } b_2\\
\text{commands} & c\in \mathscr{C}\!\!\!\!\!\! &::= \textbf{skip}\ |\ x:=a\ |\ c_1;c_2\ |\ \textbf{if }b\text{ then }c_1\text{ else }c_2\ |\ \textbf{while } b\textbf{ do }c
\end{tabular}
}
\end{align*}
where $\Real_\perp=\Real \uplus \{\perp\}$, and $\perp$ is a symbol to denote \emph{not-a-number} (NaN).

As stated in \cite{GKAT}, this language can be modelled in GKAT using actions for assignments and primitive tests for comparisons as follows:
\begin{align}
\Sigma=\set{x:=a\ |\ x\in \Variable, a\in \mathscr{A}}, \quad T=\set{a_1 < a_2\ |\ a_1,a_2\in \mathscr{A}}.
\end{align}
Following \cite{GKAT}, we interpret GKAT expressions over the space of variable assignments $\semantics{\Variable}\triangleq\Variable \rightarrow\Real_\perp$:
\begin{align*}
\eval(x:=a)&\triangleq \set{(\sigma, \sigma[x:=n])\ |\  \sigma \in \semantics{\Variable}, n=\mathscr{A}\semantics{a}_\sigma},\\
\sigma[x:=n]&\triangleq \uplambda y \ldotp 
	\begin{cases}
		n, \quad &\text{if $y=x$};\\
		\sigma(y), \quad&\text{otherwise},
	\end{cases}\\
\sat(a_1<a_2)&\triangleq \set{\sigma \in \semantics{\Variable}\ |\ \mathscr{A}\semantics{a_1}_{\sigma} < \mathscr{A}\semantics{a_2}_{\sigma}},
\end{align*}
%where $\mathscr{A}\semantics{a}_\sigma\colon \semantics{\Variable}\rightarrow \Real_\perp$ denotes arithmetic evaluation of the expression $a$ in the context of $\sigma$.
where $\mathscr{A}\semantics{a}_\sigma$ denotes the arithmetic evaluation of $a$ in the context of $\sigma$. Finally, sequential composition, conditionals and while loops are modelled by their GKAT counterparts, and $\mathbf{skip}$ is modelled by 1.



\subsection{Examples of Side Channel Attacks}

\subsection{Transformations Rules for CTP in IMP}
%The most important transformation for constant time are predication and loop unwinding. 
Branch balancing is a technique to enforce CTP at the language level. This technique adds dummy instructions to shorter branches so that both branches have the same resource consumption. We capture this notion with the following repair rule.
\begin{definition}[Branch Balancing]
\begin{align*}
\branch{b}{(x:=a_1)}{1} \leadsto \branch{b}{(x:=a_1)}{(x:=x)}
\end{align*}
\end{definition}
To balance branches, we want an instruction that has the same resource consumption as $x:=a$, but acts as an identity on $x$, hence the choice for $x:=x$.  To avoid undefined behaviour, we assume that all variables are initialised by default with a value of 0 so the assignment $x:=x$ is always well defined. 
%\todo[inline]{Maybe we can define at the theoretical level an instruction $\textbf{noOp}(:=)$?}

While branch balancing may seem like a sound solution, the compiler may remove all dummy instructions inserted since it considers them to be no-ops, taking us back to an unbalanced program at the binary level. To overcome this limitation, we rely on the correspondence between program structure and arithmetic expressions, captured by the following repair rule.
\begin{definition}[Predication]
\begin{align*}
\branch{b}{(x:=a_1)}{(x:=a_2)} \leadsto x:=b\otimes a_1+(\textbf{not }b)\otimes a_2
\end{align*}
where $\otimes$ is multiplication, but it interprets the boolean condition as 0 or 1 depending on whether the condition evaluates to $\textbf{false}$ or $\textbf{true}$, respectively. 
%Strict typing
%\begin{align*}
%\branch{b}{(x:=a_1)}{(x:=a_2)} \leadsto \left(\branch{b}{(m:=1)}{(m:=0)}\right)\cdot(x:=m\times a_1+(1-m)\times a_2)
%\end{align*}
\end{definition}
The predication rule forces the evaluation of all terms in both branches, so it is more resource intensive than branch balancing; however, predication addresses two of our objectives: one is to prevent the compiler from removing dummy assignments introduced during branch balancing, and the second is to remove branches that might be vulnerable to spectre attacks.

Loops have two problems: in general, they are both vulnerable to spectre attacks and timing side-channel attacks. The former vulnerability is due to the branching implicitly introduced by the loop's guard, and the latter is due to the variable number of executions of the loop's body. Both of these vulnerabilities can be addressed by a combination of branch balancing, predication and the following transformation rule.
%{\color{lightgray}
%\begin{definition}[Delay Termination]
%if $s$ is a secret, then we need a public constant $p$  to transform the loop so it becomes secret-independent
%\begin{align*}
%\iterate{x<s}{g} \leadsto \iterate{x<p}{\left(\branch{(x<s)}{g}{1}\right)}
%\end{align*}
%\end{definition}
%Unfortunately, this rule is still vulnerable to spectre attacks introduced by the verification loop condition $x<p$. We want to eliminate the loop altogether.}
\begin{definition}[$k$-Truncation]
%Loop unwinding is an original axiom from GKAT, represented by the equivalence
%\begin{align*}
%\iterate{b}{e} \equiv \branch{b}{(e\cdot\iterate{b}{e})}{1} 
%\end{align*}
%\begin{align*}
%\branch{b}{(e\cdot\iterate{b}{e})}{1} \equiv \branch{b}{(e\cdot(\branch{b}{(e\cdot\iterate{b}{e})}{1} ))}{1} \\
%\branch{b}{(e\cdot\iterate{b}{e})}{1} \equiv \branch{b}{(e\cdot(\branch{b}{(e\cdot\iterate{b}{e})}{1} ))}{1} \\
%\end{align*}
To $k$-truncate a loop means that we transform $\iterate{b}{e}$ into a sequence of $k$ conditional statements by unwinding the loop as follows
%\begin{align*}
%\iterate{x<n}{\left(g\cdot(x:=x+1)\right)} \leadsto \underbrace{\branch{(x<n)}{\left(g\cdot(x:=x+1)\right)}{1}}_\text{$k$ times},
%\end{align*}
%\begin{align*}
%\iterate{x<n}{(x:=x+a)} \equiv \branch{(x<n)}{\left((x:=x+a)\cdot \iterate{x<n}{(x:=x+a)}\right)}{1} 
%\end{align*}
\begin{align*}
\iterate{b}{e} \leadsto \underbrace{\left(\branch{b}{e}{1}\right)\cdot\left(\branch{b}{e}{1}\right)\ldots\left(\branch{b}{e}{1}\right)}_\text{$k$ times}
\end{align*}
An equivalence relation between $\iterate{b}{e}$ and its $k$-truncation can be obtained if $k$ is a public over-approximation of the (secret-dependent) loop bound of $\iterate{b}{e}$.

Compilers normally avoid this type of unwinding when combined with predication, because it would cause programs to slow down. However, in the context of CTP, this type of delays are desirable since they make loops safe in terms of resource consumption, and the lack of branching eliminates spectre vulnerabilities.

\todo[inline]{The programmer should tell us somehow how many times we unwind each loop that depends on a secret. FaCT forces loops to be of the form ``for x from e to e'' to fix iteration increments so they know how to unwind.}
\end{definition}

\section{Taint Analysis}
\todo[inline]{Here is an interesting problem: if we have a non-security critical section (e.g., a loop that does not depend on secret) inside a section we are applying linearisation, do we need to linearise it? Probably not! The problem is that this is not compatible with our current implementation, because the current implementation linearises everything. If we introduce breakpoints we can allow loops in the branches.}

A naive definition of Taint analysis is the following: a variable is tainted if an only if, from belongs to the transitive closure of the small-step taint procedure, which marks all uses of a tainted variable as tainted. There is no untainting. We assume that, after basic code optimisations have taken place, all uses of a variable are productive an monotonous in the security hierarchy (i.e., expressions are not constant W.R.T secret inputs, and they do not inherently lose their security label). Maybe we could empower the language with annotations that override whatever taint analysis infers?

\section{The Enforcement Algorithm}
We present a $k$-step \todo{Resolve $k$} 
enforcement algorithm that enforces CTP techniques at the level of LLVM IR (compiler level) to eliminate timing side channels introduced by the use of branching and loops. 

We implement these algorithms as LLVM passes, so it is useful to understand how code is organised in LLVM.

\subsection{Premises Behind the Enforcement Algorithm}
\begin{itemize}
\item Programmers need not worry about CTP at the language level, but rather delegate as much work as possible to the compiler. The compiler may ask for information about loop bounds and about which fields are secret (these can be annotations provided by the programmer). When compiling with the right flags, the compiler may optimise the code using standard optimisations, but must ultimately enforce CTP.
\item The sequential composition of CTP programs is CTP.  This is better understood when working with GKAT expressions: if all \emph{actions} are CTP, then their sequential composition is CTP. Even if actions are CTP, GKAT \emph{expressions} are not necessarily CTP; as we know, branching and iteration do not intrinsically satisfy CTP.
\item The attacker model is a rather strong in the sense that it can measure resource consumption of prefix executions and not only total executions, which is equivalent to an attacker that can measure the resource consumption of each individual action.
\item The enforcement algorithm is sound: a repaired version $R[P]$ of program $P$ is functionally equivalent to $P$ in terms of input and output, but $R[P]$ satisfies the secure constant time property.
\item The enforcement algorithm is probably not transparent: if a program $P$ satisfies CTP, its repaired version $R[P]$ will also satisfy CTP, but $R[P]$ and $P$ may differ in terms of resource consumption, with $R[P]$ requiring more time to execute than $P$. \todo[inline]{This is just my intuition, but I would like to get empirical evidence.}
\end{itemize}

\subsection{An LLVM Primer}
LLVM IR is a strongly typed language. We interpret an LLVM IR program as a directed graph of {basic blocks}. A \emph{basic block} (BB) is a finite sequence of non-terminating instructions followed by a single terminating instruction. In this work, a \emph{terminating} instruction is either: 1) a return instruction, which terminates the program, 2) an unconditional transition to a single BB, or 3) a conditional branching where, depending on the branching condition, there is a transition to one of two BBs.

Without loss of generality, values used inside BBs are only accessible to the BB where they are declared (this can be enforced with the \texttt{opt} optimisation flag \texttt{-reg2mem}).  We use \emph{load} and \emph{store} instructions to allow cross BB communication .

\todo[inline]{A figure/example may help?}

 that uses single-static assignment for virtual registers and has memory read/write operations to enable communication between 

A function in LLVM consists of a header BB, and, without loss of generality, a single exit BB. 

\todo[inline]{Dominator/Postdominator}
\subsection{The \texttt{Linearise} Algorithm}
We start with a function $F$ that has a single entry block and a single exit block. In a nutshell, the algorithm advances from the entry block until it finds a branch or the exit block. In the case of a branch  $\texttt{br} c A B$, the algorithm recursively linearises the branches starting in blocks $A$ and $B$ until their postdominator $\texttt{postDom}(A,B)$, and then proceeds to continue linearisation towards the exit block.

\todo[inline]{Why don't we try to describe the algorithm considering loops? It's going to be a fun exercise.}

\begin{algorithm}
    \caption{Linearisation algorithm}
    \label{Linearisation}
    \begin{algorithmic}[1] % The number tells where the line numbering should start
        \Procedure{Linearise}{$\texttt{IN},\texttt{OUT},\Delta$} 
            \If{\texttt{IN}==\texttt{OUT}}
            	\State \Return $[\texttt{IN}]$
	   \Else
	   	\State $t \gets \texttt{terminator}(\texttt{IN})$
		  \If{$t$ is \texttt{jmp NXT}}\Comment{Unconditional Jump}
		      \State  \Return $(\texttt{IN}:$\texttt{ Linearise}({$\texttt{NXT},\texttt{OUT},\Delta$}))\ \Comment{Continue from $\texttt{NXT}$}
		  \ElsIf{$t$ is \texttt{br C L R}}\Comment{Conditional Branching on Secret}
		  	\State $\texttt{PD} \gets \texttt{PostDominator}(\texttt{L},\texttt{R})$
			\State $\texttt{L'}\gets \texttt{Linearise}(\texttt{L},\texttt{PD})$
			\State $\texttt{R'}\gets \texttt{Linearise}(\texttt{R},\texttt{PD})$
			\State $\texttt{M} \gets \texttt{Predication}(\texttt{C},\texttt{L'},\texttt{R'})$
			 \State  \Return $(\texttt{IN}: \texttt{M}:$\texttt{ Linearise}({$\texttt{PD},\texttt{OUT},\Delta$}))\ \Comment{Continue from $\texttt{PD}$}
		  \EndIf
            \EndIf
        \EndProcedure
    \end{algorithmic}
\end{algorithm}


\begin{algorithm}
    \caption{Euclid's algorithm}
    \label{euclid}
    \begin{algorithmic}[1] % The number tells where the line numbering should start
        \Procedure{Euclid}{$a,b$} \Comment{The g.c.d. of a and b}
            \State $r\gets a \bmod b$
            \While{$r\not=0$} \Comment{We have the answer if r is 0}
                \State $a \gets b$
                \State $b \gets r$
                \State $r \gets a \bmod b$
            \EndWhile\label{euclidendwhile}
            \State \textbf{return} $b$\Comment{The gcd is b}
        \EndProcedure
    \end{algorithmic}
\end{algorithm}

\subsection{A Premise for Efficient Repaired Code}
The premise behind why our algorithm could compete against other algorithms that protect against spectre attacks is that fencing, the SoTA defence against Spectre attacks, prevents speculative execution (i.e., puts a halt in instruction pipeling) while our transformation promotes safe pipelining.

\subsection{Enforcing Safe Behaviour}
\todo[inline]{I originally thought that this section should be part of the GKAT rules, but after some consideration, if we were to assume that sanitisation is part of the actions (i.e. is a part of CTP practices), then this section becomes largely irrelevant for the theory. However, since LLVM can have problems when accessing arrays or dividing by zero, this should be mentioned in the LLVM section, probably.}
Dividing by zero or accessing a position outside of an array are two examples of computations that may raise exceptions during runtime. These exceptions can be mitigated by sanitising the inputs to these functions. This sanitisation is often responsibility of the programmer, who uses conditionals to ensure that the inputs are safe to use. Since our repair rules remove conditionals by forcing the execution of both branch paths, they may introduce runtime exception that were not present in the original program. This is a well-known problem when using predication (see \cite{Rane}), and to overcome it, we need to shift sanitisation of inputs to the compiler or to the architecture. 

Consider the program $\textbf{if }b \textbf{ then } y:=z/x \textbf{ else skip}$. The repair rules would transform it into the program $y:=b\otimes (z/x)+(\textbf{not }b)\otimes y$. If $b$ is a condition that rules out whether $x$ is not equal to zero, we might run into problems should we evaluate $z/x$ in the repaired program. If $\otimes$ is to follow CTP practices, its execution time must not vary on the value of its (secret) input parameters, so early termination by checking if any of the parameters is zero is not allowed. If both parameters are fully evaluated before applying multiplication, we can be certain that the expression $z/x$ is going to be evaluated, even if $x$ is equal to 0. %Fortunately, the value of $z/x$ is guarded by $(x\neq 0)$, 
To avoid raising an exception at runtime, we enforce input sanitisation at compiler level by using a variable $x_\texttt{div}$ to assign a default safe value (e.g., 1) to the parameter of the division function in case the original variable is unsafe, yielding the following repaired program
\begin{align}
x_\texttt{div}:= (x\neq 0)\otimes x+(x=0)\otimes 1\ ;\ y:=b\otimes (z/x_\texttt{div})+(\textbf{not }b)\otimes y\ .
\end{align}
Since we evaluate $z/x_\texttt{div}$ instead of $z/x$ in the repaired program, the repaired program is also safe if the original program was safe. We remark that the choice of the default safe value should be irrelevant in the context of CTP because, if the division operation also follows CTP practices, then it should not matter which values we choose as long as they do not break the functionality of the program. 

Now, consider the program $\textbf{if }b \textbf{ then } y:=A[x] \textbf{ else skip}$. Similarly to the previous example, if we repair this program naively, we obtain the program $y:=b\otimes A[x]+(\textbf{not }b)\otimes y$ which might introduce runtime exceptions should $A[x]$ be executed when $x$ is greater than the size of $A$. We can sanitise the input by 
\[x_\texttt{idx}:= (x<\textbf{size}(A))\otimes x;\ y:=b\otimes A[x_\texttt{idx}]+(\textbf{not }b)\otimes y\ ,
\]
\todo[inline]{OR}
\[x_\texttt{idx}:= x \textbf{ mod }(\textbf{size}(A));\ y:=b\otimes A[x_\texttt{idx}]+(\textbf{not }b)\otimes y\ ,
\]
to avoid undesired side effects.
\todo[inline]{I have the hunch that substituting the expression $A[x]$ for $ A[x\textbf{ mod }(\textbf{size}(A))]$ might leak via memory SC that the secret is a multiple of the array index that corresponds to $x\textbf{ mod }(\textbf{size}(A))$. However, this substitution for $A[0]$ also leaks that the secret is greater than the size of the array. Now, using a secret as an index should be frowned upon by CTP guidelines, but do we want to define the problem away? probably not. What this is missing is the memory SC mitigation, and any of the options above should be valid.}

\subsection{A GKAT to Correctly Unwind Basic Block Loops}
\todo[inline]{The objective of this section is to provide the background theory to prove that the loop transformation that we use in the CFG corresponds to an unwinding transformation of GKAT iteration expressions. For that, we show that the CFG of an LLVM IR program models a normal GKAT coalgebra. If that is the case, then, from a normal GKAT coalgebra, we can use Kleene's theorem for GKAT to obtain a GKAT expression that models the program. Once we have the GKAT expression, we can unwind loops using $k$-truncation. Somehow, this should be equivalent to unwinding a loop in the CFG (even if the loop has multiple exit points). If I am not mistaken, we want to identify two blocks: the header block and the merging block of the loop. The header is unique by definition, but the merging block is the postdominator of all exiting blocks in the loops, and that is what we will end up iterating using GKAT iteration.
}

The Control Flow Graph (CFG) of LLVM IR consists of Basic Blocks (BBs). Each BB has a single entry point and a single exit point, the latter represented by a \emph{terminator instruction}. Let us consider only two types of termination instructions for BBs:
\begin{enumerate}
\item $\texttt{return }v$, which ends the current execution and returns a value $v$; and
\item $\texttt{branch $\alpha:$ $BB_1$ ; $BB_2$}$, which states that we should continue execution from $BB_1$ if the condition $\alpha$ is valid, or continue execution from $BB_2$ otherwise.
\end{enumerate}
\todo[inline]{The $\texttt{branch}$ instruction can be generalised to a $\texttt{switch}$ instruction, but we work with binary branching since it is possible to express a switch associating on the right, so we go o a basic block on the else branch that continues resolving the cases of the switch following the classical \texttt{switch $g_i:c_i$ $\equiv$ if $g_1$ then $c_1$ elseif $g_2$ then $c_2$ elseif...} equivalence that most decent programming languages have (which is why I never use switches in C++, since they are not equivalent).}
Let us consider a GKAT where the set of actions $\Sigma$ is the set of BBs and the tests are the conditions used in \texttt{branch} termination instructions.



%\begin{definition}[Loop unwinding]
%%From \cite{GKAT}
%%\begin{align*}
%%\iterate{b}{e} \equiv \branch{b}{(e\cdot \iterate{b}{e})}{1}
%%\end{align*}
%%The programmer should tell us somehow how many times we unwind each loop that depends on a secret. A different version
%\begin{align*}
%\iterate{x<n}{(x:=x+a)} \equiv \branch{(x<n)}{\left((x:=x+a)\cdot \iterate{x<n}{(x:=x+a)}\right)}{1} 
%\end{align*}
%\todo[inline]{The programmer should tell us somehow how many times we unwind each loop that depends on a secret. FaCT forces loops to be of the form ``for x from e to e'' to fix iteration increments so they know how to unwind.}
%\end{definition}



\subsection{Types}
For our first LLVM IR sublanguage, we support the following types:
\begin{itemize}
\item Void type, which cannot be assigned to variables.
\item Integer types of the form $i_N$ where $N\in [1..2^{23}-1]$.
\end{itemize}
An element of type $i_N$ is a vector of $N$ bits; consequently, the type $i_N$ is populated by maps of the form $N\rightarrow 2$\todo{Set Notation}. There are two standard interpretations for $x\colon i_N$, an \emph{unsigned} interpretation $\usg\colon i_N\rightarrow \mathbb{N}$, where 
\begin{align}
\usg(x)= \sum_{n=0}^{N-1} x(n)\times2^{x(n)},
\end{align}
and a \emph{signed} interpretation $\sg\colon i_N\rightarrow \mathbb{Z}$, where
\begin{align}
\sg(x)= 
\begin{cases}
\ \ \ \ \sum_{n=1}^{N-1} x(n)\times2^{x(n)},\quad &\text{if $x(0)=0$};\\
-\sum_{n=1}^{N-1} x(n)\times2^{x(n)},\quad &\text{if $x(0)=1$}.
\end{cases}
\end{align}
In other words, the first bit $x(0)$ determines whether $x$ represents a positive or a negative number under the signed representation.

For the set of instructions we need to introduce the notion of variables and assignments. The set of \emph{variables} $v$ is generated by the regular expression
\begin{align}
v::= (@\ |\ \%)\cdot\letter\cdot\alphanumericP^{*}.
\end{align}
A \emph{variable assignment} is an instruction of the form $v:=e$, where $e$ is a \emph{non-void function} or an \emph{arithmetic expression}. Arithmetic expressions are 
\begin{itemize}
\item Language  
\item Arithmetic operations that use division.
\end{itemize}
\subsection{Instructions} 
\todo[inline]{Overflows}

\subsection{Central idea}
The main theoretical idea of our work is\todo{what is it?}
   %-----------------------------------
%	SUBSECTION 1
%-----------------------------------
\subsection{Subsection 1}

Nunc posuere quam at lectus tristique eu ultrices augue venenatis. Vestibulum ante ipsum primis in faucibus orci luctus et ultrices posuere cubilia Curae; Aliquam erat volutpat. Vivamus sodales tortor eget quam adipiscing in vulputate ante ullamcorper. Sed eros ante, lacinia et sollicitudin et, aliquam sit amet augue. In hac habitasse platea dictumst.

%-----------------------------------
%	SUBSECTION 2
%-----------------------------------

\subsection{Subsection 2}
Morbi rutrum odio eget arcu adipiscing sodales. Aenean et purus a est pulvinar pellentesque. Cras in elit neque, quis varius elit. Phasellus fringilla, nibh eu tempus venenatis, dolor elit posuere quam, quis adipiscing urna leo nec orci. Sed nec nulla auctor odio aliquet consequat. Ut nec nulla in ante ullamcorper aliquam at sed dolor. Phasellus fermentum magna in augue gravida cursus. Cras sed pretium lorem. Pellentesque eget ornare odio. Proin accumsan, massa viverra cursus pharetra, ipsum nisi lobortis velit, a malesuada dolor lorem eu neque.

%----------------------------------------------------------------------------------------
%	SECTION 2
%----------------------------------------------------------------------------------------

\section{Main Section 2}

Sed ullamcorper quam eu nisl interdum at interdum enim egestas. Aliquam placerat justo sed lectus lobortis ut porta nisl porttitor. Vestibulum mi dolor, lacinia molestie gravida at, tempus vitae ligula. Donec eget quam sapien, in viverra eros. Donec pellentesque justo a massa fringilla non vestibulum metus vestibulum. Vestibulum in orci quis felis tempor lacinia. Vivamus ornare ultrices facilisis. Ut hendrerit volutpat vulputate. Morbi condimentum venenatis augue, id porta ipsum vulputate in. Curabitur luctus tempus justo. Vestibulum risus lectus, adipiscing nec condimentum quis, condimentum nec nisl. Aliquam dictum sagittis velit sed iaculis. Morbi tristique augue sit amet nulla pulvinar id facilisis ligula mollis. Nam elit libero, tincidunt ut aliquam at, molestie in quam. Aenean rhoncus vehicula hendrerit. 
%\include{Chapters/Chapter2}
%\include{Chapters/Chapter3}
%\include{Chapters/Chapter4} 
%\include{Chapters/Chapter5}
%\include{Chapters/Chapter6}

%----------------------------------------------------------------------------------------
%	THESIS CONTENT - APPENDICES
%----------------------------------------------------------------------------------------

\appendix % Cue to tell LaTeX that the following "chapters" are Appendices

% Include the appendices of the thesis as separate files from the Appendices folder
% Uncomment the lines as you write the Appendices

\include{Appendices/AppendixA}
%\include{Appendices/AppendixB}
%\include{Appendices/AppendixC}
 
%----------------------------------------------------------------------------------------
%	BIBLIOGRAPHY
%----------------------------------------------------------------------------------------
 
\printbibliography[heading=bibintoc]

%----------------------------------------------------------------------------------------

\end{document} 
