\begin{figure}[t]
    \centering
    \begin{tikzcd}[column sep=large]
        X \arrow[r, dotted, "\TheBehaviourOf{\cdot}_c"] \arrow[d, "c", red] & \sigma F \arrow[d, "\simeq"] \\
        F(X) \arrow[r, dotted, "F(\TheBehaviourOf{\cdot}_c)"]     & F(\sigma F)
    \end{tikzcd}\\
    \centering
    \begin{tikzcd}[column sep=large]
        X \arrow[r, dotted, "\TheBehaviourOf{\cdot}^{m}_{c}=\TheBehaviourOf{\cdot}_{c\circ m}"] \arrow[d, "m", blue] & \sigma F \arrow[dd, "\simeq"] \\
        X \arrow[d, "c", red] & \  \\
        F(X) \arrow[r, dotted, "F\left(\TheBehaviourOf{\cdot}_{c\circ m}\right)"]     & F(\sigma F)
    \end{tikzcd}
\end{figure}
\begin{figure}[t]
    \centering
    \begin{tikzcd}
        &X 
            %\arrow[r, dotted, "\TheBehaviourOf{\cdot}^{m}_{c}=\TheBehaviourOf{\cdot}_{c\circ m}"] 
            \arrow[r, dotted, "\TheBehaviourOf{\cdot}_{c\circ m}"] 
            \arrow[d, "m", blue] 
        & \sigma F 
            \arrow[dd, "\simeq"] 
        \\
        \sigma F 
            \arrow[d, "\simeq"] 
            \arrow[rru, "w", bend left=45, purple] 
        & X 
            \arrow[d, "c", red] 
            \arrow[l, dotted, "\TheBehaviourOf{\cdot}_c"] 
            \arrow[ur, "?", olive]
            & 
        \\
        F(\sigma F)
        & F(X) 
            \arrow[l, dotted, "F(\TheBehaviourOf{\cdot}_c)"]
            \arrow[r, dotted, "F\left(\TheBehaviourOf{\cdot}_{c\circ m}\right)"]    
        & F(\sigma F)
    \end{tikzcd}
\end{figure}

\begin{figure}[t]
    % \begin{tikzcd}[column sep=large]
    %     X 
    %         %\arrow[r, dotted, "\TheBehaviourOf{\cdot}^{m}_{c}=\TheBehaviourOf{\cdot}_{c\circ m}"] 
    %         \arrow[r, dotted, "\TheBehaviourOf{\cdot}_{c\circ m}"] 
    %         \arrow[d, "m", blue] 
    %     &\sigma F   
    %         \arrow[d, "id", olive]
    %     \\
    %     X 
    %         \arrow[d, "c", red] 
    %         %\arrow[r, "\TheBehaviourOf{\cdot}_c"] 
    %     &\sigma F 
    %         \arrow[d, "\simeq"] 
    %         %\arrow[u, "w", olive]
    %     \\
    %     F(X) 
    %         %\arrow[r, dotted, "F(\TheBehaviourOf{\cdot}_{c})"]
    %         \arrow[r, dotted, "F\left(\TheBehaviourOf{\cdot}_{c\circ m}\right)"]    
    %     &F(\sigma F)
    %      %F(\sigma F)
    % \end{tikzcd}\\
    % \begin{tikzcd}[column sep=large]
    %     X 
    %         \arrow[rd, dotted, "\TheBehaviourOf{\cdot}_{c\circ m}"] 
    %         \arrow[r, "\TheBehaviourOf{\cdot}_{c}"] 
    %         \arrow[d, "m", blue] 
    %     &\sigma F   
    %         \arrow[d, "w", olive]
    %     \\
    %     X 
    %         \arrow[d, "c", red] 
    %         %\arrow[r, "\TheBehaviourOf{\cdot}_c"] 
    %     &\sigma F 
    %         \arrow[dd, "\simeq"] 
    %         %\arrow[u, "w", olive]
    %     \\
    %     F(X) 
    %         %\arrow[r, dotted, "F(\TheBehaviourOf{\cdot}_{c})"]
    %         \arrow[rd, dotted, swap, "F\left(\TheBehaviourOf{\cdot}_{c\circ m}\right)"]    
    %     &
    %     \\
    %     &
    %     F(\sigma F)
    %      %F(\sigma F)
    % \end{tikzcd}
    % \caption{This says $\TheBehaviourOf{\cdot}_{c\circ m}=w\circ\TheBehaviourOf{\cdot}_{c}$}
    % \begin{tikzcd}[column sep=large]
    %     X 
    %         \arrow[r, "\TheBehaviourOf{\cdot}_{c\circ m}"]  
    %         \arrow[d, "m", blue] 
    %     &\sigma F   
    %         %\arrow[d, "w", olive]
    %     \\
    %     X 
    %         \arrow[d, "c", red] 
    %         \arrow[r, dotted, "\TheBehaviourOf{\cdot}_{c}"]
    %     &\sigma F 
    %         \arrow[d, "\simeq"] 
    %         \arrow[u, "w", olive]
    %     \\
    %     F(X) 
    %         %\arrow[r, dotted, "F(\TheBehaviourOf{\cdot}_{c})"]
    %         \arrow[r, dotted, swap, "F\left(\TheBehaviourOf{\cdot}_{c}\right)"]    
    %     &F(\sigma F)
    %     % \\
    %     % &
    %     % F(\sigma F)
    % \end{tikzcd}
    % \caption{This says $\TheBehaviourOf{\cdot}_{c\circ m}=w\circ\TheBehaviourOf{\cdot}_{c}\circ m$. THIS IS INCORRECT. The thing is: that would be if $m$ is applied only once. }
    \begin{tikzcd}[column sep=large]
        X 
            %\arrow[r, "\TheBehaviourOf{\cdot}_{c\circ m}"]  
            \arrow[rr, dotted, bend left, "\TheBehaviourOf{\cdot}_{c\circ m}"] 
            \arrow[d, "m", blue] 
        &\sigma F   
            \arrow[d, "m'", olive]
            \arrow[r, dotted, "\TheBehaviourOf{\cdot}_{\omega\circ m'}"] 
        &\sigma F
            \arrow[dd, "\simeq", "\omega"']
        \\
        X 
            \arrow[d, "c", red] 
            \arrow[r, dotted, "\TheBehaviourOf{\cdot}_{c}"]
        &\sigma F 
            \arrow[d, "\simeq", "\omega"' ] 
            %\arrow[ur, "w"]
            %\arrow[u, "w", olive]
        &\\
        F(X) 
            %\arrow[r, dotted, "F(\TheBehaviourOf{\cdot}_{c})"]
            \arrow[r, dotted, swap, "F\left(\TheBehaviourOf{\cdot}_{c}\right)"]  
            \arrow[rr, dotted, swap, bend right, "F(\TheBehaviourOf{\cdot}_{c\circ m})"]   
        &F(\sigma F)
            \arrow[r, dotted, swap, "F(\TheBehaviourOf{\cdot}_{\omega\circ m'})"]
        &F(\sigma F)
        % \\
        % &
        % F(\sigma F)
    \end{tikzcd}
    %\caption{This says $k\circ \TheBehaviourOf{\cdot}_{c\circ m}=\TheBehaviourOf{\cdot}_{c}\circ m$. }
\end{figure}
An example:
\begin{figure}[t]
    \begin{tikzpicture}
        \node[state, accepting] (00) {$(0,0)$};
        \node[state, accepting, below right of=00] (01) {$(0,1)$};
        \node[state, accepting, above right  of=00] (10) {$(1,0)$};
        \node[state, accepting, below right of=10] (11) {$(1,1)$};
        \draw (00) edge[bend left, above] node{1} (11)
        (00) edge[bend left, above] node{0} (01)
        (01) edge[bend left, above] node{1} (11)
        (01) edge[loop above] node{0} (01)
        (10) edge[bend left, above] node{1} (11)
        (10) edge[bend left, above] node{0} (01)
        (11) edge[loop above] node{1} (11)
        (11) edge[bend left, above] node{0} (01)
        ;\end{tikzpicture}\\
    \begin{tikzcd}[column sep=large]
        (0,0) 
            %\arrow[r, "\TheBehaviourOf{\cdot}_{c\circ m}"]  
            \arrow[rr, mapsto, dotted, bend left, "\TheBehaviourOf{\cdot}_{c\circ m}"] 
            \arrow[d, mapsto, "{\texttt{const}(1,1)}", blue] 
        &\sigma F   
            \arrow[d, "\texttt{const}(\TheBehaviourOf{(1,1)})", olive]
            \arrow[r, dotted, "\TheBehaviourOf{\cdot}_{\omega\circ w}"] 
        &(0,1)^*
            \arrow[dd, "\simeq", "\omega"']
        \\
        (1,1) 
            \arrow[d, "c", red] 
            \arrow[r, dotted, "\TheBehaviourOf{\cdot}_{c}"]
        &\TheBehaviourOf{(1,1)}%1^*+(0,1)^*111^*
            \arrow[d, "\simeq", "\omega"' ] 
            %\arrow[u, "w", olive]
        &\\
        {(1,(\cdot,1))} 
            %\arrow[r, dotted, "F(\TheBehaviourOf{\cdot}_{c})"]
            \arrow[r, dotted, swap, "F\left(\TheBehaviourOf{\cdot}_{c}\right)"]  
            \arrow[rr, dotted, swap, bend right, "F(\TheBehaviourOf{\cdot}_{c\circ m})"]   
        &{(1,\TheBehaviourOf{(\cdot,1)}))}
            \arrow[r, dotted, swap, "F(\TheBehaviourOf{\cdot}_{\omega\circ w})"]
        &{(1,(0,1)^*))}
        % \\
        % &
        % F(\sigma F)
    \end{tikzcd}
    \caption{This says $k\circ \TheBehaviourOf{\cdot}_{c\circ m}=\TheBehaviourOf{\cdot}_{c}\circ m$. }
\end{figure}


% \begin{figure}[t]
%     \centering
%     \begin{tikzcd}
%         X 
%             \arrow[rrd, dotted, "\TheBehaviourOf{\cdot}^{m}_{c}=\TheBehaviourOf{\cdot}_{c\circ m}", bend left] 
%             \arrow[dr, "m", blue] 
%         &
%         & \\
%         &X
%         &\sigma F
%             %\arrow[dd, "\simeq"] \\
%         %X \arrow[d, "c", red] \arrow[l, dotted, "\TheBehaviourOf{\cdot}_c"] & \  \\
%         % F(\sigma F)& F(X) \arrow[l, dotted, "F(\TheBehaviourOf{\cdot}_c)"]    \arrow[r, dotted, "F\left(\TheBehaviourOf{\cdot}_{c\circ m}\right)"]     & F(\sigma F)
%     \end{tikzcd}
% \end{figure}
is $w$ coinductively defined? no. For $w$ to be defined coinductively, it would need to be an $F$-coalgebra homomorphism, between $(\sigma F,\simeq)$ and $(\sigma F,\simeq)$. However, since $(\sigma F,\simeq)$ is final in the category of $F$-coalgebras, the only $F$-homomorphism would be $id_{\sigma F}$, so $w$ can only be defined coinductively if $w=id_{\sigma F}$. It is, instead a transformation of $\sigma F$, defined by 
\begin{align}
    w(\TheBehaviourOf{m(x)}_{c})&\triangleq\TheBehaviourOf{x}_{c\circ m}\\
    \TheBehaviourOf{x}_{c\circ m}&=(w\circ\TheBehaviourOf{-}_{c}\circ m)(x)
\end{align}

\begin{figure}[t]
    \begin{tikzcd}[column sep=large]
        X
        \arrow[r, "m", blue] 
        %\arrow[dr, "c \circ m", red]  
        \arrow[dd, dotted, "\TheBehaviourOf{\cdot}_{c\circ m}"]
        &X 
            \arrow[d, "c", red] 
            \arrow[r, dotted, "\TheBehaviourOf{\cdot}_{c}"]
        &\sigma F 
            \arrow[d, "\simeq", "\omega"',red ] 
            %\arrow[ur, "w"]
            %\arrow[u, "w", olive]
        \\
        &
        F(X) 
            %\arrow[r, dotted, "F(\TheBehaviourOf{\cdot}_{c})"]
            \arrow[r, dotted, swap, "F\left(\TheBehaviourOf{\cdot}_{c}\right)"]  
            \arrow[d, dotted, swap, "F(\TheBehaviourOf{\cdot}_{c \circ m})"]  
        &F(\sigma F)
        \\
        \sigma F
            \arrow[r, "\simeq", "\omega"',red] 
        &F(\sigma F)
        &
        % &
        % F(\sigma F)
    \end{tikzcd}\end{figure}
\begin{figure}
    \centering
    \begin{tikzcd}[column sep=large]
        \sigma F
            \arrow[d, "\simeq","\omega"', red] 
        &X
            \arrow[r, "m", blue]
            %\arrow[rd, "c\circ m", red]
            \arrow[l, dotted, swap,"\TheBehaviourOf{\cdot}_{c\circ m}"]
        &X 
            \arrow[r, dotted, "\TheBehaviourOf{\cdot}_c"] 
            \arrow[d, "c", red] 
        & \sigma F 
            \arrow[d, "\simeq","\omega"', red] 
        \\
        F(\sigma F)
        &
        &F(X) 
            \arrow[r, dotted, "F(\TheBehaviourOf{\cdot}_c)"]
            \arrow[ll, dotted, swap,"F(\TheBehaviourOf{\cdot}_{c\circ m})"]     
        &F(\sigma F)
    \end{tikzcd}
    %\caption{This says $k\circ \TheBehaviourOf{\cdot}_{c\circ m}=\TheBehaviourOf{\cdot}_{c}\circ m$. }
\end{figure}

\begin{figure}
    \centering
    \begin{tikzcd}[column sep=large]
        \sigma F
            \arrow[d, "\simeq","\omega"'] 
        &X
            \arrow[r, "m"]
            %\arrow[rd, "c\circ m", red]
            \arrow[l, dotted, swap,"\TheBehaviourOf{\cdot}_{c\circ m}"]
        &X 
            \arrow[r, dotted, "\TheBehaviourOf{\cdot}_c"] 
            \arrow[d, "c"] 
        & \sigma F 
            \arrow[d, "\simeq","\omega"'] 
        \\
        F(\sigma F)
        &
        &F(X) 
            \arrow[r, dotted, "F(\TheBehaviourOf{\cdot}_c)"]
            \arrow[ll, dotted, swap,"F(\TheBehaviourOf{\cdot}_{c\circ m})"]     
        &F(\sigma F)
    \end{tikzcd}
    %\caption{This says $k\circ \TheBehaviourOf{\cdot}_{c\circ m}=\TheBehaviourOf{\cdot}_{c}\circ m$. }
\end{figure}

\begin{figure}
    \centering
    \begin{tikzcd}[column sep=large]
        \sigma F
            \arrow[d, "\simeq","\omega"'] 
        &X
            \arrow[r, "m"]
            %\arrow[rd, "c\circ m", red]
            \arrow[l, dotted, swap,"\TheBehaviourOf{\cdot}_{c\circ m}"]
        &X 
            \arrow[r, dotted, "\TheBehaviourOf{\cdot}_c"] 
            \arrow[d, "c"] 
        & \sigma F 
            \arrow[d, "\simeq","\omega"'] 
        \\
        F(\sigma F)
        &
        &F(X) 
            \arrow[r, dotted, "F(\TheBehaviourOf{\cdot}_c)"]
            \arrow[ll, dotted, swap,"F(\TheBehaviourOf{\cdot}_{c\circ m})"]     
        &F(\sigma F)
    \end{tikzcd}
    %\caption{This says $k\circ \TheBehaviourOf{\cdot}_{c\circ m}=\TheBehaviourOf{\cdot}_{c}\circ m$. }
\end{figure}