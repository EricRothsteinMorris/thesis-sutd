%!TEX root = ../main.tex
\chapter{Conclusion}

\todo[inline]{General directions to continue this line of work. There are mainly three directions at the moment}
\section{Applications to Security Analysis}
\todo[inline]{A wonderful assertion: no spatial transformation breaks the property behaviour of a system. This implies that an attacker needs to change the program itself, and that model is beyond state integrity.}
\section{Applications to Program Repair}
\todo[inline]{We can ourselves morph the state. If we find that some transformations, like the ones we used for program repair, enforce a behavioural property but respect functionality, then we can apply them!}
\section{Applications to Program Synthesis}
\todo[inline]{A curious problem: if you have a set of ``gadgets'' defined by an $F$-coalgebra, can you combine them to satisfy a specification?}
\section{Future work}
\todo[inline]{This should go in the last chapter!}
We would like to consider two problems related to latent behaviours: 
\begin{itemize}
\item How can we use transformations ourselves to repurpose a system that is already been defined?
\item How can an attacker use transformations to force a behaviour they want?
\end{itemize}
\todo[inline]{Note that, ultimately, both questions need a method to solve an equation for a transformation. That is, given a target behaviour and an a source coalgebra, how do you solve the problem of finding a transformation that helps you display the behaviour you want? Is it even possible?}