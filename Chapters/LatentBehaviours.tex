%!TEX root = ../main.tex
% Chapter Template


\chapter{Latent Behaviours} % Main chapter title
\label{ch:LatentBehaviours} % Change X to a consecutive number; for referencing this chapter elsewhere, use \ref{ChapterX}
\todo[inline]{Find a suitable quote?}
\begin{quote}
If you only do what you can do, you will never be more than what you are now.
\end{quote}

\section{Introduction}
\todo[inline]{Give an interesting motivational example: What is the notion of latent behaviour? Why do we care about them? How is it useful to study them?}
\section{Latent $F$-coalgebras and their Behaviours}
Informally, a system manifests a latent behaviour if and only if its states are altered but its dynamics are preserved. To alter states, we use a \emph{transformation} function.
%For the functor $\Functor$, the monad $\Monad=(\MFunctor,\eta,\mu)$, and a $\lambda$-bialgebra $\Bialgebra=(X,a,c)$, we can define the behaviour associated to $\Bialgebra$ by means of the semantic mapping in the category of $\Functor$-coalgebras, i.e., by means of $\TheBehaviourOf{\cdot}\colon \MFunctor(X)\rightarrow \sigma\Functor$. 
%The \emph{latent behaviour} given a transformation coalgebra $\mathbb{M}=(X,X\xrightarrow{(o_m,\delta_m)} F(X))$, is modelled by the function $\TheLatentBehaviourOf{\cdot}{\mathbb{M}}\colon X\rightarrow \sigma F$, which is coinductively defined, for $i\in I$, by
%\begin{align}
%o(\TheLatentBehaviourOf{x}{\mathbb{M}})&\triangleq o_m(x)\\
%\delta(\TheLatentBehaviourOf{x}{\mathbb{M}})(i)&\triangleq \TheLatentBehaviourOf{\delta_m(x)(i)}{\mathbb{M}}.
%\end{align}
%The transformation $a$ skews the behaviour of states to reflect the changes to the state. 

\begin{definition}[Consistent State Transformations]
Given an $F$-coalgebra $(X,c)$, any function of type $m\colon X\rightarrow X$ is a \emph{transformation} of $X$. %A transformation $m\colon X \rightarrow X$ has \emph{finite support} iff $m(x)\neq x$ only for a finite number of $x\in X$. We denote the set of finitely supported transformations by $X^X_\omega$. 
A transformation $m$ is \emph{(behaviourally) consistent} if and only if, whenever $x\sim y$, then $m(x)\sim m(y)$, for all $x,y \in X$. We denote the set of consistent transformations by $X^X|_\sim$. Henceforth, we consider only consistent transformations, unless explicitly mentioned otherwise.
\end{definition}


When we use a transformation function $m\colon X\rightarrow X$ to skew the normal behaviour of $\mathbb{X}$, we manifest the \emph{latent coalgebra of $\mathbb{X}$ under $m$}. 
\begin{definition}[Latent Coalgebra]
Given an $F$-coalgebra $\TheCoalgebra=(X,c)$ and a transformation $m$, the \emph{latent coalgebra under $m$} is $(X,c\circ m)$. 
% \begin{align}
%     \mathbb{X}\circ m\triangleq(X,{(o\circ m, \delta\circ m })).
% \end{align}
The function $\TheLatentBehaviourOfIn{\cdot}{m}{c}\colon X\rightarrow \sigma F$ defines the \emph{latent behaviour} under $m$. The homomorphism $\TheLatentBehaviourOfIn{\cdot}{m}{\mathbb{X}}$ corresponds to the semantic mapping of the $F$-coalgebra $(X,c\circ m)$; that is, for $x\in X$, 
\begin{align}
\TheLatentBehaviourOfIn{x}{m}{c}\triangleq\TheBehaviourOfIn{x}{{c\circ m}}
\end{align}
\todo[inline]{Latent coalgebras are coalgebras with the state space deformed.}
\end{definition}
\begin{figure}
    \centering
    \begin{tikzcd}[column sep=large]
        \sigma F
            \arrow[d, "\simeq","\omega"'] 
        &X
            \arrow[r, "m"]
            %\arrow[rd, "c\circ m", red]
            \arrow[l, dotted, swap,"\TheBehaviourOf{\cdot}_{c\circ m}"]
        &X 
            \arrow[r, dotted, "\TheBehaviourOf{\cdot}_c"] 
            \arrow[d, "c"] 
        & \sigma F 
            \arrow[d, "\simeq","\omega"'] 
        \\
        F(\sigma F)
        &
        &F(X) 
            \arrow[r, dotted, "F(\TheBehaviourOf{\cdot}_c)"]
            \arrow[ll, dotted, swap,"F(\TheBehaviourOf{\cdot}_{c\circ m})"]     
        &F(\sigma F)
    \end{tikzcd}
    \caption{$(X,c\circ s)$ is an $F$-coalgebra, so it has a unique $F$-homomorphism to the final $F$-coalgebra $(\sigma F, \omega)$. Geometrically, $s$ is a deformation of the state space.}
\end{figure}
\begin{figure}
    \centering
    \begin{tikzcd}[column sep=large]
        \sigma F
            \arrow[d, "\simeq","\omega"'] 
        &&X
            \arrow[r, "s"]
            \arrow[d, "t\circ c\circ s"] 
            %\arrow[rd, "c\circ m", red]
            \arrow[ll, dotted, swap,"\TheBehaviourOf{\cdot}_{t\circ c\circ s}"]
        &X 
            \arrow[rr, dotted, "\TheBehaviourOf{\cdot}_c"] 
            \arrow[d, "c"] 
        && \sigma F 
            \arrow[d, "\simeq","\omega"'] 
        \\
        F(\sigma F) 
        &&F(X) \arrow[ll, dotted, swap,"F(\TheBehaviourOf{\cdot}_{t\circ c\circ s})"]
        &F(X) 
            \arrow[rr, dotted, "F(\TheBehaviourOf{\cdot}_c)"]
            \arrow[l, "t"]     
        &&F(\sigma F)
    \end{tikzcd}
    \caption{Latency using both $m$ and $b$ to manifest latent behaviours: $(X,b\circ c\circ m)$ is still an $F$-coalgebra, so it has a unique $F$-homomorphism to the final $F$-coalgebra $(\sigma F, \omega)$. Consider the following metaphore: we are grading a test, and $c$ is the grading scheme; $c(x)$ is the grade assigned to $x$. A transformation $m$ would correspond to changing the answers before grading, while a transformation $b$ corresponds to changing the grade after the answers have been graded. The transformation $b$ needs not respect the properties of $c$, since it is applied after it. Thus, it would be possible to change the grade through $b$ to some value that is impossible to obtain by any possible combination of answers (i.e. for every $m$ that changes answers).}
\end{figure}

{\color{red}The deformation $b$ is not that interesting because it is too flexible. If $F=\id$ then $t$ has the same type as $s$ and they compose, so we can approximate $t.c.s$ with $c.s'$, where $s'=s.t$. If $F(X)$ has only one component, then $t=const \phi$ forces the system to have the behaviour that the attacker wants. More precisely, it can manifest any $F$-coalgebra for that carrier set. The balance would be to allow the attacker of $t$ to influence only a set of components in $F(X)$, just like we do with $s$. Consider the $F$-coalgebras of $()$ for the functor $2x\id^2$; there are only two: $c_0(())=(False,const ())$ and $c_1(())=(True,const ())$. Using state transformations we cannot manifest new behaviours given $c_0$ or $c_1$, but with with behaviour transformations we can: $t=(\texttt{not},\id)$ causes $t.c_0=c_1$ and $t.c_1=c_0$ so you can strictly do more. The question is, do we need more? 

Poetically, $m$ corresponds to a deformation of space, and $b$ corresponds to a deformation of causality.
}

\begin{example}
\label{ex:LatentBehaviour}
Consider the functor $G=2\times \texttt{id}^2$ and a $G$-coalgebra $\mathbb{X}=(2\times2,(\gamma,\delta))$ that can be used to recognise sequences of zeroes and ones that end in two consecutive ones when started at state 0., i.e. the language $(0+1)^*111^*$. We define said $G$-coalgebra $\mathbb{X}$ by
\begin{align}
\gamma(x,y)&\triangleq x \land y;\\
\delta(x,y)(i)&\triangleq(i,x).
\end{align}
Figure~\ref{fig:ExampleLatent} shows the deterministic finite automaton corresponding to this $G$-coalgebra; the coalgebra is not minimal, because the states $(0,0)$ and $(0,1)$ are bisimilar.
\begin{figure}[t]
\centering
\begin{tikzpicture}
\node[state] (00) {$(0,0)$};
\node[state, below right of=00] (01) {$(0,1)$};
\node[state, above right  of=00] (10) {$(1,0)$};
\node[state, accepting, below right of=10] (11) {$(1,1)$};
\draw (00) edge[bend left, above] node{1} (10)
(00) edge[loop above] node{0} (00)
(01) edge[bend left, above] node[left]{1} (10)
(01) edge[bend left, above] node{0} (00)
(10) edge[bend left, above] node{1} (11)
(10) edge[bend left, above] node[right]{0} (01)
(11) edge[loop above] node{1} (11)
(11) edge[bend left, above] node{0} (01)
;\end{tikzpicture}
\caption{Corresponding automaton to the $G-$coalgebra described in Example~\ref{ex:LatentBehaviour}.}
\label{fig:ExampleLatent}
\end{figure}

This $G$-coalgebra has 256 transformations, given by the $4^4$ endofunctions in $X$. We use these transformations to manifest latent behaviours.%, and every consistent transformation $m$ satisfies $m(0,0)~m(0,1)$. 
\todo[inline]{You can probably graph this with a 3 dimensional graph like a vector field whose axises are input, state, and output. We could use four dimensions so that state is 2-dimensional}
% We present the automata that correspond to latent $G-$coalgebras in Figures~\ref{fig:3.2}--\ref{fig:3.28}. Note that many of these latent coalgebras are not minimal, and many display the same behaviours.

% From 27 endofuctions and three intended behaviours, we can derive the 24 latent behaviours presented in Table~\ref{tab:ExampleLatentBehaviours} as regular expressions. 
{\color{red}
\todo[inline]{Rewrite this!!!}
The existence of latent behaviours opens a new possibility for system repurposing: if we wanted the system to recognise the language $(0+1)^*1$ (i.e., the language of sequences that end in 1), we could mutate our original system using a transformation where $0\mapsto0, 1\mapsto2$ and $2\mapsto2$ (shown in Figure~\ref{fig:3.10}), whilst preserving $0$ as the initial state. Nevertheless, if providing the transformation is within the capabilities of an adversary, it would also mean that they can repurpose our system as well. 
}


\begin{figure}[t]
\centering
\begin{tikzpicture}
\node[state] (00) {$(0,0)$};
\node[state, accepting, below right of=00] (01) {$(0,1)$};
\node[state, above right  of=00] (10) {$(1,0)$};
\node[state, below right of=10] (11) {$(1,1)$};
\draw (00) edge[bend left, above] node{1} (10)
(00) edge[loop above] node{0} (00)
(01) edge[bend right, above] node{1} (11)
(01) edge[loop left] node{0} (01)
(10) edge[loop above] node{1} (10)
(10) edge[bend left, above] node{0} (00)
(11) edge[bend left, above] node{1} (10)
(11) edge[bend left, above] node{0} (00)
;\end{tikzpicture}
\caption{Corresponding automaton to the latent $G-$coalgebra under transformation $???$ described in Example~\ref{ex:LatentBehaviour}.}
\label{fig:ExampleLatent}
\end{figure}
    
% \begin{table}[t]
% \centering
% \begin{tabular}{|l | l | }
% \hline
% $\rho_{\ref{fig:3.2}.0}$ &  $\emptyset$   \\
% $\rho_{\ref{fig:3.4}.2}$ &  $1^*$ \\
% $\rho_{\ref{fig:3.7}.0}$ &  $(0+1)^*111^*$\\%$(0+10+111^*0)^*111^*$ \\
% $\rho_{\ref{fig:3.7}.1}$ &  $\rho_{\ref{fig:3.7}.0}+1$ \\
% $\rho_{\ref{fig:3.7}.2}$ &  $\rho_{\ref{fig:3.7}.0}+1+\varepsilon$ \\
% $\rho_{\ref{fig:3.8}.1}$ &  $(00^*1+10^*1)^*$ \\
% $\rho_{\ref{fig:3.8}.0}$ &  $0^*1\rho_{\ref{fig:3.8}.1}$ \\
% $\rho_{\ref{fig:3.9}.2}$& $(0+1)^*01$\\
% $\rho_{\ref{fig:3.9}.0}$&   $\rho_{\ref{fig:3.9}.2}+1$\\
% $\rho_{\ref{fig:3.9}.1}$&   $\rho_{\ref{fig:3.9}.2}+1+\varepsilon$\\
% $\rho_{\ref{fig:3.10}.0}$ &  $(0+1)^*1$ \\
% $\rho_{\ref{fig:3.10}.1}$ &  $\rho_{\ref{fig:3.10}.0}+\varepsilon$ \\
% $\rho_{\ref{fig:3.13}.1}$ &  $1^*0\rho_{\ref{fig:3.10}.0}$ \\
% $\rho_{\ref{fig:3.17}.0}$ &  $(0+10+(11)^+(0+10))^*(11)^+$ \\
% $\rho_{\ref{fig:3.17}.1}$ &  $(11)^*+(0+10)\rho_{\ref{fig:3.17}.0}$ \\
% $\rho_{\ref{fig:3.17}.2}$ &  $0\rho_{\ref{fig:3.17}.0}+1\rho_{\ref{fig:3.17}.1}$ \\
% $\rho_{\ref{fig:3.18}.0}$ &  $\varepsilon$ \\
% $\rho_{\ref{fig:3.20}.1}$ &  $(0+1)^*0$ \\
% $\rho_{\ref{fig:3.20}.0}$ &  $\rho_{\ref{fig:3.20}.1}+\varepsilon$ \\
% $\rho_{\ref{fig:3.22}.0}$ &  $(1+0)^*$ \\
% $\rho_{\ref{fig:3.22}.1}$ &  $1^*0(1+0)^*$ \\
% $\rho_{\ref{fig:3.25}.1}$ &  $1(1+0)^*+0(1+0)^*$ \\
% $\rho_{\ref{fig:3.26}.0}$ &  $(0+10+11)^*$ \\
% $\rho_{\ref{fig:3.26}.2}$ &  $(0+1)\rho_{\ref{fig:3.26}.0}$\\
% \hline
% \end{tabular}
% \caption{Latent behaviours for all transformations for the coalgebra that recognises $(0+1)^*11$. }
% \label{tab:ExampleLatentBehaviours}
% \end{table}

% %$\rho_{\ref{fig:3.8}.0}$ &  $$ \\
% %$\rho_{\ref{fig:3.8}.1}$ &  $$ \\
% %$\rho_{\ref{fig:3.8}.2}$ &  $$ \\
% %------------------------------------------------------------------------------------------------------------------------
% \begin{figure}[t]
% \centering
% \begin{tikzpicture}
% \node[state] (q1) {$0$};
% \node[state, right of=q1] (q2) {$1$};
% \node[state, right of=q2] (q3) {$2$};
% \draw (q1) edge[loop above] node{0} (q1)
% (q1) edge[bend left, above] node{1} (q2)
% (q2) edge[bend left, above] node{0} (q1)
% (q2) edge[loop above] node{1} (q2)
% (q3) edge[bend left, above] node{1} (q2)
% (q3) edge[bend left, below] node{0} (q1);
% \end{tikzpicture}
% \caption{Latent coalgebra under graph $G(m)=\set{(0,0),(1,0),(2,0)}$. $\TheLatentBehaviourOf{0}{m}=\TheLatentBehaviourOf{1}{m}=\TheLatentBehaviourOf{2}{m}=\emptyset$ }
% \label{fig:3.2}

% \begin{tikzpicture}
% \node[state] (q1) {$0$};
% \node[state, right of=q1] (q2) {$1$};
% \node[state, right of=q2] (q3) {$2$};
% \draw (q1) edge[loop above] node{0} (q1)
% (q1) edge[bend left, above] node{1} (q2)
% (q2) edge[bend left, above] node{0} (q1)
% (q2) edge[loop above] node{1} (q2)
% (q3) edge[loop above] node{1} (q3)
% (q3) edge[bend left, below] node{0} (q1);
% \end{tikzpicture}
% \caption{Latent coalgebra under graph $G(m)=\set{(0,0),(1,0),(2,1)}$. $\TheLatentBehaviourOf{0}{m}=\TheLatentBehaviourOf{1}{m}=\TheLatentBehaviourOf{2}{m}=\emptyset$}
% \label{fig:3.3}

% \begin{tikzpicture}
% \node[state] (q1) {$0$};
% \node[state, right of=q1] (q2) {$1$};
% \node[state, accepting, right of=q2] (q3) {$2$};
% \draw (q1) edge[loop above] node{0} (q1)
% (q1) edge[bend left, above] node{1} (q2)
% (q2) edge[bend left, above] node{0} (q1)
% (q2) edge[loop above] node{1} (q2)
% (q3) edge[loop above] node{1} (q3)
% (q3) edge[bend left, below] node{0} (q1);
% \end{tikzpicture}
% \caption{Latent coalgebra under graph $G(m)=\set{(0,0),(1,0),(2,2)}$. $\TheLatentBehaviourOf{0}{m}=\TheLatentBehaviourOf{1}{m}=\emptyset$, $\TheLatentBehaviourOf{2}{m}=1^*$}
% \label{fig:3.4}
% \end{figure}

% \begin{figure}[t]
% \captionsetup{singlelinecheck=off}
% \centering
% \begin{tikzpicture}
% \node[state] (q1) {$0$};
% \node[state, right of=q1] (q2) {$1$};
% \node[state, right of=q2] (q3) {$2$};
% \draw (q1) edge[loop above] node{0} (q1)
% (q1) edge[bend left, above] node{1} (q2)
% (q2) edge[bend left, above] node{0} (q1)
% (q2) edge[bend left, above] node{1} (q3)
% (q3) edge[bend left, above] node{1} (q2)
% (q3) edge[bend left, below] node{0} (q1);
% \end{tikzpicture}
% \caption{Latent coalgebra under graph $G(m)=\set{(0,0),(1,1),(2,0)}$ . $\TheLatentBehaviourOf{0}{m}=\TheLatentBehaviourOf{1}{m}=\TheLatentBehaviourOf{2}{m}=\emptyset$}

% \begin{tikzpicture}
% \node[state] (q1) {$0$};
% \node[state, right of=q1] (q2) {$1$};
% \node[state, right of=q2] (q3) {$2$};
% \draw (q1) edge[loop above] node{0} (q1)
% (q1) edge[bend left, above] node{1} (q2)
% (q2) edge[bend left, above] node{0} (q1)
% (q2) edge[bend left, above] node{1} (q3)
% (q3) edge[loop above] node{1} (q3)
% (q3) edge[bend left, below] node{0} (q1);
% \end{tikzpicture}
% \caption{Latent coalgebra under graph $G(m)=\set{(0,0),(1,1),(2,1)}$ . $\TheLatentBehaviourOf{0}{m}=\TheLatentBehaviourOf{1}{m}=\TheLatentBehaviourOf{2}{m}=\emptyset$}

% \begin{tikzpicture}
% \node[state] (q1) {$0$};
% \node[state, right of=q1] (q2) {$1$};
% \node[state,accepting, right of=q2] (q3) {$2$};
% \draw (q1) edge[loop above] node{0} (q1)
% (q1) edge[bend left, above] node{1} (q2)
% (q2) edge[bend left, above] node{0} (q1)
% (q2) edge[bend left, above] node{1} (q3)
% (q3) edge[loop above] node{1} (q3)
% (q3) edge[bend left, below] node{0} (q1);
% \end{tikzpicture}
% \caption{Latent coalgebra under graph $G(m)=\set{(0,0),(1,1),(2,2)}$ (i.e. $m=\texttt{id}$). Latent behaviours under $m$ are the original behaviours: 
% \protect\begin{align*}
% 	\TheLatentBehaviourOf{0}{m}&=(0+1)^*111^*, \\
% 	\TheLatentBehaviourOf{1}{m}&=1+\TheLatentBehaviourOf{0}{m}\\
% 	\TheLatentBehaviourOf{2}{m}&=\varepsilon + 1+ \TheLatentBehaviourOf{0}{m}
% \protect\end{align*}
% }
% \label{fig:3.7}
% \end{figure}



% \begin{figure}[t]
% \captionsetup{singlelinecheck=off}
% \centering
% \begin{tikzpicture}
% \node[state] (q1) {$0$};
% \node[state,accepting, right of=q1] (q2) {$1$};
% \node[state, right of=q2] (q3) {$2$};
% \draw (q1) edge[loop above] node{0} (q1)
% (q1) edge[bend left, above] node{1} (q2)
% (q2) edge[bend left, above] node{0} (q1)
% (q2) edge[bend left, above] node{1} (q3)
% (q3) edge[bend left, above] node{1} (q2)
% (q3) edge[bend left, below] node{0} (q1);
% \end{tikzpicture}
% \caption{Latent coalgebra under graph $G(m)=\set{(0,0),(1,2),(2,0)}$. 
% \protect\begin{align*}
% \TheLatentBehaviourOf{0}{m}&=\TheLatentBehaviourOf{2}{m}=0^*1\TheLatentBehaviourOf{1}{m},\\
% \TheLatentBehaviourOf{1}{m}&=(00^*1+10^*1)^*
% \protect\end{align*}
% \label{fig:3.8}
% }

% \begin{tikzpicture}
% \node[state] (q1) {$0$};
% \node[state,accepting, right of=q1] (q2) {$1$};
% \node[state, right of=q2] (q3) {$2$};
% \draw (q1) edge[loop above] node{0} (q1)
% (q1) edge[bend left, above] node{1} (q2)
% (q2) edge[bend left, above] node{0} (q1)
% (q2) edge[bend left, above] node{1} (q3)
% (q3) edge[loop above] node{1} (q3)
% (q3) edge[bend left, below] node{0} (q1);
% \end{tikzpicture}
% \caption{Latent coalgebra under graph $G(m)=\set{(0,0),(1,2),(2,1)}$.
% \protect\begin{align*}
% \TheLatentBehaviourOf{0}{m}&=1+\TheLatentBehaviourOf{2}{m},\\
% \TheLatentBehaviourOf{1}{m}&=\varepsilon+1+\TheLatentBehaviourOf{2}{m},\\
% \TheLatentBehaviourOf{2}{m}&=(0+1)^*01
% \protect\end{align*}
% \label{fig:3.9}
% }

% \begin{tikzpicture}
% \node[state] (q1) {$0$};
% \node[state,accepting, right of=q1] (q2) {$1$};
% \node[state, accepting,right of=q2] (q3) {$2$};
% \draw (q1) edge[loop above] node{0} (q1)
% (q1) edge[bend left, above] node{1} (q2)
% (q2) edge[bend left, above] node{0} (q1)
% (q2) edge[bend left, above] node{1} (q3)
% (q3) edge[loop above] node{1} (q3)
% (q3) edge[bend left, below] node{0} (q1);
% \end{tikzpicture}
% \caption{Latent coalgebra under graph $G(m)=\set{(0,0),(1,2),(2,2)}$.
% \protect\begin{align*}
% \TheLatentBehaviourOf{0}{m}&=(0+1)^*1,\\ 
% \TheLatentBehaviourOf{1}{m}&=\TheLatentBehaviourOf{2}{m}=\varepsilon + \TheLatentBehaviourOf{0}{m}
% \protect\end{align*}
% \label{fig:3.10}
% }
% \end{figure}

% \begin{figure}[t]
% \captionsetup{singlelinecheck=off}
% \centering
% \begin{tikzpicture}
% \node[state] (q1) {$0$};
% \node[state, right of=q1] (q2) {$1$};
% \node[state, right of=q2] (q3) {$2$};
% \draw (q1) edge[loop above] node{0} (q1)
% (q1) edge[bend left, above] node{1} (q3)
% (q2) edge[bend left, above] node{0} (q1)
% (q2) edge[loop right] node{1} (q2)
% (q3) edge[bend left, above] node{1} (q2)
% (q3) edge[bend left, below] node{0} (q1);
% \end{tikzpicture}
% \caption{Latent coalgebra under graph $G(m)=\set{(0,1),(1,0),(2,0)}$. $\TheLatentBehaviourOf{0}{m}=
% \TheLatentBehaviourOf{1}{m}=\TheLatentBehaviourOf{2}{m}=\emptyset$}

% \begin{tikzpicture}
% \node[state] (q1) {$0$};
% \node[state, right of=q1] (q2) {$1$};
% \node[state, right of=q2] (q3) {$2$};
% \draw (q1) edge[loop above] node{0} (q1)
% (q1) edge[bend left, above] node{1} (q3)
% (q2) edge[bend left, above] node{0} (q1)
% (q2) edge[loop right] node{1} (q2)
% (q3) edge[loop above] node{1} (q3)
% (q3) edge[bend left, below] node{0} (q1);
% \end{tikzpicture}
% \caption{Latent coalgebra under graph $G(m)=\set{(0,1),(1,0),(2,1)}$. $\TheLatentBehaviourOf{0}{m}=
% \TheLatentBehaviourOf{1}{m}=\TheLatentBehaviourOf{2}{m}=\emptyset$}


% \begin{tikzpicture}
% \node[state] (q1) {$0$};
% \node[state, right of=q1] (q2) {$1$};
% \node[state,accepting, right of=q2] (q3) {$2$};
% \draw (q1) edge[loop above] node{0} (q1)
% (q1) edge[bend left, above] node{1} (q3)
% (q2) edge[bend left, above] node{0} (q1)
% (q2) edge[loop right] node{1} (q2)
% (q3) edge[loop above] node{1} (q3)
% (q3) edge[bend left, below] node{0} (q1);
% \end{tikzpicture}
% \caption{Latent coalgebra under graph $G(m)=\set{(0,1),(1,0),(2,2)}$. 
% \protect\begin{align*}
% \TheLatentBehaviourOf{0}{m}&=(0+1)^*1,\\ 
% \TheLatentBehaviourOf{1}{m}&=1^*0\TheLatentBehaviourOf{0}{m},\\
% \TheLatentBehaviourOf{2}{m}&=\varepsilon+\TheLatentBehaviourOf{0}{m}
% %\TheLatentBehaviourOf{2}{m}&=1^*+\TheLatentBehaviourOf{1}{m}
% \protect\end{align*}
% }
% \label{fig:3.13}
% \end{figure}

% \begin{figure}[t]
% \captionsetup{singlelinecheck=off}
% \centering
% \begin{tikzpicture}
% \node[state] (q1) {$0$};
% \node[state, right of=q1] (q2) {$1$};
% \node[state, right of=q2] (q3) {$2$};
% \draw (q1) edge[loop above] node{0} (q1)
% (q1) edge[bend left, above] node{1} (q3)
% (q2) edge[bend left, above] node{0} (q1)
% (q2) edge[bend left, above] node{1} (q3)
% (q3) edge[bend left, above] node{1} (q2)
% (q3) edge[bend left, below] node{0} (q1);
% \end{tikzpicture}
% \caption{Latent coalgebra under graph $G(m)=\set{(0,1),(1,1),(2,0)}$. $\TheLatentBehaviourOf{0}{m}=
% \TheLatentBehaviourOf{1}{m}=\TheLatentBehaviourOf{2}{m}=\emptyset$}


% \begin{tikzpicture}
% \node[state] (q1) {$0$};
% \node[state, right of=q1] (q2) {$1$};
% \node[state, right of=q2] (q3) {$2$};
% \draw (q1) edge[loop above] node{0} (q1)
% (q1) edge[bend left, above] node{1} (q3)
% (q2) edge[bend left, above] node{0} (q1)
% (q2) edge[bend left, above] node{1} (q3)
% (q3) edge[loop above] node{1} (q3)
% (q3) edge[bend left, below] node{0} (q1);
% \end{tikzpicture}
% \caption{Latent coalgebra under graph $G(m)=\set{(0,1),(1,1),(2,1)}$. $\TheLatentBehaviourOf{0}{m}=
% \TheLatentBehaviourOf{1}{m}=\TheLatentBehaviourOf{2}{m}=\emptyset$}


% \begin{tikzpicture}
% \node[state] (q1) {$0$};
% \node[state, right of=q1] (q2) {$1$};
% \node[state, accepting, right of=q2] (q3) {$2$};
% \draw (q1) edge[loop above] node{0} (q1)
% (q1) edge[bend left, above] node{1} (q3)
% (q2) edge[bend left, above] node{0} (q1)
% (q2) edge[bend left, above] node{1} (q3)
% (q3) edge[loop above] node{1} (q3)
% (q3) edge[bend left, below] node{0} (q1);
% \end{tikzpicture}
% \caption{Latent coalgebra under graph $G(m)=\set{(0,1),(1,1),(2,2)}$.
% \protect\begin{align*}
% \TheLatentBehaviourOf{0}{m}&=\TheLatentBehaviourOf{1}{m}=(0+1)^*1,\\ 
% \TheLatentBehaviourOf{2}{m}&=\TheLatentBehaviourOf{0}{m}+\varepsilon
% \protect\end{align*}
% }
% \end{figure}

% \begin{figure}[t]
% \centering
% \captionsetup{singlelinecheck=off}
% \begin{tikzpicture}
% \node[state] (q1) {$0$};
% \node[state,accepting, right of=q1] (q2) {$1$};
% \node[state, right of=q2] (q3) {$2$}; 
% \draw (q1) edge[loop above] node{0} (q1)
% (q1) edge[bend left, above] node{1} (q3)
% (q2) edge[bend left, above] node{0} (q1)
% (q2) edge[bend left, above] node{1} (q3)
% (q3) edge[bend left, above] node{1} (q2)
% (q3) edge[bend left, below] node{0} (q1);
% \end{tikzpicture}
% \caption{Latent coalgebra under graph $G(m)=\set{(0,1),(1,2),(2,0)}$.
% \protect\begin{align*}
% \TheLatentBehaviourOf{0}{m}&=(0+10+(11)^+(0+10))^*(11)^+ \\ %(0+1(0+1(11)^*(0+10)))^*11(11)^* ,\\ 
% \TheLatentBehaviourOf{1}{m}&=(11)^*+(0+10)\TheLatentBehaviourOf{0}{m},\\
% \TheLatentBehaviourOf{2}{m}&=0\TheLatentBehaviourOf{0}{m}+1\TheLatentBehaviourOf{1}{m}
% \protect\end{align*}
% \label{fig:3.17}
% }

% \begin{tikzpicture}
% \node[state] (q1) {$0$};
% \node[state,accepting, right of=q1] (q2) {$1$};
% \node[state, right of=q2] (q3) {$2$};
% \draw (q1) edge[loop above] node{0} (q1)
% (q1) edge[bend left, above] node{1} (q3)
% (q2) edge[bend left, above] node{0} (q1)
% (q2) edge[bend left, above] node{1} (q3)
% (q3) edge[loop above] node{1} (q3)
% (q3) edge[bend left, below] node{0} (q1);
% \end{tikzpicture}
% \caption{Latent coalgebra under graph $G(m)=\set{(0,1),(1,2),(2,1)}$.
% \protect\begin{align*}
% \TheLatentBehaviourOf{0}{m}&=\TheLatentBehaviourOf{2}{m}=\emptyset,\\ 
% \TheLatentBehaviourOf{1}{m}&=\varepsilon
% \protect\end{align*}
% \label{fig:3.18}
% }

% \begin{tikzpicture}
% \node[state] (q1) {$0$};
% \node[state,accepting, right of=q1] (q2) {$1$};
% \node[state, accepting, right of=q2] (q3) {$2$};
% \draw (q1) edge[loop above] node{0} (q1)
% (q1) edge[bend left, above] node{1} (q3)
% (q2) edge[bend left, above] node{0} (q1)
% (q2) edge[bend left, above] node{1} (q3)
% (q3) edge[loop above] node{1} (q3)
% (q3) edge[bend left, below] node{0} (q1);
% \end{tikzpicture}
% \caption{Latent coalgebra under graph $G(m)=\set{(0,1),(1,2),(2,2)}$.
% \protect\begin{align*}
% \TheLatentBehaviourOf{0}{m}&=(0+1)^*1,\\
% \TheLatentBehaviourOf{1}{m}&=\TheLatentBehaviourOf{2}{m}=\varepsilon+\TheLatentBehaviourOf{0}{m}
% \protect\end{align*}
% }
% \end{figure}

% \begin{figure}[t]
% \captionsetup{singlelinecheck=off}
% \centering
% \begin{tikzpicture}
% \node[state, accepting] (q1) {$0$};
% \node[state, right of=q1] (q2) {$1$};
% \node[state, right of=q2] (q3) {$2$};
% \draw (q1) edge[loop above] node{0} (q1)
% (q1) edge[bend left, above] node{1} (q3)
% (q2) edge[bend left, above] node{0} (q1)
% (q2) edge[loop right] node{1} (q2)
% (q3) edge[bend left, above] node{1} (q2)
% (q3) edge[bend left, below] node{0} (q1);
% \end{tikzpicture}
% \caption{Latent coalgebra under graph $G(m)=\set{(0,2),(1,0),(2,0)}$. 
% \protect\begin{align*}
% \TheLatentBehaviourOf{0}{m}&=(0+1)^*0+\varepsilon ,\\ 
% \TheLatentBehaviourOf{1}{m}&=(0+1)^*0
% \protect\end{align*} 
% \label{fig:3.20}
% }

% \begin{tikzpicture}
% \node[state, accepting] (q1) {$0$};
% \node[state, right of=q1] (q2) {$1$};
% \node[state, right of=q2] (q3) {$2$};
% \draw (q1) edge[loop above] node{0} (q1)
% (q1) edge[bend left, above] node{1} (q3)
% (q2) edge[bend left, above] node{0} (q1)
% (q2) edge[loop right] node{1} (q2)
% (q3) edge[loop above] node{1} (q3)
% (q3) edge[bend left, below] node{0} (q1);
% \end{tikzpicture}
% \caption{Latent coalgebra under graph $G(m)=\set{(0,2),(1,0),(2,1)}$.
% \protect\begin{align*}
% \TheLatentBehaviourOf{0}{m}&=(0+1)^*0+\varepsilon ,\\ 
% \TheLatentBehaviourOf{1}{m}&=(0+1)^*0
% \protect\end{align*}
% }

% \begin{tikzpicture}
% \node[state, accepting] (q1) {$0$};
% \node[state, right of=q1] (q2) {$1$};
% \node[state, accepting, right of=q2] (q3) {$2$};
% \draw (q1) edge[loop above] node{0} (q1)
% (q1) edge[bend left, above] node{1} (q3)
% (q2) edge[bend left, above] node{0} (q1)
% (q2) edge[loop right] node{1} (q2)
% (q3) edge[loop above] node{1} (q3)
% (q3) edge[bend left, below] node{0} (q1);
% \end{tikzpicture}
% \caption{Latent coalgebra under graph $G(m)=\set{(0,2),(1,0),(2,2)}$. 
% \protect\begin{align*}
% \TheLatentBehaviourOf{0}{m}&=\TheLatentBehaviourOf{2}{m}=(0+1)^*,\\ 
% \TheLatentBehaviourOf{1}{m}&=1^*0\TheLatentBehaviourOf{0}{m}
% \protect\end{align*}
% \label{fig:3.22}
% }
% \end{figure}

% \begin{figure}[t]
% \captionsetup{singlelinecheck=off}
% \centering
% \begin{tikzpicture}
% \node[state, accepting] (q1) {$0$};
% \node[state, right of=q1] (q2) {$1$};
% \node[state, right of=q2] (q3) {$2$};
% \draw (q1) edge[loop above] node{0} (q1)
% (q1) edge[bend left, above] node{1} (q3)
% (q2) edge[bend left, above] node{0} (q1)
% (q2) edge[bend left, above] node{1} (q3)
% (q3) edge[bend left, above] node{1} (q2)
% (q3) edge[bend left, below] node{0} (q1);
% \end{tikzpicture}
% \caption{Latent coalgebra under graph $G(m)=\set{(0,2),(1,1),(2,0)}$.
% \protect\begin{align*}
% \TheLatentBehaviourOf{0}{m}&=(0+1)^*0+\varepsilon,\\ 
% \TheLatentBehaviourOf{1}{m}&=\TheLatentBehaviourOf{2}{m}=(0+1)^*0\\
% \protect\end{align*}
% }

% \begin{tikzpicture}
% \node[state, accepting] (q1) {$0$};
% \node[state, right of=q1] (q2) {$1$};
% \node[state, right of=q2] (q3) {$2$};
% \draw (q1) edge[loop above] node{0} (q1)
% (q1) edge[bend left, above] node{1} (q3)
% (q2) edge[bend left, above] node{0} (q1)
% (q2) edge[bend left, above] node{1} (q3)
% (q3) edge[loop above] node{1} (q3)
% (q3) edge[bend left, below] node{0} (q1);
% \end{tikzpicture}
% \caption{Latent coalgebra under graph $G(m)=\set{(0,2),(1,1),(2,1)}$. 
% \protect\begin{align*}
% \TheLatentBehaviourOf{0}{m}&=(0+1)^*0+\varepsilon,\\ 
% \TheLatentBehaviourOf{1}{m}&=\TheLatentBehaviourOf{2}{m}=(0+1)^*0\\
% \protect\end{align*}
% }

% \begin{tikzpicture}
% \node[state, accepting] (q1) {$0$};
% \node[state, right of=q1] (q2) {$1$};
% \node[state, accepting, right of=q2] (q3) {$2$};
% \draw (q1) edge[loop above] node{0} (q1)
% (q1) edge[bend left, above] node{1} (q3)
% (q2) edge[bend left, above] node{0} (q1)
% (q2) edge[bend left, above] node{1} (q3)
% (q3) edge[loop above] node{1} (q3)
% (q3) edge[bend left, below] node{0} (q1);
% \end{tikzpicture}
% \caption{Latent coalgebra under graph $G(m)=\set{(0,2),(1,1),(2,2)}$ . 
% \protect\begin{align*}
% \TheLatentBehaviourOf{0}{m}&=\TheLatentBehaviourOf{2}{m}=(0+1)^*,\\ 
% \TheLatentBehaviourOf{1}{m}&=0\TheLatentBehaviourOf{0}{m}+1\TheLatentBehaviourOf{2}{m}
% \protect\end{align*}
% \label{fig:3.25}
% }
% \end{figure}

% \begin{figure}[t]
% \captionsetup{singlelinecheck=off}
% \centering
% \begin{tikzpicture}
% \node[state,accepting] (q1) {$0$};
% \node[state, accepting, right of=q1] (q2) {$1$};
% \node[state, right of=q2] (q3) {$2$};
% \draw (q1) edge[loop above] node{0} (q1)
% (q1) edge[bend left, above] node{1} (q3)
% (q2) edge[bend left, above] node{0} (q1)
% (q2) edge[bend left, above] node{1} (q3)
% (q3) edge[bend left, above] node{1} (q2)
% (q3) edge[bend left, below] node{0} (q1);
% \end{tikzpicture}
% \caption{Latent coalgebra under graph $G(m)=\set{(0,2),(1,2),(2,0)}$. 
% \protect\begin{align*}
% \TheLatentBehaviourOf{0}{m}&=\TheLatentBehaviourOf{1}{m}=(0+10+11)^*,\\ 
% \TheLatentBehaviourOf{2}{m}&=(0+1)\TheLatentBehaviourOf{0}{m}
% \protect\end{align*}
% \label{fig:3.26}
% }

% \begin{tikzpicture}
% \node[state,accepting] (q1) {$0$};
% \node[state, accepting, right of=q1] (q2) {$1$};
% \node[state, right of=q2] (q3) {$2$};
% \draw (q1) edge[loop above] node{0} (q1)
% (q1) edge[bend left, above] node{1} (q3)
% (q2) edge[bend left, above] node{0} (q1)
% (q2) edge[bend left, above] node{1} (q3)
% (q3) edge[loop above] node{1} (q3)
% (q3) edge[bend left, below] node{0} (q1);
% \end{tikzpicture}
% \caption{Latent coalgebra under graph $G(m)=\set{(0,2),(1,2),(2,1)}$.
% \protect\begin{align*}
% \TheLatentBehaviourOf{0}{m}&=\TheLatentBehaviourOf{1}{m}=(0+1)^*0+\varepsilon\\ 
% \TheLatentBehaviourOf{2}{m}&=(0+1)^*0
% \protect\end{align*}
% }

% \begin{tikzpicture}
% \node[state,accepting] (q1) {$0$};
% \node[state, accepting, right of=q1] (q2) {$1$};
% \node[state, accepting, right of=q2] (q3) {$2$};
% \draw (q1) edge[loop above] node{0} (q1)
% (q1) edge[bend left, above] node{1} (q3)
% (q2) edge[bend left, above] node{0} (q1)
% (q2) edge[bend left, above] node{1} (q3)
% (q3) edge[loop above] node{1} (q3)
% (q3) edge[bend left, below] node{0} (q1);
% \end{tikzpicture}
% \caption{Latent coalgebra under graph $G(m)=\set{(0,2),(1,2),(2,2)}$. 
% \protect\begin{align*}
% \TheLatentBehaviourOf{0}{m}&=\TheLatentBehaviourOf{1}{m}=\TheLatentBehaviourOf{2}{m}=(0+1)^*
% \protect\end{align*}
% }
% \label{fig:3.28}
% \end{figure}

\end{example}
\todo[inline]{If we want $a$ to change with time there is no need to do anything fancy! We can enhance the carrier by doing $X'=X\times \mathbb{N}$ or $X'=X\times [X\rightarrow 2]$; with this, $X$ is enhanced by a natural number counter or a set of conditions to make the dynamics of the transformation coalgebra more interesting. Maybe there is even no need to do changes, it all depends on how informative $X$ is. let's see}

\section{Latent Vulnerabilities}
Not all behaviours are latent for a given $F$-coalgebra. In Example~\ref{ex:LatentBehaviour}, we see that no transformation can yield the language $0^*$ as a latent behaviour. We see latent behaviours as targets for attackers, and we would like to know if there is a way for an attacker to cause the system to shift to a particular latent behaviour by means of a transformation. For that purpose, we introduce the definition of {latent vulnerability problems}.

\begin{definition}
A \emph{latent vulnerability problem} consists of a given a pointed $F$-coalgebra $\mathbb{X}=(X,c,x_0)$ and a given behaviour $\rho\in \sigma F$. To solve this problem, we need to either 1) find a transformation $m$ which proves that $\rho$ is the latent behaviour of $x_0$ under $m$ or 2) prove that no such $m$ can exist.
\end{definition}

\todo[inline]{To coinductively define $m$, we would already need to know $\mathbb{X}\circ m$, so that is not an option. Instead, we can play a game with the coalgebra as follows. We start at the initial state $x_0$, and we take turns; we play first, and we are allowed to map $x_0$ to another state (i.e. define $m(x_0)$. Once we do, the coalgebra computes $c(m(x_0))$ and, on input consumed, and if the coalgebra has transition dynamics, we can continue the game from the new state.}

\todo[inline]{Can we define  $\mathbb{X}\circ m$ based on what we know?. Let us consider the functor $O\times X^I$. We can try to solve the problem in two ways: based on bisimilarity or based on backpropagation (which are somewhat connected).}


\begin{align}
\gamma(m(x))&=\gamma_\omega(\rho)\\
\delta(m(x))(i)&\sim \delta_\omega(\rho)(i), \forall i\in I
\end{align}

\todo[inline]{This means that the problem becomes a "searching for the next candidate at every step" problem. Alternatively, you could learn to compose solutions too. }
\todo[inline]{Maybe we can also do something exciting: eventuality -> at some point in time, the behaviour you want becomes apparent, i.e. $\TheBehaviourOf{x_0}^w=\rho$ for some $w$}

 
\section{Attackers and Attacks}
There is an attacker model associated with latent vulnerabilities: attackers that can change the value of state variables, but not the program that defines the system. This later attacker would correspond to one that can arbitrarily change the coalgebra, and is in that sense a much stronger attacker. If the attacker can enforce any behaviour they want, it becomes too powerful to defend against; all coalgebras are vulnerable, and no fix is available. However, an attacker that exploits a latent vulnerability is a bit in the middle: it can change the coalgebra by means of a transformation/attack function, but only through those. In that sense, not every behaviour is at the reach of the attacker, only those who are latent.

We can do a further refinement: we can imagine attackers that control particular components of a state by partitioning states into controllable parts and observable parts
\todo[inline]{Is there a difference? We could show that every attacker that controls a component induces a transformation, and all transformations can be modelled by attackers that control a certain number of components. Right?}

\begin{definition}[Attack]
Given an $F$-coalgebra $(X,c)$, the set of its possible attacks corresponds to the set of consistent transformations of $X$, i.e., $X^X_\sim$. 
\end{definition}
Different $F$-coalgebras have different sets of attacks. In our framework, a minimal $F$-coalgebra $(X,c)$ has as many attacks as $|X|^{|X|}$.

Intuitively, an attacker can be modelled by the set of attacks they have at their disposal. However, we avoid defining attackers by enumerating their attacks, since there may be infinite, and instead we will define them by the set of components of the system that they control. 

\begin{definition}[Attacker]
Given a functor $F$ and an $F$-coalgebra $\mathbb{X}=(\vec{X},c)$, such that $\vec{X}$ has components $\Pi=\set{\pi_1, \ldots, \pi_n}$, an \emph{attacker} of $\mathbb{X}$, say $A$, is a {finite} subset of $\Pi$ whose semantics is a set of attacks defined by the function $\texttt{attacks}\colon \TheFinitePowersetOf{\Pi}\rightarrow\ThePowersetOf{X^X_\omega}$ as follows. 
For $j\in \set{1, \ldots,  n}$ and all $x\in X$,
\begin{align}
%\texttt{attacks}(A)=\set{f\in X^X_\omega | \text{ $f$ is consistent, and if $\pi_j \not\in A$, then $f(x)[\pi_j]=x[\pi_j]$}}.
\texttt{attacks}(A)=\set{f\in X^X_\omega | \text{ if $\pi_j \not\in A$, then $f(x)[\pi_j]=x[\pi_j]$}}.
\end{align}
%{\color{red}
%equivalently
%\begin{align}
%\texttt{attacks}(A)=\set{f\in X^X_\omega | \text{if $f(x)[\pi_j]\neq x[\pi_j]$, then $\pi_j \in A$, for all $x\in X$}}.
%\end{align}
%}
In other words, $A$ has access to the set of attacks that can only affect the components that $A$ has direct control of.
 \end{definition}
 \todo[inline]{Commented is the definition with ranges, but it is kinda complicated. For spectre, we define the strategy first, then we think of what attacker we are, and for vulnerabilities, we will define the attacker first, and then we find the strategy.}
% 
% \begin{definition}[Attacker]
%Given a functor $F$ and an $F$-coalgebra $\mathbb{X}=(\vec{X},c)$, such that $\vec{X}$ has components $\Pi=\set{\pi_1, \ldots, \pi_n}$, an \emph{attacker} of $\mathbb{X}$, say $A$, is a pairing of a {finite} subset of $\Pi$, namely $A[\Pi]$, and a map $A[\texttt{range}]$ that associates each component in $A[\Pi]$ with a range of possible values; formally, $A[\texttt{range}(\pi_j)]\subseteq \Pi_j$, for all $\pi_j \in A[\Pi]$. %Formally, the type of attackers is $\TheFinitePowersetOf{\Pi} \times \ThePowersetOf{\Pi \times \vec{X}}$
%
%The semantics of an attacker is given by the set of attacks defined by the function $\texttt{attacks}\colon \texttt{Attackers}\rightarrow\ThePowersetOf{X^X_\omega}$ as follows. 
%For $j\in \set{1, \ldots,  n}$,  
%\begin{align}
%\texttt{attacks}(A)&\triangleq\set{f\in X^X_\omega | \text{ if $\pi_j \not\in A$, then $f(x)[\pi_j]=x[\pi_j]$, for all $x\in X$}}\cap\\
%&\set{f\in X^X_\omega | \text{ if $\pi_j \in A$, then $f(x)[\pi_j]\in \texttt{control}(A)(\pi_j)$, for all $x\in X$}}
%\end{align}
%{\color{red}
%equivalently
%\begin{align}
%\texttt{attacks}(A)=\set{f\in X^X_\omega | \text{if $f(x)[\pi_j]\neq x[\pi_j]$, then $\pi_j \in A$, for all $x\in X$}}.
%\end{align}
%}
%In other words, we associate to $A$ the set of attacks where, if $A$ does not contain the component $\pi_j$, then the attacks cannot change the value of $\pi_j$ when mutating states.
% \end{definition}
% 
% 
 \todo[inline]{We can refine this definition of attackers with $(\pi_j, Y_j)$, with $Y_j\subseteq X_j$, instead of just $\pi_j$. HoWEVER, I highly encourage against this, because it overcomplicates things unnecessarily.}
 \todo[inline]{I love writing $j\in n$, but this means enumeration should start from 0...I can probably use that in my language though}
 \begin{example}[Attackers of Example~\ref{ex:LatentBehaviour} and their Semantics]
 The carrier $2\times 2$ has two components, $\fst$ and $\snd$, so there are four attackers: $\emptyset$, $\set{\fst}$, $\set{\snd}$ and $\set{\fst,\snd}$. The attacks of the attacker $\emptyset$ do not mutate any component, i.e., $\texttt{attacks}(\emptyset)=\set{\id}$, and the attacks of $\set{\fst}$ and $\set{\snd}$ only affect the first/second component, respectively. The attacker $\set{\fst,\snd}$ has access to all attacks.
 \end{example}
 
 \subsection{Attacks over Discrete Time Systems}
 \todo[inline]{DO NOT FORGET THAT YOU CAN ASSUME $x(t)\neq x(t+k)$ for all $k$, because otherwise, you would have taken a different decision before!!! it's like going on a loop on a chessboard.}
Recall that we model discrete time systems with pointed $\DTS$-coalgebras of the functor $\DTS=O\times X^I$ for some fixed sets $O$ and $I$. Instead of describing attacks as some general mapping over the carrier set, we might want to describe attacks as changes in the current state. We call this notion of individual transformation a \emph{strategy}, and it is formalised as follows:
\begin{definition}[Strategy]
Given a $\DTS$-coalgebra $\TheCoalgebra=(X,\gamma,\delta,x_0)$, a \emph{strategy} $\lambda$ consists of a sequence of attacks...
\todo[inline]{Why do I want this?? Strategies are useful representations of attacks...but why?}
\end{definition}

\begin{proposition}[Every Strategy is an Attack]
For all $\DTS$-coalgebras $\TheCoalgebra=(X,\gamma,\delta,x_0)$ and every strategy $\lambda\in\Lambda$\todo{complete}
\end{proposition}
 
 \section{A Partial Order of Attackers}
\todo[inline]{We know how to quantify attackers in terms of exact latent behaviours, but latent behaviours themselves have an order (they are languages i.e. sets). What about the relationship among them? Do stronger attackers need to recognise more languages, what if a language implies another? how do attackers relate there then?}
\todo[inline]{Introduce the emulation problem.}

\begin{definition}[Attack Ordering]
Given two attacks $m_1, m_2$ and an $F$-coalgebra $\TheCoalgebra$, we say that $m_1\lesssim_\TheCoalgebra m_2$ if and only if $\supp(m_1) \subseteq \supp(m_2)$, and $m_1(x)\sim_\TheCoalgebra m_2(x)$ for all $x\in \supp(m_1)$. In other words, any $x$ that $m_1$ mutates can also be mutated by $m_2$ in the same way, in the context of the coalgebra $\TheCoalgebra$. We omit the subscript $\TheCoalgebra$ when it is clear from the context, or it is irrelevant.
\end{definition}
This notion of attack ordering is a bit more general than one that uses equality (i.e., requiring $m_1(x)= m_2(x)$ instead of $m_1(x)\sim_\TheCoalgebra m_2(x)$), as it lets us focus directly on behaviour, and not on structure. If the coalgebra $\TheCoalgebra$ is minimal, then $m_1(x)= m_2(x)$ if and only if $m_1(x)\sim_\TheCoalgebra m_2(x)$, due to the coinduction proof principle.

\begin{definition}[Attack Execution]
Given an attacker $A$, an $F$-coalgebra $\TheCoalgebra=(X,c)$, and an arbitrary attack $m\in X_\omega^X$, we say that \emph{$A$ can execute the attack $m$} if and only if there exists an attack $m'\in \texttt{attacks}(A)$ such that $m\lesssim m'$.
%\begin{align}
%m' (x)\sim m (x), \quad \text{for all $x\in \supp(m)$}.
%\end{align}
%In such a case, we write $m'\lesssim m$.
\end{definition}

\begin{definition}[Attacker Ordering]
Let $\TheCoalgebra$ be an $F$-coalgebra whose components are $\Pi$. The partial order relation $\leq$ in the set of attackers $\TheFinitePowersetOf{\Pi}$ by its set inclusion, i.e. $A_1 \leq A_2$ if and only if $A_1 \subseteq A_2$. 
\end{definition}
This attacker ordering is \emph{monotonic} with respect to the \texttt{attacks} function: if $A_1 \leq A_2$, then any attack carried out by $A_1$ can be executed by $A_2$. %With it, we can define minimal attackers.

Monotonicity offers us mainly two advantages: 1) we can plan the order of attackers to be checked, and 2) we can propagate the results in case a solution fails.
\todo[inline]{Explain this better. You really need to have clear notions of solution and problems.}
% \begin{definition}[Minimal Attacker]
% \todo[inline]{Probably one of the most esoteric definitions here. It is wrt a particular problem? a particular attack, a property?}
% \end{definition}

\todo[inline]{Should we just focus on DTS? NO!  The DTS I wanted to use does not have loops, so it is clear that we can keep doing this in the current modelling.}

\begin{definition}[Emulation Problem]
We say 
\end{definition}
 
 \section{Solving Latent Vulnerability Problems}
 
 \subsection{Alternative formulations}
 \begin{definition}[Target Value]
Given an $F$-coalgebra $\TheCoalgebra=(X,c)$ with components in $\Pi$, a \emph{target value} is a proposition $x[\pi]==v$, modelled by the triple $(x,\pi,v)$ where $x\in X$, $\pi\in \Pi$ and $v$ is a value in the range of component $\pi$.
\end{definition}
 
\subsection{Exhaustive Search}
Given an attacker $A$ of an $F$-coalgebra $\TheCoalgebra=(X,c)$ and a target behaviour $\sigma$, we could perform an exhaustive search over the domain of its attacks if the carrier $X$ is finite.  A naive, exhaustive search would choose an attack $m$, mutate $\TheCoalgebra$, then perform a bisimulation check to see if there exists a state $x$ such that its latent behaviour $\TheLatentBehaviourOfIn{x}{m}{\TheCoalgebra}$ is equal to $\sigma$.

Since the search space is finite, if we cannot find an attack $m$ within the capabilities of $A$, then the system $\TheCoalgebra$ is safe with respect to $A$.

\begin{example}
\todo[inline]{For the example we can show that it can be solved}
\end{example}


\subsection{Using SMT Solvers}
Due to our formulation, we have a perfect information deterministic game, like Chomp~\cite{Chomp}. Our idea is to leverage an SMT solver to give us a winning move for the attacker, or prove that such move does not exist. For the remainder of this section, we focus on $\DTS$-coalgebras.

\todo[inline]{Rephrase this as a reachability problem. i.e. instead of a target value, use a set of goal states defined by the target values.}
\todo[inline]{HERE I AM}
\begin{definition}[Winning Latency Games]
Let $A$ be an  attacker, $\TheCoalgebra=(X,\gamma,\delta,x_0)$ be a pointed $\DTS$-coalgebra, and consider a set of target values $(\pi_1,v_1),...,(\pi_n,v_n)$. We say that the attacker $A$ \emph{wins the latency game in zero steps} if and only if there exists an attack $m$ in $\texttt{attacks}(A)$ such that $m(x_0)[p_k]=v_k$, for $k=1..n$. If a solution is found in zero steps, then all target components $\pi_1$ to $\pi_n$ must be in control of $A$, i.e., $\set{\pi_1, \ldots, \pi_n}\subseteq A$.

The attacker $A$ \emph{wins the latency game in $\tau$ steps given the inputs $(i_1, \ldots, i_\tau)$ with the attack $m$ at state $x_{\tau}$} if and only if there exists a sequence of states $(x_1, \ldots, x_\tau)$, where
\begin{align}
%m_{\tau+1}(x_{\tau +1})=x_\tau(
x_{t+1}=\delta(m (x_{t}))(i_t), \quad \text{for $t=0..{\tau-1}$}
\end{align}
and
\begin{align}
m(x_\tau)[p_k]=v_k,\quad \text{ for $k=1..n$.}
\end{align}
\end{definition}


\begin{proposition}[Composition through Emulation]
\todo[inline]{There are several ways to compose attackers: through $\delta$ and through $m$. The message I want to carry across is the following: if you know that you can solve a latency game from a state $x$, then 
 I already have a notion of winning over time, so I just need to emulate the attack of other attackers to reach my goal.}
If an attacker $A$ can win a latency game in $\tau$ steps given the inputs $i_1, \ldots i_{\tau}$ in the pointed $\DTS$-coalgebra $(X,\gamma,\delta, x_{0})$ with an attack $m$, then an attacker $A'$ can win the same latency game in $\tau+t$ steps given the input $i_1, i_2, \ldots i_{\tau+1}$ in the pointed $\DTS$-coalgebra $(X,\gamma,\delta, x_{-t})$ if and only if 
\begin{align}
\delta
\end{align}

%Given a pointed $\DTS$-coalgebra $(X,\gamma,\delta, x_0)$, an attacker $A_0$ can win a given latency game in ${\tau+1}$ steps given the inputs $(i_1, \ldots, i_\tau, i_{\tau+1})$ with some attack $m$ at some state $x_{\tau+1}$ if and only if there exists an attacker $A_{\tau}$ that can win the given latency game in one step in the pointed $\DTS$-coalgebra $(X,m\gamma,\delta, x_{\tau})$.
\end{proposition}
\begin{proof}
The main objective of $A_0$ is to reach state $x_{\tau}$ and from there 
\end{proof}


%\begin{proposition}
%\label{sec:Incompleteness}
%Given an arbitrary $F$-coalgebra $\mathbb{X}=(X,c)$ and a behaviour $\rho\in \sigma F$, there may not exist a transformation $m\colon X\rightarrow X$ such that $\rho$ is a latent behaviour of $\mathbb{X}$ under $m$.
%\end{proposition}
%\begin{proof}
%We provide a counterexample for the opposite proposition, that is, there exists a transformation $m\colon X\rightarrow X$ such that $\rho$ is a latent behaviour of $\mathbb{X}$ under $m$, for all $F$-coalgebras and all behaviours in $\sigma F$. 
%Consider, for the functor $F=2\times \texttt{id}^2$, a single-state $F$-coalgebra that accepts all sequences; this $F$-coalgebra cannot display a latent behaviour different from its original behaviour.
%\end{proof}

We would like to consider two problems related to latent behaviours: 
\begin{itemize}
\item How can we use transformations ourselves to repurpose a system that is already been defined?
\item How can an attacker use transformations to force a behaviour they want?
\end{itemize}
\todo[inline]{Note that, ultimately, both questions need a method to solve an equation for a transformation. That is, given a target behaviour and an a source coalgebra, how do you solve the problem of finding a transformation that helps you display the behaviour you want? Is it even possible?}



