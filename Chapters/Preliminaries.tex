%!TEX root = ../main.tex
% Chapter Template


\chapter{Preliminaries and Notation} % Main chapter title
\label{ch:Preliminaries} % Change X to a consecutive number; for referencing this chapter elsewhere, use \ref{ChapterX}

\todo[inline]{THIS SECTION HAS CONTENT THAT I DID NOT DEVELOP IN THE THESIS. The newer content comes in the following chapters}
%----------------------------------------------------------------------------------------
%	SECTION 1
%----------------------------------------------------------------------------------------
\section{Notation and Constructions on Sets and Functions}
\subsection{Sets and elements}
We denote \emph{sets} by upper-case letters $X,Y,Z,$ etc, and we denote their \emph{elements} by lower-case letters $x, y, z, $ etc. 

We denote the set of natural numbers by $\mathbb{N}$ and the set of natural numbers without zero by $\mathbb{N}^+$. Similarly, we denote the set of real numbers by $\mathbb{R}$ and the set of positive real numbers (excluding zero) by $\mathbb{R}^+$.

\subsection{Functions}
We denote \emph{functions} by lower-case letters $f,g,h,$ etc. A function $f\colon X\rightarrow Y$ maps the elements from the set $X$ to elements of the set $Y$. An \emph{endofunction} is a function $g\colon X\rightarrow X$ which maps a set $X$ to itself. We denote function composition by $\circ$. 
We say that an endofunction $h\colon X \rightarrow X$ has finite support iff $h(x)\neq x$ for a finite number of $x$ in $X$, even if $X$ is infinite.

\subsection{Range set}
A \emph{range set} $n$ where $n\in \mathbb{N}^+$ is a set of $n$ elements, i.e., $n=\set{0,1, \ldots, n-1}$. It is usually clear from the context when $n$ represents a number or a range set. We highlight two important range sets: the range set 1 and the range set 2.

The range set 1, which is equal to the set $\set{0}$, serves as a constant for several set operations (modulo isomorphism).
The range set 2, which is equal to $\set{0,1}$, to model booleans: 0 models $\false$ and $1$ models $\true$. 

\subsection{Exponential Set}
The \emph{exponential set} $Y^X$ is the set of all functions of type $X\rightarrow Y$. The exponential set $1\rightarrow X$ is isomorphic to $X$; $x\in X$ if and only if exists $f\colon 1\rightarrow X$ such that $f(0)=x$. (This seems overcomplicated at first, but we use this equivalence when discussing the following concepts of $F$-algebra and $F$-coalgebra.)

\subsection{Sum Set}
The \emph{sum set} $X + Y$ is the disjoint union of $X$ and $Y$, i.e. 
\begin{align*}
    X + Y\triangleq\set{ \iota_1(x) |x\in X} \cup \set{\iota_2(y) | y\in Y},
\end{align*}
where $\iota_1$ and $\iota_2$ act as tags.
\subsection{Product Set}
The \emph{product set} $X\times Y$ is the cartesian product of $X$ and $Y$, i.e. 
\begin{align*}
    X\times Y\triangleq\set{(x,y)|x\in X, y\in Y}.
\end{align*}
The \emph{finite product set} of sets $X_1, \ldots, X_n$ is their cartesian product; i.e., $X_1\times \ldots \times X_n$. We denote product sets by upper-case letters with an arrow above $\vec{X},\vec{Y}, \vec{Z}$, etc.
\subsection{Vector/Tuple}
Given a finite product set $\vec{X}=X_1\times \ldots \times X_n$, we denote the elements of $\vec{X}$ by $\vec{x}$, $\vec{y}$, $\vec{z}$, etc. We call these elements \emph{vectors} or, alternatively, \emph{tuples}. 

\subsection{Coordinates}
Given a finite product set $\vec{X}=X_1\times \ldots \times X_n$, its \emph{coordinates} are $\pi_1, \ldots, \pi_n$, where $\pi_i\colon \vec{X} \rightarrow X_i$ extract the $i$-th value of a vector; i.e., $\pi_i(x_1, \ldots, x_n)\triangleq x_i$, for $1\leq i \leq n$. We often write the expression $\vec{x}[\pi]$ instead of $\pi(\vec{x})$ when $\pi$ is a coordinate.

\subsection{Pair Function}
Given two functions $f\colon X\rightarrow Y$ and $g\colon X\rightarrow Z$, the \emph{pair function} $(f,g)\colon X\rightarrow Y\times Z$ is defined by $(f,g)(x)=(f(x),g(x))$.

\subsection{Product Function}
Given two functions $f\colon X\rightarrow Y$ and $g\colon A\rightarrow B$, the \emph{product function} $f\times g\colon X\times A\rightarrow Y\times B$ is defined by $f\times g(x,a)=(f(x),g(a))$.

\subsection{Function Mapping}
Given a set $X$ and a function $f\colon X\rightarrow Y$, the \emph{mapping of $f$ over $X$}, denoted $f(X)$, is the set defined by $f(X)\triangleq\set{f(x)| x\in X}$.

\section{Category Theory Preliminaries and Notation}

\subsection{Categories, Objects and Arrows}
A \emph{category} $\AsCategory{C}$ is a mathematical concept similar to a graph, comprised of two aspects: a collection of \emph{objects}, denoted $Obj(\AsCategory{C})$, and a collection of \emph{arrows} among those objects, denoted $Arr(\AsCategory{C})$. Each category satisfies the following  rules:
\begin{itemize}
    \item every object $X$ has an identity arrow that maps $X$ to itself, denoted $\id_X$,
    \item given the objects $X, Y$ and $Z$, and the arrows $X\xrightarrow{f}Y$ and $Y\xrightarrow{g}Z$, the arrow $X\xrightarrow{g\circ f}Z$ exists; i.e. the composition of arrows is well defined,
    \item the identity arrows act as units for arrow composition; i.e., for $X\xrightarrow{f}Y$, the equality $\id_{Y}\circ f=f=f\circ \id_{X}$ holds.
    \item the composition of arrows is associative; i.e., $(h\circ g)\circ f = h \circ (g\circ f)$.
\end{itemize}

\subsection{Commutative Diagram}

\subsection{Initial and Terminal Objects}
A category $\AsCategory{C}$ has an \emph{initial object}, often denoted by 0, if there exists a unique arrow from $0$ to every object $o$ in the category. Dually, a category $\AsCategory{C}$ has a \emph{final object}, often denoted by 1, if there exists a unique arrow from every object $o$ in the category to 1. 

\subsection{The Category of Sets and Functions}
The \emph{category of sets and functions} $\AsCategory{Set}$ is the category whose objects are sets and whose arrows are functions. 

\subsection{Functor}

\section{Universal (Co)Algebra Preliminaries and Notation}
\label{sec:Preliminaries:Coalgebras}
\todo[inline]{Make sure you create a wonderful introduction to the world of (co)algebras and explain why you want to use them. They are a wonderful way to unify the formalism of the thesis.}
\subsection{Monoids and Sequences}
\todo[inline]{Concatenation}
\subsection{Languages and Automata}
\todo[inline]{Complete here}
%\todo[inline]{*Puts an Edna Mode face* NO MONADS (unless necessary)}
%\subsection{$\Monad$-algebras, $\Functor$-coalgebras and $\lambda$-bialgebras}
\subsection{$F$-coalgebras and their Homomorphisms}
An \emph{$F$-coalgebra} for a functor $F$ is a pair $(X,X\xrightarrow{c}F(X))$ of an object $X$ and a morphism $c\colon X\rightarrow F(X)$. We use $F$-coalgebras to model dynamic systems with state. 
\begin{example}
We can model systems whose states have one component of type $O$ and that can perform state transitions on inputs of type $I$ by means of the functor $F(X)=O\times X^I$. An $F-$coalgebra is of the form $(X,X\xrightarrow{(o,\delta)}O\times X^I)$, with $o\colon X\rightarrow O$ and $\delta\colon X\rightarrow X^I$; the function $o$ lets us explore the component in the state $x$, and $\delta$ lets us perform transitions. We often write $x.o$ instead of $o(x)$, and we write $x^i$ as a shorthand for $\delta(x)(i)$. Deterministic automata that recognise sequences from an alphabet $A$ can be modelled with coalgebras of the functor $G(X)=2\times X^A$.
\end{example}
\subsection{Pointed coalgebras}
Pointed coalgebras are coalgebras with a distinguished state, usually referred to as the \emph{initial state}. Formally, for a functor $F$, a pointed $F$-coalgebra is a triple $(X,c, x_0)$ where $(X,c)$ is an $F$-coalgebra and $(X, x_0)$ is a $\Delta_X$-algebra, i.e., $x_0\colon 1\rightarrow X$, and $x_0(\star)$ characterises the initial state.
\subsection{States and Components}
We often use vectors/tuples as a states. We access the values inside states by means of projection functions. If the carrier is a product set $\vec{X}=Y_1 \times\ldots Y_n$, we can use, for $j=1..n$, the projection functions $\pi_j\colon \vec{X}\rightarrow Y_j$ defined, for $\vec{x}=(v_1,\ldots,v_n)$, by $\pi_j(\vec{x})\triangleq v_j$. In this case, we say that $\pi_1$ to $\pi_n$ are the \emph{components} of $\vec{X}$ and its elements. Since both coalgebras and components are functions from the carrier, we use bracket notation for component to distinguish them, i.e., we write $\vec{x}[\pi_j]$ instead of $\pi_j(\vec{x})$.  %, or, alternatively the \emph{state variables} of states in $X$. 
Whenever we write a state or a carrier  with arrows above them (i.e., $\vec{x}$ or $\vec{X}$), we imply that they have more than one component. Note that if the carrier $X$ has only one component, then it must be the identity function $\id\colon X\rightarrow X$. 
\todo[inline]{Consider $\vec{x}.\pi_j$ too, called dot notation}
%A \emph{$\Monad$-algebra} for a monad $\Monad=(\MFunctor,\eta,\mu)$ is a pair $(\TheSet,\Monad(\TheSet)\xrightarrow{a}\TheSet)$ of an object $\TheSet$ and a morphism $a\colon \Monad(\TheSet)\rightarrow\TheSet$ which, due to the relationship between monads and adjunctions, satisfies two properties: the \emph{multiplication square} and the \emph{unit triangle}, shown in Figure~\ref{fig:MultiplicationSquare}.
%
%%\subsubsection The compositions $a\circ \mu_X$ and $a\circ T(a)$ are equal.
%\begin{figure}[h]
%\centering
%\begin{minipage}{0.45\textwidth}
% \centering
%\begin{tikzcd}
%    T^2(X) \arrow{r}{T(a)} \arrow[swap]{d}{\mu_X} & T(X) \arrow{d}{a} \\
%    T(X) \arrow{r}{a}& X
% \end{tikzcd}
%\end{minipage}
% \begin{minipage}{0.45\textwidth}
% \centering
%\begin{tikzcd}
%    T(X) \arrow{r}{a} \arrow[swap]{dr}{1_{T(X)}} & X\arrow{d}{\eta} \\
%    & T(X)
%  \end{tikzcd}
%  \end{minipage}
%  \caption{Multiplication square (left) and unit triangle (right).}
%\label{fig:MultiplicationSquare}
%\end{figure}
%Given a state structure $S$, we can use the State monad .
%\begin{align}
%State_S(X) = (S\times X)^S 
%\end{align}
%
%\begin{align}
%\eta_X:X\rightarrow  (S\times X)^S\\
%\eta_X(x)= \lambda s \rightarrow (s,x)
%\end{align}
%
%%\begin{align}
%%>>=\colon  (S\rightarrow (T,S))\rightarrow (T\rightarrow (S\rightarrow (Y,S)) \rightarrow  (S\rightarrow (Y,S))\\
%%m >>= f = \lambda s \rightarrow \text{let $(t,s')=m(s)$ in $f(t)(s')$} 
%%\end{align}
%
%\begin{align}
%\mu_X:(S\times (S\times X)^S)^S\rightarrow  (S\times X)^S\\
%\mu_X(\mathbf{x})= \lambda s \rightarrow \text{let $(s',f)=\mathbf{x}(s)$ in f(s')}
%\end{align}
\todo[inline]{THERE IS NO NEED FOR MONADS AT THE MOMENT. Monads seem to be what maps a type $X$ to an initial/final $F$-coalgebra of a functor where $X$ is constant}
\subsection{An Intuition of States and Time}
%\subsection{Catamorphisms, Anamorphisms, Final and Initial $F$-Coalgebras}
\subsection{Anamorphisms and Final $F$-Coalgebras}
Formally, for the functor $F=O\times \texttt{id}^I$, given an $F$-coalgebra $\mathbb{X}=(X,(o,\delta))$, the behaviour of a state $x\in X$ is $\TheBehaviourOfIn{x}{\mathbb{X}}$, an element of the set $O^{I^*}$. 
\subsection{}

\section{Coinduction}
\subsection{Coinduction Proof Principle}
\todo[inline]{Coinductive Definitions?}
\subsection{Bisimulation}
% \section{Spectre}
% \label{sec:Preliminaries:Spectre}
% \todo[inline]{And Meltdown?}
% \subsection{Spectre V1}
% %\todo[inline]{}
% \subsection{Spectre V2}
\section{Cyber-Physical System} 
\todo[inline]{This section probably makes more sense in its own chapter, when we explain what the latent behaviour of a CPS is.}
\section{Side Channels}