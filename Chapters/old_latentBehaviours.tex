
\paragraph{Space-Time Duality} 
We now define the set $m(X)\triangleq\{m(x)\ |\ x\in X\}$. We define a dynamical system $(m(X),m\circ c\colon m(X)\rightarrow m(X))$, whose associated set of behaviours is $(m\circ c)_{m(X)}^\infty$.

\begin{figure}[t]
    \centering
    \begin{tikzcd}%[column sep=1.75cm, row sep=1cm]
        (X\times m(X), (c \circ m,m \circ c))
            \arrow[d,swap,"\fst"]
            \arrow[r,swap,"\snd"]
        & (m(X), m \circ c)
            \arrow[d,swap,"m\circ c_{(\cdot)}^\infty"']
        \\ 
        (X,c \circ m)
            \arrow[r,swap,"(c\circ m)_{(\cdot)}^\infty"]
        &  (X^\infty, (\cdot)')
    \end{tikzcd} 
    \caption{Pullback relating $c\circ m$ and $m\circ c$, describing the space-time duality of dynamical systems defined by $\id$-coalgebras.}
\end{figure}

\begin{theorem}[Fundamental Theorem of Latent-Behaviour Analysis]
    \label{theo:Fundamental}
    The semantic map-ping of the $F$-coalgebra $(X,c\circ m\colon X\rightarrow F(X))$ factors through the semantic mapping of the $F$-coalgerba $(m(X),c\colon m(X)\rightarrow F(m(X)))$
    \todo[inline]{This probably needs to use bounded functors.}
\end{theorem}
\begin{proof}
    The Epi-mono factorisation from Universal Coalgebra uses the kernel of homomorphisms as the quotient. Here, we replace it by $\equiv_m$, enforcing bisimilarity between $x$ and $m(x)$. Note that $\equiv_{\id}$ results in normal bisimilarity, because no enforcement takes place.

    The transformation $m\colon X\rightarrow X$ forces the appearances of equivalence classes $\equiv_m\subseteq X\times X$ with respect to $c$, where $x_1\equiv_m x_2$ iff $m(x_1)\sim m(x_2)$, which overrides bisimilarity in $X$ with respect to $c$. The set $X/\equiv_m$ is isomorphic to $m(X)$. In other words, 
\end{proof}

\begin{figure}[t]
    \centering
    \includegraphics[width=1\textwidth]{Figures/Epi-monofactorisation.png}
    \caption{Reason why the fundamental theorem works.}
    \label{fig:LambdaCurves}
    \end{figure}

\begin{figure}[t]
    \centering
    \includegraphics[width=1\textwidth]{Figures/FundamentalTheo.png}
    \caption{Reason why the fundamental theorem works.}
    \label{fig:LambdaCurves}
    \end{figure}

\begin{proposition}[Behavioural Change]
For any dynamical system described by an $F$-coalgebra 
$(X,c\circ m\colon X\rightarrow F(X))$, where $m\colon X\rightarrow X$, and  $c\colon X\rightarrow F(X)$, there exists a \emph{unique} transformation function $\Delta_m\colon \sigma F\rightarrow \sigma F$ which maps $\TheBehaviourOfIn{x}{c}$ to $\TheLatentBehaviourOfIn{x}{c}{m}$, describing the \emph{behavioural change} of $x$, for all $x\in X$.
\end{proposition}
\begin{proof}
    The function $\Delta_{m}\colon \sigma F\rightarrow \sigma F$ is an $F$-coalgebra since $F(\sigma F)\simeq \sigma F$, so it has a behaviour. 
coinductively defined by 
{\color{red}
\todo[inline]{Transform this into a commutative diagram?}
\begin{align*}
    \Delta_{m}(\TheBehaviourOfIn{x}{c}) \sim_{(c \circ m)} (x)\\
    %\Delta_{m}(c^\infty_x)[1]&\triangleq c^\infty_{m(x)}[1]\\
    \Delta_{m}(c^\infty_x)[t+1]&\triangleq \Delta_{m}\left(c^\infty_{(c\circ m)(x)}\right)[t]
\end{align*}
}
\end{proof}

\section{Dynamical Systems of the $\id$ Functor}
% Theorem~\ref{theo:Fundamental} is a far stretch from the fundamental theorem of arithmetic, which states that every natural number can be decomposed into a multiplication of primes, but it follows a similar reasoning. Normally, we ignore this decomposition, so $\delta=c$ and $m=\id$. 
% \todo[inline]{It would be nice to have ``primes" here, and they may exist. They'll probably }



The \emph{fundamental theorem of latent behaviour analysis} implies the existence of unique transformation function $\Delta_m\colon X^\infty\rightarrow X^\infty$ which maps $c_{x}^\infty$ to both $\delta_{x}^\infty$ and $c_{m(x)}^\infty$ for all $x\in X$. The function $\Delta_m$ models the behavioural change induced by $m$ when applied to the $F$-coalgebra $(X,c)$. This unique function $\delta_{x}^\infty$ and $c_{m(x)}^\infty$ is 
coinductively defined by 
% \todo[inline]{BE CAREFUL HOW YOU DEFINE THE INDICES and if you want behaviours to include $x$, in coalgberas that is not very common.}
\begin{align*}
    \Delta_{m}(c^\infty_x)[0]&\triangleq (c \circ m)(x)\\
    %\Delta_{m}(c^\infty_x)[1]&\triangleq c^\infty_{m(x)}[1]\\
    \Delta_{m}(c^\infty_x)[t+1]&\triangleq \Delta_{m}\left(c^\infty_{(c\circ m)(x)}\right)[t]
\end{align*}
This definition compounds the effect of $m$ over time.
% \begin{align}
%     \Delta_m(c_{x}^\infty)\triangleq(m\circ c)_{x}^\infty
% \end{align}
% which is not a very useful definition. However, if $m$ satisfies some linearity properties, then this function $\Delta_m(c_{x}^\infty)$ can be defined (co)inductively.

For example, consider the system $(\mathbb{N},c=(+1))$ and the operator $m=(*3)$. The corresponding definition is $\Delta_{m}$ by
\begin{align*}
    \Delta_{m}(c^\infty_x)[0]&=1\\%(+1)^\infty_{2x}[1]\\
    (\Delta_{m}(c^\infty_x))[t+1]&=\Delta_{m}(c^\infty_{3x+1})[t]
\end{align*}
%This definition compounds the effect of the operator $(*2)$. 
From $(1,2,3,\ldots)=(+1)^\infty_0$, we obtain the sequence $(1,7,13,40,121,\ldots)$. 
% Now, $\Delta_{(*2)}(\omega)[0]=0$. %, so $((+1)\circ (*2))^\infty_0[0]=0$. 
% Next, $\Delta_{(*2)}(((+1)\circ (*2))^\infty_{1})[0]=1$

%$\Delta_{(*2)}(\omega)[t+1]$ since $\omega=(+1)^\infty_{(2*0)+1}$
\begin{definition}
The operator $m$ is \emph{linear with respect to $c$} if and only if
\begin{align}
    (c\circ m)^\infty_x[t]= (m \circ c)^\infty_x[t].
\end{align}    
\end{definition}

%Functorial operators are quite rare. It means that their effects do not compound over time. 
% (Linear operators have a deep relation to bialgebras of the identity functor, where $c\circ m= m\circ c $.)
% The operator $(*2)$ is not linear with respect to $(+1)$, but the operator $+3$ is. $(0,4,8,)$.

Depending on the properties of $m$ and $\Delta_m$, we might infer whether some behavioural property was preserved for all sequences, i.e., if $P(c^\infty_x)$, then $P((c\circ m)^\infty_x)$. For example, for $m=(*3)$, and $c=(+1)$, the sequence is still strictly increasing. 



%We say that thefunction $\Delta^m$ has a solution if it can be written








THe usefulness of latent behaviour analysis is that you can decompose a behaviour-defining into several components. Each of those components 



\todo[inline]{Latent coalgebras are coalgebras with the state space deformed.}
\todo[inline]{THESE DIAGRAMS ARE WRONG. THE BEH OF C CANNOT APPEAR LIKE THISOR YOU CAN GO VIA M to the end of c}
\end{definition}
\begin{figure}
    \centering
    \begin{tikzcd}[column sep=large]
        \sigma F
            \arrow[d, "\simeq","\omega"'] 
        &X
            \arrow[r, "m"]
            %\arrow[rd, "c\circ m", red]
            \arrow[l, dotted, swap,"\TheBehaviourOf{\cdot}_{c\circ m}"]
        &X 
            \arrow[r, dotted, "\TheBehaviourOf{\cdot}_c"] 
            \arrow[d, "c"] 
        & \sigma F 
            \arrow[d, "\simeq","\omega"'] 
        \\
        F(\sigma F)
        &
        &F(X) 
            \arrow[r, dotted, "F(\TheBehaviourOf{\cdot}_c)"]
            \arrow[ll, dotted, swap,"F(\TheBehaviourOf{\cdot}_{c\circ m})"]     
        &F(\sigma F)
    \end{tikzcd}
    \caption{$(X,c\circ m)$ is an $F$-coalgebra, so it has a unique $F$-homomorphism to the final $F$-coalgebra $(\sigma F, \omega)$. Geometrically, $m$ is a transformation of the state space, so some preserve certain structural properties (e.g. a permutation) while others do not (e.g. a collapsing function to some fixed $x$).}
\end{figure}
\begin{figure}
    \centering
    \begin{tikzcd}[column sep=large]
        \sigma F
            \arrow[d, "\simeq","\omega"'] 
        &&X
            \arrow[r, "m"]
            \arrow[d, "b\circ c\circ m"] 
            %\arrow[rd, "c\circ m", red]
            \arrow[ll, dotted, swap,"\TheBehaviourOf{\cdot}_{b\circ c\circ m}"]
        &X 
            \arrow[rr, dotted, "\TheBehaviourOf{\cdot}_c"] 
            \arrow[d, "c"] 
        && \sigma F 
            \arrow[d, "\simeq","\omega"'] 
        \\
        F(\sigma F) 
        &&F(X) \arrow[ll, dotted, swap,"F(\TheBehaviourOf{\cdot}_{b\circ c\circ m})"]
        &F(X) 
            \arrow[rr, dotted, "F(\TheBehaviourOf{\cdot}_c)"]
            \arrow[l, "b"]     
        &&F(\sigma F)
    \end{tikzcd}
    \caption{Latency using both $m$ and $b$ to reveal latent behaviours: $(X,b\circ c\circ m)$ is still an $F$-coalgebra, so it has a unique $F$-homomorphism to the final $F$-coalgebra $(\sigma F, \omega)$. Metaphorically, consider the function $c$ as a causal model which transforms the cause $x$ into the consequence $c(x)$; the transformation $m$ corresponds to a change of cause, while a transformation $b$ corresponds to a change of consequence. In a more concrete example, assume that we are grading a test, and $c$ is the grading rules. The value $c(x)$ is the grade assigned to the answers $x$. A transformation $m$ would correspond to changing the answers before grading, while a transformation $b$ corresponds to changing the grade after the answers have been graded. Now, if $F(X)$ is the The transformation $b$ needs not respect the properties of $c$, since it is applied after it. Thus, it would be possible to change the grade through $b$ to some value that is impossible to obtain by any possible combination of answers (i.e. for every $m$ that changes answers).}
\end{figure}
\begin{figure}
    \centering
    \begin{tikzcd}[column sep=large]
        \sigma F
            \arrow[d, "\simeq","\omega"'] 
        &X
            \arrow[r, "m"]
            %\arrow[rd, "c\circ m", red]
            \arrow[l, dotted, swap,"\TheBehaviourOf{\cdot}_{c\circ m}"]
        &X 
            \arrow[r, dotted, "\TheBehaviourOf{\cdot}_c"] 
            \arrow[d, "c"] 
        & \sigma F 
            \arrow[d, "\simeq","\omega"'] 
        \\
        F(\sigma F)
        &
        &F(X) 
            \arrow[r, dotted, "F(\TheBehaviourOf{\cdot}_c)"]
            \arrow[ll, dotted, swap,"F(\TheBehaviourOf{\cdot}_{c\circ m})"]     
        &F(\sigma F)
    \end{tikzcd}
    \caption{$(X,T(X)\xrightarrow{c\circ a}F(X))$ is an $FT$-coalgebra, (not quite a bialgebra, but it could be). $T(X)$ defines specifications for $X$, so we could mutate those instead of $X$ directly. This is what the paper by Harrison and Goldstein does \cite{DoJudgeATestByItsCover}}
\end{figure}


{\color{red}The deformation $b$ is not that interesting because it is too flexible. If $F=\id$ then $t$ has the same type as $s$ and they compose, so we can approximate $t.c.s$ with $c.s'$, where $s'=s.t$. If $F(X)$ has only one component, then $t=const \phi$ forces the system to have the behaviour that the attacker wants. More precisely, it can reveal any $F$-coalgebra for that carrier set. The balance would be to allow the attacker of $t$ to influence only a set of components in $F(X)$, just like we do with $s$. Consider the $F$-coalgebras of $()$ for the functor $2x\id^2$; there are only two: $c_0(())=(False,const ())$ and $c_1(())=(True,const ())$. Using state transformations we cannot reveal new behaviours given $c_0$ or $c_1$, but with with behaviour transformations we can: $t=(\texttt{not},\id)$ causes $t.c_0=c_1$ and $t.c_1=c_0$ so you can strictly do more. The question is, do we need more? 

Poetically, $m$ corresponds to a deformation of space, and $b$ corresponds to a deformation of causality.
}

\begin{example}
\label{ex:LatentBehaviour}
Consider the functor $G=2\times \texttt{id}^2$ and a $G$-coalgebra $\mathbb{X}=(2\times2,(\gamma,\delta))$ 
\todo[inline]{Match with the right automaton}
. We define said $G$-coalgebra $\mathbb{X}$ by
\begin{align}
\gamma(x,y)&\triangleq x \land y;\\
\delta(x,y)(i)&\triangleq(i,x).
\end{align}
Figure~\ref{fig:ExampleLatent} shows the deterministic finite automaton corresponding to this $G$-coalgebra; the coalgebra is not minimal, because the states $(0,0)$ and $(0,1)$ are bisimilar.
% \begin{figure}[t]
% \centering
% \begin{tikzpicture}
% \node[state] (00) {$(0,0)$};
% \node[state, below right of=00] (01) {$(0,1)$};
% \node[state, above right  of=00] (10) {$(1,0)$};
% \node[state, accepting, below right of=10] (11) {$(1,1)$};
% \draw (00) edge[bend left, above] node{1} (10)
% (00) edge[loop above] node{0} (00)
% (01) edge[bend left, above] node[left]{1} (10)
% (01) edge[bend left, above] node{0} (00)
% (10) edge[bend left, above] node{1} (11)
% (10) edge[bend left, above] node[right]{0} (01)
% (11) edge[loop above] node{1} (11)
% (11) edge[bend left, above] node{0} (01)
% ;\end{tikzpicture}
% \caption{Corresponding automaton to the $G-$coalgebra described in Example~\ref{ex:LatentBehaviour}.}
% \label{fig:ExampleLatent}
% \end{figure}

This $G$-coalgebra has 256 transformations, given by the $4^4$ endofunctions in $X$. We use these transformations to reveal latent behaviours.%, and every consistent transformation $m$ satisfies $m(0,0)~m(0,1)$. 
\todo[inline]{You can probably graph this with a 3 dimensional graph like a vector field whose axises are input, state, and output. We could use four dimensions so that state is 2-dimensional}
% We present the automata that correspond to latent $G-$coalgebras in Figures~\ref{fig:3.2}--\ref{fig:3.28}. Note that many of these latent coalgebras are not minimal, and many display the same behaviours.

% From 27 endofuctions and three intended behaviours, we can derive the 24 latent behaviours presented in Table~\ref{tab:ExampleLatentBehaviours} as regular expressions. 
{\color{red}
\todo[inline]{Rewrite this!!!}
The existence of latent behaviours opens a new possibility for system repurposing: if we wanted the system to recognise the language $(0+1)^*1$ (i.e., the language of sequences that end in 1), we could mutate our original system using a transformation where $0\mapsto0, 1\mapsto2$ and $2\mapsto2$ (shown in Figure~\ref{fig:3.10}), whilst preserving $0$ as the initial state. Nevertheless, if providing the transformation is within the capabilities of an adversary, it would also mean that they can repurpose our system as well. 
}


\begin{figure}[t]
\centering
\begin{tikzpicture}
    \node[state] (00) {$(0,0)$};
    \node[state, below right of=00] (01) {$(0,1)$};
    \node[state, above right  of=00] (10) {$(1,0)$};
    \node[state, accepting, below right of=10] (11) {$(1,1)$};
    \draw (00) edge[bend left, above] node{1} (10)
    (00) edge[loop above] node{0} (00)
    (01) edge[bend left, above] node{1} (11)
    (01) edge[loop above] node{0} (01)
    (10) edge[loop above] node{1} (10)
    (10) edge[bend left, above] node{0} (00)
    (11) edge[loop above] node{1} (11)
    (11) edge[bend left, above] node{0} (01)
    ;\end{tikzpicture}
\caption{Corresponding automaton to the latent $G-$coalgebra under transformation $(a,b)\mapsto(b,a)$. This transformation is not consistent, since $(0,0)\sim (0,1)$ in the original $F$-coalgebra, but $(0,0)$ and $(1,0)$ are no longer bisimilar}% described in Example~\ref{ex:LatentBehaviour}.}
\label{fig:ExampleInconsistentLatent}
\end{figure}

\begin{figure}[t]
\centering
\begin{tikzpicture}
    \node[state] (00) {$(0,0)$};
    \node[state, below right of=00] (01) {$(0,1)$};
    \node[state, accepting, above right  of=00] (10) {$(1,0)$};
    \node[state, accepting, below right of=10] (11) {$(1,1)$};
    \draw (00) edge[bend left, above] node{1} (10)
    (00) edge[loop above] node{0} (00)
    (01) edge[bend left, above] node{1} (10)
    (01) edge[bend left, above] node{0} (00)
    (10) edge[bend left, above] node{1} (11)
    (10) edge[bend left, above] node{0} (01)
    (11) edge[loop above] node{1} (11)
    (11) edge[bend left, above] node{0} (01)
    ;\end{tikzpicture}
\caption{Corresponding automaton to the latent $G-$coalgebra under transformation $(0,x)\mapsto (0,0)$ and $(1,x)\mapsto (1,1)$. This transformation is consistent, and reveals a new behaviour from the state $(0,0)$, more precisely, the language $(0,1)^*1$ of sequences that end in $1$.}% described in Example~\ref{ex:LatentBehaviour}.}
\label{fig:ExampleConsistentLatent}
\end{figure} 
    
% \begin{table}[t]
% \centering
% \begin{tabular}{|l | l | }
% \hline
% $\rho_{\ref{fig:3.2}.0}$ &  $\emptyset$   \\
% $\rho_{\ref{fig:3.4}.2}$ &  $1^*$ \\
% $\rho_{\ref{fig:3.7}.0}$ &  $(0+1)^*111^*$\\%$(0+10+111^*0)^*111^*$ \\
% $\rho_{\ref{fig:3.7}.1}$ &  $\rho_{\ref{fig:3.7}.0}+1$ \\
% $\rho_{\ref{fig:3.7}.2}$ &  $\rho_{\ref{fig:3.7}.0}+1+\varepsilon$ \\
% $\rho_{\ref{fig:3.8}.1}$ &  $(00^*1+10^*1)^*$ \\
% $\rho_{\ref{fig:3.8}.0}$ &  $0^*1\rho_{\ref{fig:3.8}.1}$ \\
% $\rho_{\ref{fig:3.9}.2}$& $(0+1)^*01$\\
% $\rho_{\ref{fig:3.9}.0}$&   $\rho_{\ref{fig:3.9}.2}+1$\\
% $\rho_{\ref{fig:3.9}.1}$&   $\rho_{\ref{fig:3.9}.2}+1+\varepsilon$\\
% $\rho_{\ref{fig:3.10}.0}$ &  $(0+1)^*1$ \\
% $\rho_{\ref{fig:3.10}.1}$ &  $\rho_{\ref{fig:3.10}.0}+\varepsilon$ \\
% $\rho_{\ref{fig:3.13}.1}$ &  $1^*0\rho_{\ref{fig:3.10}.0}$ \\
% $\rho_{\ref{fig:3.17}.0}$ &  $(0+10+(11)^+(0+10))^*(11)^+$ \\
% $\rho_{\ref{fig:3.17}.1}$ &  $(11)^*+(0+10)\rho_{\ref{fig:3.17}.0}$ \\
% $\rho_{\ref{fig:3.17}.2}$ &  $0\rho_{\ref{fig:3.17}.0}+1\rho_{\ref{fig:3.17}.1}$ \\
% $\rho_{\ref{fig:3.18}.0}$ &  $\varepsilon$ \\
% $\rho_{\ref{fig:3.20}.1}$ &  $(0+1)^*0$ \\
% $\rho_{\ref{fig:3.20}.0}$ &  $\rho_{\ref{fig:3.20}.1}+\varepsilon$ \\
% $\rho_{\ref{fig:3.22}.0}$ &  $(1+0)^*$ \\
% $\rho_{\ref{fig:3.22}.1}$ &  $1^*0(1+0)^*$ \\
% $\rho_{\ref{fig:3.25}.1}$ &  $1(1+0)^*+0(1+0)^*$ \\
% $\rho_{\ref{fig:3.26}.0}$ &  $(0+10+11)^*$ \\
% $\rho_{\ref{fig:3.26}.2}$ &  $(0+1)\rho_{\ref{fig:3.26}.0}$\\
% \hline
% \end{tabular}
% \caption{Latent behaviours for all transformations for the coalgebra that recognises $(0+1)^*11$. }
% \label{tab:ExampleLatentBehaviours}
% \end{table}

% %$\rho_{\ref{fig:3.8}.0}$ &  $$ \\
% %$\rho_{\ref{fig:3.8}.1}$ &  $$ \\
% %$\rho_{\ref{fig:3.8}.2}$ &  $$ \\
% %------------------------------------------------------------------------------------------------------------------------
% \begin{figure}[t]
% \centering
% \begin{tikzpicture}
% \node[state] (q1) {$0$};
% \node[state, right of=q1] (q2) {$1$};
% \node[state, right of=q2] (q3) {$2$};
% \draw (q1) edge[loop above] node{0} (q1)
% (q1) edge[bend left, above] node{1} (q2)
% (q2) edge[bend left, above] node{0} (q1)
% (q2) edge[loop above] node{1} (q2)
% (q3) edge[bend left, above] node{1} (q2)
% (q3) edge[bend left, below] node{0} (q1);
% \end{tikzpicture}
% \caption{Latent coalgebra under graph $G(m)=\set{(0,0),(1,0),(2,0)}$. $\TheLatentBehaviourOf{0}{m}=\TheLatentBehaviourOf{1}{m}=\TheLatentBehaviourOf{2}{m}=\emptyset$ }
% \label{fig:3.2}

% \begin{tikzpicture}
% \node[state] (q1) {$0$};
% \node[state, right of=q1] (q2) {$1$};
% \node[state, right of=q2] (q3) {$2$};
% \draw (q1) edge[loop above] node{0} (q1)
% (q1) edge[bend left, above] node{1} (q2)
% (q2) edge[bend left, above] node{0} (q1)
% (q2) edge[loop above] node{1} (q2)
% (q3) edge[loop above] node{1} (q3)
% (q3) edge[bend left, below] node{0} (q1);
% \end{tikzpicture}
% \caption{Latent coalgebra under graph $G(m)=\set{(0,0),(1,0),(2,1)}$. $\TheLatentBehaviourOf{0}{m}=\TheLatentBehaviourOf{1}{m}=\TheLatentBehaviourOf{2}{m}=\emptyset$}
% \label{fig:3.3}

% \begin{tikzpicture}
% \node[state] (q1) {$0$};
% \node[state, right of=q1] (q2) {$1$};
% \node[state, accepting, right of=q2] (q3) {$2$};
% \draw (q1) edge[loop above] node{0} (q1)
% (q1) edge[bend left, above] node{1} (q2)
% (q2) edge[bend left, above] node{0} (q1)
% (q2) edge[loop above] node{1} (q2)
% (q3) edge[loop above] node{1} (q3)
% (q3) edge[bend left, below] node{0} (q1);
% \end{tikzpicture}
% \caption{Latent coalgebra under graph $G(m)=\set{(0,0),(1,0),(2,2)}$. $\TheLatentBehaviourOf{0}{m}=\TheLatentBehaviourOf{1}{m}=\emptyset$, $\TheLatentBehaviourOf{2}{m}=1^*$}
% \label{fig:3.4}
% \end{figure}

% \begin{figure}[t]
% \captionsetup{singlelinecheck=off}
% \centering
% \begin{tikzpicture}
% \node[state] (q1) {$0$};
% \node[state, right of=q1] (q2) {$1$};
% \node[state, right of=q2] (q3) {$2$};
% \draw (q1) edge[loop above] node{0} (q1)
% (q1) edge[bend left, above] node{1} (q2)
% (q2) edge[bend left, above] node{0} (q1)
% (q2) edge[bend left, above] node{1} (q3)
% (q3) edge[bend left, above] node{1} (q2)
% (q3) edge[bend left, below] node{0} (q1);
% \end{tikzpicture}
% \caption{Latent coalgebra under graph $G(m)=\set{(0,0),(1,1),(2,0)}$ . $\TheLatentBehaviourOf{0}{m}=\TheLatentBehaviourOf{1}{m}=\TheLatentBehaviourOf{2}{m}=\emptyset$}

% \begin{tikzpicture}
% \node[state] (q1) {$0$};
% \node[state, right of=q1] (q2) {$1$};
% \node[state, right of=q2] (q3) {$2$};
% \draw (q1) edge[loop above] node{0} (q1)
% (q1) edge[bend left, above] node{1} (q2)
% (q2) edge[bend left, above] node{0} (q1)
% (q2) edge[bend left, above] node{1} (q3)
% (q3) edge[loop above] node{1} (q3)
% (q3) edge[bend left, below] node{0} (q1);
% \end{tikzpicture}
% \caption{Latent coalgebra under graph $G(m)=\set{(0,0),(1,1),(2,1)}$ . $\TheLatentBehaviourOf{0}{m}=\TheLatentBehaviourOf{1}{m}=\TheLatentBehaviourOf{2}{m}=\emptyset$}

% \begin{tikzpicture}
% \node[state] (q1) {$0$};
% \node[state, right of=q1] (q2) {$1$};
% \node[state,accepting, right of=q2] (q3) {$2$};
% \draw (q1) edge[loop above] node{0} (q1)
% (q1) edge[bend left, above] node{1} (q2)
% (q2) edge[bend left, above] node{0} (q1)
% (q2) edge[bend left, above] node{1} (q3)
% (q3) edge[loop above] node{1} (q3)
% (q3) edge[bend left, below] node{0} (q1);
% \end{tikzpicture}
% \caption{Latent coalgebra under graph $G(m)=\set{(0,0),(1,1),(2,2)}$ (i.e. $m=\texttt{id}$). Latent behaviours under $m$ are the original behaviours: 
% \protect\begin{align*}
% 	\TheLatentBehaviourOf{0}{m}&=(0+1)^*111^*, \\
% 	\TheLatentBehaviourOf{1}{m}&=1+\TheLatentBehaviourOf{0}{m}\\
% 	\TheLatentBehaviourOf{2}{m}&=\varepsilon + 1+ \TheLatentBehaviourOf{0}{m}
% \protect\end{align*}
% }
% \label{fig:3.7}
% \end{figure}



% \begin{figure}[t]
% \captionsetup{singlelinecheck=off}
% \centering
% \begin{tikzpicture}
% \node[state] (q1) {$0$};
% \node[state,accepting, right of=q1] (q2) {$1$};
% \node[state, right of=q2] (q3) {$2$};
% \draw (q1) edge[loop above] node{0} (q1)
% (q1) edge[bend left, above] node{1} (q2)
% (q2) edge[bend left, above] node{0} (q1)
% (q2) edge[bend left, above] node{1} (q3)
% (q3) edge[bend left, above] node{1} (q2)
% (q3) edge[bend left, below] node{0} (q1);
% \end{tikzpicture}
% \caption{Latent coalgebra under graph $G(m)=\set{(0,0),(1,2),(2,0)}$. 
% \protect\begin{align*}
% \TheLatentBehaviourOf{0}{m}&=\TheLatentBehaviourOf{2}{m}=0^*1\TheLatentBehaviourOf{1}{m},\\
% \TheLatentBehaviourOf{1}{m}&=(00^*1+10^*1)^*
% \protect\end{align*}
% \label{fig:3.8}
% }

% \begin{tikzpicture}
% \node[state] (q1) {$0$};
% \node[state,accepting, right of=q1] (q2) {$1$};
% \node[state, right of=q2] (q3) {$2$};
% \draw (q1) edge[loop above] node{0} (q1)
% (q1) edge[bend left, above] node{1} (q2)
% (q2) edge[bend left, above] node{0} (q1)
% (q2) edge[bend left, above] node{1} (q3)
% (q3) edge[loop above] node{1} (q3)
% (q3) edge[bend left, below] node{0} (q1);
% \end{tikzpicture}
% \caption{Latent coalgebra under graph $G(m)=\set{(0,0),(1,2),(2,1)}$.
% \protect\begin{align*}
% \TheLatentBehaviourOf{0}{m}&=1+\TheLatentBehaviourOf{2}{m},\\
% \TheLatentBehaviourOf{1}{m}&=\varepsilon+1+\TheLatentBehaviourOf{2}{m},\\
% \TheLatentBehaviourOf{2}{m}&=(0+1)^*01
% \protect\end{align*}
% \label{fig:3.9}
% }

% \begin{tikzpicture}
% \node[state] (q1) {$0$};
% \node[state,accepting, right of=q1] (q2) {$1$};
% \node[state, accepting,right of=q2] (q3) {$2$};
% \draw (q1) edge[loop above] node{0} (q1)
% (q1) edge[bend left, above] node{1} (q2)
% (q2) edge[bend left, above] node{0} (q1)
% (q2) edge[bend left, above] node{1} (q3)
% (q3) edge[loop above] node{1} (q3)
% (q3) edge[bend left, below] node{0} (q1);
% \end{tikzpicture}
% \caption{Latent coalgebra under graph $G(m)=\set{(0,0),(1,2),(2,2)}$.
% \protect\begin{align*}
% \TheLatentBehaviourOf{0}{m}&=(0+1)^*1,\\ 
% \TheLatentBehaviourOf{1}{m}&=\TheLatentBehaviourOf{2}{m}=\varepsilon + \TheLatentBehaviourOf{0}{m}
% \protect\end{align*}
% \label{fig:3.10}
% }
% \end{figure}

% \begin{figure}[t]
% \captionsetup{singlelinecheck=off}
% \centering
% \begin{tikzpicture}
% \node[state] (q1) {$0$};
% \node[state, right of=q1] (q2) {$1$};
% \node[state, right of=q2] (q3) {$2$};
% \draw (q1) edge[loop above] node{0} (q1)
% (q1) edge[bend left, above] node{1} (q3)
% (q2) edge[bend left, above] node{0} (q1)
% (q2) edge[loop right] node{1} (q2)
% (q3) edge[bend left, above] node{1} (q2)
% (q3) edge[bend left, below] node{0} (q1);
% \end{tikzpicture}
% \caption{Latent coalgebra under graph $G(m)=\set{(0,1),(1,0),(2,0)}$. $\TheLatentBehaviourOf{0}{m}=
% \TheLatentBehaviourOf{1}{m}=\TheLatentBehaviourOf{2}{m}=\emptyset$}

% \begin{tikzpicture}
% \node[state] (q1) {$0$};
% \node[state, right of=q1] (q2) {$1$};
% \node[state, right of=q2] (q3) {$2$};
% \draw (q1) edge[loop above] node{0} (q1)
% (q1) edge[bend left, above] node{1} (q3)
% (q2) edge[bend left, above] node{0} (q1)
% (q2) edge[loop right] node{1} (q2)
% (q3) edge[loop above] node{1} (q3)
% (q3) edge[bend left, below] node{0} (q1);
% \end{tikzpicture}
% \caption{Latent coalgebra under graph $G(m)=\set{(0,1),(1,0),(2,1)}$. $\TheLatentBehaviourOf{0}{m}=
% \TheLatentBehaviourOf{1}{m}=\TheLatentBehaviourOf{2}{m}=\emptyset$}


% \begin{tikzpicture}
% \node[state] (q1) {$0$};
% \node[state, right of=q1] (q2) {$1$};
% \node[state,accepting, right of=q2] (q3) {$2$};
% \draw (q1) edge[loop above] node{0} (q1)
% (q1) edge[bend left, above] node{1} (q3)
% (q2) edge[bend left, above] node{0} (q1)
% (q2) edge[loop right] node{1} (q2)
% (q3) edge[loop above] node{1} (q3)
% (q3) edge[bend left, below] node{0} (q1);
% \end{tikzpicture}
% \caption{Latent coalgebra under graph $G(m)=\set{(0,1),(1,0),(2,2)}$. 
% \protect\begin{align*}
% \TheLatentBehaviourOf{0}{m}&=(0+1)^*1,\\ 
% \TheLatentBehaviourOf{1}{m}&=1^*0\TheLatentBehaviourOf{0}{m},\\
% \TheLatentBehaviourOf{2}{m}&=\varepsilon+\TheLatentBehaviourOf{0}{m}
% %\TheLatentBehaviourOf{2}{m}&=1^*+\TheLatentBehaviourOf{1}{m}
% \protect\end{align*}
% }
% \label{fig:3.13}
% \end{figure}

% \begin{figure}[t]
% \captionsetup{singlelinecheck=off}
% \centering
% \begin{tikzpicture}
% \node[state] (q1) {$0$};
% \node[state, right of=q1] (q2) {$1$};
% \node[state, right of=q2] (q3) {$2$};
% \draw (q1) edge[loop above] node{0} (q1)
% (q1) edge[bend left, above] node{1} (q3)
% (q2) edge[bend left, above] node{0} (q1)
% (q2) edge[bend left, above] node{1} (q3)
% (q3) edge[bend left, above] node{1} (q2)
% (q3) edge[bend left, below] node{0} (q1);
% \end{tikzpicture}
% \caption{Latent coalgebra under graph $G(m)=\set{(0,1),(1,1),(2,0)}$. $\TheLatentBehaviourOf{0}{m}=
% \TheLatentBehaviourOf{1}{m}=\TheLatentBehaviourOf{2}{m}=\emptyset$}


% \begin{tikzpicture}
% \node[state] (q1) {$0$};
% \node[state, right of=q1] (q2) {$1$};
% \node[state, right of=q2] (q3) {$2$};
% \draw (q1) edge[loop above] node{0} (q1)
% (q1) edge[bend left, above] node{1} (q3)
% (q2) edge[bend left, above] node{0} (q1)
% (q2) edge[bend left, above] node{1} (q3)
% (q3) edge[loop above] node{1} (q3)
% (q3) edge[bend left, below] node{0} (q1);
% \end{tikzpicture}
% \caption{Latent coalgebra under graph $G(m)=\set{(0,1),(1,1),(2,1)}$. $\TheLatentBehaviourOf{0}{m}=
% \TheLatentBehaviourOf{1}{m}=\TheLatentBehaviourOf{2}{m}=\emptyset$}


% \begin{tikzpicture}
% \node[state] (q1) {$0$};
% \node[state, right of=q1] (q2) {$1$};
% \node[state, accepting, right of=q2] (q3) {$2$};
% \draw (q1) edge[loop above] node{0} (q1)
% (q1) edge[bend left, above] node{1} (q3)
% (q2) edge[bend left, above] node{0} (q1)
% (q2) edge[bend left, above] node{1} (q3)
% (q3) edge[loop above] node{1} (q3)
% (q3) edge[bend left, below] node{0} (q1);
% \end{tikzpicture}
% \caption{Latent coalgebra under graph $G(m)=\set{(0,1),(1,1),(2,2)}$.
% \protect\begin{align*}
% \TheLatentBehaviourOf{0}{m}&=\TheLatentBehaviourOf{1}{m}=(0+1)^*1,\\ 
% \TheLatentBehaviourOf{2}{m}&=\TheLatentBehaviourOf{0}{m}+\varepsilon
% \protect\end{align*}
% }
% \end{figure}

% \begin{figure}[t]
% \centering
% \captionsetup{singlelinecheck=off}
% \begin{tikzpicture}
% \node[state] (q1) {$0$};
% \node[state,accepting, right of=q1] (q2) {$1$};
% \node[state, right of=q2] (q3) {$2$}; 
% \draw (q1) edge[loop above] node{0} (q1)
% (q1) edge[bend left, above] node{1} (q3)
% (q2) edge[bend left, above] node{0} (q1)
% (q2) edge[bend left, above] node{1} (q3)
% (q3) edge[bend left, above] node{1} (q2)
% (q3) edge[bend left, below] node{0} (q1);
% \end{tikzpicture}
% \caption{Latent coalgebra under graph $G(m)=\set{(0,1),(1,2),(2,0)}$.
% \protect\begin{align*}
% \TheLatentBehaviourOf{0}{m}&=(0+10+(11)^+(0+10))^*(11)^+ \\ %(0+1(0+1(11)^*(0+10)))^*11(11)^* ,\\ 
% \TheLatentBehaviourOf{1}{m}&=(11)^*+(0+10)\TheLatentBehaviourOf{0}{m},\\
% \TheLatentBehaviourOf{2}{m}&=0\TheLatentBehaviourOf{0}{m}+1\TheLatentBehaviourOf{1}{m}
% \protect\end{align*}
% \label{fig:3.17}
% }

% \begin{tikzpicture}
% \node[state] (q1) {$0$};
% \node[state,accepting, right of=q1] (q2) {$1$};
% \node[state, right of=q2] (q3) {$2$};
% \draw (q1) edge[loop above] node{0} (q1)
% (q1) edge[bend left, above] node{1} (q3)
% (q2) edge[bend left, above] node{0} (q1)
% (q2) edge[bend left, above] node{1} (q3)
% (q3) edge[loop above] node{1} (q3)
% (q3) edge[bend left, below] node{0} (q1);
% \end{tikzpicture}
% \caption{Latent coalgebra under graph $G(m)=\set{(0,1),(1,2),(2,1)}$.
% \protect\begin{align*}
% \TheLatentBehaviourOf{0}{m}&=\TheLatentBehaviourOf{2}{m}=\emptyset,\\ 
% \TheLatentBehaviourOf{1}{m}&=\varepsilon
% \protect\end{align*}
% \label{fig:3.18}
% }

% \begin{tikzpicture}
% \node[state] (q1) {$0$};
% \node[state,accepting, right of=q1] (q2) {$1$};
% \node[state, accepting, right of=q2] (q3) {$2$};
% \draw (q1) edge[loop above] node{0} (q1)
% (q1) edge[bend left, above] node{1} (q3)
% (q2) edge[bend left, above] node{0} (q1)
% (q2) edge[bend left, above] node{1} (q3)
% (q3) edge[loop above] node{1} (q3)
% (q3) edge[bend left, below] node{0} (q1);
% \end{tikzpicture}
% \caption{Latent coalgebra under graph $G(m)=\set{(0,1),(1,2),(2,2)}$.
% \protect\begin{align*}
% \TheLatentBehaviourOf{0}{m}&=(0+1)^*1,\\
% \TheLatentBehaviourOf{1}{m}&=\TheLatentBehaviourOf{2}{m}=\varepsilon+\TheLatentBehaviourOf{0}{m}
% \protect\end{align*}
% }
% \end{figure}

% \begin{figure}[t]
% \captionsetup{singlelinecheck=off}
% \centering
% \begin{tikzpicture}
% \node[state, accepting] (q1) {$0$};
% \node[state, right of=q1] (q2) {$1$};
% \node[state, right of=q2] (q3) {$2$};
% \draw (q1) edge[loop above] node{0} (q1)
% (q1) edge[bend left, above] node{1} (q3)
% (q2) edge[bend left, above] node{0} (q1)
% (q2) edge[loop right] node{1} (q2)
% (q3) edge[bend left, above] node{1} (q2)
% (q3) edge[bend left, below] node{0} (q1);
% \end{tikzpicture}
% \caption{Latent coalgebra under graph $G(m)=\set{(0,2),(1,0),(2,0)}$. 
% \protect\begin{align*}
% \TheLatentBehaviourOf{0}{m}&=(0+1)^*0+\varepsilon ,\\ 
% \TheLatentBehaviourOf{1}{m}&=(0+1)^*0
% \protect\end{align*} 
% \label{fig:3.20}
% }

% \begin{tikzpicture}
% \node[state, accepting] (q1) {$0$};
% \node[state, right of=q1] (q2) {$1$};
% \node[state, right of=q2] (q3) {$2$};
% \draw (q1) edge[loop above] node{0} (q1)
% (q1) edge[bend left, above] node{1} (q3)
% (q2) edge[bend left, above] node{0} (q1)
% (q2) edge[loop right] node{1} (q2)
% (q3) edge[loop above] node{1} (q3)
% (q3) edge[bend left, below] node{0} (q1);
% \end{tikzpicture}
% \caption{Latent coalgebra under graph $G(m)=\set{(0,2),(1,0),(2,1)}$.
% \protect\begin{align*}
% \TheLatentBehaviourOf{0}{m}&=(0+1)^*0+\varepsilon ,\\ 
% \TheLatentBehaviourOf{1}{m}&=(0+1)^*0
% \protect\end{align*}
% }

% \begin{tikzpicture}
% \node[state, accepting] (q1) {$0$};
% \node[state, right of=q1] (q2) {$1$};
% \node[state, accepting, right of=q2] (q3) {$2$};
% \draw (q1) edge[loop above] node{0} (q1)
% (q1) edge[bend left, above] node{1} (q3)
% (q2) edge[bend left, above] node{0} (q1)
% (q2) edge[loop right] node{1} (q2)
% (q3) edge[loop above] node{1} (q3)
% (q3) edge[bend left, below] node{0} (q1);
% \end{tikzpicture}
% \caption{Latent coalgebra under graph $G(m)=\set{(0,2),(1,0),(2,2)}$. 
% \protect\begin{align*}
% \TheLatentBehaviourOf{0}{m}&=\TheLatentBehaviourOf{2}{m}=(0+1)^*,\\ 
% \TheLatentBehaviourOf{1}{m}&=1^*0\TheLatentBehaviourOf{0}{m}
% \protect\end{align*}
% \label{fig:3.22}
% }
% \end{figure}

% \begin{figure}[t]
% \captionsetup{singlelinecheck=off}
% \centering
% \begin{tikzpicture}
% \node[state, accepting] (q1) {$0$};
% \node[state, right of=q1] (q2) {$1$};
% \node[state, right of=q2] (q3) {$2$};
% \draw (q1) edge[loop above] node{0} (q1)
% (q1) edge[bend left, above] node{1} (q3)
% (q2) edge[bend left, above] node{0} (q1)
% (q2) edge[bend left, above] node{1} (q3)
% (q3) edge[bend left, above] node{1} (q2)
% (q3) edge[bend left, below] node{0} (q1);
% \end{tikzpicture}
% \caption{Latent coalgebra under graph $G(m)=\set{(0,2),(1,1),(2,0)}$.
% \protect\begin{align*}
% \TheLatentBehaviourOf{0}{m}&=(0+1)^*0+\varepsilon,\\ 
% \TheLatentBehaviourOf{1}{m}&=\TheLatentBehaviourOf{2}{m}=(0+1)^*0\\
% \protect\end{align*}
% }

% \begin{tikzpicture}
% \node[state, accepting] (q1) {$0$};
% \node[state, right of=q1] (q2) {$1$};
% \node[state, right of=q2] (q3) {$2$};
% \draw (q1) edge[loop above] node{0} (q1)
% (q1) edge[bend left, above] node{1} (q3)
% (q2) edge[bend left, above] node{0} (q1)
% (q2) edge[bend left, above] node{1} (q3)
% (q3) edge[loop above] node{1} (q3)
% (q3) edge[bend left, below] node{0} (q1);
% \end{tikzpicture}
% \caption{Latent coalgebra under graph $G(m)=\set{(0,2),(1,1),(2,1)}$. 
% \protect\begin{align*}
% \TheLatentBehaviourOf{0}{m}&=(0+1)^*0+\varepsilon,\\ 
% \TheLatentBehaviourOf{1}{m}&=\TheLatentBehaviourOf{2}{m}=(0+1)^*0\\
% \protect\end{align*}
% }

% \begin{tikzpicture}
% \node[state, accepting] (q1) {$0$};
% \node[state, right of=q1] (q2) {$1$};
% \node[state, accepting, right of=q2] (q3) {$2$};
% \draw (q1) edge[loop above] node{0} (q1)
% (q1) edge[bend left, above] node{1} (q3)
% (q2) edge[bend left, above] node{0} (q1)
% (q2) edge[bend left, above] node{1} (q3)
% (q3) edge[loop above] node{1} (q3)
% (q3) edge[bend left, below] node{0} (q1);
% \end{tikzpicture}
% \caption{Latent coalgebra under graph $G(m)=\set{(0,2),(1,1),(2,2)}$ . 
% \protect\begin{align*}
% \TheLatentBehaviourOf{0}{m}&=\TheLatentBehaviourOf{2}{m}=(0+1)^*,\\ 
% \TheLatentBehaviourOf{1}{m}&=0\TheLatentBehaviourOf{0}{m}+1\TheLatentBehaviourOf{2}{m}
% \protect\end{align*}
% \label{fig:3.25}
% }
% \end{figure}

% \begin{figure}[t]
% \captionsetup{singlelinecheck=off}
% \centering
% \begin{tikzpicture}
% \node[state,accepting] (q1) {$0$};
% \node[state, accepting, right of=q1] (q2) {$1$};
% \node[state, right of=q2] (q3) {$2$};
% \draw (q1) edge[loop above] node{0} (q1)
% (q1) edge[bend left, above] node{1} (q3)
% (q2) edge[bend left, above] node{0} (q1)
% (q2) edge[bend left, above] node{1} (q3)
% (q3) edge[bend left, above] node{1} (q2)
% (q3) edge[bend left, below] node{0} (q1);
% \end{tikzpicture}
% \caption{Latent coalgebra under graph $G(m)=\set{(0,2),(1,2),(2,0)}$. 
% \protect\begin{align*}
% \TheLatentBehaviourOf{0}{m}&=\TheLatentBehaviourOf{1}{m}=(0+10+11)^*,\\ 
% \TheLatentBehaviourOf{2}{m}&=(0+1)\TheLatentBehaviourOf{0}{m}
% \protect\end{align*}
% \label{fig:3.26}
% }

% \begin{tikzpicture}
% \node[state,accepting] (q1) {$0$};
% \node[state, accepting, right of=q1] (q2) {$1$};
% \node[state, right of=q2] (q3) {$2$};
% \draw (q1) edge[loop above] node{0} (q1)
% (q1) edge[bend left, above] node{1} (q3)
% (q2) edge[bend left, above] node{0} (q1)
% (q2) edge[bend left, above] node{1} (q3)
% (q3) edge[loop above] node{1} (q3)
% (q3) edge[bend left, below] node{0} (q1);
% \end{tikzpicture}
% \caption{Latent coalgebra under graph $G(m)=\set{(0,2),(1,2),(2,1)}$.
% \protect\begin{align*}
% \TheLatentBehaviourOf{0}{m}&=\TheLatentBehaviourOf{1}{m}=(0+1)^*0+\varepsilon\\ 
% \TheLatentBehaviourOf{2}{m}&=(0+1)^*0
% \protect\end{align*}
% }

% \begin{tikzpicture}
% \node[state,accepting] (q1) {$0$};
% \node[state, accepting, right of=q1] (q2) {$1$};
% \node[state, accepting, right of=q2] (q3) {$2$};
% \draw (q1) edge[loop above] node{0} (q1)
% (q1) edge[bend left, above] node{1} (q3)
% (q2) edge[bend left, above] node{0} (q1)
% (q2) edge[bend left, above] node{1} (q3)
% (q3) edge[loop above] node{1} (q3)
% (q3) edge[bend left, below] node{0} (q1);
% \end{tikzpicture}
% \caption{Latent coalgebra under graph $G(m)=\set{(0,2),(1,2),(2,2)}$. 
% \protect\begin{align*}
% \TheLatentBehaviourOf{0}{m}&=\TheLatentBehaviourOf{1}{m}=\TheLatentBehaviourOf{2}{m}=(0+1)^*
% \protect\end{align*}
% }
% \label{fig:3.28}
% \end{figure}

\end{example}
\todo[inline]{If we want $a$ to change with time there is no need to do anything fancy! We can enhance the carrier by doing $X'=X\times \mathbb{N}$ or $X'=X\times [X\rightarrow 2]$; with this, $X$ is enhanced by a natural number counter or a set of conditions to make the dynamics of the transformation coalgebra more interesting. Maybe there is even no need to do changes, it all depends on how informative $X$ is. let's see}

\section{Latent Vulnerabilities}
\todo[inline]{This should go on a different chapter}
Not all behaviours are latent for a given $F$-coalgebra. In Example~\ref{ex:LatentBehaviour}, we see that no transformation can yield the language $0^*$ as a latent behaviour. We see latent behaviours as targets for attackers, and we would like to know if there is a way for an attacker to cause the system to shift to a particular latent behaviour by means of a transformation. For that purpose, we introduce the definition of {latent vulnerability problems}.

\begin{definition}
A \emph{latent vulnerability problem} consists of a given a pointed $F$-coalgebra $\mathbb{X}=(X,c,x_0)$ and a given behaviour $\rho\in \sigma F$. To solve this problem, we need to either 1) find a transformation $m$ which proves that $\rho$ is the latent behaviour of $x_0$ under $m$ or 2) prove that no such $m$ can exist.
\end{definition}


\begin{align}
\gamma(m(x))&=\gamma_\omega(\rho)\\
\delta(m(x))(i)&\sim \delta_\omega(\rho)(i), \forall i\in I
\end{align}